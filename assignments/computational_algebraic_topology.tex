\documentclass{article}
\usepackage{assignment}
\renewcommand{\phi}{\varphi}
\begin{document}
\title{C3.9 Mini Project: Cosheaf Homology}
\author{Franz Miltz}
\date{\today}
\maketitle

We have seen sheaves on simplicial complexes and the corresponding
sheaf cohomology. We have also seen the persistent homology
arising from a filtration of a chain complex. We aim to unify
these ideas by dualising sheaves to cosheaves and defining
the persistent homology with values in a filtration of cosheaves.
It will then be possible to contemplate corresponding Morse
complexes whose persistent homology will agree with the previous
definition.

\section{Cosheaves and homology}

We begin by defining cosheaves. As sheaves on simplicial
complexes are simply functors, so are cosheaves. Throughout
this document, fix a simplicial complex $K$.

\begin{definition}
  A \emph{cosheaf} on $K$ is a functor
  $\mathscr C:(K,\geq)\to\FVect_{\mathbb{R}}$.
\end{definition}
Let us familiarise ourselves with the notion of a cosheaf by looking
at some examples:
\begin{example}
  \begin{itemize}
    \item Let $V\in\FVect_{\mathbb{R}}$. Then the constant cosheaf
      $\underline V_K$ sends each simplex to $V$ and each inclusion
      $\tau\hookrightarrow\tau'$ to the identity $V\to V$. Note that
      this is the same as the constant sheaf $\underline V_K$, hence
      the notation is consistent.
    \item Cosheaves may be used to emphasise or deemphasise certain
      subcomplexes of $K$. For example, if we have a subcomplex
      $K'\subseteq K$ then we may define a cosheaf to be
      $\underline{\mathbb{R}}_K$ on $K'$ and $\underline{0}_K$
      everywhere else. Explicitly, we define the characteristic
      cosheaf of $K'\subseteq K$ to be
      \begin{align*}
        \mathscr I_{K'}(\tau) := \begin{cases}
          \mathbb{R} & \text{if }\tau\in K' \\
          0 &\text{otherwise}
        \end{cases}
      \end{align*}
      with $\mathscr I_{K'}(\tau\geq\sigma) = \identity$ whenever $\tau\in K'$.
      Of course, if $K=K'$ then $\mathscr I_K=\underline{\mathbb{R}}_K$.
    \item Subcomplexes also enable us to extend cosheaves to larger
      domains. Consider once again a subcomplex $K'\subseteq K$
      and a cosheaf $\mathscr C$ on $K$. By restricting
      $\mathscr C$ to a functor
      $\restrict{\mathscr C}{K'}:(K',\geq)\to\FVect_{\mathbb{R}}$
      we obtain a cosheaf on $K'$.
  \end{itemize}
\end{example}

Throught fix a cosheaf $\mathscr C$ on $K$. The chain complex
associated to a cosheaf over a simplicial complex
is analogous to sheaves and sheaf cohomology:

\begin{definition}
  For $d\geq 0$,
  define the \emph{vector space of $d$-chains of $K$ with coefficients
  in $\mathscr C$} by
  \begin{align*}
    C_d(K;\mathscr C) = \bigoplus_{\tau\in K_d} \mathscr C(\tau).
  \end{align*}
  where $K_d = \left\lbrace{\sigma \in K : \dim \sigma = d}\right\rbrace$.
\end{definition}

Observe that this is a direct generalisation of the usual chain vector
spaces as the case $\mathscr C = \underline{\mathbb{R}}_K$ yields the
equality $C_d(K;\underline{\mathbb{R}}_K)=C_d(K)$. Of
course we are not truly interested in the chains but rather in the
chain complex that they are part of.

Before we define the boundary operator, a word of caution. We are
used to thinking of maps $C_d(K)\to C_{d-1}(K)$ as matrices
with coefficients in $\mathbb{R}$. It will be fruitful to
generalise this approach to maps $f : C_d(K;\mathscr C)\to C_{d-1}(K;\mathscr C)$. However, the component
$\restrict{f}{\sigma,\tau}:\mathscr C(\tau)\to\mathscr C(\sigma)$ is no longer a scalar, which may be thought of as a linear
map $\mathbb{R}\to\mathbb{R}$, but a linear map between more
general vector spaces. Components of the composition behave
exactly in the way one would expect them to: If we have
\begin{align*}
  \bigoplus_i U_i
  \xlongrightarrow{f} \bigoplus_j V_j
  \xlongrightarrow{g} \bigoplus_k W_k
\end{align*}
then the component $U_i\to W_k$ is
\begin{align*}
  \restrict{(g\circ f)}{k,i} = \sum_j \restrict{g}{k,j}\circ\restrict{f}{j,i}.
\end{align*}

\begin{definition}
  Let $\sigma,\tau\in K$. Then the $(\sigma,\tau)$
  \emph{boundary component} is the map
  \begin{align*}
    \partial_{\sigma,\tau} = [\sigma:\tau]\mathscr C(\tau\geq\sigma) : \mathscr C(\tau) \to \mathscr C(\sigma).
  \end{align*}
  The \emph{boundary map}
  \begin{align*}
    \partial : C_d(K;\mathscr C) \to C_{d-1}(K;\mathscr C)
  \end{align*}
  is the unique linear map whose components are
  $\restrict{\partial}{\sigma,\tau}:=\partial_{\sigma,\tau}$.
\end{definition}

Observe that the component $\partial_{\sigma,\tau}$ makes sense even if $\sigma\not\leq\tau$because
then $[\sigma:\tau]=0$ so $\partial_{\sigma,\tau}$ is the zero map. Let us verify that
this does indeed deserve to be called a boundary map:

\begin{lemma}\label{thm:cosheaf_chain_complex}
  $\partial^2 = 0$.
  \begin{proof}
    We have the $\mathscr C(\tau)\to\mathscr C(\sigma)$ component
    of $\partial^2$:
    \begin{align*}
      \sum_{\delta\in K_{d-1}} \partial_{\sigma,\delta}\circ \partial_{\delta,\tau} = \left(\sum_{\delta\in K_{d-1}}[\sigma:\delta][\delta:\tau]\right)\mathscr C(\tau\geq\sigma).
    \end{align*}
    Write $\tau = \left\lbrace{v_1,\ldots,v_n}\right\rbrace$. And assume
    $\sigma = \tau \setminus \left\lbrace{v_i,v_j}\right\rbrace$ with $i<j$.
    Then $[\sigma:\delta][\delta:\tau]$  is non-zero if, and only if,
    $\sigma$ is a face of $\delta$ which is a face of $\tau$. Hence
    $\delta = \tau_{-i}$ or $\delta = \tau_{-j}$. We then note
    $\sigma = (\tau_{-j})_{-i} = (\tau_{-i})_{-(j-1)}$. It is now
    straightforward to calculate
    \begin{align*}
      [\sigma:\tau_{-i}][\tau_{-i}:\tau]=(-1)^{j-1}(-1)^i
      = -(-1)^i(-1)^j =
      -[\sigma:\tau_{-j}][\tau_{-j}:\tau]
    \end{align*}
    and hence $\partial^2=0$.
  \end{proof}
\end{lemma}

\begin{definition}
  The \emph{cosheaf homology of $K$ with $\mathscr C$-coefficients} is the
  homology of the complex
  \begin{align*}
    \cdots\xlongrightarrow{\partial} C_d(K;\mathscr C)
    \xlongrightarrow{\partial}C_{d-1}(K;\mathscr C)
    \xlongrightarrow{\partial}\cdots.
  \end{align*}
  Explicitly,
  \begin{align*}
    H_d(K;\mathscr C) = \frac{\ker(C_d(K;\mathscr C) \to C_{d-1}(K;\mathscr C))}{\im(C_{d+1}(K;\mathscr C) \to C_d(K;\mathscr C))}
  \end{align*}
\end{definition}

Once again this is a direct generalisation of singular homology as
$\partial : C_d(K;\underline{\mathbb{R}}_K)
\to C_{d-1}(K;\underline{\mathbb R}_K)$ agrees with the usual
boundary map $\partial : C_d(K)\to C_{d-1}(K)$ and thus
\begin{align*}
  H_d(K;\underline{\mathbb{R}}_K) = H_d(K).
\end{align*}

We finish this section by considering some examples:

\begin{example}
  Consider the 1-disk $K=D_1$ with vertices and edges as in
  \begin{equation*}
    % https://q.uiver.app/#q=WzAsMixbMCwwLCJcXHNpZ21hXzEiXSxbMiwwLCJcXHNpZ21hXzIiXSxbMCwxLCJcXHRhdSIsMSx7InN0eWxlIjp7ImhlYWQiOnsibmFtZSI6Im5vbmUifX19XV0=
    \begin{tikzcd}
      {\sigma_1} && {\sigma_2}
      \arrow["\tau"{description}, no head, from=1-1, to=1-3]
    \end{tikzcd}
  \end{equation*}
  Note that this is contractible, hence $H_0(K)=\mathbb{R}$ and
  $H_d(K) = 0$ for $d>0$.
  Define the chosheaf $\mathscr C\in\Cosh(K)$ by
  \begin{equation*}
    % https://q.uiver.app/#q=WzAsMyxbMCwwLCJcXG1hdGhiYiBSIl0sWzQsMCwiMCJdLFsyLDAsIlxcbWF0aGJiIFJeMiJdLFsyLDAsIih4LHkpXFxtYXBzdG8geCIsMl0sWzIsMV1d
    \begin{tikzcd}
      {\mathbb R} && {\mathbb R^2} && 0
      \arrow["{(x,y)\mapsto x}"', from=1-3, to=1-1]
      \arrow[from=1-3, to=1-5]
    \end{tikzcd}
  \end{equation*}
  It is then straightforward to calculate the homologies
  $H_0(K;\mathscr C)=0$ and $H_1(K;\mathscr C)=\mathbb{R}$.
  Thus even though $K$ is contractible its higher homologies are not
  trivial!
\end{example}

\begin{example}
  Consider the 1-sphere $K=S^1$ with vertices and edges as in
  \begin{equation*}
    % https://q.uiver.app/#q=WzAsMyxbMCwwLCJcXHNpZ21hXzEiXSxbMiwwLCJcXHNpZ21hXzIiXSxbMSwxLCJcXHNpZ21hXzMiXSxbMCwyLCJcXHRhdV8zIiwxLHsic3R5bGUiOnsiaGVhZCI6eyJuYW1lIjoibm9uZSJ9fX1dLFswLDEsIlxcdGF1XzEiLDEseyJzdHlsZSI6eyJoZWFkIjp7Im5hbWUiOiJub25lIn19fV0sWzIsMSwiXFx0YXVfMiIsMSx7InN0eWxlIjp7ImhlYWQiOnsibmFtZSI6Im5vbmUifX19XV0=
    \begin{tikzcd}
      {\sigma_1} && {\sigma_2} \\
                 & {\sigma_3}
                 \arrow["{\tau_3}"{description}, no head, from=1-1, to=2-2]
                 \arrow["{\tau_1}"{description}, no head, from=1-1, to=1-3]
                 \arrow["{\tau_2}"{description}, no head, from=2-2, to=1-3]
    \end{tikzcd}
  \end{equation*}
  Note that the homology in the first dimension is $H_1(K) = \mathbb{R}$,
  i.e. non-trivial. However, if we consider
  $\mathscr C=\underline{0}_K$ then $H_d(K;\underline{0}_K)=0$
  for all $d$. This is an instance of a simplicial complex with non-trivial
  $\mathbb{R}$-homology but trivial $\mathscr C$-homology.
\end{example}

\section{Maps}

As we are now familiar with cosheaves by themselves, it is the logical
next step to see how one cosheaf might relate to another. In particular,
we are going to be interested in defining maps between cosheaves. Before
we move on to persistent homology, we will then take some time to
look at the category of cosheaves on a particular simplicial complex.

As cosheaves are functors, it is unsurprising that maps of cosheaves
are natural transformations:

\begin{definition}
  A \emph{map of cosheaves} $\varphi:\mathscr C\to\mathscr D$ on $K$
  is
  just a natural transformation, i.e. it consists of maps
  $\varphi_\tau:\mathscr C(\tau)\to\mathscr D(\tau)$ for every simplex
  $\tau\in K$ such that, for all $\sigma\leq\tau$, the following commutes:
  \begin{equation}\label{eq:naturality}
    % https://q.uiver.app/#q=WzAsNCxbMCwwLCJcXG1hdGhzY3IgQyhcXHRhdSkiXSxbMiwwLCJcXG1hdGhzY3IgQyhcXHNpZ21hKSJdLFswLDEsIlxcbWF0aHNjciBEKFxcdGF1KSJdLFsyLDEsIlxcbWF0aHNjciBEKFxcc2lnbWEpIl0sWzAsMiwiXFxwaGlfXFx0YXUiLDJdLFsyLDMsIlxcbWF0aHNjciBEKFxcc2lnbWFcXGxlcVxcdGF1KSIsMl0sWzEsMywiXFxwaGlfXFxzaWdtYSJdLFswLDEsIlxcbWF0aHNjciBDKFxcc2lnbWFcXGxlcVxcdGF1KSJdXQ==
    \begin{tikzcd}
      {\mathscr C(\tau)} && {\mathscr C(\sigma)} \\
      {\mathscr D(\tau)} && {\mathscr D(\sigma)}
      \arrow["{\varphi_\tau}"', from=1-1, to=2-1]
      \arrow["{\mathscr D(\tau\geq\sigma)}"', from=2-1, to=2-3]
      \arrow["{\varphi_\sigma}", from=1-3, to=2-3]
      \arrow["{\mathscr C(\tau\geq\sigma)}", from=1-1, to=1-3]
    \end{tikzcd}
  \end{equation}
\end{definition}

Thus we have the category $\Cosh(K) :=
\text{Fun}((K,\geq),\FVect_{\mathbb{R}})$.
Considering a subcomplex $K'\subseteq K$ we have a on objects
$\Cosh(K')\hookrightarrow\Cosh(K)$ extending a cosheaf
$\mathscr C\in\Cosh(K')$ to a cosheaf on $K$ by setting
$\mathscr C(\tau)=0$ whenever $\tau\not\in K'$. This canonically
extends to a functor by observing that all the linear components
not specified by a map $\varphi:\mathscr C\to\mathscr D$ in $\Cosh(K')$
have signature $0\to 0$ and are thereby uniquely determined.
This very observation now also implies that we have a full subcategory
\begin{equation}\label{eq:subcomplex_full_subcategory}
  I:\Cosh(K') \longinc \Cosh(K).
\end{equation}
Note that the characteristic cosheaf of $K'$ on $K$
arises from $\underline{\mathbb{R}}_{K'}$ under this inclusion.
Moreover, by restriction we obtain a functor in the opposite
direction:
\begin{align*}
  R : \Cosh(K)\longrightarrow\Cosh(K').
\end{align*}
It is clear that these cannot form an equivalence of categories:
$K$ is a larger complex so a functor on $K$ contains more data
than its restriction to $K'$. However, we do have the next best
thing which is an adjuction:

\begin{proposition}
  There is an adjunction $R\dashv I$.
  \begin{proof}
    We have $IR = \text{Id} : \Cosh(K')\to\Cosh(K')$.
    Thus we need to construct a counit
    $\varepsilon : RI\to\text{Id}$. Consider $\mathscr C\in\Cosh(K)$.
    Define
    \begin{align*}
      \varepsilon_{\mathscr C} : RI(\mathscr C)\to\mathscr C
    \end{align*}
    by setting its component at $\tau$ to be the identity if $\tau\in K'$
    or the zero map if $\tau\not\in K'$. We need to verify naturality
    only in the case where $\sigma\leq\tau$ where $\sigma\in K'$ but
    $\tau\not\in K'$. We have the diagram
    \begin{equation*}
      % https://q.uiver.app/#q=WzAsNCxbMiwxLCJcXG1hdGhzY3IgQyhcXHNpZ21hKSJdLFsyLDAsIlxcbWF0aHNjciBDKFxcdGF1KSJdLFswLDAsIlJJXFxtYXRoc2NyIEMoXFx0YXUpIl0sWzAsMSwiUklcXG1hdGhzY3IgQyhcXHNpZ21hKSJdLFsyLDEsIlxcdmFyZXBzaWxvbiJdLFszLDAsIlxcdmFyZXBzaWxvbiIsMl0sWzEsMF0sWzIsM11d
      \begin{tikzcd}
        {RI\mathscr C(\tau)} && {\mathscr C(\tau)} \\
        {RI\mathscr C(\sigma)} && {\mathscr C(\sigma)}
        \arrow["\varepsilon", from=1-1, to=1-3]
        \arrow["\varepsilon"', from=2-1, to=2-3]
        \arrow[from=1-3, to=2-3]
        \arrow[from=1-1, to=2-1]
      \end{tikzcd}
    \end{equation*}
    which trivially commutes because $RI\mathscr C(\tau)=0$.
    Finally, it is easily verified that both of the following are
    identities:
    \begin{align*}
      R = RIR \xlongrightarrow{\varepsilon R} R,\hspace{1cm}
      I = IRI \xlongrightarrow{I \varepsilon} I.
    \end{align*}
  \end{proof}
\end{proposition}

As we are working with vector valued cosheaves, it turns out that
the categorical notions of monomorphisms, epimorphisms, and isomorphisms
agree exactly with the set-theoretic notions of injections,
surjections, and bijections.

\begin{proposition}\label{thm:epi_injective}
  A map of cosheaves is mono, epi, or iso if, and only if,
  each component is injective, surjective, or bijective, respectively.
  \begin{proof}
    This follows from a more general fact: A natural transformation
    $\varphi : F \to G : \mathcal C \to\mathcal D$ is a mono, epi, or
    iso in the functor category  $\text{Fun}(\mathcal C,\mathcal D)$
    if, and only if, each component $\varphi_C : FC \to GC$ is a
    mono, epi, or iso in $\mathcal D$, respectively. One may prove this
    using the Yoneda lemma.

    Considering the case $\mathcal C = (L,\geq)$ and
    $\mathcal D=\FVect_{\mathbb{R}}$ we find that a map of cosheaves
    $\varphi : \mathscr C\to\mathscr D$ is mono, epi, or iso if,
    and only if, each linear component $\varphi_\tau$ is
    injective, surjective, or bijective, respectively.
  \end{proof}
\end{proposition}

In particular, if we have a subcomplex $K'\subseteq K$ then the
characteristic cosheaf $\mathscr I_{K'}$ of $K'$ is a subobject of the
constant sheaf $\underline{\mathbb{R}}_K$ in $\Cosh(K)$. If one
thinks of $\underline{\mathbb{R}}_K$ as the characteristic
cosheaf of $K$ then we find that each inclusion of subcomplexes
$K''\subseteq K'\subseteq K$ gives rise to a monomorphism of
characteristic cosheaves $\mathscr I_{K''} \hookrightarrow
\mathscr I_{K'}$ in $\Cosh(K)$. We are going to make use of this
observation to generalise the notion of filtration.

Let us now fix a map of cosheaves
$\varphi : \mathscr C \to \mathscr D$ on $K$. We then have an induced
linear map on $k$-chains
$\varphi_* : C_d(K;\mathscr C) \to C_d(K;\mathscr D)$
whose components are $\varphi_\sigma : \mathscr C(\sigma) \to \mathscr D(\sigma)$.
It is unsurprising that this is the component of a chain map:
\begin{proposition}\label{thm:chain_map}
  For all $k$, the following commutes:
  \begin{equation*}
    % https://q.uiver.app/#q=WzAsNCxbMCwwLCJDX2soSztcXG1hdGhzY3IgQykiXSxbMiwwLCJDX3trKzF9KEs7XFxtYXRoc2NyIEMpIl0sWzAsMSwiQ19rKEs7XFxtYXRoc2NyIEQpIl0sWzIsMSwiQ197aysxfShLO1xcbWF0aHNjciBEKSJdLFswLDEsImQiXSxbMiwzLCJkIl0sWzAsMiwiXFxwaGlfKiIsMl0sWzEsMywiXFxwaGlfKiJdXQ==
    \begin{tikzcd}
      {C_d(K;\mathscr C)} && {C_{d+1}(K;\mathscr C)} \\
      {C_d(K;\mathscr D)} && {C_{d+1}(K;\mathscr D)}
      \arrow["\partial", from=1-1, to=1-3]
      \arrow["\partial", from=2-1, to=2-3]
      \arrow["{\varphi_*}"', from=1-1, to=2-1]
      \arrow["{\varphi_*}", from=1-3, to=2-3]
    \end{tikzcd}
  \end{equation*}
  \begin{proof}
    We observe that the following commutes:
    \begin{equation*}
      % https://q.uiver.app/#q=WzAsOCxbMCwwLCJcXGJpZ29wbHVzX3tcXGRpbSBcXHNpZ21hID0ga30gXFxtYXRoc2NyIEMoXFxzaWdtYSkiXSxbMywwLCJcXGJpZ29wbHVzX3tcXGRpbSBcXHRhdSA9IGstMX0gXFxtYXRoc2NyIEMoXFx0YXUpIl0sWzAsMywiXFxiaWdvcGx1c197XFxkaW0gXFxzaWdtYSA9IGt9IFxcbWF0aHNjciBEKFxcc2lnbWEpIl0sWzMsMywiXFxiaWdvcGx1c197XFxkaW0gXFx0YXUgPSBrLTF9IFxcbWF0aHNjciBEKFxcdGF1KSJdLFsxLDEsIlxcbWF0aHNjciBDKFxcc2lnbWEpIl0sWzEsMiwiXFxtYXRoc2NyIEQoXFxzaWdtYSkiXSxbMiwxLCJcXG1hdGhzY3IgQyhcXHRhdSkiXSxbMiwyLCJcXG1hdGhzY3IgRChcXHRhdSkiXSxbMCwxLCJkIl0sWzIsMywiZCJdLFswLDIsIlxccGhpXyoiLDJdLFsxLDMsIlxccGhpXyoiXSxbMyw3XSxbMiw1XSxbMCw0XSxbNCw1XSxbMSw2XSxbNiw3XSxbNSw3XSxbNCw2XV0=
      \begin{tikzcd}
        {\bigoplus_{\sigma\in K_d} \mathscr C({\sigma})} &&& {\bigoplus_{\tau\in K_{d-1}} \mathscr C({\tau})} \\
                                                         & {\mathscr C({\sigma})} & {\mathscr C({\tau})} \\
                                                         & {\mathscr D(\sigma)} & {\mathscr D(\tau)} \\
        {\bigoplus_{\sigma\in K_d} \mathscr D(\sigma)} &&& {\bigoplus_{\tau\in K_{d-1}} \mathscr D(\tau)}
        \arrow["\partial", from=1-1, to=1-4]
        \arrow["\partial", from=4-1, to=4-4]
        \arrow["{\varphi_*}"', from=1-1, to=4-1]
        \arrow["{\varphi_*}", from=1-4, to=4-4]
        \arrow[from=4-4, to=3-3]
        \arrow[from=4-1, to=3-2]
        \arrow[from=1-1, to=2-2]
        \arrow[from=2-2, to=3-2]
        \arrow[from=1-4, to=2-3]
        \arrow[from=2-3, to=3-3]
        \arrow[from=3-2, to=3-3]
        \arrow[from=2-2, to=2-3]
      \end{tikzcd}
    \end{equation*}
    Here the middle square is just naturality (\ref{eq:naturality})
    and the diagonal maps are the projections from the product.
    The claim now follows by the universal property of products.
  \end{proof}
\end{proposition}
Thus $\varphi : \mathscr C \to \mathscr D$ induces a
chain map $\varphi_* : C(K;\mathscr C)\to C(K;\mathscr D)$
and therefore maps between homologies
$\varphi_* : H_d(K;\mathscr C)\to H_d(K;\mathscr D)$.

\begin{example}
  Consider the map $\underline{\mathbb{R}}_K\to\mathscr C$ in $\Cosh(S^1)$
  given by
  \begin{equation*}
    % https://q.uiver.app/#q=WzAsOCxbMSwxLCJcXG1hdGhiYiBSXjIiXSxbMywxLCJcXG1hdGhiYiBSXjIiXSxbNSwxLCJcXG1hdGhiYiBSIl0sWzEsMCwiXFxtYXRoYmIgUiJdLFszLDAsIlxcbWF0aGJiIFIiXSxbNSwwLCJcXG1hdGhiYiBSIl0sWzAsMSwiXFxtYXRoc2NyIEM6Il0sWzAsMCwiXFx1bmRlcmxpbmV7XFxtYXRoYmIgUn1fSzoiXSxbMSwwXSxbMSwyLCJ4IiwyXSxbNCwzXSxbNCw1XSxbNCwxLCIoeCwwKSJdLFs1LDJdLFszLDAsIih4LDApIiwyXV0=
    \begin{tikzcd}
      {\underline{\mathbb R}_K:} & {\mathbb R} && {\mathbb R} && {\mathbb R} \\
      {\mathscr C:} & {\mathbb R^2} && {\mathbb R^2} && {\mathbb R}
      \arrow[from=2-4, to=2-2]
      \arrow["x"', from=2-4, to=2-6]
      \arrow[from=1-4, to=1-2]
      \arrow[from=1-4, to=1-6]
      \arrow["{(x,0)}", from=1-4, to=2-4]
      \arrow[from=1-6, to=2-6]
      \arrow["{(x,0)}"', from=1-2, to=2-2]
    \end{tikzcd}
  \end{equation*}
  On the level of chains we then have
  \begin{equation*}
    % https://q.uiver.app/#q=WzAsNCxbMiwxLCJcXG1hdGhiYiBSXjMiXSxbMCwwLCJcXG1hdGhiYiBSIl0sWzAsMSwiXFxtYXRoYmIgUl4yIl0sWzIsMCwiXFxtYXRoYmIgUl4yIl0sWzEsMiwiKHgsMCkiLDJdLFsxLDMsIih4LHgpIl0sWzIsMCwiKHgseSx4KSIsMl0sWzMsMCwiKHgsMCx5KSJdXQ==
    \begin{tikzcd}
      {\mathbb R} && {\mathbb R^2} \\
      {\mathbb R^2} && {\mathbb R^3}
      \arrow["{(x,0)}"', from=1-1, to=2-1]
      \arrow["{(x,x)}", from=1-1, to=1-3]
      \arrow["{(x,y,x)}"', from=2-1, to=2-3]
      \arrow["{(x,0,y)}", from=1-3, to=2-3]
    \end{tikzcd}
  \end{equation*}
  Clearly homology is non-zero at $d=0$ only. We now see that
  the induced map on homology is the unique map that makes the
  following commute:
  \begin{equation*}
    % https://q.uiver.app/#q=WzAsNCxbMCwwLCJcXG1hdGhiYiBSXjIiXSxbMiwwLCJcXG1hdGhiYiBSIl0sWzAsMSwiXFxtYXRoYmIgUl4zIl0sWzIsMSwiXFxtYXRoYmIgUiJdLFswLDEsIngteSJdLFsyLDMsIngteiIsMl0sWzAsMiwiKHgsMCx5KSIsMl0sWzEsM11d
    \begin{tikzcd}
      {\mathbb R^2} && {\mathbb R} \\
      {\mathbb R^3} && {\mathbb R}
      \arrow["{x-y}", from=1-1, to=1-3]
      \arrow["{x-z}"', from=2-1, to=2-3]
      \arrow["{(x,0,y)}"', from=1-1, to=2-1]
      \arrow[from=1-3, to=2-3]
    \end{tikzcd}
  \end{equation*}
  Here the horizontal maps are the quotient maps from
  chains to homology. Thus the induced map in homology must be the
  identity even though the map of cosheaves is not an isomorphism
  and the cosheaves are not isomorphic.
\end{example}

The previous example shows that we may have a mono
$\underline{\mathbb{R}}_K\to\mathscr C$ which induces an iso
between homologies. One may reasonably ask whether the dual
situation may arise. The answer is no.

\begin{lemma}
  Let $\varphi : \underline{\mathbb{R}}_K\to\mathscr C$ in $\Cosh(K)$
  be an epi such that
  $\varphi_* : H_*(K;\underline{\mathbb{R}}_K)\to H_*(K;\mathscr C)$
  is an iso. Then $\varphi$ is an iso.
  \begin{proof}
    Suppose $\varphi$ is not an iso. Then some component
    $\varphi_\tau : \mathbb{R}\to \mathscr C(\tau)$ is surjective but
    not bijective. Thus we must have $\mathscr C(\tau)=0$.
    Now choose a sequence $\sigma=\tau_0\leq\cdots\leq\tau_m = \tau$
    such that $\dim\tau_d = d$. Observe that naturality yields
    a commutative diagram:
    \begin{equation*}
      % https://q.uiver.app/#q=WzAsOCxbMCwwLCJcXG1hdGhiYiBSIl0sWzIsMCwiXFxtYXRoYmIgUiJdLFs2LDAsIlxcbWF0aGJiIFIiXSxbMCwyLCJcXG1hdGhzY3IgQyhcXHRhdV9tKSJdLFsyLDIsIlxcbWF0aHNjciBDKFxcdGF1X3ttLTF9KSJdLFs2LDIsIlxcbWF0aHNjciBDKFxcdGF1XzEpIl0sWzQsMiwiXFxjZG90cyJdLFs0LDAsIlxcY2RvdHMiXSxbMCwzLCJcXHZhcnBoaV97XFx0YXVfbX0iLDFdLFswLDFdLFsxLDddLFs3LDJdLFszLDRdLFs0LDZdLFs2LDVdLFsxLDQsIlxcdmFycGhpX3tcXHRhdV97bS0xfX0iLDFdLFs3LDZdLFsyLDUsIlxcdmFycGhpXzEiLDFdXQ==
      \begin{tikzcd}
        {\mathbb R} && {\mathbb R} && \cdots && {\mathbb R} \\
        \\
        {\mathscr C(\tau_m)} && {\mathscr C(\tau_{m-1})} && \cdots && {\mathscr C(\tau_1)}
        \arrow["{\varphi_{\tau_m}}"{description}, from=1-1, to=3-1]
        \arrow[from=1-1, to=1-3]
        \arrow[from=1-3, to=1-5]
        \arrow[from=1-5, to=1-7]
        \arrow[from=3-1, to=3-3]
        \arrow[from=3-3, to=3-5]
        \arrow[from=3-5, to=3-7]
        \arrow["{\varphi_{\tau_{m-1}}}"{description}, from=1-3, to=3-3]
        \arrow[from=1-5, to=3-5]
        \arrow["{\varphi_1}"{description}, from=1-7, to=3-7]
      \end{tikzcd}
    \end{equation*}
    Thus $\mathscr C(\tau_d)=0$ implies $\mathscr C(\tau_{d-1})=0$
    and hence $\mathscr C(\sigma) = 0$. Now observe that
    $[\sigma:\tau]= 0$ whenever $\sigma$ and $\tau$ belong to
    different connected components. Hence the boundary map
    decomposes into its restrictions to path-connected components.
    We may therefore assume without loss of generality that
    $K$ is path-connected, i.e. $H_0(K)=\mathbb{R}$.

    We win if we are able to show that, for all $v\in C_0(K;\mathscr C)$,
    there is a $u\in C_1(K;\mathscr C)$ such that $\partial u = v$,
    i.e. $v = 0$ in $H_0(K;\mathscr C)$.
    We do so as follows: Fix $v\in\mathscr C(\sigma')$ for some
    $\sigma'\in K_0$ and use path-connectedness
    to choose $\tau'_1,\ldots,\tau'_n\in K_1$ such that
    \begin{align*}
      \sigma = \sigma'_0 \leq \tau_1' \geq \sigma'_1 \leq \cdots \geq \sigma'_{n-1} \leq \tau'_n \geq \sigma'_n = \sigma'
    \end{align*}
    for some $\sigma'_1,\ldots,\sigma'_n \in K_0$. By relabelling $\sigma$
    to be the first $\sigma'_i$ with $\mathscr C(\sigma'_i)=0$ we may
    assume without loss of generality that $\mathscr C(\sigma'_i)\neq 0$
    and by the previous argument that $\mathscr C(\tau'_i)\neq 0$ and
    all $\mathscr C(\tau'_i \geq \sigma'_j)$ are isomorphisms.
    We are now able to choose $u_i\in\mathscr C(\tau'_i)$ such that
    $\mathscr C(\tau'_n\geq\sigma')(u_n) = v$ and
    $\mathscr C(\tau'_i\geq\sigma_i)(u_i) = \mathscr C(\tau'_{i+1}\geq\sigma_i)(u_{i+1})$. We then have the boundary
    \begin{align*}
      \partial \left({\sum_{i=1}^n u_i}\right) =
      \pm \sum_{i=1}^n (\mathscr C(\tau_i\geq \sigma_i) - \mathscr C(\tau_i\geq\sigma_{i-1}))(u_i)
    \end{align*}
    which telescopes to
    \begin{align*}
      \partial\left({\sum_{i=1}^{n} u_i}\right)
      = \pm\mathscr C(\tau_n\geq\sigma')(u_n) \mp \mathscr C(\tau_1\geq\sigma)(u_1)
      = \pm\mathscr C(\tau_n\geq\sigma')(u_n) = \pm v.
    \end{align*}
    Hence $v=0$ in $H_0(K;\mathscr C)$ and so $H_0(K;\mathscr C)=0$.
  \end{proof}
\end{lemma}

\section{Persistent homology}

Fix a simplicial complex $K$. A filtration $F^\bullet$ of $K$ is
an increasing sequence of subcomplexes
\begin{align*}
  F^1 K \subseteq F^2 K \subseteq \cdots \subseteq F^m K = K
\end{align*}
and a map of filtrations $f^\bullet : F^\bullet \to G^\bullet$ is a
sequence of simplicial maps $f^i : F^i K\to G^i K$ such that $\im(f^i)
\subseteq \im(f^{i+1})$. We thus have a category $\Fil(K)$ of filtrations
of $K$.

We have seen that subcomplexes induce subcosheaves. It is thus
natural to generalise filtrations in the following way:

\begin{definition}
  A \emph{filtration} of a cosheaf $\mathscr C\in\Cosh(K)$ is a sequence of
  cosheaves $\mathscr C^\bullet = \mathscr C^1,\ldots,\mathscr C^n$ where
  $\mathscr C^n = \mathscr C$ together with monos $\mathscr C_i
  \hookrightarrow \mathscr C_{i+1}$ for $1\leq i<n$.

  A map of filtrations $\varphi^\bullet : \mathscr C^\bullet \to \mathscr D^\bullet$ is a sequence of natural transformations
  $\varphi^i:\mathscr C^i \to\mathscr D^i$ that respects the inclusions,
  i.e. the following commutes:
  \begin{equation*}
    % https://q.uiver.app/#q=WzAsOCxbMCwwLCJcXG1hdGhzY3IgQ14xIl0sWzEsMCwiXFxtYXRoc2NyIENeMiJdLFszLDAsIlxcbWF0aHNjciBDXm0iXSxbMCwxLCJcXG1hdGhzY3IgRF4xIl0sWzEsMSwiXFxtYXRoc2NyIEReMiJdLFszLDEsIlxcbWF0aHNjciBEXm0iXSxbMiwxLCJcXGNkb3RzIl0sWzIsMCwiXFxjZG90cyJdLFswLDEsIiIsMCx7InN0eWxlIjp7InRhaWwiOnsibmFtZSI6Imhvb2siLCJzaWRlIjoidG9wIn19fV0sWzEsNywiIiwwLHsic3R5bGUiOnsidGFpbCI6eyJuYW1lIjoiaG9vayIsInNpZGUiOiJ0b3AifX19XSxbNywyLCIiLDAseyJzdHlsZSI6eyJ0YWlsIjp7Im5hbWUiOiJob29rIiwic2lkZSI6InRvcCJ9fX1dLFszLDQsIiIsMCx7InN0eWxlIjp7InRhaWwiOnsibmFtZSI6Imhvb2siLCJzaWRlIjoidG9wIn19fV0sWzQsNiwiIiwwLHsic3R5bGUiOnsidGFpbCI6eyJuYW1lIjoiaG9vayIsInNpZGUiOiJ0b3AifX19XSxbNiw1LCIiLDAseyJzdHlsZSI6eyJ0YWlsIjp7Im5hbWUiOiJob29rIiwic2lkZSI6InRvcCJ9fX1dLFswLDMsIlxcdmFycGhpXjEiLDJdLFsxLDQsIlxcdmFycGhpXjIiLDJdLFsyLDUsIlxcdmFycGhpXm0iLDJdXQ==
    \begin{tikzcd}
      {\mathscr C^1} & {\mathscr C^2} & \cdots & {\mathscr C^m} \\
      {\mathscr D^1} & {\mathscr D^2} & \cdots & {\mathscr D^m}
      \arrow[hook, from=1-1, to=1-2]
      \arrow[hook, from=1-2, to=1-3]
      \arrow[hook, from=1-3, to=1-4]
      \arrow[hook, from=2-1, to=2-2]
      \arrow[hook, from=2-2, to=2-3]
      \arrow[hook, from=2-3, to=2-4]
      \arrow["{\varphi^1}"', from=1-1, to=2-1]
      \arrow["{\varphi^2}"', from=1-2, to=2-2]
      \arrow["{\varphi^m}"', from=1-4, to=2-4]
    \end{tikzcd}
  \end{equation*}
  Denote the category of filtrations of a cosheaf $\mathscr C$ by
  $\Fil(\mathscr C)$.
\end{definition}



We previously observed that subcomplexes $K'\subseteq K$ induce
subcosheaves $\mathscr I_{K'}\hookrightarrow \mathscr I_K$. It is now
straightforwad to see that we have an inclusion of categories
\begin{align*}
  \Fil(K) \longinc \Fil(\underline{\mathbb{R}}_K)
\end{align*}
given by $F^{\bullet} \mapsto \mathscr I_{F^\bullet K}$.
Note however that $\Fil(K)$ is not a full subcategory of
$\Fil(\underline{\mathbb{R}}_K)$. For example if $K=\bullet$ is
the trivial simplicial complex then there is a unique simplicial
map $K\to K$. However, there are many linear maps
$\mathbb{R}\to\mathbb{R}$.
Moreover, there are many filtrations of cosheaves that do not
arise as a filtration of the underlying complex.

\begin{example}\label{ex:cosheaf_filtration_on_s1}
  Consider the 1-sphere $K=S^1$ with vertices and edges as in
  \begin{equation*}
    % https://q.uiver.app/#q=WzAsMyxbMCwwLCJcXHNpZ21hXzEiXSxbMiwwLCJcXHNpZ21hXzIiXSxbMSwxLCJcXHNpZ21hXzMiXSxbMCwyLCJcXHRhdV8zIiwxLHsic3R5bGUiOnsiaGVhZCI6eyJuYW1lIjoibm9uZSJ9fX1dLFswLDEsIlxcdGF1XzEiLDEseyJzdHlsZSI6eyJoZWFkIjp7Im5hbWUiOiJub25lIn19fV0sWzIsMSwiXFx0YXVfMiIsMSx7InN0eWxlIjp7ImhlYWQiOnsibmFtZSI6Im5vbmUifX19XV0=
    \begin{tikzcd}
      {\sigma_1} && {\sigma_2} \\
                 & {\sigma_3}
                 \arrow["{\tau_3}"{description}, no head, from=1-1, to=2-2]
                 \arrow["{\tau_1}"{description}, no head, from=1-1, to=1-3]
                 \arrow["{\tau_2}"{description}, no head, from=2-2, to=1-3]
    \end{tikzcd}
  \end{equation*}
  We may now define a filtration
  $\mathscr C^\bullet=\mathscr C^1\subseteq\cdots\subseteq\mathscr C^3\in\Fil(\underline{\mathbb{R}}^3_K)$ by
  \begin{align}\label{eq:s1_filtration}
    \mathscr C^i(\sigma_j) = \begin{cases}
      \mathbb{R}^{i} & \text{for }i\leq 2, \\
      \mathbb{R}^2 & \text{for }i > 2,
    \end{cases}
    \hspace{1cm}
    \text{and}
    \hspace{1cm}
    \mathscr C^i(\tau_j) = \mathbb{R}^{i-1}
  \end{align}
  where all maps $\mathscr C^i(\alpha) \hookrightarrow \mathscr C^{i+1}(\alpha)$ and $\mathscr C^i(\tau)\hookrightarrow\mathscr C^i(\sigma)$ are the
  obvious inclusions.
\end{example}

Note that a filtration $\mathscr C^\bullet$ of a cosheaf
induces a persistence module of homologies for every $d\geq 0$:
\begin{align}\label{eq:cosheaf_persistence_module}
  H_d(K;\mathscr C^0) \longinc
  H_d(K;\mathscr C^1) \longinc
  \cdots \longinc
  H_d(K;\mathscr C^n)
\end{align}

Now the $d$-th persistent homology of the filtration is just the
homology of the persistence module corresponding to the $d$-th
homology:
\begin{definition}\label{def:persistent_homology}
  The \emph{persistent homology} of a simplicial complex $K$
  with coefficients in a filtration of cosheaves $\mathscr C_i$
  is the persistent homology of the persistence module
  (\ref{eq:cosheaf_persistence_module}), i.e.
  \begin{align*}
    PH_{d,i\to j}(K;\mathscr C^\bullet) := \im\left(
      H_d(K;\mathscr C^i) \longinc
      \cdots \longinc
      H_d(K;\mathscr C^j)
    \right).
  \end{align*}
\end{definition}

\begin{example}
  Let us compute the homology groups of the cosheaf filtration
  in \ref{ex:cosheaf_filtration_on_s1}. We obtain the following
  complex of chain groups $C_i(K;\mathscr C^j)$ with $i=0,1$ from bottom to top
  and $j=1,\ldots,4$ from left to right:
  \begin{equation*}
    % https://q.uiver.app/#q=WzAsNixbMCwwLCIwIl0sWzAsMSwiXFxtYXRoYmIgUl4zIl0sWzEsMCwiXFxtYXRoYmIgUl4zIl0sWzEsMSwiXFxtYXRoYmIgUl42Il0sWzIsMSwiXFxtYXRoYmIgUl42Il0sWzIsMCwiXFxtYXRoYmIgUl42Il0sWzAsMV0sWzIsMywiZyIsMl0sWzUsNCwiaCIsMl0sWzEsMywiayIsMl0sWzMsNCwiXFxlbGwiLDJdLFswLDJdLFsyLDUsImYiXV0=
    \begin{tikzcd}
      0 & {\mathbb R^3} & {\mathbb R^6} \\
      {\mathbb R^3} & {\mathbb R^6} & {\mathbb R^6}
      \arrow[from=1-1, to=2-1]
      \arrow["g"', from=1-2, to=2-2]
      \arrow["h"', from=1-3, to=2-3]
      \arrow["k"', from=2-1, to=2-2]
      \arrow["\ell"', from=2-2, to=2-3]
      \arrow[from=1-1, to=1-2]
      \arrow["f", from=1-2, to=1-3]
    \end{tikzcd}
  \end{equation*}
  By choosing a basis $e_1,\ldots,e_n$ of $\mathbb{R}^n$ we may write
  the maps as
  \begin{align*}
    f(e_i) = e_{2i-1},\hspace{1cm}
    g(e_i) = e_{2i-1}+e_{2i+1},\hspace{1cm}
    h(e_i) = e_i + e_{i+2}, \hspace{1cm}
    k(e_i) = e_{2i-1}, \hspace{1cm}
    \ell(e_i) = e_i.
  \end{align*}
  Thus we have the complexes of homologies
  \begin{equation*}
    % https://q.uiver.app/#q=WzAsNixbMCwwLCIwIl0sWzAsMSwiXFxtYXRoYmIgUl4zIl0sWzEsMCwiXFxtYXRoYmIgUl4zIl0sWzEsMSwiXFxtYXRoYmIgUl4zIl0sWzIsMSwiMCJdLFsyLDAsIlxcbWF0aGJiIFJeNiJdLFsxLDMsImtfKiIsMl0sWzMsNCwiXFxlbGxfKiIsMl0sWzAsMl0sWzIsNSwiZl8qIl1d
    \begin{tikzcd}
      0 & {\mathbb R^3} & {\mathbb R^6} \\
      {\mathbb R^3} & {\mathbb R^3} & 0
      \arrow["{k_*}"', from=2-1, to=2-2]
      \arrow["{\ell_*}"', from=2-2, to=2-3]
      \arrow[from=1-1, to=1-2]
      \arrow["{f_*}", from=1-2, to=1-3]
    \end{tikzcd}
  \end{equation*}
  where
  \begin{align*}
    f_* = f, \hspace{1cm}
    k_* = 0, \hspace{1cm}
    \ell_* = 0.
  \end{align*}
  Therefore the only non-trivial persistent homology group is
  \begin{align*}
    PH_{1,2\to 3}(K;\mathscr C^\bullet) = \im f_* = \mathbb{R}^3.
  \end{align*}
\end{example}

\section{Acyclic partial matchings}

Consider a simplicial complex $K$. Recall that a partial matching
$\Sigma$ on $K$ is a collection of pairs $\sigma\triangleleft\tau$
such that $\sigma\leq\tau$ and no simplex appears in more than
one such pair. A $\Sigma$-path $\rho$ then is a sequence
\begin{align}\label{eq:path}
  \sigma_\rho:=\sigma_1\triangleleft\tau_1,
  \sigma_2\triangleleft\tau_2,
  \cdots,
  \sigma_m\triangleleft\tau_m=:\tau_\rho
\end{align}
such that $\tau_i\geq\sigma_{i+1}$ for all $i$.
We will write $P(\Sigma)$ to denote the set of all paths of $\Sigma$
and $|\rho|:= m$.
A partial matching $\Sigma$ is said to be acyclic
if there exists no path $\rho\in P(\Sigma)$ such that
$\sigma_\rho\triangleleft\tau_\rho\not\in\Sigma$ but
$\sigma_\rho\leq\tau_\rho$ in $K$.

Analogous to compatibility with a sheaf, we may define what it
means for an acyclic partial matching to be compatible with a cosheaf:
\begin{definition}\label{def:cosheaf_compatibility}
  A partial matching $\Sigma$ is \emph{$\mathscr C$-compatible} if,
  for all $\sigma\triangleleft\tau\in\Sigma$,
  $\mathscr C_{\sigma,\tau}$ is an iso.
\end{definition}

Thus a path (\ref{eq:path}) in a $\mathscr C$-compatible partial
matching gives rise to a map
$\partial_\rho : \mathscr C(\sigma_\rho)\to\mathscr C(\tau_\rho)$ as in
\begin{equation*}
  % https://q.uiver.app/#q=WzAsNSxbMCwwLCJcXG1hdGhzY3IgQyhcXHNpZ21hXzEpIl0sWzIsMCwiXFxtYXRoc2NyIEMoXFx0YXVfMSkiXSxbNCwwLCJcXGNkb3RzIl0sWzYsMCwiXFxtYXRoc2NyIEMoXFxzaWdtYV9tKSJdLFs4LDAsIlxcbWF0aHNjciBDKFxcdGF1X20pIl0sWzAsMSwiZF57LTF9X3tcXHNpZ21hXzEsXFx0YXVfMX0iXSxbMSwyLCJkX3tcXHNpZ21hXzIsXFx0YXVfMX0iXSxbMiwzLCJkX3tcXHNpZ21hX20sXFx0YXVfe20tMX19Il0sWzMsNCwiZF97XFxzaWdtYV9tLFxcdGF1X219XnstMX0iXV0=
  \begin{tikzcd}
    {\mathscr C(\sigma_1)} && {\mathscr C(\tau_1)} && \cdots && {\mathscr C(\sigma_m)} && {\mathscr C(\tau_m)}
    \arrow["{\partial^{-1}_{\sigma_1,\tau_1}}", from=1-1, to=1-3]
    \arrow["{\partial_{\sigma_2,\tau_1}}", from=1-3, to=1-5]
    \arrow["{\partial_{\sigma_m,\tau_{m-1}}}", from=1-5, to=1-7]
    \arrow["{\partial_{\sigma_m,\tau_m}^{-1}}", from=1-7, to=1-9]
  \end{tikzcd}
\end{equation*}
Given our discussion of cosheaf filtratations, it is now natural to
combine the notions of compatibility of a acyclic partial matching
with respect to filtrations of complexes and cosheaves, respectively,
into one:

\begin{definition}
  Let $\mathscr C^\bullet\in\Fil(\mathscr C)$. An acyclic partial
  matching $\Sigma$ is $\mathscr C^\bullet$-compatible if, for all
  $i$, it is $\mathscr C^i$-compatible.
\end{definition}

This clearly extends the notion of cosheaf compatibility from
\ref{def:cosheaf_compatibility}. Let us verify that it also extends
the notion of compatibility with a filtration of a complex.

Recall that an acyclic partial matching $\Sigma$ is said to be
compatible with a filtration $F^\bullet$ if,
\begin{align*}
  \text{$\sigma\in F^i K$ if, and only if, $\tau\in F^i K$}
\end{align*}
for all $i$ and all $\sigma\triangleleft\tau$.~\cite{nanda}
\begin{proposition}
  Let $F^\bullet \in\Fil(K)$ and $\Sigma$ an acyclic partial
  matching. Then $\Sigma$ is $F^\bullet$-compatible if, and only if,
  it is $\mathscr I_{F^\bullet K}$-compatible.
  \begin{proof}
    Assume $\Sigma$ is $F^\bullet$-compatible. Fix $i$ and some
    pair $\sigma\triangleleft\tau$. Now
    \begin{align*}
      \mathscr I_{F^i K}(\tau\geq\sigma) = \begin{cases}
        \identity : \mathbb{R}\to\mathbb{R} & \text{if }\sigma\in F^iK, \\
        0\to 0 &\text{otherwise.}
      \end{cases}
    \end{align*}
    Either way $\mathscr I_{F^i K}(\sigma\leq\tau)$ is an iso
    and thus $\Sigma$ is $\mathscr I_{F^\bullet K}$-compatible.

    Assume $\Sigma$ is $\mathscr I_{F^\bullet K}$-compatible.
    Fix $i$ and some pair $\sigma\triangleleft\tau$. If
    $\sigma\in F^i K$ then $\mathscr I_{F^i K}(\sigma) = \mathbb{R}$
    and $\mathscr I_{F^i K}(\sigma\leq\tau)$ must be an iso so
    $\mathscr I_{F^i K}(\tau) = \mathbb{R}$, i.e. $\tau\in F^i K$.
    Similarly, if
    $\sigma\not\in F^i K$ then $\mathscr I_{F^i K}(\sigma) = 0$
    and $\mathscr I_{F^i K}(\sigma\leq\tau)$ must be an iso so
    $\mathscr I_{F^i K}(\tau) = 0$, i.e. $\tau\not\in F^i K$.
    Hence $\Sigma$ is $F^\bullet$-compatible.
  \end{proof}
\end{proposition}

Before we move on to define the morse complex that such
partial matchings give rise to, we ought to consider some examples
and non-examples.

\begin{example}
  Note that every acyclic partial matching is trivially
  $\underline{V}_K$-compatible for all $V\in\FVect_{\mathbb{R}}$.
  In particular, the acyclic partial matching for the complex in
  \ref{ex:cosheaf_filtration_on_s1} given by
  \begin{align*}
    \Sigma = \left\lbrace{
        \sigma_1\triangleleft\tau_1,
        \sigma_2\triangleleft\tau_2
    }\right\rbrace
  \end{align*}
  is compatible with the cosheaf $\underline{\mathbb{R}}^3_K$ but
  not with the filtration (\ref{eq:s1_filtration}). In fact, no
  non-empty partial matching on $S^1$ can ever be compatible with
  this filtration as, for no $\sigma\leq\tau$, is
  $\mathscr C^i(\sigma\leq\tau)$ an isomorphism for all $i$.
\end{example}

\begin{example}\label{ex:acyclic_partial_matching_s1}
  Consider a different filtration $\mathscr C^\bullet\in\Fil(S^1)$ given by
  \begin{equation*}
    \mathscr C^1(S^1):
    % https://q.uiver.app/#q=WzAsNixbMCwwLCIwIl0sWzIsMiwiMCJdLFsxLDEsIjAiXSxbMywxLCIwIl0sWzIsMCwiMCJdLFs0LDAsIlxcbWF0aGJiIFIiXSxbMiwwXSxbMiwxXSxbNCwwXSxbMywxXSxbMyw1XSxbNCw1XV0=
    \begin{tikzcd}
      0 && 0 && {\mathbb R} \\
        & 0 && 0 \\
        && 0
        \arrow[from=2-2, to=1-1]
        \arrow[from=2-2, to=3-3]
        \arrow[from=1-3, to=1-1]
        \arrow[from=2-4, to=3-3]
        \arrow[from=2-4, to=1-5]
        \arrow[from=1-3, to=1-5]
    \end{tikzcd}
  \end{equation*}
  \begin{equation*}
    \mathscr C^2(S^1):
    % https://q.uiver.app/#q=WzAsNixbMCwwLCIwIl0sWzIsMiwiXFxSIl0sWzEsMSwiMCJdLFszLDEsIlxcUiJdLFsyLDAsIjAiXSxbNCwwLCJcXG1hdGhiYiBSIl0sWzIsMF0sWzIsMV0sWzQsMCwiXFx0ZXh0e2lkfSIsMV0sWzMsMSwiXFx0ZXh0e2lkfSIsMV0sWzMsNSwiXFx0ZXh0e2lkfSIsMV0sWzQsNSwiXFx0ZXh0e2lkfSIsMV1d
    \begin{tikzcd}
      0 && 0 && {\mathbb R} \\
        & 0 && \mathbb{R} \\
        && \mathbb{R}
        \arrow[from=2-2, to=1-1]
        \arrow[from=2-2, to=3-3]
        \arrow[from=1-3, to=1-1]
        \arrow["{-\text{id}}"{description}, from=2-4, to=3-3]
        \arrow["{\text{id}}"{description}, from=2-4, to=1-5]
        \arrow[from=1-3, to=1-5]
    \end{tikzcd}
  \end{equation*}
  \begin{equation*}
    \mathscr C^3(S^1):
    % https://q.uiver.app/#q=WzAsNixbMCwwLCJcXFIiXSxbMiwyLCJcXFIiXSxbMSwxLCIwIl0sWzMsMSwiXFxSIl0sWzIsMCwiXFxSIl0sWzQsMCwiXFxSXjIiXSxbMiwwXSxbMiwxXSxbNCwwLCJcXHRleHR7aWR9IiwxXSxbMywxLCIoLTEpXmlcXHRleHR7aWR9IiwxXSxbMyw1LCIiLDIseyJzdHlsZSI6eyJ0YWlsIjp7Im5hbWUiOiJob29rIiwic2lkZSI6InRvcCJ9fX1dLFs0LDUsIiIsMCx7InN0eWxlIjp7InRhaWwiOnsibmFtZSI6Imhvb2siLCJzaWRlIjoidG9wIn19fV1d
    \begin{tikzcd}
      \mathbb{R} && \mathbb{R} && {\mathbb{R}^2} \\
                 & 0 && \mathbb{R} \\
                 && \mathbb{R}
                 \arrow[from=2-2, to=1-1]
                 \arrow[from=2-2, to=3-3]
                 \arrow["{-\text{id}}"{description}, from=1-3, to=1-1]
                 \arrow["{\text{id}}"{description}, from=2-4, to=3-3]
                 \arrow[hook, from=2-4, to=1-5]
                 \arrow[hook, from=1-3, to=1-5]
    \end{tikzcd}
  \end{equation*}
  where we choose a single inclusion
  $\mathbb{R}\hookrightarrow\mathbb{R}^2$ throughout and the component
  $\mathscr C^2(\sigma_3)\hookrightarrow\mathscr C^3(\sigma_3)$
  is $-\identity$.

  If we now choose $\Sigma = \left\lbrace{\sigma_1\triangleleft\tau_1,
  \sigma_3\triangleleft\tau_2}\right\rbrace$ then $\Sigma$ is an
  acyclic partial matching that, moreover, is
  $\mathscr C^\bullet$-compatible.

  One may also wish to calculate the corresponding persistent homologies.
  We have the complex of chain groups
  \begin{equation*}
    % https://q.uiver.app/#q=WzAsNixbMCwwLCIwIl0sWzIsMCwiXFxtYXRoYmIgUiJdLFs0LDAsIlxcbWF0aGJiIFJeMiJdLFswLDIsIlxcbWF0aGJiIFIiXSxbMiwyLCJcXG1hdGhiYiBSXjIiXSxbNCwyLCJcXG1hdGhiYiBSXjQiXSxbMSwyLCIoMCx4KSJdLFsyLDUsIigteCx4K3ksMCx5KSJdLFs0LDUsIigwLHgsMCwteSkiLDJdLFsxLDQsIih4LC14KSIsMl0sWzMsNCwiKHgsMCkiLDJdLFswLDNdLFswLDFdXQ==
    \begin{tikzcd}
      0 && {\mathbb R} && {\mathbb R^2} \\
      \\
      {\mathbb R} && {\mathbb R^2} && {\mathbb R^4}
      \arrow["{(0,x)}", from=1-3, to=1-5]
      \arrow["{(-x,x+y,0,y)}", from=1-5, to=3-5]
      \arrow["{(0,x,0,-y)}"', from=3-3, to=3-5]
      \arrow["{(x,-x)}"', from=1-3, to=3-3]
      \arrow["{(x,0)}"', from=3-1, to=3-3]
      \arrow[from=1-1, to=3-1]
      \arrow[from=1-1, to=1-3]
    \end{tikzcd}
  \end{equation*}
  and thus the corresponding homology complexes:
  \begin{equation*}
    % https://q.uiver.app/#q=WzAsOCxbMSwwLCIwIl0sWzMsMCwiMCJdLFs1LDAsIjAiXSxbMSwxLCJcXG1hdGhiYiBSIl0sWzMsMSwiXFxtYXRoYmIgUiJdLFs1LDEsIlxcbWF0aGJiIFJeMiJdLFswLDAsIkhfMShTXjE7XFxtYXRoc2NyIENeXFxidWxsZXQpOiJdLFswLDEsIkhfMChTXjE7XFxtYXRoc2NyIENeXFxidWxsZXQpOiJdLFsxLDJdLFs0LDUsIiIsMix7InN0eWxlIjp7InRhaWwiOnsibmFtZSI6Imhvb2siLCJzaWRlIjoidG9wIn19fV0sWzMsNCwiXFx0ZXh0e2lkfSIsMl0sWzAsMV1d
    \begin{tikzcd}
      {H_1(S^1;\mathscr C^\bullet):} & 0 && 0 && 0 \\
      {H_0(S^1;\mathscr C^\bullet):} & {\mathbb R} && {\mathbb R} && {\mathbb R^2}
      \arrow[from=1-4, to=1-6]
      \arrow[hook, from=2-4, to=2-6]
      \arrow["{\text{id}}"', from=2-2, to=2-4]
      \arrow[from=1-2, to=1-4]
    \end{tikzcd}
  \end{equation*}
  The persistent homology groups may now simply be read off.
\end{example}

\section{Morse complex}

Recall also that the critical points of a matching are those simplices
that do not show up in any of the pairs. That is,
\begin{align*}
  \Sigma(K) = K \setminus \bigcup_{\sigma\triangleleft\tau\in\Sigma} \left\lbrace{\sigma,\tau}\right\rbrace
\end{align*}
This notation extends to $\Sigma(K_d) := K_d \cap \Sigma(K)$.
Critical simplices
are important for homology with field coefficients
because they give rise to a new chain complex whose homology agrees
with the usual homology. In particular, this means that the computation
of homology groups may be sped up by finding a good acyclic partial
matching.

We would like to take this idea and extend it to persistent homology with
cosheaf filtration coefficients. That is, we wish to construct
vector spaces $C^\Sigma_k(K;\mathscr C^i)$ and maps
\begin{align*}
  \partial^\Sigma_{d,i} : C^\Sigma_d(K;\mathscr C^i)\to C^\Sigma_{d-1}(K;\mathscr C^i)
\end{align*}
whose persistent homologies
\begin{align*}
  PH^\Sigma_{d,i\to j}(K;\mathscr C^\bullet)
  = \im(H^\Sigma_d(K;\mathscr C^i)\longinc\cdots\longinc H_d^\Sigma(K;\mathscr C^j))
\end{align*}
agree with those defined in \ref{def:persistent_homology}. Recalling that
we may regard $\mathbb{R}$-homology as $\underline{\mathbb{R}}_K$-homology,
we define the vector space of critical chains to be
\begin{align*}
  C_d^\Sigma(K;\mathscr C^i) := \bigoplus_{\sigma\in \Sigma(K_d)} \mathscr C^i(\sigma).
\end{align*}

\begin{definition}
  The \emph{$(d,i)$-Morse boundary operator} is the linear map
  \begin{align*}
    \partial_{d,i}^\Sigma : C_d^\Sigma(K;\mathscr C^i) \to C_{d-1}^\Sigma(K;\mathscr C^i)
  \end{align*}
  whose $\mathscr C^i(\tau)\to\mathscr C^i(\sigma)$ component is
  \begin{align}\label{eq:morse_boundary_component}
    \partial_{\sigma,\tau} +
    \sum_{\rho\in P(\Sigma)} (-1)^{|\rho|} \partial_{\sigma,\tau_\rho}\circ \partial_\rho \circ \partial_{\sigma_\rho,\tau}.
  \end{align}
\end{definition}

We once again remind the reader that the expression (\ref{eq:morse_boundary_component}) makes sense even for those $\rho$ where
$\sigma$ is not a face of $\tau_\rho$ or $\sigma_\rho$ is not a face of
$\tau$ as then the boundary components are simply zero. Let us verify
that these boundary operators do indeed yield a chain complex.

To do so, we are going to follow the approach taken in \cite{curry2015}.
The idea is that the Morse chain complex is obtained by adding pairs
individually to the partial ordering, i.e. removing them from the set of
critical vertices. Due to the partial ordering being acyclic, it does not
matter in which order the pairs are considered. This allows us to treat
the pairs independently and hence it suffices to prove statements about the
special case
$\Sigma=\left\lbrace{\sigma_\rho\triangleleft\tau_\rho}\right\rbrace$.

\begin{lemma}
  $\left({\partial^\Sigma}\right)^2=0$.
  \begin{proof}
    Let $\Sigma=\left\lbrace{\sigma_{\rho}\triangleleft\tau_{\rho}}\right\rbrace$.
    Consider a component
    $\mathscr C^i(\tau)\to\mathscr C^i(\sigma)$ for $\dim\tau=\dim\sigma+2$.
    This must be of the form
    \begin{align*}
      \sum_{\delta\in \Sigma(K)}
      \left(
        \partial_{\sigma,\delta} +
        \sum_{\rho\in \Sigma(K)} (-1)^{|\rho|} \partial_{\sigma,\tau_\rho}\circ \partial_\rho \circ \partial_{\sigma_\rho,\delta}
      \right) \circ
      \left(
        \partial_{\delta,\tau} +
        \sum_{\rho\in \Sigma(K)} (-1)^{|\rho|} \partial_{\delta,\tau_\rho}\circ \partial_\rho \circ \partial_{\sigma_\rho,\tau}
      \right)\\
      =
      \sum_{\delta\in \Sigma(K)}
      \left(
        \partial_{\sigma,\delta} - \partial_{\sigma,\tau_\rho}\circ \partial_\rho \circ \partial_{\sigma_\rho,\delta}
      \right) \circ
      \left(
        \partial_{\delta,\tau} - \partial_{\delta,\tau_\rho}\circ \partial_\rho \circ \partial_{\sigma_\rho,\tau}
      \right)
    \end{align*}
    By expecting dimensions we find that not both negative
    terms can be non-zero for any given $\delta$, hence we have
    \begin{align*}
      \sum_{\delta\in \Sigma(K)}
      \left(
        \partial_{\sigma,\delta}\circ \partial_{\delta,\tau}
        - \partial_{\sigma,\tau_\rho}
        \circ {\partial^{-1}_{\sigma_\rho,\tau_\rho}}
        \circ \partial_{\sigma_\rho,\delta}
        \circ \partial_{\delta,\tau}
        - \partial_{\sigma,\delta}
        \circ \partial_{\delta,\tau_\rho}
        \circ {\partial^{-1}_{\sigma_\rho,\tau_\rho}}
        \circ \partial_{\sigma_\rho,\tau}
      \right)
    \end{align*}
    Observing $\Sigma(K)=\left\lbrace{\sigma_\rho,\tau_\rho}\right\rbrace$ it is now straightforward to check that all terms cancel. By
    induction the claim follows for general $\Sigma$.
  \end{proof}
\end{lemma}

We are now justified in considering the Morse homology of
$\mathscr C^\bullet$ and $\Sigma$. We write $H^\Sigma_d(K;\mathscr C^i)$
for the $d$-th homology group of the complex
\begin{align*}
  \cdots\xlongrightarrow{\partial^\Sigma} C^\Sigma_d(K;\mathscr C^i)
  \xlongrightarrow{\partial^\Sigma}C^\Sigma_{d-1}(K;\mathscr C^i)
  \xlongrightarrow{\partial^\Sigma}\cdots.
\end{align*}


Now note that the inclusions $\mathscr C^i\hookrightarrow\mathscr C^{i+1}$
induce maps
\begin{align*}
  C^\Sigma_d(K;\mathscr C^i) \longinc C^\Sigma_d(K;\mathscr C^{i+1}).
\end{align*}
Once again we find that these are components of a chain map:

\begin{lemma}
  For all $d$ and $i$, the following commutes:
  \begin{equation*}
    % https://q.uiver.app/#q=WzAsNCxbMCwwLCJDXlxcU2lnbWFfayhLO1xcbWF0aHNjciBDXmkpIl0sWzIsMCwiQ15cXFNpZ21hX3trLTF9KEs7XFxtYXRoc2NyIENeaSkiXSxbMCwxLCJDXlxcU2lnbWFfayhLO1xcbWF0aHNjciBDXntpKzF9KSJdLFsyLDEsIkNeXFxTaWdtYV97ay0xfShLO1xcbWF0aHNjciBDXntpKzF9KSJdLFswLDEsIlxccGFydGlhbF5cXFNpZ21hIl0sWzAsMiwiIiwyLHsic3R5bGUiOnsidGFpbCI6eyJuYW1lIjoiaG9vayIsInNpZGUiOiJ0b3AifX19XSxbMSwzLCIiLDAseyJzdHlsZSI6eyJ0YWlsIjp7Im5hbWUiOiJob29rIiwic2lkZSI6InRvcCJ9fX1dLFsyLDMsIlxccGFydGlhbF5cXFNpZ21hIiwyXV0=
    \begin{tikzcd}
      {C^\Sigma_d(K;\mathscr C^i)} && {C^\Sigma_{d-1}(K;\mathscr C^i)} \\
      {C^\Sigma_d(K;\mathscr C^{i+1})} && {C^\Sigma_{d-1}(K;\mathscr C^{i+1})}
      \arrow["{\partial^\Sigma}", from=1-1, to=1-3]
      \arrow[hook, from=1-1, to=2-1]
      \arrow[hook, from=1-3, to=2-3]
      \arrow["{\partial^\Sigma}"', from=2-1, to=2-3]
    \end{tikzcd}
  \end{equation*}
  \begin{proof}
    This follows directly from the fact that $\partial^\Sigma$
    consists of boundary operators with each of which the inclusions
    commute by naturality. See \ref{thm:chain_map} for details.
  \end{proof}
\end{lemma}
Thus we have inclusions of Morse homologies
\begin{align*}
  H^\Sigma_d(K;\mathscr C^i) \longinc H^\Sigma_d(K;\mathscr C^{i+1}).
\end{align*}
Therefore, we have persistence modules of Morse homologies and hence
may define the corresponding persistent Morse homologies:
\begin{definition}
  The \emph{$d$-th persistent homology of $\Sigma$ with $\mathscr C^\bullet$-coefficients} is
  \begin{align*}
    PH^\Sigma_{d,i\to j}
    = \im\left({
        H^\Sigma_d(K;\mathscr C^i) \longinc \cdots \longinc H^\Sigma_d(K;\mathscr C^j)
    }\right)
  \end{align*}
\end{definition}

\begin{example}\label{ex:persistent_morse_homology}
  Let us now revisit \ref{ex:acyclic_partial_matching_s1}.
  Note that $\Sigma(K)=\left\lbrace{\sigma_2,\tau_3}\right\rbrace$.
  Thus we have the chain groups
  $C^\Sigma_0(K;\mathscr C^i) = \mathscr C^i(\sigma_2)$
  and
  $C^\Sigma_1(K;\mathscr C^i) = \mathscr C^i(\tau_3) = 0$.
  Thus Morse homology is only non-trivial in dimension $0$
  where we may simply read off the complex:
  \begin{equation*}
    % https://q.uiver.app/#q=WzAsNCxbMSwwLCJcXG1hdGhiYiBSIl0sWzMsMCwiXFxtYXRoYmIgUiJdLFs1LDAsIlxcbWF0aGJiIFJeMiJdLFswLDAsIkheXFxTaWdtYV8wKFNeMTtcXG1hdGhzY3IgQ15cXGJ1bGxldCk6Il0sWzEsMiwiIiwyLHsic3R5bGUiOnsidGFpbCI6eyJuYW1lIjoiaG9vayIsInNpZGUiOiJ0b3AifX19XSxbMCwxLCJcXHRleHR7aWR9IiwyXV0=
    \begin{tikzcd}
      {H^\Sigma_0(S^1;\mathscr C^\bullet):} & {\mathbb R} && {\mathbb R} && {\mathbb R^2}
      \arrow[hook, from=1-4, to=1-6]
      \arrow["{\text{id}}"', from=1-2, to=1-4]
    \end{tikzcd}
  \end{equation*}
\end{example}

\section{Equivalence}

It is of course no coincidence that the persistent homologies
in \ref{ex:persistent_morse_homology} agree with \ref{ex:acyclic_partial_matching_s1}. We constructed the Morse complex with the explicit
intention to obtain an alternative way of calculating the persistent
homology groups. The goal of this section is showing that we have
achieved our goal. We do this by constructing appropriate chain maps
and show that they are homotopic to the identity.

Consider the linear maps $\phi_d : C_d(K;\mathscr C)\to C_d^\Sigma(K;\mathscr C)$ whose component $\mathscr C(\alpha)\to\mathscr C(\omega)$ is the identity whenever $\alpha=\omega$ and
\begin{align*}
  \restrict{\phi_d}{\omega,\alpha} := \sum_{\rho\in P(\Sigma),\sigma_\rho = \alpha} (-1)^{|\rho|}\partial_{\omega,\tau_\rho}\circ\partial_\rho
\end{align*}
otherwise.

\begin{lemma}
  $\phi_\bullet : C_\bullet(K;\mathscr C) \to C^\Sigma_\bullet (K;\mathscr C)$ is a chain map.
  \begin{proof}
    We need to show $\partial^\Sigma \circ \phi_d = \phi_{d-1} \circ
    \partial$. Once again we consider $\Sigma=\left\lbrace{\sigma_\rho\triangleleft\tau_\rho}\right\rbrace$ and deduce the
    claim by induction.
    If $\alpha\not\in \Sigma(K)$ then
    \begin{align*}
      \restrict{(\partial^\Sigma\circ\phi_d)}{\omega',\alpha}
      &= \sum_{\omega\in \Sigma(K_d)} \restrict{\partial^\Sigma}{\omega',\omega}\circ \restrict{\phi_d}{\omega,\alpha}\\
      &=
      \sum_{\omega\in \Sigma(K_d)}
      \left({
          {\partial}_{\omega',\omega}
          + \sum_{\rho\in P(\Sigma)} (-1)^{|\rho|} \partial_{\omega',\tau_\rho} \circ \partial_\rho \circ \partial_{\sigma_\rho,\omega}
      }\right)
      \circ
      \left({
          \sum_{\rho\in P(\Sigma),\sigma_\rho = \alpha} (-1)^{|\rho|}\partial_{\omega,\tau_\rho}\circ\partial_\rho
      }\right).
    \end{align*}
    Using $P(\Sigma) = \left\lbrace{\sigma_\rho\triangleleft\tau_\rho}\right\rbrace$, we simplify this to
    \begin{align*}
      \sum_{\omega\in \Sigma(K_d)}
      \left({
          {\partial}_{\omega',\omega}
          - \partial_{\omega',\tau_\rho} \circ \partial^{-1}_{\sigma_\rho,\tau_\rho} \circ \partial_{\sigma_\rho,\omega}
      }\right)
      \circ
      \left({
          -\partial_{\omega,\tau_\rho}\circ{\partial}^{-1}_{\sigma_\rho,\tau_\rho}
      }\right).
    \end{align*}
    whenever $\sigma_\rho = \alpha$. For the term on the right to be non-zero we require
    $\dim\omega = \dim\sigma_\rho$. Hence
    $\partial_{\sigma_\rho,\omega}=0$ so we obtain
    \begin{align*}
      \sum_{\omega\in \Sigma(K_d)}
      - {\partial}_{\omega',\omega}
      \circ
      \partial_{\omega,\tau_\rho}\circ{\partial}^{-1}_{\sigma_\rho,\tau_\rho}
      =
      \left(
        \partial_{\omega',\sigma_\rho}\circ\partial_{\sigma_\rho,\tau_\rho}
        -
        \sum_{\omega\in K_d}
        {\partial}_{\omega',\omega}
        \circ
        \partial_{\omega,\tau_\rho}
      \right)
      \circ{\partial}^{-1}_{\sigma_\rho,\tau_\rho}
      = \partial_{\omega',\alpha}
    \end{align*}
    when $\alpha=\sigma_\rho$. If $\alpha=\tau_\rho$ we simply
    have the zero map. For $\alpha\in \Sigma(K)$, we must have
    $\restrict{\phi_d}{\omega,\alpha}=0$ whenever $\alpha\neq\omega$
    so
    \begin{align*}
      \restrict{(\partial^\Sigma\circ\phi_d)}{\omega',\alpha}
      = \restrict{\partial^\Sigma}{\omega',\alpha}
      = {\partial}_{\omega',\alpha}
      - \partial_{\omega',\tau_\rho} \circ \partial^{-1}_{\sigma_\rho,\tau_\rho} \circ \partial_{\sigma_\rho,\alpha}
      .
    \end{align*}
    We summarise
    \begin{align}\label{eq:phi_chain_cases}
      \restrict{(\partial^\Sigma\circ\phi_d)}{\omega',\alpha}
      = \begin{cases}
        \partial_{\omega',\alpha} & \alpha=\sigma_\rho, \\
        0 & \alpha=\tau_\rho, \\
        {\partial}_{\omega',\alpha}
        - \partial_{\omega',\tau_\rho} \circ \partial^{-1}_{\sigma_\rho,\tau_\rho} \circ \partial_{\sigma_\rho,\alpha} & \text{otherwise}.
      \end{cases}
    \end{align}

    On the other side we have
    \begin{align*}
      \restrict{(\phi_{d-1}\circ \partial)}{\omega',\alpha}
      &= \sum_{\alpha'\in K_{d-1}} \restrict{\phi_{d-1}}{\omega',\alpha'}\circ \restrict{\partial}{\alpha',\alpha}\\
      &=
      \partial_{\omega',\alpha}
      + \sum_{\alpha'\neq\omega'\in K_{d-1}}
      \sum_{\rho\in P(\Sigma),\sigma_\rho = \alpha'}
      (-1)^{|\rho|}
      \partial_{\omega',\tau_\rho}\circ\partial_\rho\circ\partial_{\alpha',\alpha}
    \end{align*}
    Using $\rho = \sigma_\rho\triangleleft\tau_\rho$ this simplifies
    to
    \begin{align*}
      \restrict{(\phi_{d-1}\circ \partial)}{\omega',\alpha}
      = \partial_{\omega',\alpha} - \partial_{\omega',\tau_\rho}\circ\partial_{\sigma_\rho,\tau_\rho}^{-1}\circ\partial_{\sigma_\rho,\alpha}.
    \end{align*}
    This is easily verified to equal (\ref{eq:phi_chain_cases})
    as $\partial_{\sigma_\rho,\sigma_\rho} = 0$ and
    $\partial_{\omega',\alpha}\circ\partial^{-1}_{\sigma_\rho,\alpha}\circ\partial_{\sigma_\rho,\alpha} = \partial_{\omega',\alpha}$.
  \end{proof}
\end{lemma}

In the other direction, we have the linear maps
$\psi_d : C^\Sigma_d(K;\mathscr C)\to C_d(K;\mathscr C)$
whose components $\mathscr C(\omega)\to\mathscr C(\alpha)$
are also the identity if $\alpha=\omega$ and
\begin{align*}
  \restrict{\psi_d}{\alpha,\omega}:=\sum_{\rho\in P(\Sigma),\tau_\rho=\alpha} (-1)^{|\rho|}\partial_\rho \circ \partial_{\sigma_\rho,\omega}
\end{align*}
otherwise.
\begin{lemma}
  $\psi_\bullet:C^\Sigma_\bullet(K;\mathscr C)\to C_\bullet(K;\mathscr C)$ is a chain map.
  \begin{proof}
    In analogy to $\phi_d$ we assume
    $\Sigma = \left\lbrace{\sigma_\rho\triangleleft\tau_\rho}\right\rbrace$ and compute both sides of the equation $\partial\circ\psi = \psi_{d-1}\circ\partial^\Sigma$. Thus we find
    \begin{align}\label{eq:psi_chain_case}
      \restrict{(\partial\circ\psi_d)}{\alpha,\omega}
      = \sum_{\alpha\in K_d} \restrict{\partial}{\alpha',\alpha} \circ \restrict{\psi_d}{\alpha,\omega}
      = \partial_{\alpha',\omega}
      -
      {
        \partial_{\alpha',\tau_\rho}
        \circ \partial^{-1}_{\sigma_\rho,\tau_\rho}
        \circ \partial_{\sigma_\rho,\omega}
      }.
    \end{align}
    On the other side we once again need to consider some
    cases. If $\alpha'\in \Sigma(K)$ then
    \begin{align*}
      \restrict{(\psi_{d-1} \circ \partial^\Sigma)}{\alpha,\omega}
      &= \sum_{\omega'\in \Sigma(K_{d-1})}
      \restrict{\psi_{d-1}}{\alpha',\omega'}\circ\restrict{\partial^\Sigma}{\omega',\omega} \\
      &= \partial_{\alpha',\omega}
      + \sum_{\alpha'\neq\omega'\in \Sigma(K_{d-1})}
      \restrict{\psi_{d-1}}{\alpha',\omega'}\circ\restrict{\partial^\Sigma}{\omega',\omega} \\
      &= \partial_{\alpha',\omega}.
    \end{align*}
    If $\alpha'=\tau_\rho$ we find
    \begin{align*}
      \restrict{(\psi_{d-1} \circ \partial^\Sigma)}{\alpha',\omega}
      &= \sum_{\omega'\in \Sigma(K_{d-1})}
      \restrict{\psi_{d-1}}{\alpha',\omega'}\circ\restrict{\partial^\Sigma}{\omega',\omega} \\
      &= \sum_{\omega'\in \Sigma(K_{d-1})}
      (\partial^{-1}_{\sigma_\rho,\alpha'}\circ\partial_{\sigma_\rho,\omega'})
      \circ
      \partial_{\omega',\omega}\\
      &= \partial_{\alpha',\omega}
    \end{align*}
    Finally, if $\alpha'=\sigma_\rho$ then
    \begin{align*}
      \restrict{(\psi_{d-1} \circ \partial^\Sigma)}{\alpha',\omega}
      = \sum_{\omega'\in \Sigma(K_{d-1})}
      \restrict{\psi_{d-1}}{\alpha',\omega'}\circ\restrict{\partial^\Sigma}{\omega',\omega} = 0.
    \end{align*}
    Once again it is straightforward to verify that
    this agrees with (\ref{eq:psi_chain_case}) and hence the
    claim follows by induction.
  \end{proof}
\end{lemma}

\begin{theorem}
  The complexes $(C_\bullet(K;\mathscr C),\partial)$ and
  $(C^\Sigma_\bullet(K;\mathscr C),\partial^\Sigma)$ are chain homotopy equivalent.
  \begin{proof}
    Consider $\phi_d\circ\psi_d$.
    We have $\restrict{\phi_d\circ\psi_d}{\omega,\omega}=\identity$
    and, for $\omega\neq\omega'$,
    \begin{align*}
      \restrict{(\phi_d \circ \psi_d)}{\omega',\omega}
      &= \restrict{\phi_d}{\omega',\omega}\circ\restrict{\psi_d}{\omega,\omega}+\restrict{\phi_d}{\omega',\omega'}\circ\restrict{\psi_d}{\omega',\omega}
    \end{align*}
    where we used the fact that we cannot have $\tau_\rho \triangleleft \alpha \triangleleft \sigma_{\rho'}$. Hence
    \begin{align*}
      \restrict{(\phi_d \circ \psi_d-\identity)}{\omega',\omega}
      =
      \sum_{\rho\in P(\Sigma),\sigma_\rho = \omega} \partial_{\omega',\tau_\rho}\circ\partial_\rho
      +
      \sum_{\rho\in P(\Sigma),\tau_\rho=\omega'} \partial_\rho \circ \partial_{\sigma_\rho,\omega}
    \end{align*}
    for all $\omega,\omega'\in \Sigma(K)$. But now note if
    $\omega\in \Sigma(K)$ then $\sigma_\rho\neq\omega$ for all
    $\rho\in P(\Sigma)$ and similarly for the sum on the right.
    Thus $\phi_d\circ\psi_d-\identity = 0$.

    In the other direction we assume $\Sigma=\left\lbrace{\sigma_\rho
    \triangleleft\tau_\rho}\right\rbrace$ and proceed by induction.
    We see that
    \begin{align*}
      \restrict{(\psi_d\circ\phi_d)}{\alpha',\alpha}
      = \sum_{\omega\in \Sigma(K_d)}
      \restrict{\psi_d}{\alpha',\omega}\circ\restrict{\phi_d}{\omega,\alpha}
    \end{align*}
    is zero unless $\alpha=\alpha'$, $\alpha\neq\tau_\rho=\alpha'$,
    or $\alpha=\sigma_\rho\neq\alpha'$. Computation in each
    case is straightforward and we find
    \begin{align*}
      \restrict{(\psi_d\circ\phi_d)}{\alpha',\alpha}
      = \begin{cases}
        \identity & \alpha=\alpha', \\
        -\partial_{\alpha',\tau_\rho}\circ\partial^{-1}_{\alpha,\tau_\rho}
                  & \alpha = \sigma_\rho \neq \alpha', \\
                  -\partial^{-1}_{\sigma_\rho,\alpha'} \circ \partial_{\sigma_\rho,\alpha}
                  & \alpha \neq \tau_\rho = \alpha', \\
        0         & \text{otherwise}.
      \end{cases}
    \end{align*}
    We define $\theta_d : C_d(K;\mathscr C)\to C_{d+1}(K;\mathscr C)$ by
    \begin{align*}
      \restrict{\theta_d}{\alpha',\alpha} = \begin{cases}
        {\partial}^{-1}_{\sigma_\rho,\tau_\rho} & \alpha=\sigma_\rho,\alpha'=\tau_\rho \\
        0 &\text{otherwise}
      \end{cases}
    \end{align*}
    It is straightforward to check that we have
    \begin{align*}
      \psi_d\circ\phi_d - \identity = \partial\circ\theta_d + \theta_{d-1}\circ\partial,
    \end{align*}
    hence we have constructed a chain homotopy for the
    case $|\Sigma|=1$. While induction still works, we remind
    the reader that, analogous to the non-cosheaf case,
    this depends on acylicity of $\Sigma$. This is because
    a boundary component $\restrict{\partial^\Sigma}{\tau,\sigma}$
    does not depend on any other pair $\sigma'\triangleleft\tau'$.
    Hence we may add pairs to $\Sigma$ one by one, in any
    order.
  \end{proof}
\end{theorem}

If we now consider a filtration $\mathscr C^\bullet$ it is
immediate that the homologies agree.

\begin{corollary}
  For each $i$, the complexes $(C_\bullet(K;\mathscr C^i),\partial)$
  and $(C_\bullet^\Sigma(K;\mathscr C^i),\partial^\Sigma)$ are
  chain-homotopic. Hence
  \begin{align*}
    H_\bullet (K;\mathscr C^i) \cong H_\bullet^\Sigma (K;\mathscr C^i).
  \end{align*}
\end{corollary}

While in the elementary case without cosheaves this was the
end of the story, we still require that the maps
$H_d(K;\mathscr C^i)\hookrightarrow H_d(K;\mathscr C^{i+1})$ and
$H^\Sigma_d(K;\mathscr C^i)\hookrightarrow H^\Sigma_d(K;\mathscr C^{i+1})$
agree. This will then apply that the persistent homologies
are the same.

We show that the induced inclusions are chain homotopic under the
chain homotopy equivalence. Precisely:

\begin{lemma}
  The induced map $C^\Sigma_\bullet(K;\mathscr C^i)\longinc C^\Sigma_\bullet(K;\mathscr C^{i+1})$ is chain homotopic to
  \begin{align*}
    C^\Sigma_\bullet(K;\mathscr C^i)
    \xlongrightarrow{\psi_\bullet} C_\bullet(K;\mathscr C^i)
    \longinc C_\bullet(K;\mathscr C^{i+1})
    \xlongrightarrow{\phi_\bullet} C^\Sigma_\bullet(K;\mathscr C^{i+1})
  \end{align*}
  \begin{proof}
    Fix $d$ and consider the diagram
    \begin{equation*}
      % https://q.uiver.app/#q=WzAsNCxbMCwwLCJDXlxcU2lnbWFfZChLO1xcbWF0aHNjciBDXmkpIl0sWzIsMCwiQ15cXFNpZ21hX2QoSztcXG1hdGhzY3IgQ157aSsxfSkiXSxbMCwxLCJDX2QoSztcXG1hdGhzY3IgQ15pKSJdLFsyLDEsIkNfZChLO1xcbWF0aHNjciBDXntpKzF9KSJdLFswLDIsIlxccHNpIiwyXSxbMywxLCJcXHBoaSIsMl0sWzAsMSwiXFxpb3RhXlxcU2lnbWEiLDAseyJzdHlsZSI6eyJ0YWlsIjp7Im5hbWUiOiJob29rIiwic2lkZSI6InRvcCJ9fX1dLFsyLDMsIlxcaW90YSIsMix7InN0eWxlIjp7InRhaWwiOnsibmFtZSI6Imhvb2siLCJzaWRlIjoidG9wIn19fV1d
      \begin{tikzcd}
        {C^\Sigma_d(K;\mathscr C^i)} && {C^\Sigma_d(K;\mathscr C^{i+1})} \\
        {C_d(K;\mathscr C^i)} && {C_d(K;\mathscr C^{i+1})}
        \arrow["\psi"', from=1-1, to=2-1]
        \arrow["\phi"', from=2-3, to=1-3]
        \arrow["{\iota^\Sigma}", hook, from=1-1, to=1-3]
        \arrow["\iota"', hook, from=2-1, to=2-3]
      \end{tikzcd}
    \end{equation*}
    But now observe that $\phi$ and $\psi$ are sums of
    scalar multiples of boundary components. As $\iota$
    and $\iota^\Sigma$ are chain maps, we are free to
    interchange them with boundary components. As
    $\psi$ and $\phi$ form a chain homotopy equivalence,
    the claim follows.
  \end{proof}
\end{lemma}

Note that we could have chosen the vertical maps to go in any
direction and the claim would have remained true. However, only
one such homotopy is required to establish our main theorem:

\begin{theorem}
  $PH^\Sigma_{d,i\to j}(K;\mathscr C^\bullet) = PH_{d,i\to j}(K;\mathscr C^\bullet)$
\end{theorem}

\section{Duality between sheaves and cosheaves}

We have taken great care to introduce and study cosheaves
on finite dimensional vector spaces to generalise persistent
homology. This raises two questions: Could we just have used
sheaves instead? What about infinite dimensional vector spaces?

To answer the first question, it is possible to define sheaf
cohomology in quite the same way to what we have done in this
report. For more general settings, this is indeed the usual
approach. It is then also possible to obtain a notion of
cohomology with values in a sheaf by dualising in the usual way.

Morover, for sheaves and cosheaves taking values in finite
dimensional vector spaces it truly does not matter which
one we deal with:

\begin{theorem}[{\cite[Definition 7.11]{curry2014}}]
  For every field $k$ and every simplicial complex $K$, there is a
  contravariant equivalence of categories
  \begin{align*}
    \hat V: \Sh(K;\FVect_k)^{\text{op}} \cong \Cosh(K;\FVect_k)
  \end{align*}
  between the categories of sheaves and cosheaves of finite
  dimensional vector spaces over $k$ on $K$, respectively.,
\end{theorem}

Moreover, this equivalence plays nicely with homology and
cohomology:

\begin{lemma}[{\cite[Lemma 7.12]{curry2014}}]
  For all $\mathscr F\in\Sh(K;\FVect_k)$,
  \begin{align*}
    H_\bullet(K; \hat V(\mathscr F)) \cong H^\bullet(K;\mathscr F).
  \end{align*}
\end{lemma}

Note that this duality has some significant limitations: It only
works on simplicial complexes which correspond to a very restricted
class of topological spaces and we only allow values to be
finite dimensional vector spaces.
The problem with generalising this duality to all topological
spaces lies in the fact while there is a canonical way to turn
a general presheaf into a sheaf, called sheafification,
such a process does, in general, not exist for precosheaves
and cosheaves. Unfortunately, sheafification is used to define
sheaf cohomology and hence general cosheaf homology seems
unfeasible. \cite{kishimoto2022}

Moreover, the categories of sheaves and cosheaves are not equivalent
if we allow values to be general vector spaces. This however
should hardly be a problem in a computational setting.


\printbibliography

\end{document}
