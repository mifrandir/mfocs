\documentclass{article}
\usepackage{assignment}
\begin{document}
\title{C3.9 Mini Project: Cosheaf Homology}
\author{Franz Miltz}
\date{\today}
\maketitle

\section{Cosheaves and homology}

\begin{definition}
  A \emph{cosheaf} on a simplicial complex $K$ is a functor
  $\mathscr C:(K,\geq)\to\FVect_{\mathbb{R}}$.
\end{definition}
Let us familiarise ourselves with the notion of a cosheaf by looking
at some examples:
\begin{example}
  \begin{itemize}
    \item Let $V\in\FVect_{\mathbb{R}}$. Then the constant cosheaf
      $\underline V_K$ sends each simplex to $V$ and each inclusion
      $\tau\hookrightarrow\tau'$ to the identity $V\to V$. Note that
      this is the same as the constant sheaf $\underline V_K$, hence
      the notation is consistent.
    \item Cosheaves may be used to emphasise or deemphasise certain
      subcomplexes of $K$. For example, if we have a subcomplex
      $K'\subseteq K$ then we may define a cosheaf to be
      $\underline{\mathbb{R}}_K$ on $K'$ and $\underline{0}_K$
      everywhere else. Explicitly, we define the characteristic
      cosheaf of $K'\subseteq K$ to be
      \begin{align*}
        \mathscr I_{K'}(\tau) := \begin{cases}
          \mathbb{R} & \text{if }\tau\in K' \\
          0 &\text{otherwise}
        \end{cases}
      \end{align*}
      with $\mathscr I_{K'}(\tau\geq\sigma) = \identity$ whenever $\tau\in K'$.
      Of course, if $K=K'$ then $\mathscr I_K=\underline{\mathbb{R}}_K$.
    \item Subcomplexes also enable us to extend cosheaves to larger
      domains. Consider once again a subcomplex $K'\subseteq K$
      and a cosheaf $\mathscr C$ on $K$. By restricting
      $\mathscr C$ to a functor
      $\restrict{\mathscr C}{K'}:(K',\geq)\to\FVect_{\mathbb{R}}$
      we obtain a cosheaf on $K'$.
  \end{itemize}
\end{example}

\begin{definition}
  Let $\mathscr C$ be a cosheaf on a simplicial complex $K$. Then, for $k\geq 0$,
  define the \emph{vector space of $k$-chains of $K$ with coefficients
  in $\mathscr C$} by
  \begin{align*}
    C_k(K;\mathscr C) = \bigoplus_{\tau\in K_k} \mathscr C(\tau).
  \end{align*}
  where $K_k = \left\lbrace{\sigma \in K : \dim \sigma = k}\right\rbrace$.
\end{definition}

Observe that this is a direct generalisation of the usual chain vector
spaces as the case $\mathscr C = \underline{\mathbb{R}}_K$ yields the
equality $C_k(K;\underline{\mathbb{R}}_K)=C_k(K)$. Of
course we are not truly interested in the chains but rather in the
chain complex that they are part of. It is thus natural to define
the corresponding boundary operator:

\begin{definition}
  Let $\sigma,\tau\in K$. Then the \emph{boundary component}
  is the map
  \begin{align*}
    \partial_{\sigma,\tau} = [\sigma:\tau]\mathscr C(\tau\geq\sigma) : \mathscr C(\tau) \to \mathscr C(\sigma).
  \end{align*}
  The \emph{boundary map}
  \begin{align*}
    \partial : C_k(K;\mathscr C) \to C_{k-1}(K;\mathscr C)
  \end{align*}
  is the unique linear map whose components are $\partial_{\sigma,\tau}$.
\end{definition}

Observe that the component $\partial_{\sigma,\tau}$ makes sense even if $\sigma\not\leq\tau$because
then $[\sigma:\tau]=0$ so $\partial_{\sigma,\tau}$ is the zero map. Let us verify that
this does indeed deserve to be called a boundary map:

\begin{lemma}\label{thm:cosheaf_chain_complex}
  $\partial^2 = 0$.
  \begin{proof}
    We have the $\mathscr C(\tau)\to\mathscr C(\sigma)$ component
    of $\partial^2$:
    \begin{align*}
      \sum_{\delta\in K_{k-1}} \partial_{\sigma,\delta}\circ \partial_{\delta,\tau} = \left(\sum_{\delta\in K_{k-1}}[\sigma:\delta][\delta:\tau]\right)\mathscr C(\tau\geq\sigma).
    \end{align*}
    Write $\tau = \left\lbrace{v_1,\ldots,v_n}\right\rbrace$. And assume
    $\sigma = \tau \setminus \left\lbrace{v_i,v_j}\right\rbrace$ with $i<j$.
    Then $[\sigma:\delta][\delta:\tau]$  is non-zero if, and only if,
    $\sigma$ is a face of $\delta$ which is a face of $\tau$. Hence
    $\delta = \tau_{-i}$ or $\delta = \tau_{-j}$. We then note
    $\sigma = (\tau_{-j})_{-i} = (\tau_{-i})_{-(j-1)}$. It is now
    straightforward to calculate
    \begin{align*}
      [\sigma:\tau_{-i}][\tau_{-i}:\tau]=(-1)^{j-1}(-1)^i
      = -(-1)^i(-1)^j =
      -[\sigma:\tau_{-j}][\tau_{-j}:\tau]
    \end{align*}
    and hence $\partial^2=0$.
  \end{proof}
\end{lemma}

\begin{definition}
  The \emph{cosheaf homology of $K$ with $\mathscr C$-coefficients} is the
  homology of the complex
  \begin{align*}
    \cdots\xlongrightarrow{\partial} C_k(K;\mathscr C)
    \xlongrightarrow{\partial}C_{k-1}(K;\mathscr C)
    \xlongrightarrow{\partial}\cdots.
  \end{align*}
  Explicitly,
  \begin{align*}
    H_k(K;\mathscr C) = \frac{\ker(C_k(K;\mathscr C) \to C_{k-1}(K;\mathscr C))}{\im(C_{k+1}(K;\mathscr C) \to C_k(K;\mathscr C))}
  \end{align*}
\end{definition}

Once again this is a direct generalisation of singular homology as
$\partial : C_k(K;\underline{\mathbb{R}}_K)
\to C_{k-1}(K;\underline{\mathbb R}_K)$ agrees with the usual
boundary map $\partial : C_k(K)\to C_{k-1}(K)$ and thus
\begin{align*}
  H_k(K;\underline{\mathbb{R}}_K) = H_k(K).
\end{align*}

We finish this section by considering some examples:

\begin{example}
  Consider the 1-disk $K=D_1$ with vertices and edges as in
  \begin{equation*}
    % https://q.uiver.app/#q=WzAsMixbMCwwLCJcXHNpZ21hXzEiXSxbMiwwLCJcXHNpZ21hXzIiXSxbMCwxLCJcXHRhdSIsMSx7InN0eWxlIjp7ImhlYWQiOnsibmFtZSI6Im5vbmUifX19XV0=
    \begin{tikzcd}
      {\sigma_1} && {\sigma_2}
      \arrow["\tau"{description}, no head, from=1-1, to=1-3]
    \end{tikzcd}
  \end{equation*}
  Note that this is contractible, hence $H_0(K)=\mathbb{R}$ and
  $H_i(K) = 0$ for $i>0$.
  Define the chosheaf $\mathscr C\in\Cosh(K)$ by
  \begin{equation*}
    % https://q.uiver.app/#q=WzAsMyxbMCwwLCJcXG1hdGhiYiBSIl0sWzQsMCwiMCJdLFsyLDAsIlxcbWF0aGJiIFJeMiJdLFsyLDAsIih4LHkpXFxtYXBzdG8geCIsMl0sWzIsMV1d
    \begin{tikzcd}
      {\mathbb R} && {\mathbb R^2} && 0
      \arrow["{(x,y)\mapsto x}"', from=1-3, to=1-1]
      \arrow[from=1-3, to=1-5]
    \end{tikzcd}
  \end{equation*}
  It is then straightforward to calculate the homologies
  $H_0(K;\mathscr C)=0$ and $H_1(K;\mathscr C)=\mathbb{R}$.
  Thus even though $K$ is contractible its higher homologies need
  not be trivial!
\end{example}

\begin{example}
  Consider the 1-sphere $K=S^1$ with vertices and edges as in
  \begin{equation*}
    % https://q.uiver.app/#q=WzAsMyxbMCwwLCJcXHNpZ21hXzEiXSxbMiwwLCJcXHNpZ21hXzIiXSxbMSwxLCJcXHNpZ21hXzMiXSxbMCwyLCJcXHRhdV8zIiwxLHsic3R5bGUiOnsiaGVhZCI6eyJuYW1lIjoibm9uZSJ9fX1dLFswLDEsIlxcdGF1XzEiLDEseyJzdHlsZSI6eyJoZWFkIjp7Im5hbWUiOiJub25lIn19fV0sWzIsMSwiXFx0YXVfMiIsMSx7InN0eWxlIjp7ImhlYWQiOnsibmFtZSI6Im5vbmUifX19XV0=
    \begin{tikzcd}
      {\sigma_1} && {\sigma_2} \\
                 & {\sigma_3}
                 \arrow["{\tau_3}"{description}, no head, from=1-1, to=2-2]
                 \arrow["{\tau_1}"{description}, no head, from=1-1, to=1-3]
                 \arrow["{\tau_2}"{description}, no head, from=2-2, to=1-3]
    \end{tikzcd}
  \end{equation*}
  Note that the homology in the first dimension is $H_1(K) = \mathbb{R}$,
  i.e. non-trivial. However, if we consider
  $\mathscr C=\underline{0}_K$ then $H_i(K;\underline{0}_K)=0$
  for all $i$. This is an instance of a simplicial complex with non-trivial
  $\mathbb{R}$-homology but trivial $\mathscr C$-homology.
\end{example}

\section{Maps}

As we are now familiar with cosheaves by themselves, it is the logical
next step to see how one cosheaf might relate to another. In particular,
we are going to be interested in defining maps between cosheaves. Before
we move on to persistent homology, we will then take some time to
look at the category of cosheaves on a particular simplicial complex.

As cosheaves are functors, it is unsurprising that maps of cosheaves
are natural transformations:

\begin{definition}
  A \emph{map of cosheaves} $\varphi:\mathscr C\to\mathscr D$ is
  just a natural transformation, i.e. it consists of maps
  $\varphi_\tau:\mathscr C(\tau)\to\mathscr D(\tau)$ for every simplex
  $\tau\in K$ such that, for all $\sigma\leq\tau$, the following commutes:
  \begin{equation}\label{eq:naturality}
    % https://q.uiver.app/#q=WzAsNCxbMCwwLCJcXG1hdGhzY3IgQyhcXHRhdSkiXSxbMiwwLCJcXG1hdGhzY3IgQyhcXHNpZ21hKSJdLFswLDEsIlxcbWF0aHNjciBEKFxcdGF1KSJdLFsyLDEsIlxcbWF0aHNjciBEKFxcc2lnbWEpIl0sWzAsMiwiXFxwaGlfXFx0YXUiLDJdLFsyLDMsIlxcbWF0aHNjciBEKFxcc2lnbWFcXGxlcVxcdGF1KSIsMl0sWzEsMywiXFxwaGlfXFxzaWdtYSJdLFswLDEsIlxcbWF0aHNjciBDKFxcc2lnbWFcXGxlcVxcdGF1KSJdXQ==
    \begin{tikzcd}
      {\mathscr C(\tau)} && {\mathscr C(\sigma)} \\
      {\mathscr D(\tau)} && {\mathscr D(\sigma)}
      \arrow["{\varphi_\tau}"', from=1-1, to=2-1]
      \arrow["{\mathscr D(\tau\geq\sigma)}"', from=2-1, to=2-3]
      \arrow["{\varphi_\sigma}", from=1-3, to=2-3]
      \arrow["{\mathscr C(\tau\geq\sigma)}", from=1-1, to=1-3]
    \end{tikzcd}
  \end{equation}
\end{definition}

Thus we have the category $\Cosh(K) :=
\text{Fun}((K,\geq),\FVect_{\mathbb{R}})$.
Considering a subcomplex $K'\subseteq K$ we have a on objects
$\Cosh(K')\hookrightarrow\Cosh(K)$ extending a cosheaf
$\mathscr C\in\Cosh(K')$ to a cosheaf on $K$ by setting
$\mathscr C(\tau)=0$ whenever $\tau\not\in K'$. This canonically
extends to a functor by observing that all the linear components
not specified by a map $\varphi:\mathscr C\to\mathscr D$ in $\Cosh(K')$
have signature $0\to 0$ and are thereby uniquely determined.
This very observation now also implies that we have a full subcategory
\begin{equation}\label{eq:subcomplex_full_subcategory}
  I:\Cosh(K') \longinc \Cosh(K).
\end{equation}
Note that the characteristic cosheaf of $K'$ on $K$
arises from $\underline{\mathbb{R}}_{K'}$ under this inclusion.
Moreover, by restriction we obtain a functor in the opposite
direction:
\begin{align*}
  R : \Cosh(K)\longrightarrow\Cosh(K').
\end{align*}
It is clear that these cannot form an equivalence of categories:
$K$ is a larger complex so a functor on $K$ contains more data
than its restriction to $K'$. However, we do have the next best
thing which is an adjuction:

\begin{proposition}
  There is an adjunction $R\dashv I$.
  \begin{proof}
    We have $IR = \text{Id} : \Cosh(K')\to\Cosh(K')$.
    Thus we need to construct a counit
    $\varepsilon : RI\to\text{Id}$. Consider $\mathscr C\in\Cosh(K)$.
    Define
    \begin{align*}
      \varepsilon_{\mathscr C} : RI(\mathscr C)\to\mathscr C
    \end{align*}
    by setting its component at $\tau$ to be the identity if $\tau\in K'$
    or the zero map if $\tau\not\in K'$. We need to verify naturality
    only in the case where $\sigma\leq\tau$ where $\sigma\in K'$ but
    $\tau\not\in K'$. We have the diagram
    \begin{equation*}
      % https://q.uiver.app/#q=WzAsNCxbMiwxLCJcXG1hdGhzY3IgQyhcXHNpZ21hKSJdLFsyLDAsIlxcbWF0aHNjciBDKFxcdGF1KSJdLFswLDAsIlJJXFxtYXRoc2NyIEMoXFx0YXUpIl0sWzAsMSwiUklcXG1hdGhzY3IgQyhcXHNpZ21hKSJdLFsyLDEsIlxcdmFyZXBzaWxvbiJdLFszLDAsIlxcdmFyZXBzaWxvbiIsMl0sWzEsMF0sWzIsM11d
      \begin{tikzcd}
        {RI\mathscr C(\tau)} && {\mathscr C(\tau)} \\
        {RI\mathscr C(\sigma)} && {\mathscr C(\sigma)}
        \arrow["\varepsilon", from=1-1, to=1-3]
        \arrow["\varepsilon"', from=2-1, to=2-3]
        \arrow[from=1-3, to=2-3]
        \arrow[from=1-1, to=2-1]
      \end{tikzcd}
    \end{equation*}
    which trivially commutes because $RI\mathscr C(\tau)=0$.
    Finally, it is easily verified that both of the following are
    identities:
    \begin{align*}
      R = RIR \xlongrightarrow{\varepsilon R} R,\hspace{1cm}
      I = IRI \xlongrightarrow{I \varepsilon} I.
    \end{align*}
  \end{proof}
\end{proposition}

As we are working with vector valued cosheaves, it turns out that
the categorical notions of monomorphisms, epimorphisms, and isomorphisms
agree exactly with the set-theoretic notions of injections,
surjections, and bijections.

\begin{proposition}\label{thm:epi_injective}
  A map of cosheaves is mono, epi, or iso if, and only if,
  each component is injective, surjective, or bijective, respectively.
  \begin{proof}
    This follows from a more general fact: A natural transformation
    $\varphi : F \to G : \mathcal C \to\mathcal D$ is a mono, epi, or
    iso in the functor category  $\text{Fun}(\mathcal C,\mathcal D)$
    if, and only if, each component $\varphi_C : FC \to GC$ is a
    mono, epi, or iso in $\mathcal D$, respectively. One may prove this
    using the Yoneda lemma.

    Considering the case $\mathcal C = (L,\geq)$ and
    $\mathcal D=\FVect_{\mathbb{R}}$ we find that a map of cosheaves
    $\varphi : \mathscr C\to\mathscr D$ is mono, epi, or iso if,
    and only if, each linear component $\varphi_\tau$ is
    injective, surjective, or bijective, respectively.
  \end{proof}
\end{proposition}

In particular, if we have a subcomplex $K'\subseteq K$ then the
characteristic cosheaf $\mathscr I_{K'}$ of $K'$ is a subobject of the
constant sheaf $\underline{\mathbb{R}}_K$ in $\Cosh(K)$. If one
thinks of $\underline{\mathbb{R}}_K$ as the characteristic
cosheaf of $K$ then we find that each inclusion of subcomplexes
$K''\subseteq K'\subseteq K$ gives rise to a monomorphism of
characteristic cosheaves $\mathscr I_{K''} \hookrightarrow
\mathscr I_{K'}$ in $\Cosh(K)$. We are going to make use of this
observation to generalise the notion of filtration.

Let us now fix a map of cosheaves
$\varphi : \mathscr C \to \mathscr D$ on $K$. We then have an induced
linear map on $k$-chains
$\varphi_* : C_k(K;\mathscr C) \to C_k(K;\mathscr D)$
whose components are $\varphi_\sigma : \mathscr C(\sigma) \to \mathscr D(\sigma)$.
It is unsurprising that this is the component of a chain map:
\begin{proposition}\label{thm:chain_map}
  For all $k$, the following commutes:
  \begin{equation*}
    % https://q.uiver.app/#q=WzAsNCxbMCwwLCJDX2soSztcXG1hdGhzY3IgQykiXSxbMiwwLCJDX3trKzF9KEs7XFxtYXRoc2NyIEMpIl0sWzAsMSwiQ19rKEs7XFxtYXRoc2NyIEQpIl0sWzIsMSwiQ197aysxfShLO1xcbWF0aHNjciBEKSJdLFswLDEsImQiXSxbMiwzLCJkIl0sWzAsMiwiXFxwaGlfKiIsMl0sWzEsMywiXFxwaGlfKiJdXQ==
    \begin{tikzcd}
      {C_k(K;\mathscr C)} && {C_{k+1}(K;\mathscr C)} \\
      {C_k(K;\mathscr D)} && {C_{k+1}(K;\mathscr D)}
      \arrow["\partial", from=1-1, to=1-3]
      \arrow["\partial", from=2-1, to=2-3]
      \arrow["{\varphi_*}"', from=1-1, to=2-1]
      \arrow["{\varphi_*}", from=1-3, to=2-3]
    \end{tikzcd}
  \end{equation*}
  \begin{proof}
    We observe that the following commutes:
    \begin{equation*}
      % https://q.uiver.app/#q=WzAsOCxbMCwwLCJcXGJpZ29wbHVzX3tcXGRpbSBcXHNpZ21hID0ga30gXFxtYXRoc2NyIEMoXFxzaWdtYSkiXSxbMywwLCJcXGJpZ29wbHVzX3tcXGRpbSBcXHRhdSA9IGstMX0gXFxtYXRoc2NyIEMoXFx0YXUpIl0sWzAsMywiXFxiaWdvcGx1c197XFxkaW0gXFxzaWdtYSA9IGt9IFxcbWF0aHNjciBEKFxcc2lnbWEpIl0sWzMsMywiXFxiaWdvcGx1c197XFxkaW0gXFx0YXUgPSBrLTF9IFxcbWF0aHNjciBEKFxcdGF1KSJdLFsxLDEsIlxcbWF0aHNjciBDKFxcc2lnbWEpIl0sWzEsMiwiXFxtYXRoc2NyIEQoXFxzaWdtYSkiXSxbMiwxLCJcXG1hdGhzY3IgQyhcXHRhdSkiXSxbMiwyLCJcXG1hdGhzY3IgRChcXHRhdSkiXSxbMCwxLCJkIl0sWzIsMywiZCJdLFswLDIsIlxccGhpXyoiLDJdLFsxLDMsIlxccGhpXyoiXSxbMyw3XSxbMiw1XSxbMCw0XSxbNCw1XSxbMSw2XSxbNiw3XSxbNSw3XSxbNCw2XV0=
      \begin{tikzcd}
        {\bigoplus_{\sigma\in K_k} \mathscr C({\sigma})} &&& {\bigoplus_{\tau\in K_{k-1}} \mathscr C({\tau})} \\
                                                         & {\mathscr C({\sigma})} & {\mathscr C({\tau})} \\
                                                         & {\mathscr D(\sigma)} & {\mathscr D(\tau)} \\
        {\bigoplus_{\sigma\in K_k} \mathscr D(\sigma)} &&& {\bigoplus_{\tau\in K_{k-1}} \mathscr D(\tau)}
        \arrow["\partial", from=1-1, to=1-4]
        \arrow["\partial", from=4-1, to=4-4]
        \arrow["{\varphi_*}"', from=1-1, to=4-1]
        \arrow["{\varphi_*}", from=1-4, to=4-4]
        \arrow[from=4-4, to=3-3]
        \arrow[from=4-1, to=3-2]
        \arrow[from=1-1, to=2-2]
        \arrow[from=2-2, to=3-2]
        \arrow[from=1-4, to=2-3]
        \arrow[from=2-3, to=3-3]
        \arrow[from=3-2, to=3-3]
        \arrow[from=2-2, to=2-3]
      \end{tikzcd}
    \end{equation*}
    Here the middle square is just naturality (\ref{eq:naturality})
    and the diagonal maps are the projections from the product.
    The claim now follows by the universal property of products.
  \end{proof}
\end{proposition}
Thus $\varphi : \mathscr C \to \mathscr D$ induces a
chain map $\varphi_* : C(K;\mathscr C)\to C(K;\mathscr D)$
and therefore maps between homologies
$\varphi_* : H_k(K;\mathscr C)\to H_k(K;\mathscr D)$.

\begin{example}
  Consider the map $\underline{\mathbb{R}}_K\to\mathscr C$ in $\Cosh(S^1)$
  given by
  \begin{equation*}
    % https://q.uiver.app/#q=WzAsOCxbMSwxLCJcXG1hdGhiYiBSXjIiXSxbMywxLCJcXG1hdGhiYiBSXjIiXSxbNSwxLCJcXG1hdGhiYiBSIl0sWzEsMCwiXFxtYXRoYmIgUiJdLFszLDAsIlxcbWF0aGJiIFIiXSxbNSwwLCJcXG1hdGhiYiBSIl0sWzAsMSwiXFxtYXRoc2NyIEM6Il0sWzAsMCwiXFx1bmRlcmxpbmV7XFxtYXRoYmIgUn1fSzoiXSxbMSwwXSxbMSwyLCJ4IiwyXSxbNCwzXSxbNCw1XSxbNCwxLCIoeCwwKSJdLFs1LDJdLFszLDAsIih4LDApIiwyXV0=
    \begin{tikzcd}
      {\underline{\mathbb R}_K:} & {\mathbb R} && {\mathbb R} && {\mathbb R} \\
      {\mathscr C:} & {\mathbb R^2} && {\mathbb R^2} && {\mathbb R}
      \arrow[from=2-4, to=2-2]
      \arrow["x"', from=2-4, to=2-6]
      \arrow[from=1-4, to=1-2]
      \arrow[from=1-4, to=1-6]
      \arrow["{(x,0)}", from=1-4, to=2-4]
      \arrow[from=1-6, to=2-6]
      \arrow["{(x,0)}"', from=1-2, to=2-2]
    \end{tikzcd}
  \end{equation*}
  On the level of chains we then have
  \begin{equation*}
    % https://q.uiver.app/#q=WzAsNCxbMiwxLCJcXG1hdGhiYiBSXjMiXSxbMCwwLCJcXG1hdGhiYiBSIl0sWzAsMSwiXFxtYXRoYmIgUl4yIl0sWzIsMCwiXFxtYXRoYmIgUl4yIl0sWzEsMiwiKHgsMCkiLDJdLFsxLDMsIih4LHgpIl0sWzIsMCwiKHgseSx4KSIsMl0sWzMsMCwiKHgsMCx5KSJdXQ==
    \begin{tikzcd}
      {\mathbb R} && {\mathbb R^2} \\
      {\mathbb R^2} && {\mathbb R^3}
      \arrow["{(x,0)}"', from=1-1, to=2-1]
      \arrow["{(x,x)}", from=1-1, to=1-3]
      \arrow["{(x,y,x)}"', from=2-1, to=2-3]
      \arrow["{(x,0,y)}", from=1-3, to=2-3]
    \end{tikzcd}
  \end{equation*}
  Clearly homology is non-zero at $k=0$ only. We now see that
  the induced map on homology is the unique map that makes the
  following commute:
  \begin{equation*}
    % https://q.uiver.app/#q=WzAsNCxbMCwwLCJcXG1hdGhiYiBSXjIiXSxbMiwwLCJcXG1hdGhiYiBSIl0sWzAsMSwiXFxtYXRoYmIgUl4zIl0sWzIsMSwiXFxtYXRoYmIgUiJdLFswLDEsIngteSJdLFsyLDMsIngteiIsMl0sWzAsMiwiKHgsMCx5KSIsMl0sWzEsM11d
    \begin{tikzcd}
      {\mathbb R^2} && {\mathbb R} \\
      {\mathbb R^3} && {\mathbb R}
      \arrow["{x-y}", from=1-1, to=1-3]
      \arrow["{x-z}"', from=2-1, to=2-3]
      \arrow["{(x,0,y)}"', from=1-1, to=2-1]
      \arrow[from=1-3, to=2-3]
    \end{tikzcd}
  \end{equation*}
  Here the horizontal maps are the quotient maps from
  chains to homology. Thus the induced map in homology must be the
  identity even though the map of cosheaves is not an isomorphism
  and the cosheaves are not isomorphic.
\end{example}

The previous example shows that we may have a mono
$\underline{\mathbb{R}}_K\to\mathscr C$ which induces an iso
between homologies. One may reasonably ask whether the dual
situation may arise. The answer is no.

\begin{lemma}
  Let $\varphi : \underline{\mathbb{R}}_K\to\mathscr C$ in $\Cosh(K)$
  be an epi such that
  $\varphi_* : H_*(K;\underline{\mathbb{R}}_K)\to H_*(K;\mathscr C)$
  is an iso. Then $\varphi$ is an iso.
  \begin{proof}
    Suppose $\varphi$ is not an iso. Then some component
    $\varphi_\tau : \mathbb{R}\to \mathscr C(\tau)$ is surjective but
    not bijective. Thus we must have $\mathscr C(\tau)=0$.
    Now choose a sequence $\sigma=\tau_0\leq\cdots\leq\tau_m = \tau$
    such that $\dim\tau_d = d$. Observe that naturality yields
    a commutative diagram:
    \begin{equation*}
      % https://q.uiver.app/#q=WzAsOCxbMCwwLCJcXG1hdGhiYiBSIl0sWzIsMCwiXFxtYXRoYmIgUiJdLFs2LDAsIlxcbWF0aGJiIFIiXSxbMCwyLCJcXG1hdGhzY3IgQyhcXHRhdV9tKSJdLFsyLDIsIlxcbWF0aHNjciBDKFxcdGF1X3ttLTF9KSJdLFs2LDIsIlxcbWF0aHNjciBDKFxcdGF1XzEpIl0sWzQsMiwiXFxjZG90cyJdLFs0LDAsIlxcY2RvdHMiXSxbMCwzLCJcXHZhcnBoaV97XFx0YXVfbX0iLDFdLFswLDFdLFsxLDddLFs3LDJdLFszLDRdLFs0LDZdLFs2LDVdLFsxLDQsIlxcdmFycGhpX3tcXHRhdV97bS0xfX0iLDFdLFs3LDZdLFsyLDUsIlxcdmFycGhpXzEiLDFdXQ==
      \begin{tikzcd}
        {\mathbb R} && {\mathbb R} && \cdots && {\mathbb R} \\
        \\
        {\mathscr C(\tau_m)} && {\mathscr C(\tau_{m-1})} && \cdots && {\mathscr C(\tau_1)}
        \arrow["{\varphi_{\tau_m}}"{description}, from=1-1, to=3-1]
        \arrow[from=1-1, to=1-3]
        \arrow[from=1-3, to=1-5]
        \arrow[from=1-5, to=1-7]
        \arrow[from=3-1, to=3-3]
        \arrow[from=3-3, to=3-5]
        \arrow[from=3-5, to=3-7]
        \arrow["{\varphi_{\tau_{m-1}}}"{description}, from=1-3, to=3-3]
        \arrow[from=1-5, to=3-5]
        \arrow["{\varphi_1}"{description}, from=1-7, to=3-7]
      \end{tikzcd}
    \end{equation*}
    Thus $\mathscr C(\tau_d)=0$ implies $\mathscr C(\tau_{d-1})=0$
    and hence $\mathscr C(\sigma) = 0$. Now observe that
    $[\sigma:\tau]= 0$ whenever $\sigma$ and $\tau$ belong to
    different connected components. Hence the boundary map
    decomposes into its restrictions to path-connected components.
    We may therefore assume without loss of generality that
    $K$ is path-connected, i.e. $H_0(K)=\mathbb{R}$.

    We win if we are able to show that, for all $v\in C_0(K;\mathscr C)$,
    there is a $u\in C_1(K;\mathscr C)$ such that $\partial u = v$,
    i.e. $v = 0$ in $H_0(K;\mathscr C)$.
    We do so as follows: Fix $v\in\mathscr C(\sigma')$ for some
    $\sigma'\in K_0$ and use path-connectedness
    to choose $\tau'_1,\ldots,\tau'_n\in K_1$ such that
    \begin{align*}
      \sigma = \sigma'_0 \leq \tau_1' \geq \sigma'_1 \leq \cdots \geq \sigma'_{n-1} \leq \tau'_n \geq \sigma'_n = \sigma'
    \end{align*}
    for some $\sigma'_1,\ldots,\sigma'_n \in K_0$. By relabelling $\sigma$
    to be the first $\sigma'_i$ with $\mathscr C(\sigma'_i)=0$ we may
    assume without loss of generality that $\mathscr C(\sigma'_i)\neq 0$
    and by the previous argument that $\mathscr C(\tau'_i)\neq 0$ and
    all $\mathscr C(\tau'_i \geq \sigma'_j)$ are isomorphisms.
    We are now able to choose $u_i\in\mathscr C(\tau'_i)$ such that
    $\mathscr C(\tau'_n\geq\sigma')(u_n) = v$ and
    $\mathscr C(\tau'_i\geq\sigma_i)(u_i) = \mathscr C(\tau'_{i+1}\geq\sigma_i)(u_{i+1})$. We then have the boundary
    \begin{align*}
      \partial \left({\sum_{i=1}^n u_i}\right) =
      \pm \sum_{i=1}^n (\mathscr C(\tau_i\geq \sigma_i) - \mathscr C(\tau_i\geq\sigma_{i-1}))(u_i)
    \end{align*}
    which telescopes to
    \begin{align*}
      \partial\left({\sum_{i=1}^{n} u_i}\right)
      = \pm\mathscr C(\tau_n\geq\sigma')(u_n) \mp \mathscr C(\tau_1\geq\sigma)(u_1)
      = \pm\mathscr C(\tau_n\geq\sigma')(u_n) = \pm v.
    \end{align*}
    Hence $v=0$ in $H_0(K;\mathscr C)$ and so $H_0(K;\mathscr C)=0$.
  \end{proof}
\end{lemma}

\section{Persistent homology}

Fix a simplicial complex $K$. A filtration $F^\bullet$ of $K$ is
an increasing sequence of subcomplexes
\begin{align*}
  F^1 K \subseteq F^2 K \subseteq \cdots \subseteq F^m K = K
\end{align*}
and a map of filtrations $f^\bullet : F^\bullet \to G^\bullet$ is a
sequence of simplicial maps $f^i : F^i K\to G^i K$ such that $\im(f^i)
\subseteq \im(f^{i+1})$. We thus have a category $\Fil(K)$ of filtrations
of $K$.

We have seen that subcomplexes induce subcosheaves. It is thus
natural to generalise filtrations in the following way:

\begin{definition}
  A \emph{filtration} of a cosheaf $\mathscr C\in\Cosh(K)$ is a sequence of
  cosheaves $\mathscr C^\bullet = \mathscr C^1,\ldots,\mathscr C^n$ where
  $\mathscr C^n = \mathscr C$ together with monos $\mathscr C_i
  \hookrightarrow \mathscr C_{i+1}$ for $1\leq i<n$.

  A map of filtrations $\varphi^\bullet : \mathscr C^\bullet \to \mathscr D^\bullet$ is a sequence of natural transformations
  $\varphi^i:\mathscr C^i \to\mathscr D^i$ that respects the inclusions,
  i.e. the following commutes:
  \begin{equation*}
    % https://q.uiver.app/#q=WzAsOCxbMCwwLCJcXG1hdGhzY3IgQ14xIl0sWzEsMCwiXFxtYXRoc2NyIENeMiJdLFszLDAsIlxcbWF0aHNjciBDXm0iXSxbMCwxLCJcXG1hdGhzY3IgRF4xIl0sWzEsMSwiXFxtYXRoc2NyIEReMiJdLFszLDEsIlxcbWF0aHNjciBEXm0iXSxbMiwxLCJcXGNkb3RzIl0sWzIsMCwiXFxjZG90cyJdLFswLDEsIiIsMCx7InN0eWxlIjp7InRhaWwiOnsibmFtZSI6Imhvb2siLCJzaWRlIjoidG9wIn19fV0sWzEsNywiIiwwLHsic3R5bGUiOnsidGFpbCI6eyJuYW1lIjoiaG9vayIsInNpZGUiOiJ0b3AifX19XSxbNywyLCIiLDAseyJzdHlsZSI6eyJ0YWlsIjp7Im5hbWUiOiJob29rIiwic2lkZSI6InRvcCJ9fX1dLFszLDQsIiIsMCx7InN0eWxlIjp7InRhaWwiOnsibmFtZSI6Imhvb2siLCJzaWRlIjoidG9wIn19fV0sWzQsNiwiIiwwLHsic3R5bGUiOnsidGFpbCI6eyJuYW1lIjoiaG9vayIsInNpZGUiOiJ0b3AifX19XSxbNiw1LCIiLDAseyJzdHlsZSI6eyJ0YWlsIjp7Im5hbWUiOiJob29rIiwic2lkZSI6InRvcCJ9fX1dLFswLDMsIlxcdmFycGhpXjEiLDJdLFsxLDQsIlxcdmFycGhpXjIiLDJdLFsyLDUsIlxcdmFycGhpXm0iLDJdXQ==
    \begin{tikzcd}
      {\mathscr C^1} & {\mathscr C^2} & \cdots & {\mathscr C^m} \\
      {\mathscr D^1} & {\mathscr D^2} & \cdots & {\mathscr D^m}
      \arrow[hook, from=1-1, to=1-2]
      \arrow[hook, from=1-2, to=1-3]
      \arrow[hook, from=1-3, to=1-4]
      \arrow[hook, from=2-1, to=2-2]
      \arrow[hook, from=2-2, to=2-3]
      \arrow[hook, from=2-3, to=2-4]
      \arrow["{\varphi^1}"', from=1-1, to=2-1]
      \arrow["{\varphi^2}"', from=1-2, to=2-2]
      \arrow["{\varphi^m}"', from=1-4, to=2-4]
    \end{tikzcd}
  \end{equation*}
  Denote the category of filtrations of a cosheaf $\mathscr C$ by
  $\Fil(\mathscr C)$.
\end{definition}



We previously observed that subcomplexes $K'\subseteq K$ induce
subcosheaves $\mathscr I_{K'}\hookrightarrow \mathscr I_K$. It is now
straightforwad to see that we have an inclusion of categories
\begin{align*}
  \Fil(K) \longinc \Fil(\underline{\mathbb{R}}_K)
\end{align*}
given by $F^{\bullet} \mapsto \mathscr I_{F^\bullet K}$.
Note however that $\Fil(K)$ is not a full subcategory of
$\Fil(\underline{\mathbb{R}}_K)$. For example if $K=\bullet$ is
the trivial simplicial complex then there is a unique simplicial
map $K\to K$. However, there are many linear maps
$\mathbb{R}\to\mathbb{R}$.
Moreover, there are many filtrations of cosheaves that do not
arise as a filtration of the underlying complex.

\begin{example}\label{ex:cosheaf_filtration_on_s1}
  Consider the 1-sphere $K=S^1$ with vertices and edges as in
  \begin{equation*}
    % https://q.uiver.app/#q=WzAsMyxbMCwwLCJcXHNpZ21hXzEiXSxbMiwwLCJcXHNpZ21hXzIiXSxbMSwxLCJcXHNpZ21hXzMiXSxbMCwyLCJcXHRhdV8zIiwxLHsic3R5bGUiOnsiaGVhZCI6eyJuYW1lIjoibm9uZSJ9fX1dLFswLDEsIlxcdGF1XzEiLDEseyJzdHlsZSI6eyJoZWFkIjp7Im5hbWUiOiJub25lIn19fV0sWzIsMSwiXFx0YXVfMiIsMSx7InN0eWxlIjp7ImhlYWQiOnsibmFtZSI6Im5vbmUifX19XV0=
    \begin{tikzcd}
      {\sigma_1} && {\sigma_2} \\
                 & {\sigma_3}
                 \arrow["{\tau_3}"{description}, no head, from=1-1, to=2-2]
                 \arrow["{\tau_1}"{description}, no head, from=1-1, to=1-3]
                 \arrow["{\tau_2}"{description}, no head, from=2-2, to=1-3]
    \end{tikzcd}
  \end{equation*}
  We may now define a filtration
  $\mathscr C^\bullet=\mathscr C^1\subseteq\cdots\subseteq\mathscr C^3\in\Fil(\underline{\mathbb{R}}^3_K)$ by
  \begin{align}\label{eq:s1_filtration}
    \mathscr C^i(\sigma_j) = \begin{cases}
      \mathbb{R}^{i} & \text{for }i\leq 2, \\
      \mathbb{R}^2 & \text{for }i > 2,
    \end{cases}
    \hspace{1cm}
    \text{and}
    \hspace{1cm}
    \mathscr C^i(\tau_j) = \mathbb{R}^{i-1}
  \end{align}
  where all maps $\mathscr C^i(\alpha) \hookrightarrow \mathscr C^{i+1}(\alpha)$ and $\mathscr C^i(\tau)\hookrightarrow\mathscr C^i(\sigma)$ are the
  obvious inclusions.
\end{example}

Note that a filtration $\mathscr C^\bullet$ of a cosheaf
induces a persistence module of homologies for every $k\geq 0$:
\begin{align}\label{eq:cosheaf_persistence_module}
  H_k(K;\mathscr C^0) \longinc
  H_k(K;\mathscr C^1) \longinc
  \cdots \longinc
  H_k(K;\mathscr C^n)
\end{align}

Now the $k$-th persistent homology of the filtration is just the
homology of the persistence module corresponding to the $k$-th
homology:
\begin{definition}\label{def:persistent_homology}
  The \emph{persistent homology} of a simplicial complex $K$
  with coefficients in a filtration of cosheaves $\mathscr C_i$
  is the persistent homology of the persistence module
  (\ref{eq:cosheaf_persistence_module}), i.e.
  \begin{align*}
    PH_{k,i\to j}(K;\mathscr C^\bullet) := \im\left(
      H_k(K;\mathscr C^i) \longinc
      \cdots \longinc
      H_k(K;\mathscr C^j)
    \right).
  \end{align*}
\end{definition}

\begin{example}
  Let us compute the homology groups of the cosheaf filtration
  in \ref{ex:cosheaf_filtration_on_s1}. We obtain the following
  complex of chain groups $C_i(K;\mathscr C^j)$ with $i=0,1$ from bottom to top
  and $j=1,\ldots,4$ from left to right:
  \begin{equation*}
    % https://q.uiver.app/#q=WzAsNixbMCwwLCIwIl0sWzAsMSwiXFxtYXRoYmIgUl4zIl0sWzEsMCwiXFxtYXRoYmIgUl4zIl0sWzEsMSwiXFxtYXRoYmIgUl42Il0sWzIsMSwiXFxtYXRoYmIgUl42Il0sWzIsMCwiXFxtYXRoYmIgUl42Il0sWzAsMV0sWzIsMywiZyIsMl0sWzUsNCwiaCIsMl0sWzEsMywiayIsMl0sWzMsNCwiXFxlbGwiLDJdLFswLDJdLFsyLDUsImYiXV0=
    \begin{tikzcd}
      0 & {\mathbb R^3} & {\mathbb R^6} \\
      {\mathbb R^3} & {\mathbb R^6} & {\mathbb R^6}
      \arrow[from=1-1, to=2-1]
      \arrow["g"', from=1-2, to=2-2]
      \arrow["h"', from=1-3, to=2-3]
      \arrow["k"', from=2-1, to=2-2]
      \arrow["\ell"', from=2-2, to=2-3]
      \arrow[from=1-1, to=1-2]
      \arrow["f", from=1-2, to=1-3]
    \end{tikzcd}
  \end{equation*}
  By choosing a basis $e_1,\ldots,e_n$ of $\mathbb{R}^n$ we may write
  the maps as
  \begin{align*}
    f(e_i) = e_{2i-1},\hspace{1cm}
    g(e_i) = e_{2i-1}+e_{2i+1},\hspace{1cm}
    h(e_i) = e_i + e_{i+2}, \hspace{1cm}
    k(e_i) = e_{2i-1}, \hspace{1cm}
    \ell(e_i) = e_i.
  \end{align*}
  Thus we have the complexes of homologies
  \begin{equation*}
    % https://q.uiver.app/#q=WzAsNixbMCwwLCIwIl0sWzAsMSwiXFxtYXRoYmIgUl4zIl0sWzEsMCwiXFxtYXRoYmIgUl4zIl0sWzEsMSwiXFxtYXRoYmIgUl4zIl0sWzIsMSwiMCJdLFsyLDAsIlxcbWF0aGJiIFJeNiJdLFsxLDMsImtfKiIsMl0sWzMsNCwiXFxlbGxfKiIsMl0sWzAsMl0sWzIsNSwiZl8qIl1d
    \begin{tikzcd}
      0 & {\mathbb R^3} & {\mathbb R^6} \\
      {\mathbb R^3} & {\mathbb R^3} & 0
      \arrow["{k_*}"', from=2-1, to=2-2]
      \arrow["{\ell_*}"', from=2-2, to=2-3]
      \arrow[from=1-1, to=1-2]
      \arrow["{f_*}", from=1-2, to=1-3]
    \end{tikzcd}
  \end{equation*}
  where
  \begin{align*}
    f_* = f, \hspace{1cm}
    k_* = 0, \hspace{1cm}
    \ell_* = 0.
  \end{align*}
  Therefore the only non-trivial persistent homology group is
  \begin{align*}
    PH_{1,2\to 3}(K;\mathscr C^\bullet) = \im f_* = \mathbb{R}^3.
  \end{align*}
\end{example}

\section{Acyclic partial matchings}

Consider a simplicial complex $K$. Recall that a partial matching
$\Sigma$ on $K$ is a collection of pairs $\sigma\triangleleft\tau$
such that $\sigma\leq\tau$ and no simplex appears in more than
one such pair. A $\Sigma$-path $\rho$ then is a sequence
\begin{align}\label{eq:path}
  \sigma_\rho=\sigma_1\triangleleft\tau_1,
  \sigma_2\triangleleft\tau_2,
  \cdots,
  \sigma_m\triangleleft\tau_m=\tau_\rho
\end{align}
such that $\tau_i\geq\sigma_{i+1}$ for all $i$.
We will write $P(\Sigma)$ to denote the set of all paths of $\Sigma$.
A partial matching $\Sigma$ is said to be acyclic
if there exists no path $\rho\in P(\Sigma)$ such that
$\sigma_\rho\triangleleft\tau_\rho\not\in\Sigma$ but
$\sigma_\rho\leq\tau_\rho$ in $K$.

Analogous to compatibility with a sheaf, we may define what it
means for an acyclic partial matching to be compatible with a cosheaf:
\begin{definition}\label{def:cosheaf_compatibility}
  A partial matching $\Sigma$ is \emph{$\mathscr C$-compatible} if,
  for all $\sigma\triangleleft\tau\in\Sigma$,
  $\mathscr C_{\sigma,\tau}$ is an iso.
\end{definition}

Thus a path (\ref{eq:path}) in a $\mathscr C$-compatible partial
matching gives rise to a map
$\partial_\rho : \mathscr C(\sigma_\rho)\to\mathscr C(\tau_\rho)$ as in
\begin{equation*}
  % https://q.uiver.app/#q=WzAsNSxbMCwwLCJcXG1hdGhzY3IgQyhcXHNpZ21hXzEpIl0sWzIsMCwiXFxtYXRoc2NyIEMoXFx0YXVfMSkiXSxbNCwwLCJcXGNkb3RzIl0sWzYsMCwiXFxtYXRoc2NyIEMoXFxzaWdtYV9tKSJdLFs4LDAsIlxcbWF0aHNjciBDKFxcdGF1X20pIl0sWzAsMSwiZF57LTF9X3tcXHNpZ21hXzEsXFx0YXVfMX0iXSxbMSwyLCJkX3tcXHNpZ21hXzIsXFx0YXVfMX0iXSxbMiwzLCJkX3tcXHNpZ21hX20sXFx0YXVfe20tMX19Il0sWzMsNCwiZF97XFxzaWdtYV9tLFxcdGF1X219XnstMX0iXV0=
  \begin{tikzcd}
    {\mathscr C(\sigma_1)} && {\mathscr C(\tau_1)} && \cdots && {\mathscr C(\sigma_m)} && {\mathscr C(\tau_m)}
    \arrow["{\partial^{-1}_{\sigma_1,\tau_1}}", from=1-1, to=1-3]
    \arrow["{\partial_{\sigma_2,\tau_1}}", from=1-3, to=1-5]
    \arrow["{\partial_{\sigma_m,\tau_{m-1}}}", from=1-5, to=1-7]
    \arrow["{\partial_{\sigma_m,\tau_m}^{-1}}", from=1-7, to=1-9]
  \end{tikzcd}
\end{equation*}
Given our discussion of cosheaf filtratations, it is now natural to
combine the notions of compatibility of a acyclic partial matching
with respect to filtrations of complexes and cosheaves, respectively,
into one:

\begin{definition}
  Let $\mathscr C^\bullet\in\Fil(\mathscr C)$. An acyclic partial
  matching $\Sigma$ is $\mathscr C^\bullet$-compatible if, for all
  $i$, it is $\mathscr C^i$-compatible.
\end{definition}

This clearly extends the notion of cosheaf compatibility from
\ref{def:cosheaf_compatibility}. Let us verify that it also extends
the notion of compatibility with a filtration of a complex.

Recall that an acyclic partial matching $\Sigma$ is said to be
compatible with a filtration $F^\bullet$ if,
\begin{align*}
  \text{$\sigma\in F^i K$ if, and only if, $\tau\in F^i K$}
\end{align*}
for all $i$ and all $\sigma\triangleleft\tau$.~\cite{nanda}
\begin{proposition}
  Let $F^\bullet \in\Fil(K)$ and $\Sigma$ an acyclic partial
  matching. Then $\Sigma$ is $F^\bullet$-compatible if, and only if,
  it is $\mathscr I_{F^\bullet K}$-compatible.
  \begin{proof}
    Assume $\Sigma$ is $F^\bullet$-compatible. Fix $i$ and some
    pair $\sigma\triangleleft\tau$. Now
    \begin{align*}
      \mathscr I_{F^i K}(\tau\geq\sigma) = \begin{cases}
        \identity : \mathbb{R}\to\mathbb{R} & \text{if }\sigma\in F^iK, \\
        0\to 0 &\text{otherwise.}
      \end{cases}
    \end{align*}
    Either way $\mathscr I_{F^i K}(\sigma\leq\tau)$ is an iso
    and thus $\Sigma$ is $\mathscr I_{F^\bullet K}$-compatible.

    Assume $\Sigma$ is $\mathscr I_{F^\bullet K}$-compatible.
    Fix $i$ and some pair $\sigma\triangleleft\tau$. If
    $\sigma\in F^i K$ then $\mathscr I_{F^i K}(\sigma) = \mathbb{R}$
    and $\mathscr I_{F^i K}(\sigma\leq\tau)$ must be an iso so
    $\mathscr I_{F^i K}(\tau) = \mathbb{R}$, i.e. $\tau\in F^i K$.
    Similarly, if
    $\sigma\not\in F^i K$ then $\mathscr I_{F^i K}(\sigma) = 0$
    and $\mathscr I_{F^i K}(\sigma\leq\tau)$ must be an iso so
    $\mathscr I_{F^i K}(\tau) = 0$, i.e. $\tau\not\in F^i K$.
    Hence $\Sigma$ is $F^\bullet$-compatible.
  \end{proof}
\end{proposition}

Before we move on to define the morse complex that such
partial matchings give rise to, we ought to consider some examples
and non-examples.

\begin{example}
  Note that every acyclic partial matching is trivially
  $\underline{V}_K$-compatible for all $V\in\FVect_{\mathbb{R}}$.
  In particular, the acyclic partial matching for the complex in
  \ref{ex:cosheaf_filtration_on_s1} given by
  \begin{align*}
    \Sigma = \left\lbrace{
        \sigma_1\triangleleft\tau_1,
        \sigma_2\triangleleft\tau_2
    }\right\rbrace
  \end{align*}
  is compatible with the cosheaf $\underline{\mathbb{R}}^3_K$ but
  not with the filtration (\ref{eq:s1_filtration}). In fact, no
  non-empty partial matching on $S^1$ can every be compatible with
  this filtration as, for no $\sigma\leq\tau$, is
  $\mathscr C^i(\sigma\leq\tau)$ an isomorphism for all $i$.
\end{example}

\begin{example}\label{ex:acyclic_partial_matching_s1}
  Consider a different filtration $\mathscr C^\bullet\in\Fil(S^1)$ given by
  \begin{equation*}
    \mathscr C^1(S^1):
    % https://q.uiver.app/#q=WzAsNixbMCwwLCIwIl0sWzIsMiwiMCJdLFsxLDEsIjAiXSxbMywxLCIwIl0sWzIsMCwiMCJdLFs0LDAsIlxcbWF0aGJiIFIiXSxbMiwwXSxbMiwxXSxbNCwwXSxbMywxXSxbMyw1XSxbNCw1XV0=
    \begin{tikzcd}
      0 && 0 && {\mathbb R} \\
        & 0 && 0 \\
        && 0
        \arrow[from=2-2, to=1-1]
        \arrow[from=2-2, to=3-3]
        \arrow[from=1-3, to=1-1]
        \arrow[from=2-4, to=3-3]
        \arrow[from=2-4, to=1-5]
        \arrow[from=1-3, to=1-5]
    \end{tikzcd}
  \end{equation*}
  \begin{equation*}
    \mathscr C^2(S^1):
    % https://q.uiver.app/#q=WzAsNixbMCwwLCIwIl0sWzIsMiwiXFxSIl0sWzEsMSwiMCJdLFszLDEsIlxcUiJdLFsyLDAsIjAiXSxbNCwwLCJcXG1hdGhiYiBSIl0sWzIsMF0sWzIsMV0sWzQsMCwiXFx0ZXh0e2lkfSIsMV0sWzMsMSwiXFx0ZXh0e2lkfSIsMV0sWzMsNSwiXFx0ZXh0e2lkfSIsMV0sWzQsNSwiXFx0ZXh0e2lkfSIsMV1d
    \begin{tikzcd}
      0 && 0 && {\mathbb R} \\
        & 0 && \mathbb{R} \\
        && \mathbb{R}
        \arrow[from=2-2, to=1-1]
        \arrow[from=2-2, to=3-3]
        \arrow[from=1-3, to=1-1]
        \arrow["{-\text{id}}"{description}, from=2-4, to=3-3]
        \arrow["{\text{id}}"{description}, from=2-4, to=1-5]
        \arrow[from=1-3, to=1-5]
    \end{tikzcd}
  \end{equation*}
  \begin{equation*}
    \mathscr C^3(S^1):
    % https://q.uiver.app/#q=WzAsNixbMCwwLCJcXFIiXSxbMiwyLCJcXFIiXSxbMSwxLCIwIl0sWzMsMSwiXFxSIl0sWzIsMCwiXFxSIl0sWzQsMCwiXFxSXjIiXSxbMiwwXSxbMiwxXSxbNCwwLCJcXHRleHR7aWR9IiwxXSxbMywxLCIoLTEpXmlcXHRleHR7aWR9IiwxXSxbMyw1LCIiLDIseyJzdHlsZSI6eyJ0YWlsIjp7Im5hbWUiOiJob29rIiwic2lkZSI6InRvcCJ9fX1dLFs0LDUsIiIsMCx7InN0eWxlIjp7InRhaWwiOnsibmFtZSI6Imhvb2siLCJzaWRlIjoidG9wIn19fV1d
    \begin{tikzcd}
      \mathbb{R} && \mathbb{R} && {\mathbb{R}^2} \\
                 & 0 && \mathbb{R} \\
                 && \mathbb{R}
                 \arrow[from=2-2, to=1-1]
                 \arrow[from=2-2, to=3-3]
                 \arrow["{-\text{id}}"{description}, from=1-3, to=1-1]
                 \arrow["{\text{id}}"{description}, from=2-4, to=3-3]
                 \arrow[hook, from=2-4, to=1-5]
                 \arrow[hook, from=1-3, to=1-5]
    \end{tikzcd}
  \end{equation*}
  where we choose a single inclusion
  $\mathbb{R}\hookrightarrow\mathbb{R}^2$ throughout and the component
  $\mathscr C^2(\sigma_3)\hookrightarrow\mathscr C^3(\sigma_3)$
  is $-\identity$.

  If we now choose $\Sigma = \left\lbrace{\sigma_1\triangleleft\tau_1,
  \sigma_3\triangleleft\tau_2}\right\rbrace$ then $\Sigma$ is an
  acyclic partial matching that, moreover, is
  $\mathscr C^\bullet$-compatible.

  One may also wish to calculate the corresponding persistent homologies.
  We have the complex of chain groups
  \begin{equation*}
    % https://q.uiver.app/#q=WzAsNixbMCwwLCIwIl0sWzIsMCwiXFxtYXRoYmIgUiJdLFs0LDAsIlxcbWF0aGJiIFJeMiJdLFswLDIsIlxcbWF0aGJiIFIiXSxbMiwyLCJcXG1hdGhiYiBSXjIiXSxbNCwyLCJcXG1hdGhiYiBSXjQiXSxbMSwyLCIoMCx4KSJdLFsyLDUsIigteCx4K3ksMCx5KSJdLFs0LDUsIigwLHgsMCwteSkiLDJdLFsxLDQsIih4LC14KSIsMl0sWzMsNCwiKHgsMCkiLDJdLFswLDNdLFswLDFdXQ==
    \begin{tikzcd}
      0 && {\mathbb R} && {\mathbb R^2} \\
      \\
      {\mathbb R} && {\mathbb R^2} && {\mathbb R^4}
      \arrow["{(0,x)}", from=1-3, to=1-5]
      \arrow["{(-x,x+y,0,y)}", from=1-5, to=3-5]
      \arrow["{(0,x,0,-y)}"', from=3-3, to=3-5]
      \arrow["{(x,-x)}"', from=1-3, to=3-3]
      \arrow["{(x,0)}"', from=3-1, to=3-3]
      \arrow[from=1-1, to=3-1]
      \arrow[from=1-1, to=1-3]
    \end{tikzcd}
  \end{equation*}
  and thus the corresponding homology complexes:
  \begin{equation*}
    % https://q.uiver.app/#q=WzAsOCxbMSwwLCIwIl0sWzMsMCwiMCJdLFs1LDAsIjAiXSxbMSwxLCJcXG1hdGhiYiBSIl0sWzMsMSwiXFxtYXRoYmIgUiJdLFs1LDEsIlxcbWF0aGJiIFJeMiJdLFswLDAsIkhfMShTXjE7XFxtYXRoc2NyIENeXFxidWxsZXQpOiJdLFswLDEsIkhfMChTXjE7XFxtYXRoc2NyIENeXFxidWxsZXQpOiJdLFsxLDJdLFs0LDUsIiIsMix7InN0eWxlIjp7InRhaWwiOnsibmFtZSI6Imhvb2siLCJzaWRlIjoidG9wIn19fV0sWzMsNCwiXFx0ZXh0e2lkfSIsMl0sWzAsMV1d
    \begin{tikzcd}
      {H_1(S^1;\mathscr C^\bullet):} & 0 && 0 && 0 \\
      {H_0(S^1;\mathscr C^\bullet):} & {\mathbb R} && {\mathbb R} && {\mathbb R^2}
      \arrow[from=1-4, to=1-6]
      \arrow[hook, from=2-4, to=2-6]
      \arrow["{\text{id}}"', from=2-2, to=2-4]
      \arrow[from=1-2, to=1-4]
    \end{tikzcd}
  \end{equation*}
  The persistent homology groups may now simply be read off.
\end{example}

\section{Morse complex}

Recall also that the critical points of a matching are those simplices
that do not show up in any of the pairs. That is,
\begin{align*}
  C(\Sigma) = K_0 \setminus \bigcup_{\sigma\triangleleft\tau\in\Sigma} \left\lbrace{\sigma,\tau}\right\rbrace
\end{align*}
Such simplices are important for homology with field coefficients
because they give rise to a new chain complex whose homology agrees
with the usual homology. In particular, this means that the computation
of homology groups may be sped up by finding a good acyclic partial
matching.

We would like to take this idea and extend it to persistent homology with
cosheaf filtration coefficients. That is, we wish to construct
vector spaces $C^\Sigma_k(K;\mathscr C^i)$ and maps
\begin{align*}
  \partial^\Sigma_{k,i} : C^\Sigma_k(K;\mathscr C^i)\to C^\Sigma_{k-1}(K;\mathscr C^i)
\end{align*}
whose persistent homologies
\begin{align*}
  PH^\Sigma_{k,i\to j}(K;\mathscr C^\bullet)
  = \im(H^\Sigma_k(K;\mathscr C^i)\longinc\cdots\longinc H_k^\Sigma(K;\mathscr C^j))
\end{align*}
agree with those defined in \ref{def:persistent_homology}. Recalling that
we may regard $\mathbb{R}$-homology as $\underline{\mathbb{R}}_K$-homology,
we define the vector space of critical chains to be
\begin{align*}
  C_k^\Sigma(K;\mathscr C^i) := \bigoplus_{\sigma\in C(\Sigma)\cap K_k} \mathscr C^i(\sigma).
\end{align*}

\begin{definition}
  The \emph{$(k,i)$-Morse boundary operator} is the linear map
  \begin{align*}
    \partial_{k,i}^\Sigma : C_k^\Sigma(K;\mathscr C^i) \to C_{k-1}^\Sigma(K;\mathscr C^i)
  \end{align*}
  whose $\mathscr C^i(\tau)\to\mathscr C^i(\sigma)$ component is
  \begin{align}\label{eq:morse_boundary_component}
    \partial_{\sigma,\tau} +
    \sum_{\rho\in C(\Sigma)} (-1)^{|\rho|} \partial_{\sigma,\tau_\rho}\circ \partial_\rho \circ \partial_{\sigma_\rho,\tau}.
  \end{align}
\end{definition}

We once again remind the reader that the expression (\ref{eq:morse_boundary_component}) makes sense even for those $\rho$ where
$\sigma$ is not a face of $\tau_\rho$ or $\sigma_\rho$ is not a face of
$\tau$ as then the boundary components are simply zero. Let us verify
that these boundary operators do indeed yield a chain complex.

\begin{lemma}
  $\left({\partial^\Sigma}\right)^2=0$.
  \begin{proof}
    Consider a component
    $\mathscr C^i(\tau)\to\mathscr C^i(\sigma)$ for $\dim\tau=\dim\sigma+2$.
    This must be of the form
    \begin{align*}
      \sum_{\delta}
      \left(
        \partial_{\sigma,\delta} +
        \sum_{\rho\in C(\Sigma)} (-1)^{|\rho|} \partial_{\sigma,\tau_\rho}\circ \partial_\rho \circ \partial_{\sigma_\rho,\delta}
      \right) \circ
      \left(
        \partial_{\delta,\tau} +
        \sum_{\rho\in C(\Sigma)} (-1)^{|\rho|} \partial_{\delta,\tau_\rho}\circ \partial_\rho \circ \partial_{\sigma_\rho,\tau}
      \right)\\
      =
      \sum_{\rho,{\rho'}\in C(\Sigma)}
      (-1)^{|\rho|+|{\rho'}|}
      \sum_{\delta}
      \left(
        \partial_{\sigma,\delta} +
        \partial_{\sigma,\tau_\rho}\circ \partial_\rho \circ \partial_{\sigma_\rho,\delta}
      \right) \circ
      \left(
        \partial_{\delta,\tau} +
        \partial_{\delta,\tau_{\rho'}}\circ \partial_{\rho'} \circ \partial_{\sigma_{\rho'},\tau}
      \right)
    \end{align*}
    Let us consider the inner sum. Isolating the parts that
    depend on $\delta$, this splits into four components
    \begin{align*}
      &\sum_{\delta} \partial_{\sigma,\delta}\circ\partial_{\delta,\tau},
      \hspace{1cm}
      &&\left(\sum_\delta \partial_{\sigma,\delta} \circ  \partial_{\delta,\tau_{\rho'}}\right)\circ \partial_{\rho'} \circ \partial_{\sigma_{\rho'},\tau},
      \\
      &
      \partial_{\sigma,\tau_\rho}\circ \partial_\rho \circ
      \left(\sum_\delta \partial_{\sigma_\rho,\delta} \circ \partial_{\delta,\tau}\right),
      \hspace{1cm}
      &&
      \partial_{\sigma,\tau_\rho}\circ \partial_\rho \circ
      \left(
        \sum_\delta \partial_{\sigma_\rho,\delta} \circ
        \partial_{\delta,\tau_{\rho'}}
      \right)
      \circ \partial_{\rho'} \circ \partial_{\sigma_{\rho'},\tau}.
    \end{align*}
    By \ref{thm:cosheaf_chain_complex} we know that each of these
    sums must yield the zero map, hence the claim follows.
  \end{proof}
\end{lemma}

We are now justified in considering the Morse homology of
$\mathscr C^\bullet$ and $\Sigma$. We write $H^\Sigma_k(K;\mathscr C^i)$
for the $k$-th homology group of the complex
\begin{align*}
  \cdots\xlongrightarrow{\partial^\Sigma} C^\Sigma_k(K;\mathscr C^i)
  \xlongrightarrow{\partial^\Sigma}C^\Sigma_{k-1}(K;\mathscr C^i)
  \xlongrightarrow{\partial^\Sigma}\cdots.
\end{align*}


Now note that the inclusions $\mathscr C^i\hookrightarrow\mathscr C^{i+1}$
induce maps
\begin{align*}
  C^\Sigma_k(K;\mathscr C^i) \longinc C^\Sigma_k(K;\mathscr C^{i+1}).
\end{align*}
Once again we find that these are components of a chain map:

\begin{lemma}
  For all $k$ and $i$, the following commutes:
  \begin{equation*}
    % https://q.uiver.app/#q=WzAsNCxbMCwwLCJDXlxcU2lnbWFfayhLO1xcbWF0aHNjciBDXmkpIl0sWzIsMCwiQ15cXFNpZ21hX3trLTF9KEs7XFxtYXRoc2NyIENeaSkiXSxbMCwxLCJDXlxcU2lnbWFfayhLO1xcbWF0aHNjciBDXntpKzF9KSJdLFsyLDEsIkNeXFxTaWdtYV97ay0xfShLO1xcbWF0aHNjciBDXntpKzF9KSJdLFswLDEsIlxccGFydGlhbF5cXFNpZ21hIl0sWzAsMiwiIiwyLHsic3R5bGUiOnsidGFpbCI6eyJuYW1lIjoiaG9vayIsInNpZGUiOiJ0b3AifX19XSxbMSwzLCIiLDAseyJzdHlsZSI6eyJ0YWlsIjp7Im5hbWUiOiJob29rIiwic2lkZSI6InRvcCJ9fX1dLFsyLDMsIlxccGFydGlhbF5cXFNpZ21hIiwyXV0=
    \begin{tikzcd}
      {C^\Sigma_k(K;\mathscr C^i)} && {C^\Sigma_{k-1}(K;\mathscr C^i)} \\
      {C^\Sigma_k(K;\mathscr C^{i+1})} && {C^\Sigma_{k-1}(K;\mathscr C^{i+1})}
      \arrow["{\partial^\Sigma}", from=1-1, to=1-3]
      \arrow[hook, from=1-1, to=2-1]
      \arrow[hook, from=1-3, to=2-3]
      \arrow["{\partial^\Sigma}"', from=2-1, to=2-3]
    \end{tikzcd}
  \end{equation*}
  \begin{proof}
    This follows directly from the fact that $\partial^\Sigma$
    consists of boundary operators with each of which the inclusions
    commute by naturality. See \ref{thm:chain_map} for details.
  \end{proof}
\end{lemma}
Thus we have inclusions of Morse homologies
\begin{align*}
  H^\Sigma_k(K;\mathscr C^i) \longinc H^\Sigma_k(K;\mathscr C^{i+1}).
\end{align*}
Therefore, we have persistence modules of Morse homologies and hence
may define the corresponding persistent Morse homologies:
\begin{definition}
  The \emph{$k$-th persistent homology of $\Sigma$ with $\mathscr C^\bullet$-coefficients} is
  \begin{align*}
    PH^\Sigma_{k,i\to j}
    = \im\left({
        H^\Sigma_k(K;\mathscr C^i) \longinc \cdots \longinc H^\Sigma_k(K;\mathscr C^j)
    }\right)
  \end{align*}
\end{definition}

\begin{example}\label{ex:persistent_morse_homology}
  Let us now revisit \ref{ex:acyclic_partial_matching_s1}.
  Note that $C(\Sigma)=\left\lbrace{\sigma_2,\tau_3}\right\rbrace$.
  Thus we have the chain groups
  $C^\Sigma_0(K;\mathscr C^i) = \mathscr C^i(\sigma_2)$
  and
  $C^\Sigma_1(K;\mathscr C^i) = \mathscr C^i(\tau_3) = 0$.
  Thus Morse homology is only non-trivial in dimension $0$
  where we may simply read off the complex:
  \begin{equation*}
    % https://q.uiver.app/#q=WzAsNCxbMSwwLCJcXG1hdGhiYiBSIl0sWzMsMCwiXFxtYXRoYmIgUiJdLFs1LDAsIlxcbWF0aGJiIFJeMiJdLFswLDAsIkheXFxTaWdtYV8wKFNeMTtcXG1hdGhzY3IgQ15cXGJ1bGxldCk6Il0sWzEsMiwiIiwyLHsic3R5bGUiOnsidGFpbCI6eyJuYW1lIjoiaG9vayIsInNpZGUiOiJ0b3AifX19XSxbMCwxLCJcXHRleHR7aWR9IiwyXV0=
    \begin{tikzcd}
      {H^\Sigma_0(S^1;\mathscr C^\bullet):} & {\mathbb R} && {\mathbb R} && {\mathbb R^2}
      \arrow[hook, from=1-4, to=1-6]
      \arrow["{\text{id}}"', from=1-2, to=1-4]
    \end{tikzcd}
  \end{equation*}
\end{example}

\section{Equivalence}

It is of course no coincidence that the persistent homologies
in \ref{ex:persistent_morse_homology} agree with \ref{ex:acyclic_partial_matching_s1}. We constructed the Morse complex with the explicit
intention to obtain an alternative way of calculating the persistent
homology groups. The goal of this section is showing that we have
achieved our goal. We do this by constructing appropriate chain
homotopies.

\begin{lemma}
  If $\Sigma=\left\lbrace{\sigma\triangleleft\tau}\right\rbrace$
  is compatible with $\mathscr C\in\Cosh(K)$ then the complexes
  $(C_\bullet(K;\mathscr C),\partial)$ and
  $(C^\Sigma_\bullet(K;\mathscr C),\partial^\Sigma)$ are chain homotopic.
  \begin{proof}
    Consider the linear maps $\phi_d : C_d(K;\mathscr C)\to C_d^\Sigma(K;\mathscr C)$ whose component $\mathscr C(\alpha)\to\mathscr C(\omega)$ is the identity whenever $\alpha=\omega$ and
    \begin{align*}
      \restrict{\phi_d}{\alpha,\omega} = \sum_{\rho\in P(\Sigma),\sigma_\rho = \alpha} \partial_{\omega,\tau_\rho}\circ\partial_\rho
    \end{align*}
    otherwise. In the other direction, we have the linear map
    $\psi_d : C^\Sigma_d(K;\mathscr C)\to C_d(K;\mathscr C)$
    whose component $\mathscr C(\omega)\to\mathscr C(\alpha)$
    also is the identity if $\alpha=\omega$ and
    \begin{align*}
      \restrict{\psi_d}{\omega,\alpha}=\sum_{\rho\in P(\Sigma),\tau_\rho=\alpha} \partial_\rho \circ \partial_{\sigma_\rho,\omega}
    \end{align*}
    otherwise.

    Let us verify that we have thus defined chain maps. \missingproof

    Consider $\phi_d\circ\psi_d$. We have the component
    $\mathscr C(\omega)\to\mathscr C(\omega')$ which is the identity when $\omega=\omega'$ and otherwise is given by
    \begin{align*}
      \restrict{(\phi_d \circ \psi_d)}{\omega,\omega'}
      = \restrict{\phi_d}{\omega,\omega'}\circ\restrict{\psi_d}{\omega',\omega'}+\restrict{\phi_d}{\omega,\omega}\circ\restrict{\psi_d}{\omega,\omega'}
    \end{align*}
    where we used the fact that we cannot have $\tau_\rho \triangleleft \alpha \triangleleft \sigma_{\rho'}$. Hence
    \begin{align*}
      \restrict{(\phi_d \circ \psi_d-\identity)}{\omega,\omega'}
      =
      \sum_{\rho\in P(\Sigma),\sigma_\rho = \omega} \partial_{\omega',\tau_\rho}\circ\partial_\rho
      +
      \sum_{\rho\in P(\Sigma),\tau_\rho=\omega'} \partial_\rho \circ \partial_{\sigma_\rho,\omega}
    \end{align*}
    for all $\omega,\omega'\in C(\Sigma)$. But now note if
    $\omega\in C(\Sigma)$ then $\sigma_\rho\neq\omega$ for all
    $\rho\in P(\Sigma)$ and similarly for the sum on the right.
    Thus $\phi_d\circ\psi_d-\identity = 0$.

    In the other direction, we may calculate
    \begin{align*}
      \restrict{(\psi_d\circ\phi_d)}{\alpha,\alpha'}
      &= \sum_{\omega\in C(\Sigma)\cap K_d}
      \restrict{\psi_d}{\omega,\alpha'}\circ\restrict{\phi_d}{\alpha,\omega}\\
      &= \sum_{\omega\in C(\Sigma)\cap K_d}
      \left({
          \sum_{\rho\in P(\Sigma),\tau_\rho=\alpha'} \partial_\rho \circ \partial_{\sigma_\rho,\omega}
      }\right)
      \circ
      \left({
          \sum_{\rho\in P(\Sigma),\sigma_\rho = \alpha} \partial_{\omega,\tau_\rho}\circ\partial_\rho
      }\right)
    \end{align*}
    %   Now define
    %   $\theta_d : C_d(K;\mathscr C)\to C_{d+1}(K;\mathscr C)$
    %   by \todo{probably no $-1$ here so things cancel below}
    %   \begin{align*}
    %     \restrict{\theta_d}{\sigma,\tau}
    %     = \sum_{\rho\in P(\Sigma),\sigma_\rho = \sigma,\tau_\rho=\tau}
    %     (-1)^{\left\vert{\rho}\right\vert} \partial_\rho.
    %   \end{align*}
    %   It is now straightforward to calculate
    %   \begin{align*}
    %     \restrict{(\partial^\Sigma\circ\theta_d)}{\omega,\omega'}
    %     &= \sum_{\tau\in C(\Sigma)\cap K_{d+1}} \restrict{\partial^\Sigma}{\tau,\omega'} \circ \restrict{\theta}{\omega,\tau}\\
    %     &= \sum_{\tau\in C(\Sigma)\cap K_{d+1}}
    %     \left({
    %         \partial_{\omega',\tau} + \sum_{\rho\in P(\Sigma)} (-1)^{\left\vert{\rho}\right\vert}\partial_{\omega',\tau_\rho}\circ\partial_\rho\circ\partial_{\sigma_\rho,\tau}
    %     }\right)
    %     \circ
    %     \left({
    %         \sum_{\rho\in P(\Sigma),\sigma_\rho = \omega,\tau_\rho=\tau}(-1)^{\left\vert{\rho}\right\vert}\partial_\rho
    %     }\right)\\
    %     &= \sum_{\tau\in C(\Sigma)\cap K_{d+1}}
    %     \left({
    %         \sum_{\rho\in P(\Sigma),\sigma_\rho = \omega,\tau_\rho=\tau}(-1)^{|\rho|}\partial_{\omega',\tau}\circ\partial_\rho + \sum_{\rho\in P(\Sigma),\sigma_\rho=\omega} (-1)^{\left\vert{\rho}\right\vert}\partial_{\omega',\tau_\rho}\circ\partial_\rho
    %     }\right)\\
    %     &=
    %     \sum_{\rho\in P(\Sigma),\sigma_\rho = \omega}(-1)^{|\rho|}\partial_{\omega',\tau_\rho}\circ\partial_\rho + \sum_{\rho\in P(\Sigma),\sigma_\rho=\omega} (-1)^{\left\vert{\rho}\right\vert}\partial_{\omega',\tau_\rho}\circ\partial_\rho
    % \end{align*}
    % and
    % \begin{align*}
    % \restrict{(\theta_{d-1}\circ\partial^\Sigma)}{\omega,\omega'}
    %     &= \sum_{\tau\in C(\Sigma)\cap K_{d-1}}
    %     \restrict{\theta}{\tau,\omega'}\circ\restrict{\partial^\Sigma}{\omega,\tau}\\
    %     &= \sum_{\tau\in C(\Sigma)\cap K_{d-1}}
    %     \left({
    %         \sum_{\rho\in P(\Sigma),\sigma_\rho=\tau,\tau_\rho=\omega'}
    %         (-1)^{\left\vert{\rho}\right\vert}\partial_\rho
    %     }\right)
    %     \circ
    %     \left({
    %         \partial_{\tau,\omega} +
    %         \sum_{\rho\in P(\Sigma)} (-1)^{\left\vert{\rho}\right\vert}
    %         \partial_{\tau,\tau_\rho}\circ\partial_\rho\circ\partial_{\sigma_\rho,\omega}
    %     }\right)
    %   \end{align*}
    \missingproof
  \end{proof}
\end{lemma}

\begin{theorem}
  For each $i$, the complexes $(C_\bullet(K;\mathscr C^i),\partial)$
  and $(C_\bullet^\Sigma(K;\mathscr C^i),\partial^\Sigma)$ are
  chain-homotopic.
  \begin{proof}
    \missingproof
  \end{proof}
\end{theorem}

\begin{corollary}
  $PH^\Sigma_{k,i\to j}(K;\mathscr C^\bullet) = PH_{k,i\to j}(K;\mathscr C^\bullet)$
\end{corollary}

\section{Linear duality}

\todo{write about the equivalence $\op{\Sh(K)}\cong\Cosh(K)$, if
there is time}

\printbibliography

\end{document}
