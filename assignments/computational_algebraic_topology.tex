\documentclass{article}
\usepackage{assignment}
\begin{document}
\title{C3.9 Mini Project: Cosheaf Homology}
\author{Franz Miltz}
\date{\today}
\maketitle

\section{Cosheaves and homology}

\begin{definition}
  A \emph{cosheaf} on a simplicial complex $K$ is a functor
  $\mathscr C:(K,\geq)\to\FVect_{\mathbb{R}}$.
\end{definition}

If one more generally define a sheaf on any poset to be a
functor $P\to\FVect_{\mathbb{R}}$ then it is straightforward to
see that a cosheaf on $K$ is just a sheaf on the cocomplex
$\mathcal P(K_0)\setminus K$ and we have an equivalence of categories
\begin{align*}
  \Cosh(K) \cong \Sh(\mathcal P(K_0)\setminus K)
\end{align*}

\begin{example}
  \begin{itemize}
    \item Let $V\in\FVect_{\mathbb{R}}$. Then the constant cosheaf
      $\underline V_K$ sends each simplex to $V$ and each inclusion
      $\tau\hookrightarrow\tau'$ to the identity $V\to V$. Note that
      this is the same as the constant sheaf $\underline V_K$, hence
      the notation is consistent.
  \end{itemize}
\end{example}

\begin{definition}
  Let $\mathscr C$ be a cosheaf on a simplicial complex $K$. Then, for $k\geq 0$,
  define the \emph{vector space of $k$-chains of $K$ with coefficients
  in $\mathscr C$} by
  \begin{align*}
    C_k(K;\mathscr C) = \bigoplus_{\dim \tau = k} \mathscr C(\tau).
  \end{align*}
\end{definition}

\begin{definition}
  The \emph{boundary map}
  \begin{align*}
    d : C_k(K;\mathscr C) \to C_{k-1}(K;\mathscr C)
  \end{align*}
  is given by
  \begin{align*}
    v \in \mathscr C(\sigma) \mapsto \bigoplus_{\dim\tau=k-1} [\tau:\sigma] \mathscr C(\tau\leq\sigma)(v)
  \end{align*}
\end{definition}

\begin{definition}
  The \emph{cosheaf homology of $K$ with $\mathscr C$-coefficients} is the
  homology of the complex
  \begin{align*}
    \cdots\xlongrightarrow{d} C_k(K;\mathscr C)
    \xlongrightarrow{d}C_{k-1}(K;\mathscr C)
    \xlongrightarrow{d}\cdots.
  \end{align*}
  Explicitly,
  \begin{align*}
    H_k(K;\mathscr C) = \frac{\ker(C_k(K;\mathscr C) \to C_{k-1}(K;\mathscr C))}{\im(C_{k+1}(K;\mathscr C) \to C_k(K;\mathscr C))}
  \end{align*}
\end{definition}

\section{Examples of cosheaf homology}

\begin{lemma}
  Consider the chosheaf $\mathscr C\in\Cosh(K)$ given by
  \begin{align*}
    \{x\}\hookrightarrow\{x,y\}\hookleftarrow\{y\}
    \hspace{1cm}
    \mapsto
    \hspace{1cm}
    \mathbb{R}\xleftarrow{\pi}\mathbb{R}^2\rightarrow 0
  \end{align*}
  Then
  \begin{align*}
    H_*(K;\underline{\mathbb{R}}_K) = H_*(*;\underline{\mathbb{R}}_K)
    \hspace{1cm}
    \text{and}
    \hspace{1cm}
    H_*(K;\mathscr C) \neq H_*(*;\mathscr C)
  \end{align*}
\end{lemma}

\begin{lemma}
  Consider the simplicial complex $K$ with two disconnected points
  and $\mathscr C=\underline{0}_K$. Then
  \begin{align*}
    H_*(K;\underline{\mathbb{R}}_K) \neq H_*(*;\underline{\mathbb{R}}_K)
    \hspace{1cm}
    \text{and}
    \hspace{1cm}
    H_*(K;\mathscr C) = H_*(*;\mathscr C)
  \end{align*}
\end{lemma}

\section{Maps}

\begin{definition}
  A \emph{map of cosheaves} $\phi:\mathscr C\to\mathscr D$ is
  just a natural transformation, i.e. it consists of maps
  $\phi_\tau:\mathscr C(\tau)\to\mathscr D(\tau)$ for every simplex
  $\tau\in K$ such that, for all $\sigma\leq\tau$, the following commutes:
  \begin{equation}\label{eq:naturality}
    % https://q.uiver.app/#q=WzAsNCxbMCwwLCJcXG1hdGhzY3IgQyhcXHRhdSkiXSxbMiwwLCJcXG1hdGhzY3IgQyhcXHNpZ21hKSJdLFswLDEsIlxcbWF0aHNjciBEKFxcdGF1KSJdLFsyLDEsIlxcbWF0aHNjciBEKFxcc2lnbWEpIl0sWzAsMiwiXFxwaGlfXFx0YXUiLDJdLFsyLDMsIlxcbWF0aHNjciBEKFxcc2lnbWFcXGxlcVxcdGF1KSIsMl0sWzEsMywiXFxwaGlfXFxzaWdtYSJdLFswLDEsIlxcbWF0aHNjciBDKFxcc2lnbWFcXGxlcVxcdGF1KSJdXQ==
    \begin{tikzcd}
      {\mathscr C(\tau)} && {\mathscr C(\sigma)} \\
      {\mathscr D(\tau)} && {\mathscr D(\sigma)}
      \arrow["{\phi_\tau}"', from=1-1, to=2-1]
      \arrow["{\mathscr D(\sigma\leq\tau)}"', from=2-1, to=2-3]
      \arrow["{\phi_\sigma}", from=1-3, to=2-3]
      \arrow["{\mathscr C(\sigma\leq\tau)}", from=1-1, to=1-3]
    \end{tikzcd}
  \end{equation}
\end{definition}

Thus we have the category $\Cosh(K) := \text{Fun}((K,\geq),\FVect_{\mathbb{R}})$.

\begin{itemize}
  \item There are two notable examples of the constant cosheaf:
    $\underline{0}_K$ and $\underline{\mathbb{R}}_K$ which
    are initial and terminal in $K$, respectively.
\end{itemize}

\begin{proposition}
  A map of cosheaves is mono, epi, or iso if, and only if,
  each component is injective, surjective, or bijective, respectively.
  \begin{proof}
    \missingproof
  \end{proof}
\end{proposition}

Fix a simplicial complex $K$ and a map of cosheaves
$\phi : \mathscr C \to \mathscr D$ on $K$. We then have an induced
linear map on $k$-chains given by
\begin{align*}
  \phi_* : C_k(K;\mathscr C) &\to C_k(K;\mathscr D) \\
  v \in \mathscr C(\sigma) &\mapsto \phi_\sigma(v)
\end{align*}
It is unsurprising that this does indeed induce a chain map:
\begin{proposition}
  For all $K$, $k$, and $\phi : \mathscr C\to\mathscr D$,
  the following commutes:
  \begin{equation}
    % https://q.uiver.app/#q=WzAsNCxbMCwwLCJDX2soSztcXG1hdGhzY3IgQykiXSxbMiwwLCJDX3trKzF9KEs7XFxtYXRoc2NyIEMpIl0sWzAsMSwiQ19rKEs7XFxtYXRoc2NyIEQpIl0sWzIsMSwiQ197aysxfShLO1xcbWF0aHNjciBEKSJdLFswLDEsImQiXSxbMiwzLCJkIl0sWzAsMiwiXFxwaGlfKiIsMl0sWzEsMywiXFxwaGlfKiJdXQ==
    \begin{tikzcd}
      {C_k(K;\mathscr C)} && {C_{k+1}(K;\mathscr C)} \\
      {C_k(K;\mathscr D)} && {C_{k+1}(K;\mathscr D)}
      \arrow["d", from=1-1, to=1-3]
      \arrow["d", from=2-1, to=2-3]
      \arrow["{\phi_*}"', from=1-1, to=2-1]
      \arrow["{\phi_*}", from=1-3, to=2-3]
    \end{tikzcd}
  \end{equation}
  \begin{proof}
    We observe that the following commutes:
    \begin{equation}
      % https://q.uiver.app/#q=WzAsOCxbMCwwLCJcXGJpZ29wbHVzX3tcXGRpbSBcXHNpZ21hID0ga30gXFxtYXRoc2NyIEMoXFxzaWdtYSkiXSxbMywwLCJcXGJpZ29wbHVzX3tcXGRpbSBcXHRhdSA9IGstMX0gXFxtYXRoc2NyIEMoXFx0YXUpIl0sWzAsMywiXFxiaWdvcGx1c197XFxkaW0gXFxzaWdtYSA9IGt9IFxcbWF0aHNjciBEKFxcc2lnbWEpIl0sWzMsMywiXFxiaWdvcGx1c197XFxkaW0gXFx0YXUgPSBrLTF9IFxcbWF0aHNjciBEKFxcdGF1KSJdLFsxLDEsIlxcbWF0aHNjciBDKFxcc2lnbWEpIl0sWzEsMiwiXFxtYXRoc2NyIEQoXFxzaWdtYSkiXSxbMiwxLCJcXG1hdGhzY3IgQyhcXHRhdSkiXSxbMiwyLCJcXG1hdGhzY3IgRChcXHRhdSkiXSxbMCwxLCJkIl0sWzIsMywiZCJdLFswLDIsIlxccGhpXyoiLDJdLFsxLDMsIlxccGhpXyoiXSxbMyw3XSxbMiw1XSxbMCw0XSxbNCw1XSxbMSw2XSxbNiw3XSxbNSw3XSxbNCw2XV0=
      \begin{tikzcd}
        {\bigoplus_{\dim \sigma = k} \mathscr C(\sigma)} &&& {\bigoplus_{\dim \tau = k-1} \mathscr C(\tau)} \\
                                                         & {\mathscr C(\sigma)} & {\mathscr C(\tau)} \\
                                                         & {\mathscr D(\sigma)} & {\mathscr D(\tau)} \\
        {\bigoplus_{\dim \sigma = k} \mathscr D(\sigma)} &&& {\bigoplus_{\dim \tau = k-1} \mathscr D(\tau)}
        \arrow["d", from=1-1, to=1-4]
        \arrow["d", from=4-1, to=4-4]
        \arrow["{\phi_*}"', from=1-1, to=4-1]
        \arrow["{\phi_*}", from=1-4, to=4-4]
        \arrow[from=4-4, to=3-3]
        \arrow[from=4-1, to=3-2]
        \arrow[from=1-1, to=2-2]
        \arrow[from=2-2, to=3-2]
        \arrow[from=1-4, to=2-3]
        \arrow[from=2-3, to=3-3]
        \arrow[from=3-2, to=3-3]
        \arrow[from=2-2, to=2-3]
      \end{tikzcd}
    \end{equation}
    Here the middle square is just naturality (\ref{eq:naturality})
    and the diagonal maps are the projections from the product.
    The claim now follows by the universal property of products.
  \end{proof}
\end{proposition}

\section{Persistent homology}

\begin{definition}
  A \emph{filtration} of a cosheaf $\mathscr C\in\Cosh(K)$ is a sequence of
  cosheaves $\mathscr C_\bullet = \mathscr C^0,\ldots,\mathscr C^n$ where
  $\mathscr C_n = \mathscr C$ together with monomorphisms $\iota^i : \mathscr C_i
  \hookrightarrow \mathscr C_{i+1}$ for $0\leq i<n$.
\end{definition}

Note that a filtration $\mathscr C_\bullet$ of a cosheaf
induces a persistence module of homologies for every $k\geq 0$:
\begin{align}\label{eq:cosheaf_persistence_module}
  H_k(K;\mathscr C_0) \xlongrightarrow{\iota_*}
  H_k(K;\mathscr C_1) \xlongrightarrow{\iota_*}
  \cdots \xlongrightarrow{\iota_*}
  H_k(K;\mathscr C_n)
\end{align}

Now the $k$-th persistent homology of the filtration is just the
homology of the $k$-th persistence module:

\begin{definition}
  The \emph{persistent homology groups} of a simplicial complex $K$
  with coefficients in a filtration of cosheaves $\mathscr C_i$
  is the persistent homology of the persistence module
  (\ref{eq:cosheaf_persistence_module}), i.e.
  \begin{align*}
    PH_{k,i\to j}(K;\mathscr C_\bullet) = \im\left(
      H_k(K;\mathscr C_i) \xlongrightarrow{\iota_*}
      \cdots \xlongrightarrow{\iota_*}
      H_k(K;\mathscr C_j)
    \right).
  \end{align*}
\end{definition}


\printbibliography

\end{document}
