\documentclass{article}
\usepackage{assignment}
\begin{document}
\title{C2.6 Mini Project: Homogeneous Spectra}
\author{Franz Miltz}
\date{\today}
\maketitle

\section{Construction}

One may construct projective space by gluing affine schemes.
This is analogous to constructing classical projective space by covering
using affine charts. Given that projective space and its subschemes
play a central role in algebraic geometry, we would like to construct
it more directly. This is possible using a new construction, the
homogeneous spectrum, which extends the classical approach of restricting
the attention to homogeneous polynomials to a more general class of rings.

We will begin by introducing some prerequisites and then continue by
constructing a topological space and its structure sheaf in
complete analogy to $\Spec$.

\subsection{Graded rings}

A classical projective variety is given by homogeneous polynomial
equations. Here homogeneous means that all summands of the polynomial
have the same degree. In particular, one may observe that given a field $k$,
there exists a decomposition
\begin{align*}
  k[x_1,\ldots,x_n] = \bigoplus_{d=0}^\infty k[x_1,\ldots,x_n]_d
\end{align*}
where each $k[x_1,\ldots,x_n]_d$ consists of homogeneous degree $d$
polynomials. Note the crucial property that, for homogeneous polynomials $f$
and $g$ of degree $d$ and $d'$, we obtain a homogeneous polynomial $fg$ of
degree $d+d'$.

In general we are not always dealing with polynomial rings. In order
to identify the homogeneous elements, we need to assume some
additional structure:

\begin{definition}
  A ring $S$ is graded if there exists a decomposition
  $S = \bigoplus_{d=0}^\infty S_d$
  such that $S_d S_{d'} \subseteq S_{d+d'}$.
  A map of rings $\phi : S\to S'$ is graded if it
  respects the grading, i.e. $\phi(S_d) \subseteq S'_d$.
  \question{do we need maps of graded rings?}
\end{definition}

\begin{example}
  Every polynomial ring $R[x_1,\ldots,x_n]$ admits a grading into
  homogeneous polynomials of degree $d$ and the inclusion
  $R[x_1,\ldots,x_n] \hookrightarrow R[x_1,\ldots,x_{n+1}]$
  is a map of graded rings.
\end{example}

In light of the motivating example, we call an element
$f$ of a graded ring $S$ homogeneous, if $f\in S_d$ for some $d$
and we write $\deg f = d$.
Such homogeneous elements are going to play a key role. In particular,
the points of $\Proj S$ are going to correspond to prime ideals of $S$
that are generated by homogeneous elements.

\begin{definition}
  An ideal $I$ of a graded ring $S$ is \emph{homogeneous} if it splits into
  additive subgroups $I=\oplus_{d=0}^\infty I_d$ such that
  $I_d\subseteq S_d$ and $S_d I_{d'} \subseteq I_{d+d'}$.
\end{definition}

In other words, if $f\in I$ with decomposition into homogeneous parts
$f=f_1+\cdots + f_m$ then $f_i\in I$ for all $i$. Thus an ideal is
homogeneous if, and only if, it is generated by homogeneous elements.
\cite{boer1961}

\begin{example}
  We define the \emph{irrelevant ideal} of a graded ring $S$
  to be $S_+ = \oplus_{d=1}^\infty S_d$. It is straightforward to
  see that this is indeed a graded ideal.
\end{example}

\subsection{Topological space}

We now have all the ingredients to construct the homogeneous spectrum
of a graded ring. Topologically this will simply be a subspace of
the usual spectrum where we exclude all non-homogeneous ideals and the irrelevant one.

Fix a graded ring $S$.

\begin{definition}
  The \emph{homogeneous spectrum} $\Proj S$
  is the topological subspace of $\Spec S$ containing all
  homogeneous prime ideals that do not contain $S_+$. I.e.
  \begin{align*}
    \Proj S := \left\lbrace{ \text{$p$ homogeneous}, S_+ \not\subseteq p }\right\rbrace \subseteq \Spec S.
  \end{align*}
\end{definition}

We obtain distinguished homogeneous opens
\begin{align*}
  D_+(f) := D(f) \cap \Proj S
\end{align*}
for $f\in S$ homogeneous.
Unsurprisingly, these do indeed form a basis of the homogeneous spectrum.
To see this, some preliminary calculation is required.

\begin{lemma}
  Let $g_0\in S_0$. Then
  \begin{align}\label{eq:distinguished_zero_union}
    D(g_0) \cap \Proj S = \bigcup_{f\in S_d, d\geq 1} D_+(g_0f).
  \end{align}
  \begin{proof}
    ($\supseteq$) We observe, for all $f\in S_d$ with $d\geq 1$,
    \begin{align*}
      D_+(g_0f) \subseteq D(g_0 f) \subseteq D(g_0).
    \end{align*}

    ($\subseteq$) Assume $p\in\Proj S$ such that $g_0\not\in p$, but whenever
    $f\in S_d$ with $d\geq 1$ we have $g_0f\in p$. We then see that $f\in p$
    for all $f\in S_d$ with $d\geq 1$, i.e. $S_+\subseteq p$. This is a
    contradiction. Thus $g_0\in p$ implies $g_0 f\not\in p$
    for some $f\in S_d$ with $d\geq 1$.
  \end{proof}
\end{lemma}

\begin{proposition}
  The set of distinguished opens
  \begin{align*}
    \left\lbrace{ D_+(f) : \text{$f\in S$ homogeneous}}\right\rbrace
  \end{align*}
  forms a basis of $\Proj S$.
  \begin{proof}
    Let $g=g_1+\cdots+g_m\in S$ with $g_i\in S_i$. Then
    \begin{align*}
      D(g)\cap\Proj S
      &= D\left({\sum_i g_i}\right) \cap \Proj S \\
      &= \bigcup_i D(g_i) \cap\Proj S \\
      &= (D(g_0)\cap\Proj S)\cup \bigcup_{i\geq 1} D_+(g_i).
    \end{align*}
    Using (\ref{eq:distinguished_zero_union}) we may write each element
    of the basis
    \begin{align*}
      \left\lbrace{D(g) \cap \Proj S : g\in S}\right\rbrace
    \end{align*}
    as a union of distinguished opens $D_+(f)$.
  \end{proof}
\end{proposition}

\begin{example}
  Consider $p\in\Proj R[x]$ for some $R$.
  Note that $R[x]_+\not\subseteq p$ and
  hence $x\not\in p$. If $ax^n\in p$ for some $a\in R$ and $n>0$
  we must have $a\in p$. Thus the map $\Proj R[x] \to \Spec R$
  sending $p \mapsto p \cap R$ is a bijection with inverse
  $p_0 \mapsto p_0 R[x]$. It is straightfoward to verify that both
  directions are continuous, hence we have a homeomorphism
  \begin{align*}
    \Proj R[x] \simeq \Spec R.
  \end{align*}
\end{example}

\subsection{Structure sheaf}


Consider $f\in R$ and recall the localisation $R_f$. This is the ring
that corresponds to a distinguished open $D(f)$. In particular,
there is a homeomorphism
\begin{align*}
  \Spec R_f \simeq D(f).
\end{align*}

Once again fix a graded ring $S$.
If $A\subseteq S$ is a non-empty multiplicatively closed subset
of homogeneous elements then the localisation $S_A$ has a
natural grading \todo{this is actually a $\mathbb{Z}$-grading and thus something we haven't properly defined}
where the homogeneous elements are of the form $s/a$ for $s\in S$
and $a\in A$ and $\deg(s/a) = \deg s - \deg a$.

\begin{definition}
  Let $A\subseteq S$ a homogeneous submonoid. Then the \emph{homogeneous
  localisation} is the subring of $S_A$ consisting of homogeneous degree zero
  elements $S_{(A)} = (S_A)_0$.
\end{definition}

In close analogy to the regular spectrum construction, there are two important
values of $A$ that we will encounter. Firstly, if $f\in S$ is homogeneous of
positive degree and $I=(f)$ the homogeneous ideal generated by $f$ then we
write $S_{(f)}:=S_{(I)}$. Secondly, if $p\subseteq S$ is a homogeneous prime
ideal and $A=S\setminus p$ then we write $S_{(p)} := S_{(S)}$.

It turns out that this notion of localisation is precisely what is
required to work with the homogeneous spectrum. Let us observe some
first evidence of this:

\begin{lemma}\label{lem:homogeneous_localisation}
  Let $f\in S$ be homogeneous of positive degree. Then
  \begin{align*}
    \Spec S_{(f)} = D_+(f).
  \end{align*}
  \begin{proof}
    By construction $D_+(f)$ is a topological subspace of
    $D(f)\simeq\Spec S_f$.
    \missingproof
  \end{proof}
\end{lemma}


Now consider the case $D_+(g)\subseteq D_+(f)$. In particular,
by Lemma~\ref{lem:homogeneous_localisation}, we have an inclusion
$R_{(f)}\hookrightarrow R_{(g)}$. If $D_+(g)=D_+(f)$ then
$\Spec R_{(g)}$ and $\Spec R_{(f)}$ are homeomorphic so
$R_{(f)}$ and $R_{(g)}$ must be isomorphic.
Thus we may define the following sheaf of rings on the $\Proj S$:

\begin{definition}
  The \emph{structure sheaf} $\mathcal O_{\Proj S}$ is the unique sheaf
  on $\Proj S$ satisfying
  \begin{align}\label{eq:structure_sheaf}
    \mathcal O_{\Proj S}(D_+(f)) = S_{(f)}
  \end{align}
  with the restrictions corresponding to the map
  \begin{align*}
    D_+(g) = \Spec S_{(g)} \hookrightarrow \Spec S_{(f)} = D_+(f).
  \end{align*}
\end{definition}

It is straightforward to see that (\ref{eq:structure_sheaf})
defines a presheaf on the basis of standard opens of $\Proj S$.
However, it remains to check the sheaf conditions:

\begin{proposition}
  The structure sheaf $\mathcal O_{\Proj S}$ is a sheaf.
  In particular, it satisfies the sheaf conditions.
  \begin{proof}
    Consider a cover $D_+(f) = \bigcup_i D_+(g_i)$. Recall that the
    sheaf conditions of the standard basis are equivalent to the
    exactness of the following sequence:
    \begin{align*}
      0 \longrightarrow
      S_{(f)} \longrightarrow
      \prod S_{(fg_i)} \longrightarrow
      \prod S_{(fg_ig_j)}
    \end{align*}
    \missingproof
  \end{proof}
\end{proposition}

Recall that the stalk of the spectrum of a ring at a prime ideal
is the localisation at that prime ideal. Thus we expect the stalks
of the homogeneous spectrum to be the homogeneous localisations.
This is indeed the case.

\begin{proposition}\label{thm:stalks}
  Suppose $x\in\Proj S$
  corresponds to a homogeneous prime ideal $p\subseteq S$. Then the
  stalk of $\mathcal O_{\Proj S}$ at $x$ is
  \begin{align*}
    \mathcal O_{\Proj S,x} = S_{(p)}.
  \end{align*}
  \begin{proof}
    \missingproof
  \end{proof}
\end{proposition}

We now make the following simple observation:
\begin{lemma}
  If $S$ is local then so is $S_0$.
  \begin{proof}
    Let $m\subseteq S$ be the unique maximal ideal. Write
    $m_0 = m \cap S_0$. This is an ideal in $S_0$. We may calculate
    $S_0\setminus m_0 = (S\setminus m)\cap S_0 = S^\times\cap S_0 = S_0^\times$.
  \end{proof}
\end{lemma}
Thus, in particular, the homogeneous localisation at a prime ideal
is local.

\begin{proposition}
  $(\Proj S, \mathcal O_{\Proj S})$ is a scheme.
  \begin{proof}
    \missingproof
  \end{proof}
\end{proposition}

\section{Properties}

We have now defined our scheme $\Proj S$ corresponding to a
graded ring $S$. Before we use this to construct projective space,
we take some time to study properties of both $\Proj$ as an operation
as well as the schemes that are thus obtained.

\subsection{Functoriality}

Recall that the spectrum construction establishes an equivalence
\begin{align*}
  \Spec : \op{\Ring} \cong \AffSch.
\end{align*}
Naively, one might hope that the map $\Proj$ establishes a similar
equivalence between graded rings and projective schemes.
However, it is not hard to see that such an equivalence cannot exist
because $\Proj$ fails to be functorial: For a map of graded rings
$\phi : S \to S'$ there need not be a canonical map of homogeneous
spectra $\Proj S' \to \Proj S$. This is because the preimage of
$p\in\Proj S'$ may contain the irrelevant ideal $S_+$. Hence
${\phi}^{-1}(p)\not\in\Proj S$.

\todo{there are more things to say here so maybe we should}

\subsection{Separatedness}

Let us now focus our attention towards projective schemes, i.e. schemes
of the form $\Proj S$ for some graded ring $S$. The first property that
one is generally interested in is separatedness. We know that affine schemes are separated. The same is true for projective schemes. In proving
this we will follow \cite[\href{https://stacks.math.columbia.edu/tag/01KP}{Tag 01KP}]{stacks-project} and \cite[\href{https://stacks.math.columbia.edu/tag/01M3}{Tag 01M3}]{stacks-project}, although in a more linear fashion.

We intend to show that the diagonal map
\begin{align}\label{eq:diagonal_map}
  \Delta : \Proj S \to \Proj S \times \Proj S
\end{align}
is a closed immersion. We begin by establishing an affine open
cover of the product $\Proj S\times \Proj S$:
\begin{lemma}
  The schemes $D_+(f)\times D_+(g)$ for homogeneous $f,g\in S$
  of positive degree form an affine open cover of $\Proj S\times \Proj S$.
  \begin{proof}
    We follow \cite[\href{https://stacks.math.columbia.edu/tag/01JS}{Tag 01JS}]{stacks-project}.
    Consider the projections
    \begin{align*}
      \Proj S \xlongleftarrow{p} \Proj S\times \Proj S \xlongrightarrow{q} \Proj S.
    \end{align*}
    It is clear that the sets ${p}^{-1}(D_+(f))\cap{q}^{-1}(D_+(g))$
    form an open cover of $\Proj S\times \Proj S$. We claim there is an isomorphism of
    schemes
    \begin{align}\label{eq:product_intersection_iso}
      D_+(f)\times D_+(g) \cong {p}^{-1}(D_+(f))\cap{q}^{-1}(D_+(g)).
    \end{align}
    To establish this let $T$ be a scheme and consider maps
    $a:T\to D_+(f)\subseteq \Proj S$ and $b:T\to D_+(g)\subseteq \Proj S$.
    Then there is a unique map $T \to \Proj S\times \Proj S$ that makes the
    following commute:
    \begin{equation}
      % https://q.uiver.app/#q=WzAsNixbMCwyLCJYIl0sWzQsMiwiWCJdLFsyLDIsIlhcXHRpbWVzIFgiXSxbMCwxLCJEXysoZikiXSxbNCwxLCJEXysoZykiXSxbMiwwLCJUIl0sWzIsMCwicCJdLFsyLDEsInEiLDJdLFszLDAsIiIsMix7InN0eWxlIjp7InRhaWwiOnsibmFtZSI6Imhvb2siLCJzaWRlIjoiYm90dG9tIn19fV0sWzQsMSwiIiwxLHsic3R5bGUiOnsidGFpbCI6eyJuYW1lIjoiaG9vayIsInNpZGUiOiJib3R0b20ifX19XSxbNSwzLCJhIiwyXSxbNSwyLCIiLDIseyJzdHlsZSI6eyJib2R5Ijp7Im5hbWUiOiJkYXNoZWQifX19XSxbNSw0LCJiIl1d
      \begin{tikzcd}
  && T \\
        {D_+(f)} &&&& {D_+(g)} \\
        \Proj S && {\Proj S\times \Proj S} && \Proj S
        \arrow["p", from=3-3, to=3-1]
        \arrow["q"', from=3-3, to=3-5]
        \arrow[hook', from=2-1, to=3-1]
        \arrow[hook', from=2-5, to=3-5]
        \arrow["a"', from=1-3, to=2-1]
        \arrow[dashed, from=1-3, to=3-3]
        \arrow["b", from=1-3, to=2-5]
      \end{tikzcd}
    \end{equation}
    The image of this map must lie in ${p}^{-1}(D_+(f))\cap{q}^{-1}(D_+(g))$. Hence ${p}^{-1}(D_+(f))\cap{q}^{-1}(D_+(g))$ is a product of the
    open subschemes $D_+(f)$ and $D_+(g)$ and the isomorphism
    (\ref{eq:product_intersection_iso}) follows by uniqueness of products.
  \end{proof}
\end{lemma}

We may now show that the diagonal map restricts to a closed immersion
for each of the open affines in the cover. This implies that the
map as a whole is a closed immersion, a fact which is straightforward
to check.

\begin{theorem}\label{thm:separated}
  Let $S$ be a graded ring. Then $\Proj S$ is separated.
  \begin{proof}
    Note ${\Delta}^{-1}(D_+(f)\times D_+(g)) = D_+(f)\cap D_+(g) = D_+(fg)$.
    To show that
    \begin{align*}
      \Delta : D_+(fg) \to D_+(f)\times D_+(g)
    \end{align*}
    is a closed immersion we need to show that
    $S_{(f)}\otimes S_{(g)}\to S_{(fg)}$ is surjective. To this end,
    consider $s\in S_{(fg)}$. This is of the form
    $s=h/(f^m g^n)$ for some homogeneous $h\in S$ and
    $\deg h = m\deg f + n\deg g$. Without loss of generality we assume
    $m = m'\deg g$ and $n=n'\deg f$ to write
    \begin{align*}
      s = \frac{h}{f^{(m'+n')\deg g}} \cdot \frac{f^{n'\deg g}}{g^{n'\deg g}}
    \end{align*}
    where the factors are elements of $S_{(f)}$ and $S_{(g)}$,
    respectively. Thus the map (\ref{eq:diagonal_map}) is a closed
    immersion on an affine open cover of $\Proj S\times \Proj S$
    and hence in its own right.
  \end{proof}
\end{theorem}

One may now recall the following standard result:

\begin{lemma}\label{thm:composition_of_separted_maps}
  Let $f:X\to Y$ and $g:Y\to Z$ be maps of schemes. If $g\circ f$ is
  separated then so is $f$.
  \begin{proof}
    See e.g. {\cite[\href{https://stacks.math.columbia.edu/tag/01KV}{Tag 01KV}]{stacks-project}}.
  \end{proof}
\end{lemma}

This immediately lets us extend \ref{thm:separated} to a wider
selection of base schemes:

\begin{corollary}
  Let $S$ be a graded $R$-algebra.
  Then $\Proj S$ is separated over $R$.
  \begin{proof}
    We have shown that the map
    \begin{align*}
      \Proj S \longrightarrow \Spec R \longrightarrow \Spec\mathbb{Z}
    \end{align*}
    is separated. The claim follows by \ref{thm:composition_of_separted_maps}.
  \end{proof}
\end{corollary}

In particular every graded ring $S$ is an $S_0$-algebra and hence
separated over $\Spec S_0$. This observation will directly imply
that $\projective{n}{k}$ is separated over $k$.

\subsection{The noetherian case}

\begin{example}
  % https://math.stackexchange.com/questions/4394865/if-operatornameproj-s-is-noetherian-must-s-be-noetherian
  $S_0 = \mathbb{Z}$ and $S_d = \mathbb{C}[x]_d$.
  Then $\Proj S = \mathbb{C}[x]$ but $S_+$ not fg.
  \missingexample
\end{example}

Thus we require stronger conditions for $\Proj S$ to be noetherian.

\section{Projective space}

Our main motivation for constructing the homogeneous spectrum and
its structure sheaf was to be able to construct projective space
more directly. In this section we are going to define projective space
as the homogeneous spectrum of a polynomial ring and verify that
this definition agrees with the gluing of affine schemes.

Moreover, we are going to use our new construction to study the
classical case over an algebraically closed field and show that this
does indeed yield a complete variety, as one would expect.

\subsection{Construction}

\begin{definition}
  Let $n\geq 0$. Then $n$-dimensional \emph{projective space}
  is the scheme
  \begin{align*}
    \projective{n}{} = \Proj \mathbb{Z}[x_0,\ldots,x_n].
  \end{align*}
\end{definition}

We have already encountered projective space as a gluing of $n+1$
copies of $n$-dimensional affine space along the maps
\begin{align*}
  \phi^\sharp_{ij} :
  \mathbb{Z}[\ldots,\hat x_i,\ldots,x_j^\pm,\ldots]
  &\to \mathbb{Z}[\ldots,x_i^\pm,\ldots,\hat x_j,\ldots] \\
  x_j &\mapsto {x^{-1}_i}
\end{align*}

We now need to verify that both notions agree.

\begin{lemma}\label{lem:affine_cover}
  There is a covering of $\projective{n}{}$ with opens
  $U_0,\ldots,U_n$ with isomorphisms
  \begin{align*}
    \phi_i : U_i \to \Spec\mathbb{Z}[x_0,\ldots,\hat x_i,\ldots,x_n],
  \end{align*}
  and
  \begin{align*}
    \phi_{ij} : U_i \cap U_j \to \Spec\mathbb{Z}[x_0,\ldots,\hat x_i,\ldots,x_j^\pm,\ldots,x_n]
  \end{align*}
  such that $\phi_{ij}\circ{\phi}^{-1}_{ji}(x_i) = {x}^{-1}_j$.
  \todo{i suspect that this does not quite work... i believe one needs
  to normalise all the coordinates, not just one; help may be found in the Bonn notes 10.3 or by thinking about affine charts where one of the coordinate is fixed to 1}
  \begin{proof}
    \missingproof
  \end{proof}
\end{lemma}

More generally, we may define projective space over any ring and
indeed any field. This is achieved as usual through base change.

\begin{definition}
  Let $n\geq 0$ and $T$ a scheme. Then $n$-dimensional
  \emph{projective space over $T$} is the base change
  \begin{align*}
    \projective{n}{T} := (\projective{n}{})_T = \projective{n}{}\times T.
  \end{align*}
  If $T=\Spec R$ we write $\projective{n}{R}=\projective{n}{T}$.
\end{definition}

\subsection{Points}

The affine cover from \label{lem:affine_cover} allows us to denote the
points of $\projective{n}{}$ more conveniently.

\subsection{Properness}

\begin{lemma}
  $\projective{n}{}$ is of finite type.
  \begin{proof}
    \missingproof
  \end{proof}
\end{lemma}

\begin{theorem}\label{thm:proper}
  $\projective{n}{S}$ is proper over $S$.
  \begin{proof}
    As base change preserves properness it suffices to show
    that $\projective{n}{}$ is proper. We have shown that
    $\projective{n}{}$ is separated and of finite type
    and are thus justified in applying the valuative criterion
    of properness.

    It remains to show that for all valuation rings
    $V$ with fraction fields $K=\Frac V$ there exists a unique
    map $\Spec V \to \projective{n}{}$ that makes the following commute:
    \begin{equation}
      % https://q.uiver.app/#q=WzAsNCxbMCwwLCJcXHRleHR7U3BlYyB9SyJdLFswLDIsIlxcdGV4dHtTcGVjIH1WIl0sWzIsMCwiXFxtYXRoYmYgUF5uIl0sWzIsMiwiXFx0ZXh0e1NwZWMgfVxcbWF0aGJiIFoiXSxbMCwxXSxbMSwzXSxbMiwzXSxbMCwyXSxbMSwyLCIiLDEseyJzdHlsZSI6eyJib2R5Ijp7Im5hbWUiOiJkYXNoZWQifX19XV0=
      \begin{tikzcd}
        {\text{Spec }K} && {\mathbf P^n} \\
        \\
        {\text{Spec }V} && {\text{Spec }\mathbb Z}
        \arrow[from=1-1, to=3-1]
        \arrow[from=3-1, to=3-3]
        \arrow[from=1-3, to=3-3]
        \arrow[from=1-1, to=1-3]
        \arrow[dashed, from=3-1, to=1-3]
      \end{tikzcd}
    \end{equation}
    \missingproof
  \end{proof}
\end{theorem}

\subsection{Integrality}

In this final section we will show that, over a field $k$,
projective space is a variety which, by the previous section,
is complete. The key property that we are missing is integrality.
We are going to show that $\projective{n}{k}$ is reduced and irreducible.


Fix $k$ and $n$ and write $X=\projective{n}{k}$.
It rather straightforward to see that $X$
is indeed reduced.
\begin{lemma}\label{thm:reduced}
  $X$ is reduced.
  \begin{proof}
    By~\ref{thm:stalks} we have
    $\mathcal O_{X,x} = k[x_0,\ldots,x_n]_{(p_x)}$. Now note
    that $S=k[x_0,\ldots,x_n]$ is reduced. Thus the localisation
    $S_{p_x}$ is reduced and so is its subring $S_{(p_x)}$.
  \end{proof}
\end{lemma}

\begin{theorem}\label{thm:irreducible}
  $X$ is irreducible.
  \begin{proof}
    We follow the proof of \cite[\href{https://stacks.math.columbia.edu/tag/01OL}{Tag 01OL}]{stacks-project}.
    We have shown that $X$ may be covered by $U_1,\ldots,U_n$ with
    $U_i \cong \affine{n}{k}$ and $U_i\cap U_j\neq\emptyset$. Now write
    $X=X_1 \cup X_2$ for some closed $X_1,X_2\subseteq X$.
    Fix $i$. As $U_i$ is irreducible we must have
    $U_i\subseteq X_1$ or $U_i\subseteq X_2$. Assume without loss
    of generality $U_i\subseteq Z_1$. Now, for all $j$,
    we observe
    \begin{align*}
      U_j = \overline{U_i\cap U_j}\cup U_j\cap(X\setminus U_i)
    \end{align*}
    where both sides are closed and the set on the right is a proper
    subset because $U_i\cap U_j$ is non-empty. As $U_j$ is irreducible,
    $U_i\cap U_j$ is dense in $U_j$. Moreover, $U_i \cap U_j$ is
    contained in the closed subset $X_1$. Thus $U_j\subseteq X_1$
    for all $j$ and hence $X=X_1$.
  \end{proof}
\end{theorem}

\begin{corollary}
  $\projective{n}{k}$ is a complete variety.
  \begin{proof}
    From (\ref{thm:proper}) we have properness which means separated,
    universally closed, and of finite type. By (\ref{thm:reduced})
    and (\ref{thm:irreducible}) $X$ is reduced and irreducible, i.e.
    integral.
  \end{proof}
\end{corollary}

\printbibliography

\end{document}
