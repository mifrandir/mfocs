\documentclass{article}
\usepackage{assignment}
\addbibresource{higgs.bib}
\title{Proposal: Stable bundles and Higgs bundles}

\begin{document}

Hitchin introduced Higgs bundles as solutions to the self-duality
equations on compact Riemann surfaces. \cite{hitchin1986} The
corresponding moduli spaces form quasi-projective varieties and
hyper-K\"ahler manifolds \cite{mccarthy2020}, hinting at the rich relationship to complex
and algebraic geometry.
Since their
introduction in in 1986, Higgs fields have appeared in various areas  of research such
as non-abelian Hodge theory, mirror symmetry, and geometric Langlands duality.
\cite{bradlow2007}

Constructing moduli spaces of Higgs bundles is non-trivial. One approach
is to follow the construction of the extensively studied spaces of
stable bundles by using suitable stability conditions and symplectic
quotients. This viewpoint relies on results from geometric
invariant theory (GIT). \cite{neitzke2021} The relationship between Higgs bundles and stable
bundles goes further, however. It turns out that covectors of stable
bundles correspond to Higgs bundles and, moreover, form an open
dense subset of the moduli space of Higgs bundles.
Beyond this, the non-abelian Hodge correspondence
allows us to view Higgs bundles simultaneously as projective flat
connections as well as complex representations of the fundamental
group.

The aim of this project is to make sense of Higgs bundles and stable
bundles by considering both the geometric and algebraic perspectives.
This includes rigorously constructing the moduli spaces as geometric
quotients. To this end it will be necessary to review introductory
material on complex and symplectic geometry, GIT, the theory of
moduli spaces, and possibly Grothendieck's Quot schemes.

Following this we intend to prove that the cotangent space of stable bundles
forms a dense open subset of the space of Higgs bundles. Possible extensions
include reviewing the applications of Higgs bundles to algebraic geometry and
theoretical physics, particularly within the subjects of mirror symmetry and
the geometric Langlands program.


\printbibliography

\end{document}
