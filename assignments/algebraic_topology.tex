\documentclass{article}
\usepackage{assignment}
\begin{document}
\title{C3.1 Algebraic Topology Mini Project: De Rahm Cohomology}
\author{Franz Miltz}
\date{\today}
\maketitle
\tableofcontents

\section{Manifolds}

\todo{write recap}

\section{The exterior algebra}

Recall that the tensor product $U\otimes V$ of two vector spaces $U$ and $V$ is, up to isomorphism, defined by the universal property that any bilinear map $f:U\times V\to W$ determines a unique linear map $\tilde f : U\otimes V\to W$ making the following diagram commute:

\begin{equation*}
  % https://q.uiver.app/#q=WzAsMyxbMCwwLCJVXFx0aW1lcyBWIl0sWzEsMSwiVVxcb3RpbWVzIFYiXSxbMiwwLCJXIl0sWzAsMiwiZiJdLFsxLDIsIlxcdGlsZGUgZiIsMl0sWzAsMSwiXFxwaSIsMl1d
  \begin{tikzcd}
    {U\times V} && W \\
                & {U\otimes V}
                \arrow["f", from=1-1, to=1-3]
                \arrow["{\tilde f}"', from=2-2, to=1-3]
                \arrow["\pi"', from=1-1, to=2-2]
  \end{tikzcd}
\end{equation*}

This is associative up to a canonical isomorphism and hence we may write
\begin{align*}
  \tensor{k}{V} = V \otimes \cdots \otimes V
\end{align*}
for the $k$-fold tensor product of $V$.

\begin{definition}
  The \emph{tensor algebra} of a vector space $V$ is the set
  \begin{align*}
    \tensor{}{V} = \bigcup_{k=1}^\infty \tensor{k}{V}
  \end{align*}
  equipped with the tensor products
  \begin{align*}
    \otimes : \tensor{k}{V} \times \tensor{\ell}{V}\to \tensor{k+\ell} V.
  \end{align*}
\end{definition}

This is an $\mathbb{R}$-algebra in the usual sense. It will prove fruitful to define the ideal spanned by elements of the form $v\otimes v$ in $\tensor{}{V}$. Let us write
\begin{align*}
  I(V) = \langle\left\lbrace{v\otimes v : v \in V}\right\rbrace.
\end{align*}
Then an element $a\in I(V)\subseteq \tensor{}{V}$ is a linear combination
of multiples of $v\otimes v$ for some $v\in V$. There are a few
logical consequences worth highlighting:
\begin{itemize}
  \item For all $u,v\in V$, $u\otimes v + v \otimes u \in I(V)$
    because
    \begin{align*}
      u\otimes v + v \otimes u = (u+v)\otimes(u+v)-u\otimes u-v\otimes v.
    \end{align*}
  \item For all $a,b\in\tensor{}{V}$, $a\otimes b+b\otimes a$ due to
    a similar argument as the above. \todo{wrong}
  \item For all $a,b\in V$, $a\otimes b\otimes a\in I(V)$ because
    \begin{align*}
      a\otimes b\otimes a = (a\otimes b\otimes a + a \otimes a \otimes b) - a\otimes a\otimes b.
    \end{align*}
\end{itemize}

This lets us define a construction that we will be studying for the remainder
of this project:

\begin{definition}
  Let $V$ be an $n$-dimensional vector space. The \emph{exterior algebra of
  $V$} is the quotient
  \begin{align*}
    \exterior{}{V} = \tensor{}{V}/I(V).
  \end{align*}
  We write $\exterior{k}{V}$ for the image of $\tensor{k}{V}$ under the quotient
  map.
\end{definition}

\begin{lemma}
  Let $V$ be an $n$-dimensional vector space.
  Then for all $\alpha\in\exterior{k}{V}$ and $\beta\in\exterior{\ell}{V}$,
  \begin{align*}
    \alpha\wedge\beta = (-1)^{k\ell}\beta\wedge\alpha.
  \end{align*}
  \begin{proof}
    We have seen $u\wedge v + v\wedge u = 0$ for any $u,v\in V$.
    By linearity of the quotient map $\tensor{}{V}\to\exterior{}{V}$,
    it follows that interchaning any adjacent elements in an expression
    $u_1\wedge\cdots\wedge u_k$ changes the sign. Now,
    for $u_1,\ldots,u_k,v_1,\ldots,v_\ell\in V$,
    \begin{align*}
      (u_1\wedge\cdots\wedge u_k)\wedge(v_1\wedge\cdots\wedge v_\ell)
      = (-1)^{k\ell}(v_1\wedge\cdots\wedge v_\ell)\wedge(u_1\wedge\cdots\wedge u_k)
    \end{align*}
    by moving each $v_i$ individually. The claim then follows
    by writing elements of $\exterior{k}{V}$ as linear combinations
    of $k$-fold wedges $v_1\wedge\cdots\wedge v_k$.
  \end{proof}
\end{lemma}

\begin{lemma}
  Let $V$ be an $n$-dimensional vector space and $v_1,\ldots,v_n$ a basis.
  Then
  \begin{align*}
    \left\lbrace{v_{j_1}\wedge\cdots\wedge v_{j_k} : 1\leq j_1<\cdots<j_k\leq n }\right\rbrace
  \end{align*}
  is a basis for $\exterior{k}{V}$. Thus $\dim\exterior{k}{V} = \binom{n}{k}$.
  \begin{proof}
  \end{proof}
\end{lemma}

To simplify notation, we will use multi-indices. A multi-index $I$ of size
$k$ and dimension $n$ is a tuple $(i_1,\ldots,i_k)$ such that
$1\leq i_1<\cdots<i_k\leq n$. We will write $v_I = v_{i_1}\wedge \cdots \wedge
v_{i_k}$ for the basis vectors of $\exterior{k}{V}$ so a general
element $x\in\exterior{k}{V}$ will be written as
$x = \sum_I x_I v_I$ where the $I$ is understood to range over
all the $\binom{n}{k}$ multi-indices of size $k$ and dimension $n$.

\begin{example}
  We are going to be interested in studying the exterior
  algebra of the cotangent spaces $T_p^* X$ of a manifold $X$.
  Let $\alpha\in \exterior{k}{T^*_p X}$. Then there is a corresponding $a\in
  \tensor{k}{T_p^* X}$. Now
  \begin{align*}
    \tensor{k}{T_p^* X} = \tensor{k}{\Hom(T_p X,\mathbb{R})} \cong \Hom(\tensor{k}{T_p X}, \tensor{k}{\mathbb{R}}) \cong \Hom(\tensor{k}{T_p X},\mathbb{R}).
  \end{align*}
  Hence $a$ may be understood as a map $\tensor{k}{T_p X}\to\mathbb{R}$
  which is just a multilinear map $T_p X\times\cdots\times T_p X\to\mathbb{R}$. It is not difficult to see that the elements of $\exterior{k}{T^*_p X}$
  correspond exactly to those multilinear maps that are alternating,
  i.e. for all $1\leq i < j\leq k$,
  \begin{align*}
    a(x_1,\ldots,x_i,\ldots,x_j,\ldots x_k) = -a(x_1,\ldots,x_{i-1},x_j,x_{i+1},\ldots,x_{j-1},x_i,x_{j+1},\ldots,x_k)
  \end{align*}
\end{example}

\section{Vector bundles}

\begin{definition}
  Let $X$ be an $n$-manifold and $k\geq 0$.
  A \emph{rank $k$ vector bundle on $X$} consists of
  \begin{enumerate}
    \item an $(n+k)$-manifold $E$,
    \item a smooth surjective map $\pi : E\to X$, and
    \item for each $x\in X$, a $k$-dimensional real vector space structure on ${\pi}^{-1}\left\lbrace{x}\right\rbrace$,
  \end{enumerate}
  such that, for each $x\in X$ there is an open neighbourhood $U\ni x$
  together with a diffeomorphism $\psi : {\pi}^{-1} U \to U\times\mathbb{R}^k$ that restricts to an isomorphism $\psi : {\pi}^{-1}\left\lbrace{u}\right\rbrace\to \left\lbrace{u}\right\rbrace\times\mathbb{R}^k$ for all $u\in U$.
\end{definition}

\begin{example}\label{ex:bundles}
  Let $X$ be an $n$-manifold. Then there are several obvious vector bundles:
  \begin{itemize}
    \item The manifold $X\times\mathbb{R}^m$ together with the projection $\pi : X\times\mathbb{R}^m\to X$ trivially forms a vector bundle.
    \item Define
      \begin{align*}
        TX = \bigsqcup_{p\in X} T_p X = \left\lbrace{(p, v) : x \in X, v \in T_p X}\right\rbrace
      \end{align*}
      where $T_p X$ denotes the tangent space of $X$ at $p$. The manifold
      structure on $TX$ is induced by the manifold structure on $X$ in the
      obvious way: For each chart $\phi:U\to X$, define a map
      $\psi:U\times\mathbb{R}^n\to TX$ by
      \begin{align*}
        \psi(p,y_1,\ldots y_n)
        = \sum_{j=1}^k y_j \res{\frac{\partial}{\partial x_j}}{p}.
      \end{align*}
      This turns out to be a chart \todo{prove this?} and, moreover, gives rise
      to an atlas on $TX$. We thus have a $2n$-manifold $TX$ which, together with the projection $(p,v)\mapsto p$, forms a vector bundle called the \emph{tangent bundle}.
    \item One may similarly define the \emph{cotangent bundle} $T^*X$.
      More generally, the exterior powers $\exterior{k}{T^* X}$ are going
      to be of interest to us. Define
      \begin{align*}
        \exterior{k}{T^*X} = \bigsqcup_{p\in X} \exterior{k}{T^*_p X}
      \end{align*}
      where $T^*_p X$ is the dual $(T_pX)^*$, i.e.~the space of linear
      maps $T_p X\to\mathbb{R}$. Once again, we extend a chart
      $\phi:U\to X$ to a chart $\psi : U\times\mathbb{R}^{\binom{n}{k}}\to \exterior{k}{T^*X}$ in the obvious way:
      \begin{align*}
        \psi(p,y) = \sum_I y_I \res{dx_I}{p}
      \end{align*}
      where $\res{dx_1}{p},\ldots,\res{dx_n}{p}$ are the usual basis vectors of
      $T^*_p X$. Direct computation shows that this is a chart and
      thus we have a manifold $\exterior{k}{T^*_p X}$ of dimension
      $n+\binom{n}{k}$ which forms a vector bundle with the
      projection $(p,v)\mapsto p$.
  \end{itemize}
\end{example}

\begin{definition}
  A  (smooth) \emph{section} of a vector bundle $\pi:E\to X$ is a (smooth)
  section of the map $\pi$, i.e. a smooth map $\sigma : X\to E$ such that $\pi
  \circ \sigma = \identity_X$. Denote by $\Gamma(E)$ the set of all sections of
  a vector bundle $E$.
\end{definition}

\begin{example}
  Let us inspect the sections of the bundles in~\ref{ex:bundles}.
  \begin{itemize}
    \item A section of the trivial bundle is a map $p\mapsto (p,\phi(p))$ where $\phi:X\to\mathbb{R}^m$ is smooth.
    \item A section of the tangent bundle is a smooth map of the form
      $p\mapsto (p,v_p)$ where $v_p\in T_p X$. Such a section is called
      a \emph{vector field}.
    \item Similarly, a section of $\exterior{k}{T^* X}$ smoothly assigns to each
      $p\in X$ an element $\alpha\in\exterior{k}{T^*_p X}$. That is,
      if we have coordinates $(x_1,\ldots,x_n)$ on $U\subseteq X$,
      then
      \begin{align}\label{eq:local_differential_form}
        \res{\alpha}{U} = \sum_I \alpha_I dx_I
      \end{align}
      where each $\alpha_I$ is a smooth function $U\to \mathbb{R}$.
  \end{itemize}
\end{example}

\section{Differential forms}

\begin{definition}
  Let $X$ be a manifold. For $k\geq 1$, the set of \emph{differential
  $k$-forms} is $\Omega^k(X) = \Gamma^\infty(\exterior{k}{T^* X})$.
  Further define $\Omega^0(X)$ to be the set of smooth functions
  $X\to\mathbb{R}$.
\end{definition}

\todo{note how $k>n$ leads to zeros}

We saw in (\ref{eq:local_differential_form}) that, locally, a differential
form is just a smooth linear combination of $dx_I$. This turns out
to be a good way to think about differential forms.

While differential forms are useful in their own right, in particular
to generalise the notions of integration and orientation to general
manifolds, we will use them to define a co-chain complex whose cohomology will be the de Rahm cohomology, which itself is equivalent to singular
cohomology with real coefficients. In this sense, differential forms
may be thought of an analog to chains.

The objects in our co-chain complex will be the vector spaces $\Omega^k(X)$. Let us now define the boundary map. Recall that, in local coordinates,
the derivative of a smooth function $f:X\to\mathbb{R}$ is
\begin{align*}
  df = \sum_{i=1}^n \frac{\partial f}{\partial x_i} dx_i.
\end{align*}
That is, differentiating a function yields a 1-form. This is easily
generalised to a derivative of $k$-forms by considering a smooth
multiple $a dx_I$ on an open subset $U\subseteq X$. Hence
\begin{align}\label{eq:local_exterior_derivative}
  d(adx_I) = da\wedge dx_I = \sum_{i=1}^n \frac{\partial a}{\partial x_i} dx_i\wedge dx_I.
\end{align}
Note that this is a strong candidate for a boundary map:
\begin{align*}
  d^2(adx_I) &= \sum_{i=1}^{n} d\left({\frac{\partial a}{\partial x_i}}\right)\wedge dx_i \wedge dx_I\\
             &= \sum_{i=1}^{n} \sum_{j=1}^{n} \frac{\partial^2 a}{\partial x_i \partial x_j} dx_j\wedge dx_i \wedge dx_I \\
             &= 0
\end{align*}
where the last step holds because $\partial^2/\partial x_i\partial x_j$ is symmetric in $i$ and $j$ but $dx_i\wedge dx_j = -dx_j \wedge dx_i$.

This derivative of $k$-forms also inherits one of the essential properties of the usual derivative. Consider $\alpha=adx_I\in\Omega^k(X)$ and $\beta=bdx_J\in\Omega^\ell(X)$. Then
\begin{align*}
  d(\alpha\wedge\beta) &= d(abdx_I\wedge dx_J) \\
                       &= d(ab)\wedge dx_I\wedge dx_J \\
                       &= (adb + bda)\wedge dx_I\wedge dx_J \\
                       &=(-1)^k a dx_I\wedge db \wedge dx_J
                       + da \wedge dx_I\wedge bdx_J \\
                       &= (-1)^k \alpha\wedge d\beta + d\alpha\wedge\beta.
\end{align*}

We avoid referencing local coordinates in our definition by taking these
properties as axioms:

\begin{definition}
  Let $X$ be an $n$-manifold. The \emph{exterior derivative} is the
  unique family of linear maps
  \begin{align*}
    d : \Omega^k(X) \to \Omega^{k+1}(X)
  \end{align*}
  such that
  \begin{enumerate}
    \item for $f\in\Omega^0(X)$, $df\in\Omega^1(X)$ is the usual derivative,
    \item $d^2 = 0$,
    \item for $\alpha\in\Omega^k(X)$, $d(\alpha\wedge\beta) = d\alpha\wedge\beta + (-1)^k \alpha\wedge d\beta$.
  \end{enumerate}
\end{definition}

We have already seen that (\ref{eq:local_exterior_derivative}) satisfies
these conditions. Moreover, taking $f\in\Omega^0(X)$ and $\alpha\in\Omega^k$
with $\res{\alpha}{U} = dx_I$ for some $U\subseteq X$, e.g. by considering a partition of unity, we find
\begin{align*}
  \res{d(f\alpha)}{U}
  &= \res{d(f\wedge\alpha)}{U}\\
  &= \res{(df\wedge\alpha + f\wedge d\alpha)}{U}\\
  &= \res{(df\wedge\alpha)}{U} + \res{f}{U}\wedge \res{d\alpha}{U}\\
  &= \res{(df\wedge\alpha)}{U} + \res{f}{U}\wedge \res{d^2x_I}{U}\\
  &= \res{(df\wedge\alpha)}{U}.
\end{align*}
Hence our local expression (\ref{eq:local_exterior_derivative}) must
hold for any family of maps satisfying the definition. In particular,
this shows that the exterior derivative is in fact unique and given by
the expression (\ref{eq:local_exterior_derivative}) in any coordinates.

By setting $\Omega^k(X)=0$ for $k<0$, we have a cochain complex
\begin{align*}
  \cdots
  \xrightarrow{d} \Omega^k(X)
  \xrightarrow{d} \Omega^{k+1}(X)
  \xrightarrow{d} \cdots
\end{align*}
This is the \emph{de Rahm complex}.

\section{De Rahm cohomology}

\section{Homotopy invariance}

\section{K\"unneth's theorem}

\section{Poincar\'e duality}

\end{document}
