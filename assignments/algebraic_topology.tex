\documentclass{article}
\usepackage{assignment}
\begin{document}
\title{De Rahm Cohomology}
\author{Franz Miltz}
% TODO date
\date{\today}
\maketitle


\section{The exterior algebra}

Recall that the tensor product $U\otimes V$ of two vector spaces $U$ and $V$ is, up to isomorphism, defined by the universal property that any bilinear map $f:U\times V\to W$ determines a unique linear map $\tilde f : U\otimes V\to W$ making the following diagram commute:

\begin{equation*}
  % https://q.uiver.app/#q=WzAsMyxbMCwwLCJVXFx0aW1lcyBWIl0sWzEsMSwiVVxcb3RpbWVzIFYiXSxbMiwwLCJXIl0sWzAsMiwiZiJdLFsxLDIsIlxcdGlsZGUgZiIsMl0sWzAsMSwiXFxwaSIsMl1d
  \begin{tikzcd}
    {U\times V} && W \\
                & {U\otimes V}
                \arrow["f", from=1-1, to=1-3]
                \arrow["{\tilde f}"', from=2-2, to=1-3]
                \arrow["\pi"', from=1-1, to=2-2]
  \end{tikzcd}
\end{equation*}

This is associative up to a canonical isomoprhism and hence we may write
\begin{align*}
  \bigotimes^k V = V \otimes \cdots \otimes V
\end{align*}
for the $k$-fold tensor product of $V$.

\begin{definition}
  Let $V$ be an $n$-dimensional vector space. The \emph{exterior algebra of
  $V$} is the quotient
  \begin{align*}
    \Lambda V = \frac{\bigotimes V}{\langle\left\lbrace{v\otimes v : v \in V}\right\rbrace\rangle}.
  \end{align*}
  We write $\Lambda^k V$ for the image of $\bigotimes^k V$ under the quotient
  map.
\end{definition}

\begin{lemma}
  Let $V$ be an $n$-dimensional vector space and $v_1,\ldots,v_n$ a basis.
  Then
  \begin{align*}
    \left\lbrace{v_{j_1}\wedge\cdots\wedge v_{j_k} : 1\leq j_1<\cdots<j_k\leq n }\right\rbrace
  \end{align*}
  is a basis for $\Lambda^k V$.
  \begin{proof}
    \todo{proof}
  \end{proof}
\end{lemma}

To simplify notation, we will use multi-indices. A multi-index $I$ of size
$k$ and dimension $n$ is a tuple $(i_1,\ldots,i_k)$ such that
$1\leq i_1<\cdots<i_k\leq n$. We will write $v_I = v_{i_1}\wedge \cdots \wedge
v_{i_k}$ for the basis vectors of $\Lambda^k V$ so a general
element $x\in\Lambda^k V$ will be written as
$x = \sum_I x_I v_I$ where the $I$ is understood to range over
all the $\binom{n}{k}$ multi-indices of size $k$ and dimension $n$.

\begin{example}
  We are going to be interested in studying the exterior
  algebra of the cotangent spaces $T_p^* X$ of a manifold $X$.
  Let $\alpha\in \Lambda^k T^*_p X$. Then there is a corresponding $a\in
  \bigotimes^k T_p^* X$. Now
  \begin{align*}
    \bigotimes^k T_p^* X = \bigotimes^k \Hom(T_p X,\mathbb{R}) \cong \Hom(\bigotimes^k T_p X, \bigotimes^k \mathbb{R}) \cong \Hom(\bigotimes^k T_p X,\mathbb{R}).
  \end{align*}
  Hence $a$ may be understood as a map $\bigotimes^k T_p X\to\mathbb{R}$
  which is just a $k$-linear map $T_p X\times\cdots\times T_p X\to\mathbb{R}$. Now any two representatives $a,a'\in\bigotimes^k T_p^* X$ differ
  at most by
\end{example}

\section{Vector bundles}

\begin{definition}
  Let $X$ be an $n$-manifold and $k\geq 0$.
  A \emph{rank $k$ vector bundle on $X$} consists of
  \begin{enumerate}
    \item an $(n+k)$-manifold $E$,
    \item a smooth surjective map $\pi : E\to X$, and
    \item for each $x\in X$, a $k$-dimensional real vector space structure on ${\pi}^{-1}\left\lbrace{x}\right\rbrace$,
  \end{enumerate}
  such that, for each $x\in X$ there is an open neighbourhood $U\ni x$
  together with a diffeomorphism $\psi : {\pi}^{-1} U \to U\times\mathbb{R}^k$ that restricts to an isomorphism $\psi : {\pi}^{-1}\left\lbrace{u}\right\rbrace\to \left\lbrace{u}\right\rbrace\times\mathbb{R}^k$ for all $u\in U$.
\end{definition}

\begin{example}\label{ex:bundles}
  Let $X$ be an $n$-manifold. Then there are several obvious vector bundles:
  \begin{itemize}
    \item The manifold $X\times\mathbb{R}^m$ together with the projection $\pi : X\times\mathbb{R}^m\to X$ trivially forms a vector bundle.
    \item Define
      \begin{align*}
        TX = \bigsqcup_{p\in X} T_p X = \left\lbrace{(p, v) : x \in X, v \in T_p X}\right\rbrace
      \end{align*}
      where $T_p X$ denotes the tangent space of $X$ at $p$. The manifold
      structure on $TX$ is induced by the manifold structure on $X$ in the
      obvious way: For each chart $\phi:U\to X$, define a map
      $\psi:U\times\mathbb{R}^n\to TX$ by
      \begin{align*}
        \psi(p,y_1,\ldots y_n)
        = \sum_{j=1}^k y_j \res{\frac{\partial}{\partial x_j}}{p}.
      \end{align*}
      This turns out to be a chart \todo{prove this?} and, moreover, gives rise
      to an atlas on $TX$. We thus have a $2n$-manifold $TX$ which, together with the projection $(p,v)\mapsto p$, forms a vector bundle called the \emph{tangent bundle}.
    \item One may similarly define the \emph{cotangent bundle} $T^*X$.
      More generally, the exterior powers $\Lambda^k T^* X$ are going
      to be of interest to us. Define
      \begin{align*}
        \Lambda^k T^*X = \bigsqcup_{p\in X} \Lambda^k T^*_p X
      \end{align*}
      where $T^*_p X$ is the dual $(T_pX)^*$, i.e.~the space of linear
      maps $T_p X\to\mathbb{R}$. Once again, we extend a chart
      $\phi:U\to X$ to a chart $\psi : U\times\mathbb{R}^{\binom{n}{k}}\to \Lambda^k T^*X$ in the obvious way:
      \begin{align*}
        \psi(p,y) = \sum_I y_I \res{dx_I}{p}
      \end{align*}
      where $\res{dx_1}{p},\ldots,\res{dx_n}{p}$ are the usual basis vectors of
      $T^*_p X$. Once again, direct computation shows that this is a chart and
      thus we have a manifold $\Lambda^k T^*_p X$ of dimension
      $n+\binom{n}{k}$ which forms a vector bundle with the
      projection $(p,v)\mapsto p$.
  \end{itemize}
\end{example}

\begin{definition}
  A  (smooth) \emph{section} of a vector bundle $\pi:E\to X$ is a (smooth)
  section of the map $\pi$, i.e. a smooth map $\sigma : X\to E$ such that $\pi
  \circ \sigma = \identity_X$. Denote by $\Gamma(E)$ the set of all sections of
  a vector bundle $E$.
\end{definition}

\begin{example}
  Let us inspect the sections of the bundles in~\ref{ex:bundles}.
  \begin{itemize}
    \item A section of the trivial bundle is a map $p\mapsto (p,\phi(p))$ where $\phi:X\to\mathbb{R}^m$ is smooth.
    \item A section of the tangent bundle is a smooth map of the form
      $p\mapsto (p,v_p)$ where $v_p\in T_p X$. Such a section is called
      a \emph{vector field}.
    \item Similarly, a section of $\Lambda^k T^* X$ smoothly assigns to each
      $p\in X$ an element $\alpha\in\Lambda^k T^*_p X$.
  \end{itemize}
\end{example}
\todo{example}

\section{Differential forms}

\begin{definition}
  Let $X$ be a manifold. The set of \emph{differential $k$-forms} is
  $\Omega^k(X) = \Gamma^\infty(\Lambda^k T^* X)$.
\end{definition}

\end{document}
