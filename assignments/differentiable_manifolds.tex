\documentclass{article}
\usepackage{assignment}
\begin{document}
\title{C3.3 Differentiable Manifolds Mini Project: Principal Bundles}
\author{Candidate 1082394}
\date{\today}
\maketitle

\section{Vector bundles}

Recall the definition of a vector bundle:

\begin{definition}
  Let $X$ be an $n$-manifold and $k\geq 0$.
  A \emph{rank $k$ vector bundle on $X$} consists of
  \begin{enumerate}
    \item an $(n+k)$-manifold $E$,
    \item a smooth surjective map $\pi : E\to X$, and
    \item for each $x\in X$, a $k$-dimensional real vector space structure on $E_x = {\pi}^{-1}\left\lbrace{x}\right\rbrace$,
  \end{enumerate}
  such that, for each $x\in X$ there is an open neighbourhood $U\ni x$
  together with a diffeomorphism ${\pi}^{-1}(U)\cong U\times\mathbb{R}^k$ that restricts to an isomorphism ${\pi}^{-1}\left\lbrace{y}\right\rbrace\cong\left\lbrace{y}\right\rbrace\times\mathbb{R}^k$ for all $y\in U$.
\end{definition}

We write $E_x$ and $E_U$ for the pre-images of
$\left\lbrace{x}\right\rbrace\subseteq X$ and $U\subseteq X$ under the
projection, respectively. We have a category. Here are the morphisms:

\begin{definition}
  Let $E\to X$ and $F\to Y$ be vector bundles.
  A morphism $E\to F$ consists of smooth maps $\phi:E\to F$ and
  $f:X\to Y$ making the following commute:
  \begin{equation*}
    % https://q.uiver.app/#q=WzAsNCxbMCwwLCJFIl0sWzIsMCwiRiJdLFswLDEsIlgiXSxbMiwxLCJZIl0sWzAsMSwiXFxwaGkiXSxbMiwzLCJmIl0sWzAsMiwiXFxwaSIsMl0sWzEsMywiXFxwaSJdXQ==
    \begin{tikzcd}
      E && F \\
      X && Y
      \arrow["\phi", from=1-1, to=1-3]
      \arrow["f", from=2-1, to=2-3]
      \arrow["\pi"', from=1-1, to=2-1]
      \arrow["\pi", from=1-3, to=2-3]
    \end{tikzcd}
  \end{equation*}
\end{definition}

\section{Principal bundles}

\begin{definition}
  Let $X$ be a manifold and $G$ a Lie group. A \emph{principal $G$-bundle
  $P\to X$} consists of
  \begin{enumerate}
    \item a manifold $P$,
    \item a smooth, free right $G$-action $\mu:P\times G\to P$,
    \item a $G$-invariant smooth projection $\pi:P\to X$
  \end{enumerate}
  such that there exists a covering of $X$ by open sets
  $U$ with diffeomorphisms ${\pi}^{-1}(U)\cong U\times G$
  identifying $\mu : {\pi}^{-1}(U)\times G\to{\pi}^{-1}(U)$
  and $\pi:{\pi}^{-1}(U)\to U$ with the maps
  $((x,g),h)\mapsto (x,gh)$ and $(x,g)\mapsto x$, respectively.
\end{definition}

As usual, we will denote the fibre over $x\in X$ by $P_x = {\pi}^{-1}\left\lbrace{x}\right\rbrace$. Similarly, write $P_U = {\pi}^{-1}(U)$.
Let us do a little warm-up:

\begin{lemma}
  Let $P\to X$ be a principal $G$-bundle and $U\subseteq X$ any open
  subset. Then $P_U\to X$ is a principal $G$-bundle.
  \begin{proof}
    The manifold structure $P_U$ is inherited from $P$,
    we have $\mu(P_U\times G)$ by using the identification
    of $\mu$ with $((x,g),h)\mapsto (x,gh)$, $\pi:P_U\to U$
    remains smooth and $G$-invariant, and the covering and the
    diffeomorphisms are obtained by intersection and restriction, respectively.
  \end{proof}
\end{lemma}

Like most mathematical objects, principal bundles form categories.
To make this precise, we need to define morphisms of principal bundles.
This turns out to be entirely analogous to morphisms of vector
bundles:

\begin{definition}
  Let $P \to X$ and $Q\to Y$ be principal $G$-bundles.
  A morphism $P\to Q$ is a pair of smooth maps $\phi:P\to Q$ and $f:X\to Y$
  such that $\phi$ is $G$-equivariant and the following commute:
  \begin{equation*}
    % https://q.uiver.app/#q=WzAsNCxbMCwwLCJQIl0sWzIsMCwiUSJdLFswLDEsIlgiXSxbMiwxLCJZIl0sWzAsMSwiXFxwaGkiXSxbMiwzLCJmIl0sWzAsMiwiXFxwaSIsMl0sWzEsMywiXFxwaSJdXQ==
    \begin{tikzcd}
      P && Q \\
      X && Y
      \arrow["\phi", from=1-1, to=1-3]
      \arrow["f", from=2-1, to=2-3]
      \arrow["\pi"', from=1-1, to=2-1]
      \arrow["\pi", from=1-3, to=2-3]
    \end{tikzcd}
  \end{equation*}
\end{definition}

\section{Associated bundles}

The similarities between vector bundles and principal bundles are
hard to miss. This is not an accident. One way to think of principal
bundles is as a generalisation of vector bundles. The following
constructions are standard, see e.g.~\cite{mitchell2001}.

If a group $G$ acts on $A$ from the right and $B$ from the left, we have the
diagonal action of $G$ on $A\times B$ given by $g(a,b)=(ag,{g}^{-1}b)$. We then write the quotient $(A\times B)/G$ under this action as
\begin{align*}
  A\times_G B = \left\lbrace{[a,b]=G(a,b) : a\in A, b\in B}\right\rbrace.
\end{align*}

One property that we are going to make good use of is the following:
\begin{lemma}\label{lem:fibre_product_space}
  Let $V$ be a representation of $G$. Then $G\times_G V \cong V$
  as vector spaces.
  \begin{proof}
    Define $\phi:G\times_G V\to V$ by $[g,v]\mapsto gv$. This is well-defined
    as
    \begin{align*}
      \phi[gh,{h}^{-1}v]=gh{h}^{-1}v=gv=\phi[g,v].
    \end{align*}
    There clearly is an inverse $v\mapsto [e,v]$. Moreover,
    taking the vector space structure on $G\times_G V$ to be
    \begin{align}\label{eq:fibre_product_space}
      \lambda[g,u] + \mu[h,v] = [gh,\lambda u + \mu v]
    \end{align}
    it is easy to check that this agres with the the vector space
    structure on $V$ under the isomorphism $\phi$.
  \end{proof}
\end{lemma}
We will use this in the case where $P$ is a principal bundle and $V$ is
a representation of $G$. Hence we have $P\times_G V$. Using
\ref{lem:fibre_product_space} we obtain a manifold structure on
$P\times_G V$: For trivial neighbourhoods $U$ we have
$P_U\times_G V\cong (U\times G)\times_G V\cong U\times V$
so charts and trivialisations of $P$ combine to cover $P\times_G V$
with charts.

\begin{definition}
  Let $P\to X$ be a principal $G$-bundle and let $V$ be a finite dimensional
  representation of $G$. The \emph{associated vector bundle} consists of
  \begin{enumerate}
    \item the manifold $E(P;V)=P\times_G V$,
    \item the map $E(P;V)\to X$ given by $[p,v]\mapsto \pi(p)$
      where $\pi:P\to X$,
    \item the vector space structure on $E_x$ is inherited
      from $V$ through $E(P;V)_x = P_x \times_G V \cong G\times_G V \cong V$.
  \end{enumerate}
\end{definition}

We write $E(P)=E(P,\mathbb{R}^r)$ for principal $GL(r,\mathbb{R})$-bundles $P$
and the standard representation $\mathbb{R}^r$.
Note that the projection $E(P;V)\to X$ is well-defined as $P\to X$ is
$G$-invariant. To see that this defines a vector bundle, we need to find the
trivialisations. Consider $x\in X$. Take an open $U\subseteq X$ such that
$P_U\cong U\times G$. Thus
\begin{align*}
  E(P;V)_U
  = P_U\times_G V
  \cong (U\times G)\times_G V
\end{align*}
Noting that $G$ acts on $U\times G$ by $(u,g)h =(u,gh)$ we thus have
a diffeomorphism
\begin{align*}
  E(P;V)_U \cong U \times(G\times_G V) \cong U\times V
\end{align*}
where it is straightforward to check that the restrictions
to the fibres behave as expected.

\begin{definition}
  Let $P\to X$ and $Q\to Y$ be principal $G$-bundles, $V$ a representation,
  and consider a bundle map $(\phi,f):P\to Q$. The \emph{associated
  bundle map} $E(\phi,f):E(P;V)\to E(Q;V)$ consists of
  \begin{enumerate}
    \item the smooth map $E(\phi;V):E(P;V)\to E(Q;V)$ given by
      $[p,v]\mapsto [\phi(p),v]$,
    \item the smooth map $E(f;V)=f:X\to Y$.
  \end{enumerate}
\end{definition}

We verify that this is well-defined. Let $[p,u]=[q,v]\in E(P;V)$.
Then there is a $g\in G$ such that $p=gq$ and $u={g}^{-1}v$.
By $G$-equivariance of $\phi$, we have $\phi(p)=\phi(q)g$ and
hence $[\phi(p),u]=[\phi(q),v]$ in $E(Q;V)$. This defines a bundle map
as one may easily calculate:
\begin{align*}
  \pi(E(\phi;V)\left[{p,u}\right])
  =\pi_E\left[{\phi(p),u}\right]
  =\pi_P(\phi(p))
  = f(\pi_P(p))
  = E(f;V)(\pi_P(p))
  = E(f;V)(\pi_E\left[p,u\right]).
\end{align*}

\section{Frame bundles}

It is also possible to move in the other direction, that is to define
a principal bundle using a vector bundle. To do this,
one defines the frame bundle. Consider a rank $r$ vector bundle $E\to X$.
A frame of $E$ at $x$ is an ordered basis of $E_x$. Thus a frame is
a linear isomorphism $\mathbb{R}^r\to E_x$ where $x\in X$.
Write $F_x E$ for the set of all frames on at $x$. Note that
$GL(r,\mathbb{R})$ naturally acts on $F_x E$ by $eg = e\circ g$.
Thus we have a principal bundle structure:

\begin{definition}
  Let $E\to X$ be a rank $r$ vector bundle. The \emph{frame bundle} consists of
  \begin{enumerate}
    \item the manifold $P(E)=\bigsqcup_{x\in X} F_x E$ of all frames on
      $E$,
    \item the action $(x,e)g=(x,e\circ g)$ of $g\in GL(r,\mathbb{R})$,
    \item the projection $(x,e)\mapsto x$.
  \end{enumerate}
\end{definition}

There is an obvious manifold structure on $P(E)$ that makes $P(E)\to X$ into a
principal $GL(r,\mathbb{R})$-bundle. To obtain this, consider a trivialisation
$E_U \cong U\times\mathbb{R}^r$. For each $U$ we then obtain a
trivialisations $P(E)_U \cong U\times GL(r,\mathbb{R})$ by noting
$P(E)_{U} = \bigsqcup_{x\in U} F_x E$. The charts that cover $P(E)_{U}$ now
arise by using charts on $GL(r,\mathbb{R})$ and noting that we have chosen a
region $U\subseteq X$ that trivialises $E$.

\begin{definition}
  Let $E\to X$ and $F\to Y$ be rank $r$ vector bundles and
  consider a bundle map $(\phi,f):E\to F$. The \emph{frame bundle
  map} $P(\phi,f):P(E)\to P(F)$ consists of
  \begin{enumerate}
    \item the smooth map $P(\phi):P(E)\to P(F)$ given by
      $(x,e)\mapsto (f(x),\phi\circ e)$, and
    \item the smooth map $P(f)=f:X\to Y$.
  \end{enumerate}
\end{definition}

This is easily checked to be a map of principal bundles:
\begin{align*}
  \pi(P(\phi)(x,e)) = \pi(f(x),\phi\circ e) = f(x) = f(\pi(x,e)).
\end{align*}
and
\begin{align*}
  P(\phi)((x,e)g)
  = P(\phi)(x, e\circ g)
  = (f(x),\phi\circ e\circ g)
  = (f(x),\phi\circ e)g
  = P(\phi)(x,e)g.
\end{align*}

\section{Vector bundles are $GL(r,\mathbb{R})$-principal bundles}

Consider the categories $\textbf{VB}_r$ of rank $r$ vector bundles
and $\textbf{PB}(G)$ of principal $G$-bundles, both up to isomorphism.
The associated bundle is a vector bundle corresponding to a
principal $GL(r,\mathbb{R})$-bundle. In the other direction,
we may use the standard representation
$GL(r,\mathbb{R})\to\Aut(\mathbb{R}^r)$ to construct from each
vector bundle a frame bundle, i.e. principal $GL(r,\mathbb{R})$-bundle.

Thus we have functors
\begin{equation*}
  % https://q.uiver.app/#q=WzAsMixbMCwwLCJcXHRleHRiZntQQn0oRykiXSxbMiwwLCJcXHRleHRiZntWQn1fayJdLFswLDEsIkUiLDAseyJjdXJ2ZSI6LTJ9XSxbMSwwLCJQIiwwLHsiY3VydmUiOi0yfV1d
  \begin{tikzcd}
    {\textbf{PB}(GL(r,\mathbb{R}))} && {\textbf{VB}_r}
    \arrow["E", curve={height=-12pt}, from=1-1, to=1-3]
    \arrow["P", curve={height=-12pt}, from=1-3, to=1-1]
  \end{tikzcd}
\end{equation*}

These functors turn out to be inverses:

\begin{theorem}
  For all $r\geq 0$, there is an isomorphism of categories
  \begin{align*}
    E : \textbf{PB}(GL(r,\mathbb{R})) \cong \textbf{VB}_k : P.
  \end{align*}
  \begin{proof}
    We use functor $E$ with respect to the standard representation
    of $GL(r,\mathbb{R})$ on $\mathbb{R}^r$.

    Consider a principal $GL(r,\mathbb{R})$-bundle $P\to X$. Then
    \begin{align*}
      P(E(P)) = P\left({P\times_{GL(r,\mathbb{R})}\mathbb{R}^r}\right)
      = \bigsqcup_{x\in X} F_x \left({P\times_{GL(r,\mathbb{R})}\mathbb{R}^r}\right).
    \end{align*}
    Now a frame of $E(P)$ at $x$ is a linear isomorphism
    $\mathbb{R}^r \to E(P)_x$, i.e. a linear isomorphism
    \begin{align}\label{eq:frame_linear_iso}
      \mathbb{R}^r\to P_x\times_{GL(r,\mathbb{R})}\mathbb{R}^r
      \cong \left(\left\lbrace{x}\right\rbrace\times GL(r,\mathbb{R})\right)\times_{GL(r,\mathbb{R})}\mathbb{R}^r \cong \mathbb{R}^r.
    \end{align}
    That is, an element of $P(E(P))_x$ is a linear isomorphism
    $\mathbb{R}^r\to\mathbb{R}^r$, i.e. an element of $GL(r,\mathbb{R})$.
    Thus we have a bijection of fibres $\phi_x : P(E(P))_x \to P_x$ which,
    through the local trivialisations, extends to a Lie group diffeomorphsim
    $GL(r,\mathbb{R})\to GL(r,\mathbb{R})$.

    The smoothness follows from the smoothness of the action
    of $GL(r,\mathbb{R})$ on $\mathbb{R}^r$. It remains to show that
    the induced map $X\to\Diff(GL(r,\mathbb{R}))$ is smooth.
    However, the diffeomorphisms $\phi_x$ may be described as
    the trivialisations $P_x\cong\left\lbrace{x}\right\rbrace\times
    GL(r,\mathbb{R})$ composed with various smooth maps. These trivialisations
    vary smoothly in $x$ by definition of a principal bundle and
    hence we have an isomorphism of principal $GL(r,\mathbb{R})$-bundles
    $P\to P(E(P))$.

    Conversely, consider a rank $r$ vector bundle $E\to X$. Then
    \begin{align*}
      E(P(E)) = \bigsqcup_{x\in X} F_x E \times_{GL(r,\mathbb{R})}\mathbb{R}^r.
    \end{align*}
    Thus
    \begin{align*}
      E(P(E))_x = F_x E\times_{GL(r,\mathbb{R})}\mathbb{R}^r.
    \end{align*}
    Noting that an element $e\in F_x E$ is a linear isomorphism
    $\mathbb{R}^r\to E_x$ and that the action of $g\in GL(r,\mathbb{R})$
    on $F_x E$ is given by $e\mapsto e\circ g$,
    we have a well-defined isomorphism $E(P(E))_x\to E_x$ given by
    $[e,u]\mapsto e(u)$ whose inverse may be constructed by fixing
    a non-zero $u^*\in\mathbb{R}^r$ and choosing,
    for each $v\in E_x$, a frame $e\in F_x E$ that sends
    $u^*\mapsto v$. All such $e$ are then related by a linear isomorphism
    with eigenvector $u^*$ and hence represent the same class in
    $E(P(E))_x$.  Now we need to verify that this extends to
    a diffeomorphism $E(P(E))\to E$. This can be done by considering
    trivialisations on $U\subseteq E(P(E))$ and $V\subseteq E$ respectively and
    concluding that $E(P(E))_{U\cap V}\to E_{U\cap V}$ is a diffeomorphism.
    Finally, this is a bundle map by construction so $E(P(E))\cong E$
    as vector bundles.

    One may similarly show that $E\circ P$ and $P\circ E$ fix equivalence
    classes of bundle maps and hence prove the claim.
  \end{proof}
\end{theorem}

\section{Tangent frame bundles}

While taking the standard representation $GL(r,\mathbb{R})$
of $\mathbb{R}^r$ leads to $E=E(P(E))$, the same is not true in general.
It is thus interesting to take the standard vector bundle on a manifold,
the tangent bundle, and study where it gets sent by the functor $E(P(-); V)$.
This leads to the observation that all tensor bundles
are obtained in this way. \cite{joyce2019}

\begin{theorem}
  Consider the representation of $GL(r,\mathbb{R})$ on
  $\tensor{}{k}{\mathbb{R}}^r\otimes\tensor{}{\ell}{(\mathbb{R}^r)^*}$,
  and let $X$ be a manifold. Then
  \begin{align*}
    E(P(TX);\tensor{}{k}{\mathbb{R}}^r\otimes\tensor{}{\ell}{(\mathbb{R}^r)^*}) = \tensor{}{k}{TX}\otimes\tensor{}{\ell}{T^*X}.
  \end{align*}
  \begin{proof}
    Fix $k,\ell\geq 0$. Write
    \begin{align*}
      \tensor{\ell}{k}{\mathbb{R}^r} = \tensor{}{k}{\mathbb{R}^r}\otimes\tensor{}{\ell}\left({\mathbb{R}^r}\right)^*
    \end{align*}
    and similarly for bundles. Further, write $E=E(-;\tensor{\ell}{k}\mathbb{R}^r)$. Then
    \begin{align*}
      E(P(TX)) = P(TX) \times_{GL(r,\mathbb{R})} \tensor{\ell}{k}{\mathbb{R}^r}.
    \end{align*}
    In particular, we have the trivialisations
    \begin{align*}
      E(P(TX))_U &=  P(TX)_U \times_{GL(r,\mathbb{R})} \tensor{\ell}{k}{\mathbb{R}^r}\\
                 &\cong U\times GL(r,\mathbb{R})\times_{GL(r,\mathbb{R})}\tensor{\ell}{k}{\mathbb{R}^r} \\
                 &\cong U\times \tensor{\ell}{k}{\mathbb{R}^r}.
    \end{align*}
    It is now routine to verify that the manifold structure on
    $E(P(TX))$ agrees with that on $\tensor{\ell}{k} TX$, e.g.
    by comparing charts.
  \end{proof}
\end{theorem}

\section{$G$-structures}

In addition to giving rise to tensor bundles, the tangent frame bundle
allows to endow manifolds with further geometric structures.

\begin{definition}
  Let $X$ be an $n$-manifold and $G$ be a Lie subgroup of $GL(n,\mathbb{R})$.
  A $G$-structure on $X$ is a subbundle of the tangent frame bundle
  $P(TX)$. That is, there is a submanifold $P\subseteq P(TX)$ such that
  $\res{\pi}{P}:P\to X$ defines a principal $G$-bundle and $GP=P$.
\end{definition}

One particular structure on manifolds are Riemannian metrics. Recall the
definition:

\begin{definition}
  A \emph{Riemannian metric} on a manifold $X$ is a section
  $g\in\Gamma^\infty(T^*X\otimes T^*X)$ which is symmetric and positive
  definite, i.e., for $x\in X$ and $u,v\in T^*_x X$,
  $g_x(u,v)=g_x(v,u)$ and $g_x(u,v)\geq 0$ with $g_x(u,v)=0$
  if, and only if, $u=v$.
\end{definition}

It turns out that Riemannian metrics are a special type of $G$-structure.

\begin{theorem}\label{thm:metrics_are_structures}
  Let $X$ be a manifold. There is a 1-1 correspondence between Riemannian
  metrics and $O(n)$-structures on $X$.
  \begin{proof}
    We follow \cite{cap2021}. Riemannian metrics induce $O(n)$-structures by
    \begin{align*}
      P(g) = \left\lbrace{(x,e) \in P(TX) : \text{$e$ is orthonormal w.r.t $g_x$}}\right\rbrace.
    \end{align*}
    A frame $e:\mathbb{R}^n\to T_x X$ is orthonormal if $g_x(e(v_i),e(v_j))=\delta_{ij}$ where $v_1,\ldots,v_n$ is the standard basis of $\mathbb{R}^n$.
    By choosing an isomorphism $T_x X \cong \mathbb{R}^n$, each $g_x$ defines
    a symmetric, positive definite $A\in GL(n,\mathbb{R})$ such that $g_x(u,v)=u^\top A v$. In particular, there is a $B\in GL(n,\mathbb{R})$ such that
    $A=B^\top B$, i.e. $g_x(u,v)=\langle Bu,Bv\rangle$.
    Similarly, a frame $e:\mathbb{R}^n\to T_x X$ defines
    a matrix $E\in GL(n,\mathbb{R})$ by $e(u) = Eu$. Now $e$ is orthonormal if
    \begin{align*}
      g_x(e(v_i),e(v_j)) = \langle Be(v_i), Be(v_j)\rangle = \langle v_i, v_j \rangle,
    \end{align*}
    i.e. $I=E^\top B^\top B E$. Thus $BE\in O(n)$. Thus
    we have a Lie group isomorphism $P(g)_x \cong\left\lbrace{x}\right\rbrace\times O(n)$. This
    extends via trivialisations $P(TX)_U \cong U\times GL(n,\mathbb{R})$
    to a diffeomorphisms $P(g)_U \cong U\times O(n)$. Thus one sees that
    $P(g)$ is indeed a submanifold of $P(TX)$ as well as a principal
    $O(n)$-bundle. It remains to check that $O(n)P(g)=P(g)$. This is
    obvious as, for all $C\in O(n)$ and $E\in P(g)$,
    $C^\top E^\top B^\top BEC = C^\top C = I$ and hence $CE\in P(g)$.

    Now suppose we have an $O(n)$-structure $P$. Then, for all $x\in X$,
    $P_x$ consists of frames $e:\mathbb{R}^r\to T_x X$. We may choose
    one such frame $e\in P_x$ to define
    \begin{align*}
      g^P_x(u,v) = \langle {e}^{-1}(u),{e}^{-1}(v)\rangle.
    \end{align*}
    Note that $P_x \cong O(n)$ and hence for all other $e'=e\circ A$ for
    some $A\in O(n)$. Thus the $g^P_x$ does not depend on the choice of
    $e$. Now observe that we have diffeomorphisms $P_U \cong U\times O(n)$
    which allow us to define a smooth section $U\to T^*X\otimes T^*X$ by
    $x\to g^P_x$. By covering $P$ with $\left\lbrace{U_i}\right\rbrace$
    that have isomorphisms $P_{U_i}\cong O(n)\times U_i$ and a subordinate
    partition of unity $\left\lbrace{\eta_i}\right\rbrace$ we may define
    a smooth section $X\to T^*X \otimes T^*X$ by $x\mapsto \sum_i \eta_i g^P_x = g^P_x$.
    We have thus constructed a smooth section $X\to T^*X\otimes T^*X$ whose
    values are symmetric and positive definite, i.e. a Riemannian metric.

    We observe
    \begin{align*}
      g^{P(g)}_x(u,v) = \langle {e}^{-1}(u),{e}^{-1}(v)\rangle
      = g_x(e\circ{e}^{-1}(u), e\circ{e}^{-1}(v))
    \end{align*}
    so $g^{P(g)}=g$ for all Riemannian manifolds $(X,g)$.

    The inverse holds, too. Consider $e\in P(g^P)_x$. By construction,
    \begin{align*}
      g^P_x(e(v_i),e(v_j))=\langle v_i, v_j\rangle
    \end{align*}
    and $g^P_x(u,v)=\langle{f}^{-1}(u),{f}^{-1}(v)\rangle$,
    for all $f\in P_x$. Combining these we find
    \begin{align*}
      \langle {f}^{-1}\circ e(v_i), {f}^{-1}\circ e(v_j)\rangle
      = \langle v_i, v_j\rangle.
    \end{align*}
    Thus ${f}^{-1}\circ e\in O(n)$ which implies $e\in P_x$.

    Conversely, if $e\in P_x$ then
    \begin{align*}
      g_x^P(e(v_i),e(v_j)) = \langle {e}^{-1}\circ e(v_i),{e}^{-1}\circ e(v_j)\rangle = \langle v_i,v_j\rangle
    \end{align*}
    so $e\in P(g^P)_x$. Now both $P$ and $P(g^P)$ are subbundles
    of $P(TX)$ with equal fibres, proving $P=P(g^P)$ and hence the claim.
  \end{proof}
\end{theorem}

One may similarly show that $GL_+(n,\mathbb{R})$-structures correspond
to orientations and hence $SO(n)=GL_+(n,\mathbb{R})\cap O(n)$ structures
correspond to combinations of Riemannian metrics and orientations:

\begin{theorem}
  Let $X$ be a manifold. There is a one-to-one correspondence between
  orientations on $X$ and $GL_+(n,\mathbb{R})$-structures on $X$
  where $GL_+(n,\mathbb{R})$ is the group of positive determinant
  matrices.
  \begin{proof}[Sketch of proof.]
    Analogous to the above by taking a volume form $\omega\in\Omega^n(X)$
    to induce a $GL_+(r,\mathbb{R})$-structure
    \begin{align*}
      P(\omega) = \left\lbrace{(x,e) : \omega(e(v_1),\ldots,e(v_n))>0}\right\rbrace
    \end{align*}
    and a $GL_+(r,\mathbb{R})$-structure to induce a volume form
    by choosing a frame for each trivialisation, constructing a
    top degree form locally, and then combining them using partitions
    of unity.
  \end{proof}
\end{theorem}

\begin{corollary}
  Let $X$ be a manifold. There is a one-to-one correspondence between
  $SO(n)$-structures on $X$ and pairs of orientations and Riemannian
  metrics.
\end{corollary}

\section{Levi-Civita connection}

We closely follow \cite{lee2018} in this section.
Connections may be thought of as a way of taking directional derivatives
of vector fields.

\begin{definition}
  Let $E\to X$ be a vector bundle. A \emph{connection in $E$} is a map
  \begin{align*}
    \nabla : \Gamma(TX) \times \Gamma(E) &\to \Gamma(E) \\
    (u,s) &\mapsto \nabla_u s
  \end{align*}
  such that
  \begin{enumerate}
    \item $\nabla_u s$ is $C^\infty(X)$-linear in $u$,
    \item $\nabla_u s$ is $\mathbb{R}$-linear in $s$, and
    \item for $f\in C^\infty(X)$,
      \begin{align*}
        \nabla_u (fs) = f\nabla_u s + u(f)s.
      \end{align*}
  \end{enumerate}
\end{definition}

As the canonical non-trivial vector field on a manifold,
we are going to be particularly interested in connections on the
tangent bundle $TX$. Connections of the tangent bundle extend to
general tensor bundles.

The tangent bundle is even more relevant for
Riemannian manifolds as the metric defines a sort of inner product
on vector fields, i.e. sections of the tangent bundle.
It is thus reasonable to require that the connection, thought of
as differentiation, respects the metric, thought of as the product,
by satisfying the product rule:

\begin{definition}
  Let $(X,g)$ be a Riemannian manifold. A connection $\nabla$ in $TX$ is
  \emph{metric} if, and only if, for all $u,v,w\in\Gamma(TX)$,
  \begin{align*}
    \nabla_u g(v,w) = g(\nabla_u v, w) + g(v,\nabla_u w)
  \end{align*}
  where we write $\nabla_u g(v,w) = u g(v,w)$
\end{definition}

The notation $\nabla_u g(v,w)$ is motivated by the fact that any
connection in $TX$ uniquely extends to connections in any tensor bundle
$\tensor{}{k}{TX}\otimes\tensor{}{\ell}{T^*X}$. In particular, $g(v,w)$
is a smooth function $X\to\mathbb{R}$, i.e. a tensor with $k=\ell=0$.

Our goal is to obtain a canonical connection on a Riemannian manifold.
We will use this to define the holonomy group in the next chapter.
However, manifolds admit many metric connections. Therefore we require
a stronger condition:

\begin{definition}
  Let $X$ be a manifold. A connection $\nabla$ in $TX$ is
  \emph{symmetric} if, for all $u,v\in\Gamma(TX)$,
  \begin{align*}
    \nabla_u v - \nabla_v u = [u,v].
  \end{align*}
\end{definition}

Here $[u,v]$ is the Lie-bracket or the Lie-derivative $\mathcal L_u v$
which is given in coordinates by
\begin{align*}
  [u,v] = u(v_i)\frac{\partial }{\partial y_i} - v(u_i)\frac{\partial}{\partial x_i}
\end{align*}
where we have written $v=\sum_i u_i\frac{\partial}{\partial x_i}$
and $v=\sum_i v_i\frac{\partial}{\partial y_i}$.
This condition turns out to be the correct one to obtain a canonical connection in the tangent bundle of a Riemannian manifold:

\begin{theorem}[Fundamental Theorem of Riemannian Geometry]
  Let $(X,g)$ be a Riemannian manifold. Then there exists a unique
  symmetric metric connection in $TX$.
\end{theorem}

We call this connection the \emph{Levi-Civita connection}.

\section{Holonomy group of a connection}

It is common in geometry and toplogy to associate algebraic structures
to geometric objects. We are now going to define the holonomy group
of a connection. This is a standard construction, see e.g. \cite{joyce2007}.

Given a smooth curve $\gamma:[0,1]\to X$, we may pull back vector bundles
on $E\to X$ to vector bundles $\gamma^*E\to [0,1]$ by precomposition.
Note that $[0,1]$ has boundry, but the definitions of vector bundles
extend to this case. Alternatively, one may consider the smooth extension
of $\gamma$ to $(-\varepsilon,1+\varepsilon)$ for some $\varepsilon>0$.
Similarly, a connection $\nabla$ in $E$ pulls back to a connection
$\gamma^*\nabla$ in $\gamma^* E$ by
\begin{align*}
  (\gamma^*\nabla)_v s = \nabla_v (s\circ {\gamma}^{-1})\circ\gamma
\end{align*}
where $v:[0,1]\to T[0,1]$ is a vector field and $s:[0,1]\to E$
is a section of $\gamma^* E$. Note that $(\gamma^*\nabla)_v$ depends
only on the values of $v$ along $\gamma$. This follows from the
$C^\infty(X)$-linearity of connections by using bump functions around
$\gamma$. Hence it makes sense to write $(\gamma^*\nabla)_v s$
even when $v$ is only defined for $x\in\gamma[0,1]$.

We use this pull-back to define the core ingredient of the holonomy group:

\begin{definition}
  Let $E\to X$ be a vector bundle, $\nabla$ a connection in $E$,
  and $\gamma:I\to X$ a smooth curve. Then,
  for each $e\in E_{\gamma(0)}$, there is a unique section $s_e\in\Gamma(\gamma^* E)$
  such that $s_e(0)=e$ and, for all $t\in I$, $\nabla_{\dot\gamma} s(t) = 0$.
  The \emph{parallel transport} along $\gamma$ is the linear map
  \begin{align*}
    P_\gamma : E_{\gamma(0)} &\to E_{\gamma(1)} \\
    e &\mapsto s_e(1)
  \end{align*}
\end{definition}

Here $\dot\gamma$ is the tangent vector field of $\gamma$ along the curve.
In local coordinates, this is
\begin{align*}
  \dot\gamma = \sum_i \frac{\partial x_i}{\partial t} \frac{\partial}{\partial x_i}
\end{align*}
where $\gamma(t) = (x_1(t),\ldots,x_n(t))$.

We call a curve $\gamma : I\to X$ a loop at $x\in X$ if $\gamma(0)=\gamma(1)=x$.

\begin{definition}
  Let $E\to X$ be a vector bundle, $\nabla$ a connection in $E$,
  and $x\in X$. Then the \emph{holonomy group of $\nabla$ at $x$}
  is
  \begin{align*}
    \Hol_x(\nabla) = \left\lbrace{P_\gamma : \text{$\gamma$ is a loop at $x$}}\right\rbrace.
  \end{align*}
\end{definition}

The group sturcture on $\Hol_x(\nabla)$ is inherited from the group $GL(E_x)$ of linear automorphisms on $E_x$. To see that this is well-defined, note
that, for loops $\gamma,\delta : I \to X$ at $x$, we have
\begin{align*}
  P_\gamma\circ P_\delta = P_{\gamma\delta}, \hspace{1cm}
  {P_\gamma}^{-1} = P_{{\gamma}^{-1}}
\end{align*}
where $\gamma\delta$ is the usual concatenation of loops and ${\gamma}^{-1}$
is the reverse of the loop. Moreover, $\Hol_x(\nabla)$ inherits the
manifold structure of $GL(E_x)$ and is thus a Lie subgroup.

As an abstract group, the holonomy group of a connection is independent
of the base point:

\begin{lemma}
  Let $\nabla$ be a connection in a vector bundle $E\to X$, let $x,y\in X$,
  and let $\gamma$ be a path from $x$ to $y$. Then
  \begin{align*}
    \Hol_y(\nabla)=P_\gamma \Hol_x(\nabla){P_\gamma}^{-1}.
  \end{align*}
  \begin{proof}
    This follows because, if $\gamma$ is a path from $x$ to $y$ and
    $\delta$ is a path from $y$ to $z$ then $P_\delta \circ P_\gamma = P_{\delta\gamma}$ where $\delta\gamma$ is the usual concatenation path from
    $x$ to $z$. Thus, for all $P_\delta\in\Hol_x(\nabla)$ and $\gamma$
    as above, $\gamma\delta{\gamma}^{-1}$ is a loop at $y$ and hence
    defines a $P_{\gamma\delta{\gamma}^{-1}}\in\Hol_y(\nabla)$. The
    same argument then works in the reverse direction.
  \end{proof}
\end{lemma}

Thus, if $E$ has rank $k$ then one may regard $\Hol_x(\nabla)$ as
a Lie subgroup of $GL(k,\mathbb{R})$ up to conjugation.

\section{Holonomy groups of Riemannian metrics}

Every Riemannian manifold $(X,g)$ has a canonical connection
$\nabla$ in $TX$, the Levi-Citiva connection. Thus a metric $g$
defines a holonomy group $\Hol_x(g) = \Hol_x(\nabla)\subseteq GL(T_x X)$.
If one is to study the geometry of Riemannian manifolds, it is natural
to study Riemannian holonomy groups. See e.g. \cite{joyce2007} for
details.

We begin by observing that Riemannian holonomy groups depend only on the
metric, not on the base point.
A section $s\in\Gamma(E)$ is constant if $\nabla s = 0$, i.e. $\nabla_v s=0$, for all $v\in\Gamma(TX)$. We mentioned earlier that connections in
$TX$ extend to all tensor bundles. This defines the notion of a constant
tensor. Note that, by definition of the Levi-Civita connection,
$\nabla g = 0$ so $g$ is a constant tensor.
Next, observe that $\Hol_x(g)\subseteq GL(T_x X)$ acts naturally
on $T_x X$ and thus on all tensors. It is a property of this action that
constant tensors are fixed by this action. In particular,
$\Hol_x(g)$ preserves $g_x$. We showed in \ref{thm:metrics_are_structures} that the subgroup of $GL(T_x X)$ that preserves $g_x$ is exactly
$O(n)$, making $\Hol_x(g)$ a subgroup of $O(n)$. This is indepedent of
$x$ up to conjugation, hence we may simply write $\Hol(g)$.

We are now able to fully classify Riemannian holonomy groups.
Firstly, observe that, if $(X,g)$ and $(Y,h)$ are Riemannian
manifolds then $(g\times h)_{x,y} = g_x + h_y$ defines a Riemannian
metric on $X\times Y$. Hence we may define
the product of Riemannian manifolds $(X\times Y,g\times h)$.
The holonomy groups then satisfy $\Hol_{x,y}(g\times
h)=\Hol_x(g)\times\Hol_y(h)$. Thus the representation of
$\Hol_{x,y}(g\times h)$ on $T_{x,y}(X\times Y)$ is reducible.

The converse is not immediately true: If the representation of
$Hol_x(g)$ is reducible it is in general not true that the metric
may be written as $g=g_1\times g_2$. Thus we require the notion of
local reduicibility:

\begin{definition}
  A Riemannian manifold $(X,g)$ is \emph{locally reducible} if, for
  all $x\in X$, there is a neighbourhood $U\ni x$, Riemannian manifolds
  $(V,g^V)$, $(W,g^W)$, and a diffeomorphism $U\cong V\times W$ such that
  $\res{g}{U}$ is isometric to $g_X\times g_Y$.
  A metric that is not locally reducible is called \emph{irreducible}.
\end{definition}

This is the correct notion of irreducibility as the representations
of the holonomy groups of an irreducible metric are guaranteed to
be irreducible. So, to classify Riemannian holonomy groups, it suffices
to focus our attention on irreducible Riemannian manifolds.

We then distinguish two types of Riemannian manifolds: symmetric
and non-symmetric.

\begin{definition}
  A Riemannian manifold $(X,g)$ is symmetric if, for every point $x\in X$,
  there is an isometry $s_x : X\to X$ with isolated fixed point $x$ which is
  an involution, i.e. $s_x^2 = \identity$.
\end{definition}

Here isolated means that there exists an open neighbourhood $U\ni x$
such that $x$ is the only fixed point of $\res{s_x}{U}$.
The full list of holonomy groups of irreducible symmetric Riemannian
manifolds may be found in \cite{besse1987}.

The non-symmetric case is more compact:

\begin{theorem}[Berger]
  If $(X,g)$ is a simply-connected, non-symmetric, irreducible Riemannian
  $n$-manifold then $\Hol_x(g)$ is one of
  \begin{align*}
    SO(n), \hspace{1cm}
    U(n/2), \hspace{1cm}
    SU(n/2), \hspace{1cm}
    Sp(n/4), \hspace{1cm}
    Sp(n/4)Sp(1), \hspace{1cm}
    G_2, \hspace{1cm}
    \text{Spin}(7).
  \end{align*}
\end{theorem}

\section{Constructing manifolds with specific holonomy groups}

Berger's theorem tells us exactly which groups arise as holonomy groups of
Riemannian manifolds. However, it is not immediately obvious and in fact
non-trivial to construct Riemannian manifolds that yield each of the listed
holonomy groups. Particularly interesting are the two exceptional
holonomy groups $G_2$ and $\text{Spin}(7)$.

Fix $G$ to be one of the groups on Berger's list. Then one way
construct a Riemannian manifold $(X,G)$ with $\Hol_x(g)=G$ by
first constructing a $G$-structure. If this structure is not torsion
free, then it may be possible to deform it to a torsion free $G$-structure
with the same holonomy group. One may then need to check
that the structure is topologically well-behaved to obtain a $G$-structure.
For example, in the case $G=G_2$, it is required that $\pi_1(X)$ is
finite to find that the $G_2$-structure does indeed yield a manifold
with holonomy $G_2$. For full examples for the cases $G=G_2$
and $G=\text{Spin}(7)$ see \cite{joyce2007}.

\printbibliography{}

\end{document}
