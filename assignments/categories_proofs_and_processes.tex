\documentclass{article}
\usepackage{assignment}
\begin{document}
\title{Bicategories and Formal Monads}
\author{Franz Miltz}
\date{\today}
\maketitle

In \cite{pawel2017} three mathematical equivalences are described.
We aim to formalise this. Before we can start in earnest, we need
to establish some mathematical background. In particular, we aim
to familiarise ourselves with monoidal categories and bicategories.
We then encounter a non-obvious example of a bicategory which
we will require later on.

\section{Bicategories and monoidal categories}

We begin this report by revisiting the notion of a monoidal category.
Recall that a monoidal structure equips a category with a multiplication
which is associative and unital, up to a canonical isomorphism. Here
canonical means that between any two objects there can be at most one
isomorphism arising from the monoidal structure. This is made precise
by specifying a few elementary isomorphisms which correspond to
unitality and associativity and requiring that they satisfy some coherence axioms.

We then extend these ideas to bicategories, a generalisation of categories that introduces maps between maps. We will study bicategories with
some care, introducing the span bicategory which will come in handy later
on. Towards the end of this section we see that monoidal categories
may indeed be thought of as single object bicategories.

\subsection{Monoidal categories}

Before we state the definition, we remind ourselves of some useful notation which we will utilise heavily throughout this report. Let $\mathcal C$
and $\mathcal D$ be categories. Recall that we
then have a product category $\mathcal C\times\mathcal D$ whose objects
are pairs $(C,D)$ for $C\in\mathcal C$ and $D\in\mathcal D$ and whose
maps are pairs $(f,g):(C,D)\to (C',D')$ where $f:C\to C'$ in $\mathcal C$
and $g:D\to D'$ in $\mathcal D$. If we now have functors
$F:\mathcal C\to\mathcal C'$ and $G:\mathcal D\to\mathcal D'$ there is
a functor $F\times G:\mathcal C\times\mathcal D\to\mathcal C'\times\mathcal D'$ which is given by $(x,y)\mapsto (Fx,Gy)$ for objects and maps alike.
In particular, for any functor $F : \mathcal C\to \mathcal C'$ we have
a functor $\mathcal C\times \mathcal D\to\mathcal C'\times\mathcal D$
which is the product of $F$ with the identity on $\mathcal D$.
We will denote this by $F\times\mathcal D$.

\begin{definition}\label{def:monoidal_category}
  A \emph{monoidal category} consists of a category $\mathcal C$,
  an object $I\in\mathcal C$, a functor
  $\otimes:\mathcal C\times\mathcal C\to\mathcal C$, and
  natural isomorphisms with components
  \begin{align*}
    \alpha_{A,B,C} : (A\otimes B)\otimes C \to A\otimes(B\otimes C),
    \hspace{1cm}
    \lambda_A : I\otimes A\to A,
    \hspace{1cm}
    \rho_A : A\otimes I\to A
  \end{align*}
  such that
  \begin{equation}\label{eq:monoidal_pentagon}
    % https://q.uiver.app/#q=WzAsNSxbMCwwLCIoKEFcXG90aW1lcyBCKVxcb3RpbWVzIEMpXFxvdGltZXMgRCJdLFswLDIsIihBXFxvdGltZXMoQlxcb3RpbWVzIEMpKVxcb3RpbWVzIEQiXSxbNCwyLCJBXFxvdGltZXMoKEJcXG90aW1lcyBDKVxcb3RpbWVzIEQpIl0sWzQsMCwiQVxcb3RpbWVzKEJcXG90aW1lcyhDXFxvdGltZXMgRCkpIl0sWzIsMCwiKEFcXG90aW1lcyBCKVxcb3RpbWVzKENcXG90aW1lcyBEKSJdLFswLDEsIlxcYWxwaGFfe0EsQixDfVxcb3RpbWVzIEQiLDJdLFsxLDIsIlxcYWxwaGFfe0EsQlxcb3RpbWVzIEMsRH0iLDJdLFsyLDMsIkFcXG90aW1lcyBcXGFscGhhX3tCLEMsRH0iLDJdLFswLDQsIlxcYWxwaGFfe0FcXG90aW1lcyBCLEMsRH0iXSxbNCwzLCJcXGFscGhhX3tBLEIsQ1xcb3RpbWVzIER9Il1d
    \begin{tikzcd}
      {((A\otimes B)\otimes C)\otimes D} && {(A\otimes B)\otimes(C\otimes D)} && {A\otimes(B\otimes(C\otimes D))} \\
      \\
      {(A\otimes(B\otimes C))\otimes D} &&&& {A\otimes((B\otimes C)\otimes D)}
      \arrow["{\alpha_{A,B,C}\otimes D}"', from=1-1, to=3-1]
      \arrow["{\alpha_{A,B\otimes C,D}}"', from=3-1, to=3-5]
      \arrow["{A\otimes \alpha_{B,C,D}}"', from=3-5, to=1-5]
      \arrow["{\alpha_{A\otimes B,C,D}}", from=1-1, to=1-3]
      \arrow["{\alpha_{A,B,C\otimes D}}", from=1-3, to=1-5]
    \end{tikzcd}
  \end{equation}
  and
  \begin{equation}\label{eq:monoidal_triangle}
    % https://q.uiver.app/#q=WzAsMyxbMCwwLCIoQVxcb3RpbWVzIEkpXFxvdGltZXMgQiJdLFsxLDEsIkFcXG90aW1lcyBCIl0sWzIsMCwiQVxcb3RpbWVzKElcXG90aW1lcyBCKSJdLFswLDEsIlxccmhvX3tBfVxcb3RpbWVzIEIiLDJdLFsyLDEsIkFcXG90aW1lc1xcbGFtYmRhX0IiXSxbMCwyLCJcXGFscGhhX3tBLEksQn0iXV0=
    \begin{tikzcd}
      {(A\otimes I)\otimes B} && {A\otimes(I\otimes B)} \\
                              & {A\otimes B}
                              \arrow["{\rho_{A}\otimes B}"', from=1-1, to=2-2]
                              \arrow["{A\otimes\lambda_B}", from=1-3, to=2-2]
                              \arrow["{\alpha_{A,I,B}}", from=1-1, to=1-3]
    \end{tikzcd}
  \end{equation}
  commute.
\end{definition}

The most common examples of monoidal categories arise from
cartesian and cocartesian structures.

%%%%%%%%%%%%%%%%%%%%%%%%%%%%%%%%%
\subsection{Bicategories}
%%%%%%%%%%%%%%%%%%%%%%%%%%%%%%%%%

Category theory may be used to generalise many mathematical
concepts. Categories themselves are structures that have been
observed to arise very naturally in many areas of mathematics.
In this section we are going to define the notion of a bicategory.
This is a fairly involved task which we will try and motivate
in two distinct ways.

Firstly, its has long been observed that, rather than studying mathematical structures
by themselves, it is often fruitful to think about the structure
preserving maps between them. This observation motivates the very
existence of category theory. Taking this philosophy one step further
one might wonder whether, rather than studying structure preserving
maps between structures, one ought to study structure preserving
maps between structure preserving maps between structures. This is
in fact an approach that category theorists have adopted: Natural
transformations are structure preserving maps between functors which
themselves are structure preserving maps between categories. Yet,
the usual category of categories does not contain any natural
transformations. The bicategory of categories, functors, and natural
transformations addresses the problem.

Secondly, many genearlisations arise by regarding a certain mathematical
structure as a single-object category and extending the definition
to categories with more than one object. For example, groups are
single object groupoids, rings are single object additive categories,
monoids are single object categories (see \ref{sec:monoids_are_categories}). Observing this pattern leads to the study of enriched categories
which we will not delve into. However, we will see that bicategories
may be thought of as such a generalisation of monoidal categories.


With this knowledge fresh in our memories, we are ready to follow
the definition of a bicategory (see e.g. \cite{leinster1998}):

\begin{definition}\label{def:bicategory}
  A \emph{bicategory} $\mathbf C$ consists of
  \begin{itemize}
    \item a class of objects $\text{Obj}_{\mathbf{C}}$ where
      we will write $A\in\mathbf C$ to mean $A\in\text{Obj}_{\mathbf C}$;
    \item for all $A,B\in\mathbf C$, a 1-category
      $\Hom_{\mathbf C}[A,B]$ whose objects $f:A\to B$ are
      called 1-cells and whose morphisms are called 2-cells;
    \item for all $A\in\mathbf{C}$, a functor
      $\identity_A:\mathbf 1\to \Hom_{\mathbf C}[A,A]$ which we identify
      with its unique value $\identity_A  : A\to A$;
    \item for all $A,B,C\in\mathbf{C}$, a composition functor
      \begin{align*}
        \circ_{A,B,C} : \Hom_{\mathbf C}[B,C]\times\Hom_{\mathbf C}[A,B] \to \Hom_{\mathbf C}[A,C]
      \end{align*}
      whose value on 1-cells $f\in\Hom_{\mathbf C}[A,B]$ and
      $g\in\Hom_{\mathbf C}[B,C]$ we will denote by
      $g\circ f := \circ_{A,B,C}(g,f)$;
    \item for all $A,B,C,D\in\mathbf C$, a natural isomorphism
      with components
      \begin{align*}
        \alpha_{f,g,h} : (h\circ g) \circ f \to h\circ(g\circ f), \hspace{1cm}
        \rho_f : f \circ \identity_A \to f, \hspace{1cm}
        \lambda_f : \identity_B\circ f \to f
      \end{align*}
      at $f\in\Hom_{\mathbf C}[A,B]$, $g\in\Hom_{\mathbf C}[B,C]$, and
      $h\in\Hom_{\mathbf C}[C,D]$;
  \end{itemize}
  such that
  \begin{itemize}
    \item for suitable 1-cells $f,g,h,k$, the following commutes
      \begin{equation}\label{eq:pentagon}
        % https://q.uiver.app/#q=WzAsNSxbMCwwLCIoKGtcXGNpcmMgaClcXGNpcmMgZylcXGNpcmMgZiJdLFswLDIsIihrXFxjaXJjIGgpXFxjaXJjKGdcXGNpcmMgZikiXSxbNCwyLCJrXFxjaXJjKGhcXGNpcmMoZ1xcY2lyYyBmKSkiXSxbMiwwLCIoa1xcY2lyYyhoXFxjaXJjIGcpKVxcY2lyYyBmIl0sWzQsMCwia1xcY2lyYyAoKGhcXGNpcmMgZylcXGNpcmMgZikiXSxbMCwxLCJcXGFscGhhX3tmLGcsa1xcY2lyYyBofSIsMl0sWzEsMiwiXFxhbHBoYV97Z1xcY2lyYyBmLGgsa30iLDJdLFs0LDIsImtcXGNpcmMgXFxhbHBoYV97ZixnLGh9Il0sWzAsMywiXFxhbHBoYV97ZyxoLGt9XFxjaXJjIGYiXSxbMyw0LCJcXGFscGhhX3tmLGdcXGNpcmMgaCxrfSJdXQ==
        \begin{tikzcd}
          {((k\circ h)\circ g)\circ f} && {(k\circ(h\circ g))\circ f} && {k\circ ((h\circ g)\circ f)} \\
          \\
          {(k\circ h)\circ(g\circ f)} &&&& {k\circ(h\circ(g\circ f))}
          \arrow["{\alpha_{f,g,k\circ h}}"', from=1-1, to=3-1]
          \arrow["{\alpha_{g\circ f,h,k}}"', from=3-1, to=3-5]
          \arrow["{k\circ \alpha_{f,g,h}}", from=1-5, to=3-5]
          \arrow["{\alpha_{g,h,k}\circ f}", from=1-1, to=1-3]
          \arrow["{\alpha_{f,g\circ h,k}}", from=1-3, to=1-5]
        \end{tikzcd}
      \end{equation}
    \item for all 1-cells $A\xlongrightarrow{f}B\xlongrightarrow{g} C$
      the following commutes:
      \begin{equation}\label{eq:triangle}
        % https://q.uiver.app/#q=WzAsMyxbMCwwLCIoZ1xcY2lyYyBcXHRleHR7aWR9X0IpXFxjaXJjIGYiXSxbMiwwLCJnXFxjaXJjKFxcdGV4dHtpZH1fQlxcY2lyYyBmKSJdLFsxLDEsImdcXGNpcmMgZiJdLFswLDIsIlxccmhvX2cgXFxjaXJjIGYiLDJdLFsxLDIsImdcXGNpcmMgXFxsYW1iZGFfZiJdLFswLDEsIlxcYWxwaGFfe2YsXFx0ZXh0e2lkfV9CLGd9Il1d
        \begin{tikzcd}
          {(g\circ \text{id}_B)\circ f} && {g\circ(\text{id}_B\circ f)} \\
                                        & {g\circ f}
                                        \arrow["{\rho_g \circ f}"', from=1-1, to=2-2]
                                        \arrow["{g\circ \lambda_f}", from=1-3, to=2-2]
                                        \arrow["{\alpha_{f,\text{id}_B,g}}", from=1-1, to=1-3]
        \end{tikzcd}
      \end{equation}
  \end{itemize}
\end{definition}

Throughout this definition we have been rigorous in identifying the
precise functors and natural transformation involved. Such a level
of detail is usually unnecessary and, moreover, tends to obscure
interesting details. For this reason, we will drop subscripts whenever
it is possible to infer them from context.

Next, we ought to make clear what the natural isomorphisms
$\alpha$, $\rho$, and $\lambda$ actually are. We know that their
components are 2-cells, i.e. maps in some hom-category. But what
are the corresponding functors? Fix $A,B,C,D$, then the
corresponding $\alpha$ fits into the diagram
\begin{equation*}
  % https://q.uiver.app/#q=WzAsNCxbMCwwLCJcXHRleHR7SG9tfV97XFxtYXRoYmYgQ31bQyxEXVxcdGltZXNcXHRleHR7SG9tfV97XFxtYXRoYmYgQ31bQixDXVxcdGltZXNcXHRleHR7SG9tfV97XFxtYXRoYmYgQ31bQSxCXSJdLFswLDEsIlxcdGV4dHtIb219X3tcXG1hdGhiZiBDfVtCLERdXFx0aW1lc1xcdGV4dHtIb219X3tcXG1hdGhiZiBDfVtBLEJdIl0sWzMsMCwiXFx0ZXh0e0hvbX1fe1xcbWF0aGJmIEN9W0MsRF1cXHRpbWVzXFx0ZXh0e0hvbX1fe1xcbWF0aGJmIEN9W0EsQ10iXSxbMywxLCJcXHRleHR7SG9tfV97XFxtYXRoYmYgQ31bQSxEXSJdLFsyLDMsIlxcY2lyY197QSxDLER9Il0sWzEsMywiXFxjaXJjX3tBLEIsRH0iLDJdLFswLDEsIlxcY2lyY197QixDLER9XFx0aW1lc3tcXHRleHR7SG9tfV97XFxtYXRoYmYgQ31bQSxCXX0iLDJdLFswLDIsIntcXHRleHR7SG9tfV97XFxtYXRoYmYgQ31bQyxEXX1cXHRpbWVzXFxjaXJjX3tBLEIsQ30iXSxbNiw0LCJcXGFscGhhIiwwLHsic2hvcnRlbiI6eyJzb3VyY2UiOjIwLCJ0YXJnZXQiOjIwfX1dXQ==
  \begin{tikzcd}
    {\text{Hom}_{\mathbf C}[C,D]\times\text{Hom}_{\mathbf C}[B,C]\times\text{Hom}_{\mathbf C}[A,B]} &&& {\text{Hom}_{\mathbf C}[C,D]\times\text{Hom}_{\mathbf C}[A,C]} \\
    {\text{Hom}_{\mathbf C}[B,D]\times\text{Hom}_{\mathbf C}[A,B]} &&& {\text{Hom}_{\mathbf C}[A,D]}
    \arrow[""{name=0, anchor=center, inner sep=0}, "{\circ_{A,C,D}}", from=1-4, to=2-4]
    \arrow["{\circ_{A,B,D}}"', from=2-1, to=2-4]
    \arrow[""{name=1, anchor=center, inner sep=0}, "{\circ_{B,C,D}\times{\text{Hom}_{\mathbf C}[A,B]}}"', from=1-1, to=2-1]
    \arrow["{{\text{Hom}_{\mathbf C}[C,D]}\times\circ_{A,B,C}}", from=1-1, to=1-4]
    \arrow["\alpha", shorten <=40pt, shorten >=40pt, Rightarrow, from=1, to=0]
  \end{tikzcd}
\end{equation*}
One may similarly write diagrams for $\lambda$ and $\rho$, but
it will suffice to consider the components and their naturality
in their arguments. See \cite{leinster1998} for details. Naturality
of $\alpha$ then means that, for any suitable 2-cells, the following
commutes:
\begin{equation*}
  % https://q.uiver.app/#q=WzAsNCxbMCwwLCIoaFxcY2lyYyBnKVxcY2lyYyBmIl0sWzIsMCwiaFxcY2lyYyhnXFxjaXJjIGYpIl0sWzAsMSwiKGgnXFxjaXJjIGcnKVxcY2lyYyBmJyJdLFsyLDEsImgnXFxjaXJjKGcnXFxjaXJjIGYnKSJdLFswLDEsIlxcYWxwaGEiXSxbMiwzLCJcXGFscGhhIl0sWzAsMiwiKFxccGhpXFxjaXJjXFxwc2kpXFxjaXJjIFxcdGhldGEiLDJdLFsxLDMsIlxccGhpXFxjaXJjKFxccHNpXFxjaXJjXFx0aGV0YSkiXV0=
  \begin{tikzcd}
    {(h\circ g)\circ f} && {h\circ(g\circ f)} \\
    {(h'\circ g')\circ f'} && {h'\circ(g'\circ f')}
    \arrow["\alpha", from=1-1, to=1-3]
    \arrow["\alpha", from=2-1, to=2-3]
    \arrow["{(\phi\circ\psi)\circ \theta}"', from=1-1, to=2-1]
    \arrow["{\phi\circ(\psi\circ\theta)}", from=1-3, to=2-3]
  \end{tikzcd}
\end{equation*}
It will suffice that this holds for $\phi$, $\psi$, and $\theta$
individually, taking the other 2-cells to be identities. This is
the approach that we are going to take.

Let us now see how one might define $\textbf{CAT}$, the bicategory with
objects all categories (as always, up to some suitable notion of
universe to avoid set theoretic problems). For categories
$\mathcal C$ and
$\mathcal D$, the hom-category $\Hom[\mathbb C,\mathbb D]$ is just
the functor category $\text{Fun}(\mathbb C,\mathbb D)$ and the functor
$\identity_{\mathbb C} : \mathbf 1 \to [\mathbb C,\mathbb C]$
maps to the identity functor $\text{Id}:\mathbb C\to\mathbb C$.
We note that composition of functors is associative and respects
identities as units, hence we may take $\alpha$, $\lambda$, and $\rho$
to be the identities. Thus the conditions (\ref{eq:pentagon}) and
(\ref{eq:triangle}) are trivially satisfied.

%%%%%%%%%%%%%%%%%%%%%%%%%%%%%%%%%
\subsection{Spans}\label{sec:spans}
%%%%%%%%%%%%%%%%%%%%%%%%%%%%%%%%%

The previous example is a little bit underwhelming: All the structure
and conditions related to associativity and unitality is trivial.
This leads to the question of when this might not be the case. The answer
is straightforward: whenever limits or colimits are involved. It is
well-known that limits are not unique, but whenever one has two limits
of the same diagram then there exists a unique isomorphism. This makes
them perfect candidates for constructing bicategories whose structural
isomorphisms are not identities.
We will, in part, follow \cite{rebro} to construct such an example.

Fix a category $\mathcal C$ which we assume to have all limits, although
this is not strictly required.


\begin{definition}
  Consider $A,B\in\mathcal C$. A \emph{span of $A$ and $B$} is
  a pair of maps $A \leftarrow S \rightarrow B$ which we will
  refer to simply as $S$. A map of spans is a map $f:S\to S'$
  that makes the following commute:
  \begin{equation}
    % https://q.uiver.app/#q=WzAsNCxbMCwxLCJBIl0sWzIsMSwiQiJdLFsxLDAsIlMiXSxbMSwyLCJTJyJdLFsyLDMsImYiLDJdLFsyLDBdLFszLDBdLFszLDFdLFsyLDFdXQ==
    \begin{tikzcd}
  & S \\
      A && B \\
        & {S'}
        \arrow["f"', from=1-2, to=3-2]
        \arrow[from=1-2, to=2-1]
        \arrow[from=3-2, to=2-1]
        \arrow[from=3-2, to=2-3]
        \arrow[from=1-2, to=2-3]
    \end{tikzcd}
  \end{equation}
  Thus we have a category $\Span(A,B)$ of spans and span
  maps.
\end{definition}

The bicategory that we are about to construct will have the same
objects as $\mathcal C$ and $\Span(A,B)$ as its hom-categories.
This is the main difference between our construction and the one
presented in \cite{rebro} where the 2-cells are are also spans,
up to isomorphism. This suggests that it is possible to repeat
this indefinitely, defining $n$-cells to be spans between
$(n-1)$-cells. We will not pursue this idea futher.

What we require now is a way to compose spans. We will use pullbacks:

\begin{definition}
  Consider maps $f:A\to X$ and $g:B\to X$ in $\mathcal C$.
  A \emph{pullback} of $f$ and $g$ is a limit of the diagram
  \begin{align*}
    A \xlongrightarrow{f} X \xlongleftarrow{g} B.
  \end{align*}
\end{definition}

In detail, a pullback consists of a span $P$ of $A$ and $B$ such that
the following commutes
\begin{equation}\label{eq:pullback}
  % https://q.uiver.app/#q=WzAsNCxbMCwwLCJQIl0sWzAsMiwiQSJdLFsyLDAsIkIiXSxbMiwyLCJYIl0sWzEsMywiZiIsMl0sWzIsMywiZyIsMl0sWzAsMV0sWzAsMl0sWzAsMywiIiwxLHsic3R5bGUiOnsibmFtZSI6ImNvcm5lciJ9fV1d
  \begin{tikzcd}
    P && B \\
    \\
    A && X
    \arrow["f"', from=3-1, to=3-3]
    \arrow["g"', from=1-3, to=3-3]
    \arrow[from=1-1, to=3-1]
    \arrow[from=1-1, to=1-3]
    \arrow["\lrcorner"{anchor=center, pos=0.125}, draw=none, from=1-1, to=3-3]
  \end{tikzcd}
\end{equation}
and universal in the sense that, for any other span $P'$
satisfying (\ref{eq:pullback}) there is a unique map of spans $P'\to P$
making the following commute:
\begin{equation*}
  % https://q.uiver.app/#q=WzAsNSxbMSwxLCJQIl0sWzEsMywiQSJdLFszLDEsIkIiXSxbMywzLCJYIl0sWzAsMCwiUCciXSxbMSwzLCJmIiwyXSxbMiwzLCJnIiwyXSxbMCwxXSxbMCwyXSxbMCwzLCIiLDEseyJzdHlsZSI6eyJuYW1lIjoiY29ybmVyIn19XSxbNCwxXSxbNCwyXSxbNCwwLCIiLDIseyJzdHlsZSI6eyJib2R5Ijp7Im5hbWUiOiJkYXNoZWQifX19XV0=
  \begin{tikzcd}
    {P'} \\
  & P && B \\
  \\
  & A && X
  \arrow["f"', from=4-2, to=4-4]
  \arrow["g"', from=2-4, to=4-4]
  \arrow[from=2-2, to=4-2]
  \arrow[from=2-2, to=2-4]
  \arrow["\lrcorner"{anchor=center, pos=0.125}, draw=none, from=2-2, to=4-4]
  \arrow[from=1-1, to=4-2]
  \arrow[from=1-1, to=2-4]
  \arrow[dashed, from=1-1, to=2-2]
  \end{tikzcd}
\end{equation*}

\begin{lemma}
  For all $A,B,C$, there is a functor
  \begin{align}\label{eq:span_composition}
    \circ_{A,B,C}:\Span(B,C)\times\text{Span}(A,B)\to\text{Span}(A,C)
  \end{align}
  given by taking pullbacks.
  \begin{proof}
    Consider span maps $f:S\to S'$ in $\Span(A,B)$ and $g:T\to T'$
    in $\Span(B,C)$. By the universal property of $S'\times_B T'$
    these induce a unique span map $S\times_B T\to S'\times_B T'$
    as in the diagram:
    \begin{equation*}
      % https://q.uiver.app/#q=WzAsOSxbMCwwLCJTXFx0aW1lc19CIFQnIl0sWzAsNCwiUydcXHRpbWVzX0IgVCciXSxbMSwxLCJTIl0sWzEsMywiUyciXSxbMiwyLCJCIl0sWzAsMiwiQSJdLFszLDEsIlQiXSxbMywzLCJUJyJdLFs0LDIsIkMiXSxbMiwzLCJmIl0sWzAsMl0sWzEsM10sWzAsMSwiZlxcdGltZXNfQiBnIiwyLHsiY3VydmUiOjUsInN0eWxlIjp7ImJvZHkiOnsibmFtZSI6ImRhc2hlZCJ9fX1dLFszLDRdLFsyLDRdLFsyLDVdLFszLDVdLFs2LDcsImciXSxbNiw4XSxbNyw4XSxbMCw2XSxbMSw3XSxbNiw0XSxbNyw0XV0=
      \begin{tikzcd}
        {S\times_B T'} \\
  & S && T \\
        A && B && C \\
          & {S'} && {T'} \\
          {S'\times_B T'}
          \arrow["f", from=2-2, to=4-2]
          \arrow[from=1-1, to=2-2]
          \arrow[from=5-1, to=4-2]
          \arrow["{f\times_B g}"', curve={height=30pt}, dashed, from=1-1, to=5-1]
          \arrow[from=4-2, to=3-3]
          \arrow[from=2-2, to=3-3]
          \arrow[from=2-2, to=3-1]
          \arrow[from=4-2, to=3-1]
          \arrow["g", from=2-4, to=4-4]
          \arrow[from=2-4, to=3-5]
          \arrow[from=4-4, to=3-5]
          \arrow[from=1-1, to=2-4]
          \arrow[from=5-1, to=4-4]
          \arrow[from=2-4, to=3-3]
          \arrow[from=4-4, to=3-3]
      \end{tikzcd}
    \end{equation*}
    Functoriality follows thus: If $f:S\to S$ and $g:T\to T$ are
    identities then $f\times_B g:S\times_B T\to S\times_B T$ is
    also the identity. Further, if $f,f'\in\Span(A,B)$ and
    $g,g'\in\Span(B,C)$ are composable span maps then
    $(f\times_B g)\circ(f'\times_B g')$ is a span map which must
    agree with the span map induced by $(f\circ f')\times_B(g\circ g')$
    by universality.
  \end{proof}
\end{lemma}

Let us now make sure that taking pull-backs is associative,
at least up to canonical isomorphism:

\begin{lemma}
  $(R\times_B S)\times_C T$ is a limit of the diagram
  \begin{equation}\label{eq:triple_span}
    % https://q.uiver.app/#q=WzAsNyxbMSwwLCJBIl0sWzMsMCwiQiJdLFs1LDAsIkMiXSxbMiwxLCJYIl0sWzQsMSwiWSJdLFswLDEsIlciXSxbNiwxLCJaIl0sWzAsM10sWzEsM10sWzEsNF0sWzIsNF0sWzAsNV0sWzIsNl1d
    \begin{tikzcd}
  & R && S && T \\
      A && B && C && D
      \arrow[from=1-2, to=2-3]
      \arrow[from=1-4, to=2-3]
      \arrow[from=1-4, to=2-5]
      \arrow[from=1-6, to=2-5]
      \arrow[from=1-2, to=2-1]
      \arrow[from=1-6, to=2-7]
    \end{tikzcd}
  \end{equation}
  \begin{proof}
    Consider any object and maps that make the following commute:
    \begin{equation*}
      % https://q.uiver.app/#q=WzAsOCxbMSwyLCJBIl0sWzMsMiwiQiJdLFs1LDIsIkMiXSxbMiwzLCJYIl0sWzQsMywiWSJdLFszLDAsIlAiXSxbMCwzLCJXIl0sWzYsMywiWiJdLFswLDNdLFsxLDNdLFsxLDRdLFsyLDRdLFs1LDJdLFswLDZdLFsyLDddLFs1LDFdLFs1LDBdXQ==
      \begin{tikzcd}
  &&& P \\
  \\
  & R && S && T \\
        A && B && C && D
        \arrow[from=3-2, to=4-3]
        \arrow[from=3-4, to=4-3]
        \arrow[from=3-4, to=4-5]
        \arrow[from=3-6, to=4-5]
        \arrow[from=1-4, to=3-6]
        \arrow[from=3-2, to=4-1]
        \arrow[from=3-6, to=4-7]
        \arrow[from=1-4, to=3-4]
        \arrow[from=1-4, to=3-2]
      \end{tikzcd}
    \end{equation*}
    As the left square commutes, there is a unique map
    $P\to R\times_B S$ through which the maps $P\to R$ and
    $P\to S$ factor. Thus we have a diagram
    \begin{equation*}
      % https://q.uiver.app/#q=WzAsOSxbMSwyLCJBIl0sWzMsMiwiQiJdLFs1LDIsIkMiXSxbMiwzLCJYIl0sWzQsMywiWSJdLFszLDAsIlAiXSxbMCwzLCJXIl0sWzYsMywiWiJdLFsyLDEsIkFcXHRpbWVzX1ggQiJdLFswLDNdLFsxLDNdLFsxLDRdLFsyLDRdLFs1LDJdLFswLDZdLFsyLDddLFs1LDhdLFs4LDBdLFs4LDFdXQ==
      \begin{tikzcd}
  &&& P \\
  && {R\times_B S} \\
  & R && S && T \\
        A && B && C && D
        \arrow[from=3-2, to=4-3]
        \arrow[from=3-4, to=4-3]
        \arrow[from=3-4, to=4-5]
        \arrow[from=3-6, to=4-5]
        \arrow[from=1-4, to=3-6]
        \arrow[from=3-2, to=4-1]
        \arrow[from=3-6, to=4-7]
        \arrow[from=1-4, to=2-3]
        \arrow[from=2-3, to=3-2]
        \arrow[from=2-3, to=3-4]
      \end{tikzcd}
    \end{equation*}
    The pentagon on the right yields a canonical morphism of spans
    $P\to (R\times_B S)\times_C T$ through which the maps
    $P\to R\times_B S$ and $P\to T$ factor.
  \end{proof}
\end{lemma}

Note that, by symmetry, $R\times_B (S\times_C T)$ is also the
limit of (\ref{eq:triple_span}) and hence we have a canonical isomorphism
\begin{align}\label{eq:pullback_associator}
  \alpha_{R,S,T}:(R\times_B S)\times_C T \cong R\times_B (S\times_C T)
\end{align}
in $\Span(A,D)$. This map is characterised by commuting with the
projections in to the cone (\ref{eq:triple_span}).
We ought to make sure that our associator is natural.

\begin{lemma}
  The map (\ref{eq:pullback_associator}) is natural in its arguments.
  \begin{proof}
    Consider a span map $f:R\to R'$ and note that
    ${\alpha}^{-1}\circ (f\circ (S\circ T)) \circ \alpha$ provides
    us with a span map as in the diagram:
    \begin{equation*}
      % https://q.uiver.app/#q=WzAsNyxbMSwwLCIoUlxcY2lyYyBTKVxcY2lyYyBUIl0sWzEsMSwiUlxcY2lyYyhTXFxjaXJjIFQpIl0sWzEsNCwiKFInXFxjaXJjIFMpXFxjaXJjIFQiXSxbMSwzLCJSJ1xcY2lyYyhTXFxjaXJjIFQpIl0sWzAsMSwiUiJdLFswLDMsIlInIl0sWzIsMiwiU1xcY2lyYyBUIl0sWzAsMSwiXFxhbHBoYSJdLFszLDIsIlxcYWxwaGFeey0xfSIsMl0sWzEsMywiZlxcY2lyYyAoU1xcY2lyYyBUKSIsMV0sWzEsNF0sWzQsNSwiZiIsMV0sWzMsNV0sWzAsNF0sWzIsNV0sWzEsNl0sWzMsNl0sWzAsNl0sWzIsNl1d
      \begin{tikzcd}
  & {(R\circ S)\circ T} \\
        R & {R\circ(S\circ T)} \\
          && {S\circ T} \\
        {R'} & {R'\circ(S\circ T)} \\
             & {(R'\circ S)\circ T}
             \arrow["\alpha", from=1-2, to=2-2]
             \arrow["{\alpha^{-1}}"', from=4-2, to=5-2]
             \arrow["{f\circ (S\circ T)}"{description}, from=2-2, to=4-2]
             \arrow[from=2-2, to=2-1]
             \arrow["f"{description}, from=2-1, to=4-1]
             \arrow[from=4-2, to=4-1]
             \arrow[from=1-2, to=2-1]
             \arrow[from=5-2, to=4-1]
             \arrow[from=2-2, to=3-3]
             \arrow[from=4-2, to=3-3]
             \arrow[from=1-2, to=3-3]
             \arrow[from=5-2, to=3-3]
      \end{tikzcd}
    \end{equation*}
    This respects $f$ and hence, by universality of $(R'\circ S)\circ T$,
    must agree with the map $(f\circ S)\circ T$. I.e. the following
    commutes:
    \begin{equation*}
      % https://q.uiver.app/#q=WzAsNCxbMCwwLCIoUlxcY2lyYyBTKVxcY2lyYyBUIl0sWzIsMCwiUlxcY2lyYyhTXFxjaXJjIFQpIl0sWzAsMSwiKFInXFxjaXJjIFMpXFxjaXJjIFQiXSxbMiwxLCJSJ1xcY2lyYyhTXFxjaXJjIFQpIl0sWzAsMSwiXFxhbHBoYSJdLFsyLDMsIlxcYWxwaGEiLDJdLFsxLDMsImZcXGNpcmMgKFNcXGNpcmMgVCkiXSxbMCwyLCIoZlxcY2lyYyBTKVxcY2lyYyBUIiwyXV0=
      \begin{tikzcd}
        {(R\circ S)\circ T} && {R\circ(S\circ T)} \\
        {(R'\circ S)\circ T} && {R'\circ(S\circ T)}
        \arrow["\alpha", from=1-1, to=1-3]
        \arrow["\alpha"', from=2-1, to=2-3]
        \arrow["{f\circ (S\circ T)}", from=1-3, to=2-3]
        \arrow["{(f\circ S)\circ T}"', from=1-1, to=2-1]
      \end{tikzcd}
    \end{equation*}
    This proves naturality in the first argument. The others follow
    similarly.
  \end{proof}
\end{lemma}

Next, we need to check coherence as in (\ref{eq:pentagon}).
\begin{lemma}
  The natural isomorphism (\ref{eq:pullback_associator}) satisfies
  (\ref{eq:pentagon}).
  \begin{proof}
    Consider the diagram
    \begin{equation*}
      % https://q.uiver.app/#q=WzAsOSxbMCwwLCIoKFJcXGNpcmMgUylcXGNpcmMgVClcXGNpcmMgVSJdLFswLDMsIihSXFxjaXJjIFMpXFxjaXJjKFRcXGNpcmMgVSkiXSxbNCwzLCJSXFxjaXJjKFNcXGNpcmMoVFxcY2lyYyBVKSkiXSxbMiwwLCIoUlxcY2lyYyhTXFxjaXJjIFQpKVxcY2lyYyBVIl0sWzQsMCwiUlxcY2lyYyAoKFNcXGNpcmMgVClcXGNpcmMgVSkiXSxbMywyLCJSIl0sWzEsMiwiUlxcY2lyYyBTIl0sWzEsMSwiKFJcXGNpcmMgUylcXGNpcmMgVCJdLFsyLDEsIlJcXGNpcmMoU1xcY2lyYyBUKSJdLFswLDEsIlxcYWxwaGEiLDJdLFsxLDIsIlxcYWxwaGEiLDJdLFs0LDIsIlJcXGNpcmMgXFxhbHBoYSJdLFswLDMsIlxcYWxwaGFcXGNpcmMgVSJdLFszLDQsIlxcYWxwaGEiXSxbMiw1XSxbMCw3XSxbMSw2XSxbNCw1XSxbNiw1XSxbNyw2XSxbMyw4XSxbNyw4LCJcXGFscGhhIl0sWzgsNV1d
      \begin{tikzcd}
        {((R\circ S)\circ T)\circ U} && {(R\circ(S\circ T))\circ U} && {R\circ ((S\circ T)\circ U)} \\
                                     & {(R\circ S)\circ T} & {R\circ(S\circ T)} \\
                                     & {R\circ S} && R \\
        {(R\circ S)\circ(T\circ U)} &&&& {R\circ(S\circ(T\circ U))}
        \arrow["\alpha"', from=1-1, to=4-1]
        \arrow["\alpha"', from=4-1, to=4-5]
        \arrow["{R\circ \alpha}", from=1-5, to=4-5]
        \arrow["{\alpha\circ U}", from=1-1, to=1-3]
        \arrow["\alpha", from=1-3, to=1-5]
        \arrow[from=4-5, to=3-4]
        \arrow[from=1-1, to=2-2]
        \arrow[from=4-1, to=3-2]
        \arrow[from=1-5, to=3-4]
        \arrow[from=3-2, to=3-4]
        \arrow[from=2-2, to=3-2]
        \arrow[from=1-3, to=2-3]
        \arrow["\alpha", from=2-2, to=2-3]
        \arrow[from=2-3, to=3-4]
      \end{tikzcd}
    \end{equation*}
    It is straightforward to check that each face commutes, either by
    functoriality of the pullback or by definition of the isomorphism
    $\alpha$. One may construct similar diagrams with sinks $S$, $T$,
    and $U$, respectively. Thus the outer maps define identical cones
    and hence, by universality of the limit, must agree.
  \end{proof}
\end{lemma}

It is straightforward to see that $A\leftarrow A\rightarrow A$
is an identity for pullbacks. This is because whenever we have
a commuting diagram
\begin{equation*}
  % https://q.uiver.app/#q=WzAsNixbMCwyLCJYIl0sWzIsMiwiWCJdLFs0LDIsIlkiXSxbMSwxLCJYIl0sWzMsMSwiQSJdLFsyLDAsIlAiXSxbMywwLCIiLDIseyJsZXZlbCI6Miwic3R5bGUiOnsiaGVhZCI6eyJuYW1lIjoibm9uZSJ9fX1dLFszLDEsIiIsMCx7ImxldmVsIjoyLCJzdHlsZSI6eyJoZWFkIjp7Im5hbWUiOiJub25lIn19fV0sWzQsMV0sWzQsMl0sWzUsNF0sWzUsM11d
  \begin{tikzcd}
  && P \\
  & A && S \\
    A && A && B
    \arrow[Rightarrow, no head, from=2-2, to=3-1]
    \arrow[Rightarrow, no head, from=2-2, to=3-3]
    \arrow[from=2-4, to=3-3]
    \arrow[from=2-4, to=3-5]
    \arrow[from=1-3, to=2-4]
    \arrow[from=1-3, to=2-2]
  \end{tikzcd}
\end{equation*}
the map $P\to A$ factors through $S$ and hence there is a unique
map of spans $P\to S$. Thus we obtain canonical isomorphisms
\begin{align}\label{eq:pullback_unitors}
  \lambda_S : A\times_A S \cong S,\hspace{1cm}
  \rho_S : S\times_A A \cong S
\end{align}
in $\Span(A,B)$. We have thus assembled all the ingredients to
define the bicategory with spans as 1-cells. Noting that the
isomorphisms (\ref{eq:pullback_unitors}) are the projections into
the components of the pullback, naturality follows easily. The final
thing we need to check is the second coherence condition:

\begin{lemma}
  The natural isomorphisms (\ref{eq:pullback_associator}) and
  (\ref{eq:pullback_unitors}) satisfy (\ref{eq:triangle}).
  \begin{proof}
    Once again we postcompose with the projections. Thus we have
    \begin{equation*}
      % https://q.uiver.app/#q=WzAsNSxbMCwwLCIoU1xcY2lyYyBBKVxcY2lyYyBUIl0sWzQsMCwiU1xcY2lyYyhBXFxjaXJjIFQpIl0sWzIsMywiU1xcY2lyYyBUIl0sWzIsMSwiUyJdLFsxLDEsIlNcXGNpcmMgQSJdLFswLDIsIlxccmhvXFxjaXJjIFQiLDIseyJjdXJ2ZSI6NX1dLFsxLDIsIlNcXGNpcmNcXGxhbWJkYSIsMCx7ImN1cnZlIjotNX1dLFswLDEsIlxcYWxwaGEiXSxbMCw0XSxbNCwzLCJcXHJobyJdLFsyLDNdLFsxLDNdXQ==
      \begin{tikzcd}
        {(S\circ A)\circ T} &&&& {S\circ(A\circ T)} \\
                            & {S\circ A} & S \\
                            \\
                            && {S\circ T}
                            \arrow["{\rho\circ T}"', curve={height=30pt}, from=1-1, to=4-3]
                            \arrow["S\circ\lambda", curve={height=-30pt}, from=1-5, to=4-3]
                            \arrow["\alpha", from=1-1, to=1-5]
                            \arrow[from=1-1, to=2-2]
                            \arrow["\rho", from=2-2, to=2-3]
                            \arrow[from=4-3, to=2-3]
                            \arrow[from=1-5, to=2-3]
      \end{tikzcd}
    \end{equation*}
    Where the top square is compatibility of $\alpha$ with the
    projection into $S$ and the remaining faces hold by definition
    of $\rho\circ T$ and $S\circ\lambda$, respectively.
    Postcomposing with $S\circ T\to T$ is completely analogous.
  \end{proof}
\end{lemma}

\begin{definition}\label{def:span_bicategory}
  $\Span(\mathcal C)$ is the bicategory where
  \begin{itemize}
    \item objects the same as in $\mathcal C$;
    \item hom-categories are $\Hom[A,B]:=\Span(A,B)$;
    \item the identity functors takes $A$ to the trivial span
      $A\leftarrow A \rightarrow A$;
    \item composition is given by the functor (\ref{eq:span_composition});
    \item structural 2-cells $\alpha$, $\lambda$, $\rho$ as
      in (\ref{eq:pullback_associator}) and (\ref{eq:pullback_unitors}).
  \end{itemize}
\end{definition}

It is interesting to observe that the objects in $\Span(\mathcal C)$
are objects in $\mathcal C$ and the 2-cells are maps in $\mathcal C$.
Thus $\Span(\mathcal C)$ does not introduce maps of maps in the way that
natural transformations map between functors, but rather we have introduced additional structure between objects and maps.

%%%%%%%%%%%%%%%%%%%%%%%%%%%%%%%%%
\subsection{Monoidal categories are bicategories}
%%%%%%%%%%%%%%%%%%%%%%%%%%%%%%%%%

As promised, we are now going to see that it is possible to make
the resemblance between monoidal categories and bicategories precise.
In particular, monoidal categories are single-object bicatgories.

\begin{definition}
  The \emph{bicategory $\mathbf C$ induced by a monoidal category
  $(\mathcal C,\otimes, I,\alpha,\lambda,\rho)$} consists of
  a single object $*$, the unique hom-category $\Hom[*,*]:=\mathcal C$,
  the identity functor $\identity_*$ mapping to $I$, the composition functor
  \begin{align*}
    \circ_{*,*,*} : \Hom[*,*]\times\Hom[*,*]\to\Hom[*,*]
  \end{align*}
  given by $\otimes : \mathcal C\times\mathcal C\to\mathcal C$,
  and the natural isomorphism corresponding to the unique triple
  of objects $*,*,*$ are $\alpha$, $\rho$ and $\lambda$.
\end{definition}

Thus the 1-cells of $\mathbf C$ are the objects of $\mathcal C$ and
their composition is the monoidal product. Thus, for $A,B,C,D\in\mathcal
C$, the conditions (\ref{eq:pentagon}) and (\ref{eq:triangle}) become
exactly the coherence conditions (\ref{eq:monoidal_pentagon}) and
(\ref{eq:monoidal_triangle}) of monoidal categories! Thus they are
obviously satisfied. It is not hard to see that it is easy to
go in the other direction, i.e. if $\mathbf C$ is a bicategory with a single object $*$ then $\Hom[*,*]$ has a monoidal structure given
by composition and the identity 1-cell. This is a partial inverse
in the sense that, given a monoidal category, turning it into
a bicategory, and then turning it into a monoidal category takes us
back to where we started.

There is more to be said on this subject: One may define pseudofunctors,
i.e. maps of bicategories, and hence a category of bicategories
$\textbf{Bicat}$. In particular, there is a full subcategory
$\textbf{1Bicat}\hookrightarrow\textbf{Bicat}$ of single-object bicategories. Moreover, there is the notion of monoidal functors,
i.e. maps of monoidal categories. These may be used to form the
category $\textbf{MonCat}$ of monoidal categories. One then obtains
an equivalence $\textbf{1Bicat}\cong\textbf{MonCat}$. This statement
is stronger than what we have discussed in this section in the sense
that it proves that there truly is no difference between single-object
bicategories and monoidal categories because their structure preserving
morphisms agree. Unfortunately,
rigorous pursuit of this idea is beyond the scope of this
report.

\section{Monoids are categories}

We have seen that one may generalise monoidal categories to bicategories.
Let us now turn our attention to a simpler example. It turns out that
categories are a generalisation of monoids. In particular, we will show
that there is an equivalence of categories between single-object
categories and monoids.

Recall that a monoid consists of an underlying set equipped with
a unital associative multiplication. More precisely:

\begin{definition}
  A \emph{monoid} $(M,e,m)$ consists of
  a set $M$,
  a distinguished element $e\in M$, and
  a map $m : M\times M\to M$
  such that, for all $x,y,z\in M$,
  $m(e,x) = x = m(x,e)$ and $m(x,m(y,z)) = m(m(x,y),z)$.
\end{definition}

Now a map of monoids is a map of underlying sets that preserves
identities and multiplication:

\begin{definition}
  A \emph{map of monoids} $f:(M,e,m)\to(M',e',m')$ is a map
  $f:M\to M'$ such that, for all $x,y\in M$,
  $f(m(x,y)) = m(f(x),f(y))$ and $f(e)=e'$.
\end{definition}

It is easy to check that composition of monoid maps yields another
monoid map. Hence we have a category of monoids which we will
denote by $\textbf{Monoid}$.
On the other hand, we have the full subcategory
$\textbf{1Cat}\hookrightarrow\textbf{Cat}$ of small categories with
a unique object. That is, $\textbf{1Cat}$ contains all small single-object
categories and all functors between them.

Note that the two conditions that we impose on the multiplication of a monoid
are associativity and unitality. Those are exactly the conditions we require
the composition of a category to satisfy. We may thus consider a monoid
as a category:

\begin{definition}
  Let $(M,e,m)$ be a monoid. The induced category $I(M,e,m)$ has
  has a single object $*$ with hom-set $\Hom(*,*):= M$, identity
  $\identity_* := e$, and composition $x\circ y := m(x,y)$ for
  $x,y\in\Hom(*,*)$.

  Let $f:(M,e,m) \to (M',e',m')$ be a monoid map. The induced functor
  $If:I(M,e,m)\to I(M',e',m')$ is given by the map on objects $*\mapsto *$
  and the map on hom-sets $x \mapsto f(x)$ for $x\in\Hom(*,*)$.
\end{definition}

Let us make sure that this is well-defined. In particular, $I(M,e,m)$
is a category as, for $x,y,z\in\Hom(*,*)$,
\begin{align*}
  x\circ(y\circ z) = m(x,m(y,z)) = m(m(x,y),z) = (x\circ y)\circ z
\end{align*}
and
\begin{align*}
  x \circ \identity_* = m(x,e) = x = m(e,x) = \identity_* \circ x.
\end{align*}
Further, $If$ is a functor as $If(\identity_*) = f(e) = e' = \identity_*$
and, for $x,y\in\Hom(*,*)$,
\begin{align*}
  If(x\circ y) = f(m(x,y)) = m(f(x),f(y)) = If(x) \circ If(y).
\end{align*}
Our notation suggests that we have defined a functor from
monoids to single object categories. This is indeed the case.
\begin{lemma}
  $I:\textbf{Monoid}\to\textbf{1Cat}$ is a functor.
  \begin{proof}
    Consider a monoid map $f:(M,e,m)\to(M',e',m')$. We then
    have $If(\identity_*) = f(e) = e' = \identity_*$, i.e. $I$
    preserves identities. Moreover, for $x,y,z\in M$,
    $If(x\circ y) = f(m(x,y)) = m(f(x),f(y)) = If(x)\circ If(y)$ by
    definition of a monoid map. Thus $I$ is compatible with composition,
    too.
  \end{proof}
\end{lemma}

We may also go the other way:

\begin{lemma}
  Let $\mathbb C,\mathbb D$ be small categories with unique objects
  $X\in\mathbb{C}$ and $Y\in\mathbb D$ and let $F:\mathbb{C}\to\mathbb D$
  be a functor. Then $J\mathbb C := (\Hom(X,X),\identity_X,\circ)$ is a
  monoid and $x\mapsto Fx$ defines a monoid map $JF : J\mathbb{C}\to J\mathbb D$.
  \begin{proof}
    We verify that $J\mathbb{C}$ is a monoid. For $x\in J\mathbb{C}$
    we find
    \begin{align*}
      m(e,x) = \identity_X \circ x = x = x \circ \identity_X = m(x,e).
    \end{align*}
    Similarly, for $x,y,z\in J\mathbb{C}$,
    \begin{align*}
      m(x,m(y,z)) = x\circ (y\circ z) = (x\circ y)\circ z = m(m(x,y),z).
    \end{align*}
    Moreover, for $x,y\in J\mathbb{C}$,
    \begin{align*}
      JF(m(x,y)) = F(x\circ y) = Fx \circ Fy = m(JF(x), JF(y))
    \end{align*}
    and $JF(e) = F(\identity_X) = \identity_Y = e'$. Thus
    $JF$ is a monoid map.
  \end{proof}
\end{lemma}

Once again, this is a functor:

\begin{lemma}
  $J:\textbf{1Cat}\to\textbf{Monoid}$ is a functor.
  \begin{proof}
    If $\text{Id}:\mathbb C\to\mathbb C$ is the identity functor
    then $J(\text{Id})$ must be the identity map which is the
    identity on $J\mathbb C$. Moreover, if we have functors
    \begin{align*}
      \mathbb C \xlongrightarrow{F} \mathbb D \xlongrightarrow{G} \mathbb E
    \end{align*}
    in $\textbf{1Cat}$ then, for $x\in J\mathbb C$,
    \begin{align*}
      JG\circ JF(x) = G(F(x)) = (G\circ F)(x) = J(G\circ F)(x).
    \end{align*}
    Thus $J$ respects composition and hence defines a functor.
  \end{proof}
\end{lemma}

One might think that $I$ and $J$ are mutually inverse. However, this
is not so: Consider $\mathbb C,\mathbb C'\in\textbf{1Cat}$ which are
isomorphic through $X\mapsto Y$ on objects and $x\mapsto x$ on maps.
Then $\mathbb{C}\neq\mathbb{C}'$ but $J\mathbb{C}=J\mathbb{C}'$.
This is a common phenomenon in category theory. Hence one rarely
speaks about isomorphic categories but rather about equivalent ones:

\begin{definition}
  An \emph{equivalence of categories $\mathbb C\cong\mathbb D$} consists
  of functors $F:\mathbb{C}\to\mathbb D$ and $G:\mathbb D\to\mathbb C$
  and natural isormopshims $\eta : \text{Id}\to GF$ and
  $\epsilon:FG\to \text{Id}$ as in the diagram
  \begin{equation*}
    % https://q.uiver.app/#q=WzAsNCxbMCwxLCJcXG1hdGhiYiBDIl0sWzEsMCwiXFxtYXRoYmIgRCJdLFsyLDEsIlxcbWF0aGJiIEMiXSxbMywwLCJcXG1hdGhiYiBEIl0sWzAsMSwiRiJdLFsxLDIsIkciLDFdLFswLDIsIiIsMix7ImxldmVsIjoyLCJzdHlsZSI6eyJoZWFkIjp7Im5hbWUiOiJub25lIn19fV0sWzEsMywiIiwyLHsibGV2ZWwiOjIsInN0eWxlIjp7ImhlYWQiOnsibmFtZSI6Im5vbmUifX19XSxbMiwzLCJGIiwyXSxbNiwxLCJcXGV0YSIsMCx7InNob3J0ZW4iOnsic291cmNlIjoyMH19XSxbMiw3LCJcXGVwc2lsb24iLDIseyJzaG9ydGVuIjp7InRhcmdldCI6MjB9fV1d
    \begin{tikzcd}
  & {\mathbb D} && {\mathbb D} \\
      {\mathbb C} && {\mathbb C}
      \arrow["F", from=2-1, to=1-2]
      \arrow["G"{description}, from=1-2, to=2-3]
      \arrow[""{name=0, anchor=center, inner sep=0}, Rightarrow, no head, from=2-1, to=2-3]
      \arrow[""{name=1, anchor=center, inner sep=0}, Rightarrow, no head, from=1-2, to=1-4]
      \arrow["F"', from=2-3, to=1-4]
      \arrow["\eta", shorten <=3pt, Rightarrow, from=0, to=1-2]
      \arrow["\epsilon"', shorten >=3pt, Rightarrow, from=2-3, to=1]
    \end{tikzcd}
  \end{equation*}
\end{definition}

That is, functors $F:\mathbb{C}\to\mathbb D:G$ form an equivalence
if they are inverses up to isomorphism. Our functors $I$ and $J$
satisfy this condition:

\begin{proposition}
  There is an equivalence of categories $\textbf{Monoid}\cong\textbf{1Cat}$.
  \begin{proof}
    It is straightforward to verify
    $JI=\text{Id}:\textbf{Monoid}\to\textbf{Monoid}$. We need
    to define a natural isormophism $\epsilon:IJ\to\text{Id}$.
    Consider $X\in\mathbb C\in\textbf{1Cat}$. Define
    $\epsilon_{\mathbb{C}}(*)=X$ and $\epsilon_{\mathbb{C}}(x) = x$ for
    $x\in\Hom(X,X)$. This is natural in $\mathbb{C}$ as, for all
    $F:\mathbb{C}\to\mathbb D$ in $\textbf{1Cat}$,
    $\epsilon_{\mathbb D}(IJF(*))$ and $IJF(\epsilon_{\mathbb{C}})$
    must be the unique object in $\mathbb D$ and on arrows $\varepsilon$
    is the identity so there is nothing to check. Moreover,
    $\epsilon_{\mathbb{C}}$ is a bijection on objects and arrows so
    it is an isomorphism.
  \end{proof}
\end{proposition}

\section{Monads are monoids}

In this section we divert slightly from \cite{pawel2017}. There
monads are first generalised to bicategories and then shown to be
equivalent to monoid objects in the hom-categories. We reverse
the order because it allows us to more intuitively define
the category of monads on a category, it simplifies
the definition of a monad in a bicategory, and it leaves the
least obvious result for last. Note that this change does
not really save us much work, if any.

We have seen that categories are one generalisation of monoids. However,
one may go in a different direction. Observe that an element $e\in M$
may be identified with a map $e : 1 \to M$ where $1=\left\lbrace{*}\right\rbrace$ is a singleton set and we abuse notation to write $e(*) = e$.
One may now ask what is special about $\textbf{Set}$? We could define
a monoid to consists of maps $m:M\times M\to M$ and $e:1\to M$ in
any cartesian category. We are going to go one step further and
replace the cartesian structure with an arbitrary monoidal structure:

\begin{definition}
  Let $(\mathcal C,I,\otimes,\alpha,\lambda,\rho)$ be a monoidal
  category. A
  \emph{monoid object} $(M,e,m)$ in $\mathcal C$ consists of
  an object $M\in\mathcal C$ and maps
  $e:I\to M$ and $m:M\otimes M\to M$ in $\mathcal C$
  such that the following commute:
  \begin{equation}\label{eq:monoid_axioms}
    % https://q.uiver.app/#q=WzAsNCxbMCwwLCJJXFxvdGltZXMgTSJdLFsyLDAsIk1cXG90aW1lcyBNIl0sWzIsMiwiTSJdLFs0LDAsIk1cXG90aW1lcyBJIl0sWzAsMSwiZVxcb3RpbWVzIE0iXSxbMSwyLCJtIl0sWzAsMiwiXFxsYW1iZGEiLDJdLFszLDEsIk1cXG90aW1lcyBlIiwyXSxbMywyLCJcXHJobyJdXQ==
    \begin{tikzcd}
      {I\otimes M} && {M\otimes M} && {M\otimes I} \\
      \\
                   && M
                   \arrow["{e\otimes M}", from=1-1, to=1-3]
                   \arrow["m", from=1-3, to=3-3]
                   \arrow["\lambda"', from=1-1, to=3-3]
                   \arrow["{M\otimes e}"', from=1-5, to=1-3]
                   \arrow["\rho", from=1-5, to=3-3]
    \end{tikzcd}, \hspace{1cm}
    % https://q.uiver.app/#q=WzAsNSxbMCwwLCIoTVxcb3RpbWVzIE0pXFxvdGltZXMgTSJdLFswLDEsIk1cXG90aW1lcyBNIl0sWzEsMiwiTSJdLFsyLDEsIk1cXG90aW1lcyBNIl0sWzIsMCwiTVxcb3RpbWVzKE1cXG90aW1lcyBNKSJdLFswLDEsIm1cXG90aW1lcyBNIiwyXSxbMSwyLCJtIiwyXSxbMywyLCJtIl0sWzAsNCwiXFxhbHBoYSJdLFs0LDMsIk1cXG90aW1lcyBtIl1d
    \begin{tikzcd}
      {(M\otimes M)\otimes M} && {M\otimes(M\otimes M)} \\
      {M\otimes M} && {M\otimes M} \\
                   & M
                   \arrow["{m\otimes M}"', from=1-1, to=2-1]
                   \arrow["m"', from=2-1, to=3-2]
                   \arrow["m", from=2-3, to=3-2]
                   \arrow["\alpha", from=1-1, to=1-3]
                   \arrow["{M\otimes m}", from=1-3, to=2-3]
    \end{tikzcd}
  \end{equation}
\end{definition}

Consider the monoidal structure on $\Set$ given by the cartesian structure.
Then a monoid object in $\Set$ is just a monoid in the usual sense: We have a
set $M$, a map $e:1\to M$, i.e. an element $e\in M$, and a multiplication
$m:M\times M\to M$. The conditions \ref{eq:monoid_axioms} are just
unitality and associativity, respectively.

\begin{definition}
  A \emph{map of monoid objects} $f:(M,e,m)\to(M',e',m')$ in
  a monoidal category $(\mathcal C, I,\otimes)$
  is a map $f:M\to M'$ in $\mathcal C$ such that the following commute:
  \begin{equation}\label{eq:monoid_map_axioms}
    % https://q.uiver.app/#q=WzAsNCxbMCwwLCJNXFxvdGltZXMgTSJdLFsyLDAsIk0nXFxvdGltZXMgTSciXSxbMCwxLCJNIl0sWzIsMSwiTSciXSxbMCwxLCJmXFxvdGltZXMgZiJdLFsxLDMsIm0nIl0sWzAsMiwibSIsMl0sWzIsMywiZiIsMl1d
    \begin{tikzcd}
      {M\otimes M} && {M'\otimes M'} \\
      M && {M'}
      \arrow["{f\otimes f}", from=1-1, to=1-3]
      \arrow["{m'}", from=1-3, to=2-3]
      \arrow["m"', from=1-1, to=2-1]
      \arrow["f"', from=2-1, to=2-3]
    \end{tikzcd}\hspace{1cm}
    % https://q.uiver.app/#q=WzAsMyxbMCwxLCJNIl0sWzIsMSwiTSciXSxbMSwwLCJJIl0sWzIsMCwiZSIsMl0sWzIsMSwiZSciXSxbMCwxLCJmIiwyXV0=
    \begin{tikzcd}
  & I \\
      M && {M'}
      \arrow["e"', from=1-2, to=2-1]
      \arrow["{e'}", from=1-2, to=2-3]
      \arrow["f"', from=2-1, to=2-3]
    \end{tikzcd}
  \end{equation}
\end{definition}

It is easily verified that the composite of monoid maps
\begin{align*}
  (M,e,m)\xlongrightarrow{f}(M',e',m')\xlongrightarrow{g}(M'',e'',m'')
\end{align*}
is itself a monoid map as
\begin{equation*}
  % https://q.uiver.app/#q=WzAsNixbMCwxLCJNIl0sWzIsMSwiTSciXSxbMiwwLCJNJ1xcb3RpbWVzIE0nIl0sWzAsMCwiTVxcb3RpbWVzIE0iXSxbNCwwLCJNJydcXG90aW1lcyBNJyciXSxbNCwxLCJNJyciXSxbMCwxLCJmIiwyXSxbMiwxLCJtJyJdLFszLDIsImZcXG90aW1lcyBmIl0sWzMsMCwibSIsMl0sWzIsNCwiZ1xcb3RpbWVzIGciXSxbMSw1LCJnIiwyXSxbNCw1LCJtJyciLDFdLFszLDQsIihnXFxjaXJjIGYpXFxvdGltZXMgKGdcXGNpcmMgZikiLDAseyJjdXJ2ZSI6LTV9XV0=
  \begin{tikzcd}
    {M\otimes M} && {M'\otimes M'} && {M''\otimes M''} \\
    M && {M'} && {M''}
    \arrow["f"', from=2-1, to=2-3]
    \arrow["{m'}", from=1-3, to=2-3]
    \arrow["{f\otimes f}", from=1-1, to=1-3]
    \arrow["m"', from=1-1, to=2-1]
    \arrow["{g\otimes g}", from=1-3, to=1-5]
    \arrow["g"', from=2-3, to=2-5]
    \arrow["{m''}"{description}, from=1-5, to=2-5]
    \arrow["{(g\circ f)\otimes (g\circ f)}", curve={height=-30pt}, from=1-1, to=1-5]
  \end{tikzcd}
\end{equation*}
commutes by functoriality of $\otimes:\mathcal C\times\mathcal C\to\mathcal C$. If $f=\identity$, $M=M'$, $e=e'$, and $m=m'$ then the conditions
are trivially satisfied. Hence we have a category
$\textbf{Monoid}(\mathcal C,I,\otimes)$ of monoid objects in
a monoidal category $(\mathcal C,I,\otimes)$ where we surpress
the structural isomorphisms $\alpha$, $\lambda$, and $\rho$.

Other than generalisation for the sake of generalisation, it is not
immediately obvious what we have achieved. However, one need only
consider the category of $k$-vector spaces $\textbf{Vect}_k$ with
its monoidal structure given by the tensor product. A
monoid object in $\textbf{Vect}_k$ is a $k$-algebra! In particular, we do
not require any additional compatibility conditions between the monoid
multiplication and the scalar product of a vector space.

A less obvious example of monoid objects are monads. Recall
the definition:
\begin{definition}
  Let $\mathcal C$ be a category. A \emph{monad} on $\mathcal C$
  consists of a functor $T:\mathcal C\to\mathcal C$ and natural
  transformations
  $\eta:\text{Id}\to T$ and $\mu:T^2\to T$
  such that the following commute:
  \begin{equation}\label{eq:monad_laws}
    % https://q.uiver.app/#q=WzAsNCxbMiwwLCJUXjIiXSxbMiwyLCJUIl0sWzAsMCwiVCJdLFswLDIsIlReMiJdLFswLDEsIlxcbXUiXSxbMiwwLCJUXFxldGEiXSxbMiwxLCIiLDEseyJsZXZlbCI6Miwic3R5bGUiOnsiaGVhZCI6eyJuYW1lIjoibm9uZSJ9fX1dLFsyLDMsIlxcZXRhIFQiLDJdLFszLDEsIlxcbXUiLDJdXQ==
    \begin{tikzcd}
      T && {T^2} \\
      \\
      {T^2} && T
      \arrow["\mu", from=1-3, to=3-3]
      \arrow["T\eta", from=1-1, to=1-3]
      \arrow[Rightarrow, no head, from=1-1, to=3-3]
      \arrow["{\eta T}"', from=1-1, to=3-1]
      \arrow["\mu"', from=3-1, to=3-3]
    \end{tikzcd}\hspace{1cm}
    % https://q.uiver.app/#q=WzAsNCxbMCwwLCJUXjMiXSxbMiwwLCJUXjIiXSxbMCwyLCJUXjIiXSxbMiwyLCJUIl0sWzAsMSwiVFxcbXUiXSxbMCwyLCJcXG11IFQiLDJdLFsyLDMsIlxcbXUiLDJdLFsxLDMsIlxcbXUiXV0=
    \begin{tikzcd}
      {T^3} && {T^2} \\
      \\
      {T^2} && T
      \arrow["T\mu", from=1-1, to=1-3]
      \arrow["{\mu T}"', from=1-1, to=3-1]
      \arrow["\mu"', from=3-1, to=3-3]
      \arrow["\mu", from=1-3, to=3-3]
    \end{tikzcd}
  \end{equation}
\end{definition}

Here the natural transormation $\eta T : T\to T^2$ has
components $(\eta T)_X = \eta_{TX}$ and similarly
for $\mu T$.

We have seen many examples of monads. Hence we
are ready to move on to defining the category of monads.
As usual, a map of monads is a map of underlying objects that plays
well with the additional structure:

\begin{definition}
  A \emph{map of monads} $\phi:(T,\eta,\mu)\to(T',\eta',\mu')$
  on $\mathcal C$ is a natural transformation $\phi:T\to T'$
  such that
  \begin{equation}\label{eq:monad_map_laws}
    % https://q.uiver.app/#q=WzAsNyxbMCwwLCJUXjIiXSxbMCwxLCJUIl0sWzIsMSwiVCciXSxbMiwwLCIoVCcpXjIiXSxbNSwwLCJcXHRleHR7SWR9Il0sWzQsMSwiVCJdLFs2LDEsIlQnIl0sWzAsMywiXFxwaGleMiJdLFsxLDIsIlxccGhpIl0sWzAsMSwiXFxtdSIsMl0sWzMsMiwiXFxtdSciXSxbNCw1LCJcXGV0YSIsMl0sWzUsNiwiXFxwaGkiLDJdLFs0LDYsIlxcZXRhJyJdXQ==
    \begin{tikzcd}
      {T^2} && {(T')^2} &&& {\text{Id}} \\
      T && {T'} && T && {T'}
      \arrow["{\phi^2}", from=1-1, to=1-3]
      \arrow["\phi", from=2-1, to=2-3]
      \arrow["\mu"', from=1-1, to=2-1]
      \arrow["{\mu'}", from=1-3, to=2-3]
      \arrow["\eta"', from=1-6, to=2-5]
      \arrow["\phi"', from=2-5, to=2-7]
      \arrow["{\eta'}", from=1-6, to=2-7]
    \end{tikzcd}
  \end{equation}
\end{definition}

Here the natural transformation $\phi^2 : T^2 \to (T')^2$ has
components
\begin{align*}
  T^2 X = T(TX)
  \xlongrightarrow{\phi_{TX}} T'(TX)
  \xlongrightarrow{T'\phi_X} T'(T'X) = (T')^2 X.
\end{align*}
It is straightforward to verify that the composition of monad
maps yields a monad map. Hence we have a category $\textbf{Monad}(\mathcal C)$.

Now recall that we have a bicategory $\textbf{CAT}$ of
categories, functors, and natural transformations. Moreover, we
showed that, for each object $X$ in a bicategory $\mathbf{C}$,
there is an induced monoidal structure on $\Hom(X,X)$. So for
each category $\mathcal C\in\textbf{CAT}$ we obtain a monoidal structure
$(\Hom[\mathcal C,\mathcal C],\text{Id},\circ)$.

The monad laws (\ref{eq:monad_laws}) look similar to
the monoid axioms (\ref{eq:monoid_axioms}). The monoidal structure
on $\Hom[\mathcal C,\mathcal C]$ above allows us to regard monads
as monoid objects and thus obtain an equivalence of categories.

\begin{proposition}
  There is an equivalence of categories
  \begin{align*}
    \textbf{Monoid}(\Hom[\mathcal C,\mathcal C],\text{Id},\circ)
    \cong \textbf{Monad}(\mathcal C).
  \end{align*}
  \begin{proof}
    Consider a monad $(T,\eta,\mu)$ on $\mathcal C$. Note that no
    modification is required to turn this into a monoid object:
    We have $T\in\Hom[\mathcal C,\mathcal C]$, $\eta : \text{Id}\to T$
    and $\mu : T \circ T \to T$. Moreover, the natural isomorphisms
    in the bicategory $\textbf{CAT}$ are identities, hence the
    monoid axioms (\ref{eq:monoid_axioms}) are precisely the monad
    laws (\ref{eq:monad_laws}). Note that we have not changed
    anything about the triple $(T,\eta,\mu)$, hence we have a bijection
    of objects between $\textbf{Monad}(\mathcal C)$ and
    $\textbf{Monoid}(\Hom[\mathcal C,\mathcal C],\text{Id},\circ)$.

    It is now even more straightforward to see that
    (\ref{eq:monoid_map_axioms}) and (\ref{eq:monad_map_laws})
    are identical for monoid object maps in $(\Hom[\mathcal C,\mathcal
    C],\text{Id},\circ)$ and maps of monads on $\mathcal C$,
    respectively.
  \end{proof}
\end{proposition}

\section{Categories are monads}

We have seen that monads are monoids in the category of endofunctors.
One may now observe that functor categories are the hom-categories of
the bicategory $\mathbf{CAT}$. Thus we may, more generally, refer
to a monoid object in $T\in\Hom[A,A]$ for any object $A$ in any
bicategory $\mathbf C$ as an monad.

\begin{definition}
  Let $\mathbf C$ be a bicategory. A \emph{formal monad in
  $\mathbf C$} is a quadruple $(A,T,\eta,\mu)$ such that
  $A\in\mathbf C$ and $(T,\eta,\mu)$ is a monoid object in
  $\Hom[A,A]$.
\end{definition}

So far we have only defined
maps of such monads in the case where the underlying objects agrees.
We now rectify this:

\begin{definition}
  A \emph{map of formal monads}
  $(H,\phi):(A,T,\eta,\mu)\to(A',T',\eta',\mu')$ consists of
  a 1-cell $H:A\to A'$ and a 2-cell $\phi : H\circ T\to T'\circ H $
  such that the following commute:
  \begin{equation*}
    % https://q.uiver.app/#q=WzAsOSxbMCwxLCJIXFxjaXJjIFQiXSxbMiwxLCJUJ1xcY2lyYyBIIl0sWzQsMCwiSFxcY2lyYyBUXFxjaXJjIFQiXSxbOCwwLCJUJ1xcY2lyYyBUJ1xcY2lyYyBIIl0sWzgsMSwiVCdcXGNpcmMgSCJdLFs0LDEsIkhcXGNpcmMgVCJdLFs2LDAsIlQnXFxjaXJjIEhcXGNpcmMgVCJdLFsyLDAsIkEnXFxjaXJjIEgiXSxbMCwwLCJIXFxjaXJjIEEiXSxbMCwxLCJcXHBoaSIsMl0sWzMsNCwiXFxtdSdcXGNpcmMgSCJdLFsyLDUsIkhcXGNpcmMgXFxtdSIsMl0sWzUsNCwiXFxwaGkiLDJdLFs2LDMsIlQnXFxjaXJjIFxccGhpIl0sWzIsNiwiXFxwaGlcXGNpcmMgVCJdLFs3LDEsIlxcZXRhJ1xcY2lyYyBIIl0sWzgsMCwiSFxcY2lyY1xcZXRhIiwyXSxbOCw3LCJcXGNvbmciXV0=
    \begin{tikzcd}
      {H\circ A} && {A'\circ H} && {H\circ T\circ T} && {T'\circ H\circ T} && {T'\circ T'\circ H} \\
      {H\circ T} && {T'\circ H} && {H\circ T} &&&& {T'\circ H}
      \arrow["\phi"', from=2-1, to=2-3]
      \arrow["{\mu'\circ H}", from=1-9, to=2-9]
      \arrow["{H\circ \mu}"', from=1-5, to=2-5]
      \arrow["\phi"', from=2-5, to=2-9]
      \arrow["{T'\circ \phi}", from=1-7, to=1-9]
      \arrow["{\phi\circ T}", from=1-5, to=1-7]
      \arrow["{\eta'\circ H}", from=1-3, to=2-3]
      \arrow["H\circ\eta"', from=1-1, to=2-1]
      \arrow["\cong", from=1-1, to=1-3]
    \end{tikzcd}
  \end{equation*}
\end{definition}

Note that this is only one of two possible definitions: We could
just as easily have chosen the 2-cell to go the other way.
This would have resulted
in a fundamentally different notion. Our definition of a
monad map is sometimes referred to as \emph{colax} whereas the
version with the reversed 2-cell $\psi : T'\circ H\to H\circ T$
is referred to as \emph{lax}. \cite{clarke2019}

We omit the proofs that composition of formal monad maps yields
a formal monad map. We thus have a category of formal monads
$\textbf{Monad}(\mathbf C)$. Let us investigate this category
a little.

In \ref{sec:spans} we constructed the bicategory $\Span(\mathcal
C)$. The structure of $\Span(\mathcal C)$ depends heavily on our choice
of $\mathcal C$. We are going to consider the case $\mathcal C=\Set$,
a category which we know to have limits. What are formal monads
in $\Span(\Set)$? They consist of a set $A$, a span
$(s,t):S\to A\times A$, and morphisms of spans
$\eta:A\to S$ and $\mu:S\times_A S\to S$.

Now recall that our underlying category is $\Set$, hence we are justified
in thinking about elements of objects. What do we know about elements
$f$ of $S$?
Firstly, the span $A\xleftarrow{s} S\xrightarrow{t} A$ provides
us with two elements $s(f)$ and $t(f)$ of $A$. Moreover, if we
have $f,g\in S$ such that $t(f) = s(g)$ then we have another element
$\mu(f,g)\in S$. Note that $s(\mu(f,g))=s(f)$ and $t(\mu(f,g))=t(g)$.
Moreover, each $a\in A$ provides us with a distinct $\eta(a)\in S$
which, due to $\eta$ being a span map, must satisfy
$s(\eta(a)) = t(\eta(a)) = a$.

It is now not hard to see that this provides us with the structure of
a small category. This correspondence between formal monads in
$\Span(\Set)$
and small categories is in fact a bijection. It is also possible to
translate a functor of small categories into a colax map of formal
monads. However, it is not the case that every colax map of formal
monads
corresponds to a functor and hence we fail to obtain an equivalence
of categories. However, we have the next best thing:

\begin{proposition}
  There is an inclusion of categories
  \begin{align*}
    \textbf{Cat}\longinc\textbf{Monad}(\Span(\Set)).
  \end{align*}
  which is a bijection on objects.
  \begin{proof}
    Consider a small category $\mathbb C$. The corresponding
    formal monad $M(\mathbb C)$ in $\Span(\Set)$ is given by the span
    \begin{align*}
      \text{Hom}_{\mathbb C} := \bigsqcup_{A,B\in\mathbb C} \Hom(A,B) &\to \text{Obj}_{\mathbb C} \times \text{Obj}_{\mathbb C} \\
      (f:A\to B) &\mapsto (A,B)
    \end{align*}
    and the span maps $\eta(A) := \identity_A$ and
    $\mu(f,g) := g\circ f$. Let us briefly verify the monad
    axioms. We calculate
    \begin{align*}
      \mu(f,\eta(B)) = \identity_B \circ f = f = f\circ\identity_A = \mu(\eta(A), f)
    \end{align*}
    and
    \begin{align*}
      \mu(\mu(f,g),h) = h\circ g\circ f =\mu(f,\mu(g,h)).
    \end{align*}
    It is straightforward to see what the
    inverse of this operation must be. Thus we have a bijection
    on objects.

    Somewhat less obvious is the behaviour of maps under this
    inclusion.
    Consider a functor $F:\mathbb C\to\mathbb D$. We want
    to construct a map of formal monads
    $M(F) : M(\mathbb C)\to M(\mathbb D)$. Write $F_0$ for the map
    on objects and $F_1$ for the map on maps. We then define
    $M(F)$ to consists of the span
    \begin{align*}
      H=(\identity, F_0):\text{Obj}_{\mathbb C} &\to \text{Obj}_{\mathbb C}\times\text{Obj}_{\mathbb D} \\
      A &\mapsto (A,FA)
    \end{align*}
    and the span map
    \begin{align*}
      \phi:\Hom_{\mathbb C} \times_{\text{Obj}} H &\to H\times_{\text{Obj}} \Hom_{\mathbb D} \\
      (f:A\to B, B) &\mapsto (A, Ff : FA \to FB)
    \end{align*}
    Let us make sure that $\phi$ is a span map. We begin by
    observing that elements of $H\times_{\text{Obj}}\Hom_{\mathbb C}$
    and $\Hom_{\mathbb D}\times_{\text{Obj}} H$ are fully determined
    by their projectons in $\Hom_{\mathbb C}$ and $\Hom_{\mathbb D}$,
    respectively.
    Under this isomorphism of spans, $\phi$ is just $F_1$ and the
    condition on formal monad maps becomes:
    \begin{equation*}
      % https://q.uiver.app/#q=WzAsNixbNCwxLCJcXHRleHR7SG9tfV97XFxtYXRoYmIgRH0iXSxbMywyLCJcXHRleHR7T2JqfV97XFxtYXRoYmIgRH0iXSxbMywwLCJcXHRleHR7T2JqfV97XFxtYXRoYmIgRH0iXSxbMSwwLCJcXHRleHR7T2JqfV97XFxtYXRoYmIgQ30iXSxbMSwyLCJcXHRleHR7T2JqfV97XFxtYXRoYmIgQ30iXSxbMCwxLCJcXHRleHR7SG9tfV97XFxtYXRoYmIgQ30iXSxbMCwxLCJ0IiwxXSxbMCwyLCJzIiwxXSxbNCwxLCJGXzAiLDFdLFs1LDMsInMiLDFdLFs1LDQsInQiLDFdLFszLDIsIkZfMCIsMV0sWzUsMCwiRl8xIiwxXV0=
      \begin{tikzcd}
  & {\text{Obj}_{\mathbb C}} && {\text{Obj}_{\mathbb D}} \\
        {\text{Hom}_{\mathbb C}} &&&& {\text{Hom}_{\mathbb D}} \\
                                 & {\text{Obj}_{\mathbb C}} && {\text{Obj}_{\mathbb D}}
                                 \arrow["t"{description}, from=2-5, to=3-4]
                                 \arrow["s"{description}, from=2-5, to=1-4]
                                 \arrow["{F_0}"{description}, from=3-2, to=3-4]
                                 \arrow["s"{description}, from=2-1, to=1-2]
                                 \arrow["t"{description}, from=2-1, to=3-2]
                                 \arrow["{F_0}"{description}, from=1-2, to=1-4]
                                 \arrow["{F_1}"{description}, from=2-1, to=2-5]
      \end{tikzcd}
    \end{equation*}
    which is just the condition $F(\Hom(A,B)) = \Hom(FA,FB)$ which
    is satisfied by any functor $F$.

    Finally, we need to verify that $M(F)$ is a map of formal monads.
    The conditions reduce to $\identity_{F_0 A} = F_1(\identity_A)$
    and $F_1(g\circ f) = F_1(g)\circ F_1(f)$, i.e. functoriality.

    In the interest of space, we leave it to the reader to verify
    that $M$ is a functor. Once that is done, it is rather
    straightforward to see that $M$ is injective on both objects
    and maps.
  \end{proof}
\end{proposition}

So which formal monad map correspond to functors? In other words,
what is the largest subcategory of
$\textbf{Monad}(\Span(\Set))$ is $\textbf{Cat}$
equivalent to? The necessary and sufficient condition on $(H,\phi)$
is that the span $H$ needs to be of the form $(f,g)$ for some
bijection $f$. More generally, this means that $H$ is a left-adjoint
in $\Span(\Set)$. \cite{lack2010} Similarly, the equivalence then
extends extends to the study of internal categories.
\cite{clarke2019}

\printbibliography

\end{document}
