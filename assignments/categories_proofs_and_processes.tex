\documentclass{article}
\usepackage{assignment}
\begin{document}
\title{Categories, Proofs, and Processes Mini Project}
\author{Franz Miltz}
\date{\today}
\maketitle

\section{Monoidal categories}

We begin this report by revisiting the notion of a monoidal category.

\begin{definition}
  \missingdefinition{monoidal category}
\end{definition}

\section{Monoidal categories are bicategories}

Category theory may be used to generalise many mathematical
concepts. Categories themselves are structures that have been
observed to arise very naturally in many areas of mathematics.
In this section we are going to define the notion of a bicategory.
This is a fairly involved task which we will try and motivate
in two distinct ways.

Firstly, its has long been observed that, rather than studying mathematical structures
by themselves, it is often fruitful to think about the structure
preserving maps between them. This observation motivates the very
existence of category theory. Taking this philosophy one step further
one might wonder whether, rather than studying structure preserving
maps between structures, one ought to study structure preserving
maps between structure preserving maps between structures. This is
in fact an approach that category theorists have adopted: Natural
transformations are structure preserving maps between functors which
themselves are structure preserving maps between categories. Yet,
the usual category of categories does not contain any natural
transformations. The bicategory of categories, functors, and natural
transformations addresses the problem.

Secondly, many genearlisations arise by regarding a certain mathematical
structure as a single-object category and extending the definition
to categories with more than one object. For example, groups are
single object groupoids, rings are single object additive categories,
monoids are single object categories (see \ref{sec:monoids_are_categories}). Observing this pattern leads to the study of enriched categories
which we will not delve into. However, we will see that bicategories
may be thought of as such a generalisation of monoidal categories.

Before we state the definition, we remind ourselves of some useful notation. Let $\mathcal C$ and $\mathcal D$ be categories. Recall that we
then have a product category $\mathcal C\times\mathcal D$ whose objects
are pairs $(C,D)$ for $C\in\mathcal C$ and $D\in\mathcal D$ and whose
maps are pairs $(f,g):(C,D)\to (C',D')$ where $f:C\to C'$ in $\mathcal C$
and $g:D\to D'$ in $\mathcal D$. If we now have functors
$F:\mathcal C\to\mathcal C'$ and $G:\mathcal D\to\mathcal D'$ there is
a functor $F\times G:\mathcal C\times\mathcal D\to\mathcal C'\times\mathcal D'$ which is given by $(x,y)\mapsto (Fx,Gy)$ for objects and maps alike.
In particular, for any functor $F : \mathcal C\to \mathcal C'$ we have
a functor $\mathcal C\times \mathcal D\to\mathcal C'\times\mathcal D$
which is the product of $F$ with the identity on $\mathcal D$.
We will denote this by $F\times\mathcal D$.

Moreover, if we have functors $F,G:\mathcal C\to\mathcal D$ and
a natural transformation $\varphi : F\Rightarrow G$ we may combine
all these components in a diagram

\begin{equation*}
  % https://q.uiver.app/#q=WzAsMixbMCwwLCJcXG1hdGhjYWwgQyJdLFsyLDAsIlxcbWF0aGNhbCBEIl0sWzAsMSwiRiIsMCx7ImN1cnZlIjotM31dLFswLDEsIkciLDIseyJjdXJ2ZSI6M31dLFsyLDMsIlxcdmFycGhpIiwxLHsic2hvcnRlbiI6eyJzb3VyY2UiOjIwLCJ0YXJnZXQiOjIwfX1dXQ==
  \begin{tikzcd}
    {\mathcal C} && {\mathcal D}
    \arrow[""{name=0, anchor=center, inner sep=0}, "F", curve={height=-18pt}, from=1-1, to=1-3]
    \arrow[""{name=1, anchor=center, inner sep=0}, "G"', curve={height=18pt}, from=1-1, to=1-3]
    \arrow["\varphi"{description}, shorten <=5pt, shorten >=5pt, Rightarrow, from=0, to=1]
  \end{tikzcd}
\end{equation*}

With this knowledge fresh in our memories, we are ready to follow
the definition of a bicategory:

\begin{definition}\label{def:bicategory}
  A \emph{bicategory} $\mathbf C$ consists of
  \begin{itemize}
    \item a class of objects $\text{Obj}_{\mathbf{C}}$ where
      we will write $A\in\mathbf C$ to mean $A\in\text{Obj}_{\mathbf C}$;
    \item for all $A,B\in\mathbf C$, a 1-category
      $\Hom_{\mathbf C}[A,B]$ with
      composition $\bullet$ whose objects $f:A\to B$ are called 1-cells and
      whose morphisms $\mathbf u:f\Rightarrow g$ are called 2-cells;
    \item for all $A\in\mathbf{C}$, a functor
      $\identity_A:\mathbf 1\to \Hom_{\mathbf C}[A,A]$ which we identify
      with a 1-cell $\identity_A : A\to A$;
    \item for all $A,B,C\in\mathbf{C}$, a composition functor
      \begin{align*}
        \circ_{A,B,C} : \Hom_{\mathbf C}[B,C]\times\Hom_{\mathbf C}[A,B] \to \Hom_{\mathbf C}[A,C]
      \end{align*}
      whose value on 1-cells $f\in\Hom_{\mathbf C}[A,B]$ and
      $g\in\Hom_{\mathbf C}[B,C]$ we will denote by
      $g\circ f := \circ_{A,B,C}(g,f)$;
    \item for all $A,B,C,D\in\mathbf C$, a natural transformation
      \begin{equation*}
        % https://q.uiver.app/#q=WzAsNCxbMCwwLCJcXHRleHR7SG9tfV97XFxtYXRoYmYgQ31bQyxEXVxcdGltZXNcXHRleHR7SG9tfV97XFxtYXRoYmYgQ31bQixDXVxcdGltZXNcXHRleHR7SG9tfV97XFxtYXRoYmYgQ31bQSxCXSJdLFswLDIsIlxcdGV4dHtIb219X3tcXG1hdGhiZiBDfVtCLERdXFx0aW1lc1xcdGV4dHtIb219X3tcXG1hdGhiZiBDfVtBLEJdIl0sWzMsMCwiXFx0ZXh0e0hvbX1fe1xcbWF0aGJmIEN9W0MsRF1cXHRpbWVzXFx0ZXh0e0hvbX1fe1xcbWF0aGJmIEN9W0EsQ10iXSxbMywyLCJcXHRleHR7SG9tfV97XFxtYXRoYmYgQ31bQSxEXSJdLFsyLDMsIlxcY2lyY197QSxDLER9Il0sWzEsMywiXFxjaXJjX3tBLEIsRH0iLDJdLFswLDEsIlxcY2lyY197QixDLER9XFx0aW1lc3tcXHRleHR7SG9tfV97XFxtYXRoYmYgQ31bQSxCXX0iLDJdLFswLDIsIntcXHRleHR7SG9tfV97XFxtYXRoYmYgQ31bQyxEXX1cXHRpbWVzXFxjaXJjX3tBLEIsQ30iXSxbNiw0LCJcXGFscGhhX3tBLEIsQyxEfSIsMCx7InNob3J0ZW4iOnsic291cmNlIjoyMCwidGFyZ2V0IjoyMH19XV0=
        \begin{tikzcd}
          {\text{Hom}_{\mathbf C}[C,D]\times\text{Hom}_{\mathbf C}[B,C]\times\text{Hom}_{\mathbf C}[A,B]} &&& {\text{Hom}_{\mathbf C}[C,D]\times\text{Hom}_{\mathbf C}[A,C]} \\
          \\
          {\text{Hom}_{\mathbf C}[B,D]\times\text{Hom}_{\mathbf C}[A,B]} &&& {\text{Hom}_{\mathbf C}[A,D]}
          \arrow[""{name=0, anchor=center, inner sep=0}, "{\circ_{A,C,D}}", from=1-4, to=3-4]
          \arrow["{\circ_{A,B,D}}"', from=3-1, to=3-4]
          \arrow[""{name=1, anchor=center, inner sep=0}, "{\circ_{B,C,D}\times{\text{Hom}_{\mathbf C}[A,B]}}"', from=1-1, to=3-1]
          \arrow["{{\text{Hom}_{\mathbf C}[C,D]}\times\circ_{A,B,C}}", from=1-1, to=1-4]
          \arrow["{\alpha_{A,B,C,D}}", shorten <=40pt, shorten >=40pt, Rightarrow, from=1, to=0]
        \end{tikzcd}
      \end{equation*}
      whose components at $f\in\Hom_{\mathbf C}[A,B]$,
      $g\in\Hom_{\mathbf C}[B,C]$, and $h\in\Hom_{\mathbf C}[C,D]$
      we denote by
      \begin{align*}
        \alpha_{f,g,h} : (h\circ g) \circ f \Rightarrow h\circ(g\circ f)
      \end{align*}
    \item for all $A,B\in\mathbf C$, a natural transformation
      \begin{equation*}
        % https://q.uiver.app/#q=WzAsMyxbMCwwLCJcXHRleHR7SG9tfV97XFxtYXRoYmYgQ31bQSxCXVxcdGltZXNcXG1hdGhiZiAxIl0sWzAsMiwiXFx0ZXh0e0hvbX1fe1xcbWF0aGJmIEN9W0EsQl1cXHRpbWVzXFx0ZXh0e0hvbX1fe1xcbWF0aGJmIEN9W0EsQV0iXSxbMiwyLCJcXHRleHR7SG9tfV97XFxtYXRoYmYgQ31bQSxCXSJdLFswLDEsIlxcdGV4dHtIb219X3tcXG1hdGhiZiBDfVtBLEJdXFx0aW1lcyBJX0EiLDJdLFsxLDIsIlxcY2lyY197QSxBLEJ9IiwyXSxbMCwyLCJcXGNvbmciLDFdLFszLDUsIlxccmhvX3tBLEJ9IiwyLHsic2hvcnRlbiI6eyJzb3VyY2UiOjIwLCJ0YXJnZXQiOjIwfX1dXQ==
        \begin{tikzcd}
          {\text{Hom}_{\mathbf C}[A,B]\times\mathbf 1} \\
          \\
          {\text{Hom}_{\mathbf C}[A,B]\times\text{Hom}_{\mathbf C}[A,A]} && {\text{Hom}_{\mathbf C}[A,B]}
          \arrow[""{name=0, anchor=center, inner sep=0}, "{\text{Hom}_{\mathbf C}[A,B]\times \identity_A}"', from=1-1, to=3-1]
          \arrow["{\circ_{A,A,B}}"', from=3-1, to=3-3]
          \arrow[""{name=1, anchor=center, inner sep=0}, "\cong"{description}, from=1-1, to=3-3]
          \arrow["{\rho_{A,B}}"', shorten <=11pt, shorten >=11pt, Rightarrow, from=0, to=1]
        \end{tikzcd}
      \end{equation*}
      whose component at $f\in\Hom_{\mathbf C}[A,B]$ we will denote by
      \begin{align*}
        \rho_f : f \circ \identity_A \Rightarrow f;
      \end{align*}
    \item for all $A,B\in\mathbf C$, a natural transformation
      \begin{equation*}
        % https://q.uiver.app/#q=WzAsMyxbMCwwLCJcXG1hdGhiZiAxXFx0aW1lc1xcdGV4dHtIb219X3tcXG1hdGhiZiBDfVtBLEJdIl0sWzAsMiwiXFx0ZXh0e0hvbX1fe1xcbWF0aGJmIEN9W0IsQl1cXHRpbWVzXFx0ZXh0e0hvbX1fe1xcbWF0aGJmIEN9W0EsQl0iXSxbMiwyLCJcXHRleHR7SG9tfV97XFxtYXRoYmYgQ31bQSxCXSJdLFswLDEsIklfQlxcdGltZXNcXHRleHR7SG9tfV97XFxtYXRoYmYgQ31bQSxCXSIsMl0sWzEsMiwiXFxjaXJjX3tBLEIsQn0iLDJdLFswLDIsIlxcY29uZyIsMV0sWzMsNSwiXFxsYW1iZGFfe0EsQn0iLDIseyJzaG9ydGVuIjp7InNvdXJjZSI6MjAsInRhcmdldCI6MjB9fV1d
        \begin{tikzcd}
          {\mathbf 1\times\text{Hom}_{\mathbf C}[A,B]} \\
          \\
          {\text{Hom}_{\mathbf C}[B,B]\times\text{Hom}_{\mathbf C}[A,B]} && {\text{Hom}_{\mathbf C}[A,B]}
          \arrow[""{name=0, anchor=center, inner sep=0}, "{\identity_B\times\text{Hom}_{\mathbf C}[A,B]}"', from=1-1, to=3-1]
          \arrow["{\circ_{A,B,B}}"', from=3-1, to=3-3]
          \arrow[""{name=1, anchor=center, inner sep=0}, "\cong"{description}, from=1-1, to=3-3]
          \arrow["{\lambda_{A,B}}"', shorten <=12pt, shorten >=12pt, Rightarrow, from=0, to=1]
        \end{tikzcd}
      \end{equation*}
      whose component at $f\in\Hom_{\mathbf C}[A,B]$ we will denote by
      \begin{align*}
        \lambda_f : \identity_B\circ f \Rightarrow f;
      \end{align*}
  \end{itemize}
  such that
  \begin{itemize}
    \item for all 1-cells
      $ A \xlongrightarrow{f} B \xlongrightarrow{g} C \xlongrightarrow{h} D \xlongrightarrow{k} E $
      the following commutes
      \begin{equation}\label{eq:pentagon}
        % https://q.uiver.app/#q=WzAsNSxbMCwwLCIoKGtcXGNpcmMgaClcXGNpcmMgZylcXGNpcmMgZiJdLFswLDIsIihrXFxjaXJjIGgpXFxjaXJjKGdcXGNpcmMgZikiXSxbNCwyLCJrXFxjaXJjKGhcXGNpcmMoZ1xcY2lyYyBmKSkiXSxbMiwwLCIoa1xcY2lyYyhoXFxjaXJjIGcpKVxcY2lyYyBmIl0sWzQsMCwia1xcY2lyYyAoKGhcXGNpcmMgZylcXGNpcmMgZikiXSxbMCwxLCJcXGFscGhhX3tmLGcsa1xcY2lyYyBofSIsMix7ImxldmVsIjoyfV0sWzEsMiwiXFxhbHBoYV97Z1xcY2lyYyBmLGgsa30iLDIseyJsZXZlbCI6Mn1dLFs0LDIsImtcXGNpcmMgXFxhbHBoYV97ZixnLGh9IiwwLHsibGV2ZWwiOjJ9XSxbMCwzLCJcXGFscGhhX3tnLGgsa31cXGNpcmMgZiIsMCx7ImxldmVsIjoyfV0sWzMsNCwiXFxhbHBoYV97ZixnXFxjaXJjIGgsa30iLDAseyJsZXZlbCI6Mn1dXQ==
        \begin{tikzcd}
          {((k\circ h)\circ g)\circ f} && {(k\circ(h\circ g))\circ f} && {k\circ ((h\circ g)\circ f)} \\
          \\
          {(k\circ h)\circ(g\circ f)} &&&& {k\circ(h\circ(g\circ f))}
          \arrow["{\alpha_{f,g,k\circ h}}"', Rightarrow, from=1-1, to=3-1]
          \arrow["{\alpha_{g\circ f,h,k}}"', Rightarrow, from=3-1, to=3-5]
          \arrow["{k\circ \alpha_{f,g,h}}", Rightarrow, from=1-5, to=3-5]
          \arrow["{\alpha_{g,h,k}\circ f}", Rightarrow, from=1-1, to=1-3]
          \arrow["{\alpha_{f,g\circ h,k}}", Rightarrow, from=1-3, to=1-5]
        \end{tikzcd}
      \end{equation}
    \item for all 1-cells $A\xlongrightarrow{f}B\xlongrightarrow{g} C$
      the following commutes:
      \begin{equation}\label{eq:triangle}
        % https://q.uiver.app/#q=WzAsMyxbMCwwLCIoZ1xcY2lyYyBcXHRleHR7aWR9X0IpXFxjaXJjIGYiXSxbMiwwLCJnXFxjaXJjKFxcdGV4dHtpZH1fQlxcY2lyYyBmKSJdLFsxLDEsImdcXGNpcmMgZiJdLFswLDIsIlxccmhvX2cgXFxjaXJjIGYiLDIseyJsZXZlbCI6Mn1dLFsxLDIsImdcXGNpcmMgXFxsYW1iZGFfZiIsMCx7ImxldmVsIjoyfV0sWzAsMSwiXFxhbHBoYV97ZixcXHRleHR7aWR9X0IsZ30iLDAseyJsZXZlbCI6Mn1dXQ==
        \begin{tikzcd}
          {(g\circ \text{id}_B)\circ f} && {g\circ(\text{id}_B\circ f)} \\
                                        & {g\circ f}
                                        \arrow["{\rho_g \circ f}"', Rightarrow, from=1-1, to=2-2]
                                        \arrow["{g\circ \lambda_f}", Rightarrow, from=1-3, to=2-2]
                                        \arrow["{\alpha_{f,\text{id}_B,g}}", Rightarrow, from=1-1, to=1-3]
        \end{tikzcd}
      \end{equation}
  \end{itemize}
\end{definition}

Throughout this definition we have been rigorous in identifying the
precise functors and natural transformation involved. Such a level
of detail is usually unnecessary and, moreover, tends to obscure
interesting details. For this reason, we will drop subscripts whenever
it is possible to infer them from context.

Let us now see how one might define $\textbf{Cat}$, the bicategory with
objects all small categories. For small categories $\mathbb C$ and
$\mathbb D$, the hom-category $\Hom[\mathbb C,\mathbb D]$ is just
the functor category $[\mathbb C,\mathbb D]$ and the functor
$\identity_{\mathbb C} : \mathbf 1 \to [\mathbb C,\mathbb C]$
maps to the identity functor $\text{Id}:\mathbb C\to\mathbb C$.
We note that composition of functors is associative and respects
identities as units, hence we may take $\alpha$, $\lambda$, and $\rho$
to be the identities. Thus the conditions (\ref{eq:pentagon}) and
(\ref{eq:triangle}) are trivially satisfied.

The previous example is a little bit underwhelming: All the structure
and conditions related to associativity and unitality is trivial.
Let us construct a less trivial example:

\begin{definition}
  Let $\mathcal C$ be a category. The bicategory $\text{Span}(\mathcal C)$
  consists of
  \begin{itemize}
    \item x
  \end{itemize}
\end{definition}

We need to check that this is well-defined. In particular, one needs to
verify the coherence conditions. \missingproof

We will come back to this bicategory later on. For now, let us now make sure
that bicategories with a single object are indeed the same as monoidal
categories. Consider a monoidal
category $(\mathcal C,\otimes, I,\alpha,\lambda,\rho)$.
Then the corresponding bicategory $\mathbf C$ consists of
\begin{itemize}
  \item a single object $*$;
  \item the unique hom-category $\Hom[*,*]=\mathcal C$;
  \item the identity functor $\identity_*$ mapping to $I$;
  \item the composition functor
    \begin{align*}
      \circ_{*,*,*} : \Hom[*,*]\times\Hom[*,*]\to\Hom[*,*]
    \end{align*}
    given by $\otimes : \mathcal C\times\mathcal C\to\mathcal C$;
  \item the natural transformations $\alpha_{*,*,*}=\alpha$,
    $\rho_{*,*}=\rho$ and $\lambda_{*,*}=\lambda$.
\end{itemize}
Thus the 1-cells of $\mathbf C$ are the objects of $\mathcal C$ and
their composition is the monoidal product. Thus, for $A,B,C,D\in\mathcal C$, the conditions (\ref{eq:pentagon}) and (\ref{eq:triangle}) become

\begin{equation*}
  % https://q.uiver.app/#q=WzAsNSxbMCwwLCIoKGtcXGNpcmMgaClcXGNpcmMgZylcXGNpcmMgZiJdLFswLDIsIihrXFxjaXJjIGgpXFxjaXJjKGdcXGNpcmMgZikiXSxbNCwyLCJrXFxjaXJjKGhcXGNpcmMoZ1xcY2lyYyBmKSkiXSxbMiwwLCIoa1xcY2lyYyhoXFxjaXJjIGcpKVxcY2lyYyBmIl0sWzQsMCwia1xcY2lyYyAoKGhcXGNpcmMgZylcXGNpcmMgZikiXSxbMCwxLCJcXGFscGhhX3tmLGcsa1xcY2lyYyBofSIsMix7ImxldmVsIjoyfV0sWzEsMiwiXFxhbHBoYV97Z1xcY2lyYyBmLGgsa30iLDIseyJsZXZlbCI6Mn1dLFs0LDIsImtcXGNpcmMgXFxhbHBoYV97ZixnLGh9IiwwLHsibGV2ZWwiOjJ9XSxbMCwzLCJcXGFscGhhX3tnLGgsa31cXGNpcmMgZiIsMCx7ImxldmVsIjoyfV0sWzMsNCwiXFxhbHBoYV97ZixnXFxjaXJjIGgsa30iLDAseyJsZXZlbCI6Mn1dXQ==
  \begin{tikzcd}
    {((A\otimes B)\otimes C)\otimes D} && {(A\otimes(B\otimes C))\otimes D} && {A\otimes ((B\otimes C)\otimes D)} \\
    \\
    {(A\otimes B)\otimes(C\otimes D)} &&&& {A\otimes(B\otimes(C\otimes D))}
    \arrow["{\alpha}"', Rightarrow, from=1-1, to=3-1]
    \arrow["{\alpha}"', Rightarrow, from=3-1, to=3-5]
    \arrow["{A\otimes \alpha}", Rightarrow, from=1-5, to=3-5]
    \arrow["{\alpha\otimes D}", Rightarrow, from=1-1, to=1-3]
    \arrow["{\alpha}", Rightarrow, from=1-3, to=1-5]
  \end{tikzcd}
\end{equation*}
and
\begin{equation*}
  % https://q.uiver.app/#q=WzAsMyxbMCwwLCIoZ1xcY2lyYyBcXHRleHR7aWR9X0IpXFxjaXJjIGYiXSxbMiwwLCJnXFxjaXJjKFxcdGV4dHtpZH1fQlxcY2lyYyBmKSJdLFsxLDEsImdcXGNpcmMgZiJdLFswLDIsIlxccmhvX2cgXFxjaXJjIGYiLDIseyJsZXZlbCI6Mn1dLFsxLDIsImdcXGNpcmMgXFxsYW1iZGFfZiIsMCx7ImxldmVsIjoyfV0sWzAsMSwiXFxhbHBoYV97ZixcXHRleHR7aWR9X0IsZ30iLDAseyJsZXZlbCI6Mn1dXQ==
  \begin{tikzcd}
    {(A\otimes I)\otimes B} && {A\otimes(I\otimes B)} \\
                            & {A\otimes B}
                            \arrow["{\rho \otimes B}"', Rightarrow, from=1-1, to=2-2]
                            \arrow["{A\otimes \lambda}", Rightarrow, from=1-3, to=2-2]
                            \arrow["{\alpha}", Rightarrow, from=1-1, to=1-3]
  \end{tikzcd}
\end{equation*}
But these are exactly the coherence conditions of monoidal categories! Thus they are obviously satisfied. It is not hard to see that it is easy to
go in the other direction, i.e. if $\mathbf C$ is a bicategory with a single object $*$ then $\Hom[*,*]$ has a monoidal structure given
by composition and the identity 1-cell.

There is more to be said on this subject: One may define pseudofunctors,
i.e. maps of bicategories, and hence a category of bicategories
$\textbf{Bicat}$. In particular, there is a full subcategory
$\textbf{1Bicat}\hookrightarrow\textbf{Bicat}$ of single-object bicategories. Moreover, there is the notion of monoidal functors,
i.e. maps of monoidal categories. These may be used to form the
category $\textbf{MonCat}$ of monoidal categories. One then obtains
an equivalence $\textbf{1Bicat}\cong\textbf{MonCat}$. This statement
is stronger than what we have discussed in this section in the sense
that it proves that there truly is no difference between single-object
bicategories and monoidal categories because their structure preserving
morphisms agree. Unfortunately,
rigorous pursuit of this idea is beyond the scope of this
report.

\section{Monoids are categories}

We have seen that one may generalise monoidal categories to bicategories.
Let us now turn our attention to a simpler example. It turns out that
categories are a generalisation of monoids.
Recall that a monoid consists of an underlying set equipped with
a unital associative multiplication. More precisely:

\begin{definition}
  A \emph{monoid} $(M,e,m)$ consists of
  a set $M$,
  a distinguished element $e\in M$, and
  a map $m : M\times M\to M$
  such that, for all $x,y,z\in M$,
  $m(e,x) = x = m(x,e)$ and $m(x,m(y,z)) = m(m(x,y),z)$.
\end{definition}

Now a map of monoids is a map of underlying sets that preserves
identities and multiplication:

\begin{definition}
  A \emph{map of monoids} $f:(M,e,m)\to(M',e',m')$ is a map
  $f:M\to M'$ such that, for all $x,y\in M$,
  $f(m(x,y)) = m(f(x),f(y))$ and $f(e)=e'$.
\end{definition}

It is easy to check that composition of monoid maps yields another
monoid map. Hence we have a category of monoids which we will
denote by $\textbf{Mon}$.
On the other hand, we have the full subcategory
$\textbf{1Cat}\hookrightarrow\textbf{Cat}$ of small categories with
a unique object. That is, $\textbf{1Cat}$ contains all small single-object
categories and all functors between them.

Note that the two conditions that we impose on the multiplication of a monoid
are associativity and unitality. Those are exactly the conditions we require
the composition of a category to satisfy. We may thus consider a monoid
as a category:

\begin{definition}
  Let $(M,e,m)$ be a monoid. The induced category $I(M,e,m)$ has
  has a single object $*$ with hom-set $\Hom(*,*):= M$, identity
  $\identity_* := e$, and composition $x\circ y := m(x,y)$ for
  $x,y\in\Hom(*,*)$.

  Let $f:(M,e,m) \to (M',e',m')$ be a monoid map. The induced functor
  $If:I(M,e,m)\to I(M',e',m')$ is given by the map on objects $*\mapsto *$
  and the map on hom-sets $x \mapsto f(x)$ for $x\in\Hom(*,*)$.
\end{definition}

Let us make sure that this is well-defined. In particular, $I(M,e,m)$
is a category as, for $x,y,z\in\Hom(*,*)$,
\begin{align*}
  x\circ(y\circ z) = m(x,m(y,z)) = m(m(x,y),z) = (x\circ y)\circ z
\end{align*}
and
\begin{align*}
  x \circ \identity_* = m(x,e) = x = m(e,x) = \identity_* \circ x.
\end{align*}
Further, $If$ is a functor as $If(\identity_*) = f(e) = e' = \identity_*$
and, for $x,y\in\Hom(*,*)$,
\begin{align*}
  If(x\circ y) = f(m(x,y)) = m(f(x),f(y)) = If(x) \circ If(y).
\end{align*}
Our notation suggests that we have defined a functor from
monoids to single object categories. This is indeed the case.
\begin{lemma}
  $I:\textbf{Mon}\to\textbf{1Cat}$ is a functor.
  \begin{proof}
    Consider a monoid map $f:(M,e,m)\to(M',e',m')$. We then
    have $If(\identity_*) = f(e) = e' = \identity_*$, i.e. $I$
    preserves identities. Moreover, for $x,y,z\in M$,
    $If(x\circ y) = f(m(x,y)) = m(f(x),f(y)) = If(x)\circ If(y)$ by
    definition of a monoid map. Thus $I$ is compatible with composition,
    too.
  \end{proof}
\end{lemma}

We may also go the other way:

\begin{lemma}
  Let $\mathbb C,\mathbb D$ be small categories with unique objects
  $X\in\mathbb{C}$ and $Y\in\mathbb D$ and let $F:\mathbb{C}\to\mathbb D$
  be a functor. Then $J\mathbb C := (\Hom(X,X),\identity_X,\circ)$ is a
  monoid and $x\mapsto Fx$ defines a monoid map $JF : J\mathbb{C}\to J\mathbb D$.
  \begin{proof}
    We verify that $J\mathbb{C}$ is a monoid. For $x\in J\mathbb{C}$
    we find
    \begin{align*}
      m(e,x) = \identity_X \circ x = x = x \circ \identity_X = m(x,e).
    \end{align*}
    Similarly, for $x,y,z\in J\mathbb{C}$,
    \begin{align*}
      m(x,m(y,z)) = x\circ (y\circ z) = (x\circ y)\circ z = m(m(x,y),z).
    \end{align*}
    Moreover, for $x,y\in J\mathbb{C}$,
    \begin{align*}
      JF(m(x,y)) = F(x\circ y) = Fx \circ Fy = m(JF(x), JF(y))
    \end{align*}
    and $JF(e) = F(\identity_X) = \identity_Y = e'$. Thus
    $JF$ is a monoid map.
  \end{proof}
\end{lemma}

Once again, this is a functor:

\begin{lemma}
  $J:\textbf{1Cat}\to\textbf{Mon}$ is a functor.
  \begin{proof}
    If $\text{Id}:\mathbb C\to\mathbb C$ is the identity functor
    then $J(\text{Id})$ must be the identity map which is the
    identity on $J\mathbb C$. Moreover, if we have functors
    \begin{align*}
      \mathbb C \xlongrightarrow{F} \mathbb D \xlongrightarrow{G} \mathbb E
    \end{align*}
    in $\textbf{1Cat}$ then, for $x\in J\mathbb C$,
    \begin{align*}
      JG\circ JF(x) = G(F(x)) = (G\circ F)(x) = J(G\circ F)(x).
    \end{align*}
    Thus $J$ respects composition and hence defines a functor.
  \end{proof}
\end{lemma}

One might think that $I$ and $J$ are mutually inverse. However, this
is not so: Consider $\mathbb C,\mathbb C'\in\textbf{1Cat}$ which are
isomorphic through $X\mapsto Y$ on objects and $x\mapsto x$ on maps.
Then $\mathbb{C}\neq\mathbb{C}'$ but $J\mathbb{C}=J\mathbb{C}'$.
This is a common phenomenon in category theory. Hence one rarely
speaks about isomorphic categories but rather about equivalent ones:

\begin{definition}
  An \emph{equivalence of categories $\mathbb C\cong\mathbb D$} consists
  of functors $F:\mathbb{C}\to\mathbb D$ and $G:\mathbb D\to\mathbb C$
  and natural isormopshims $\eta : \text{Id}\to GF$ and
  $\epsilon:FG\to \text{Id}$ as in the diagram
  \begin{equation*}
    % https://q.uiver.app/#q=WzAsNCxbMCwxLCJcXG1hdGhiYiBDIl0sWzEsMCwiXFxtYXRoYmIgRCJdLFsyLDEsIlxcbWF0aGJiIEMiXSxbMywwLCJcXG1hdGhiYiBEIl0sWzAsMSwiRiJdLFsxLDIsIkciLDFdLFswLDIsIiIsMix7ImxldmVsIjoyLCJzdHlsZSI6eyJoZWFkIjp7Im5hbWUiOiJub25lIn19fV0sWzEsMywiIiwyLHsibGV2ZWwiOjIsInN0eWxlIjp7ImhlYWQiOnsibmFtZSI6Im5vbmUifX19XSxbMiwzLCJGIiwyXSxbNiwxLCJcXGV0YSIsMCx7InNob3J0ZW4iOnsic291cmNlIjoyMH19XSxbMiw3LCJcXGVwc2lsb24iLDIseyJzaG9ydGVuIjp7InRhcmdldCI6MjB9fV1d
    \begin{tikzcd}
  & {\mathbb D} && {\mathbb D} \\
      {\mathbb C} && {\mathbb C}
      \arrow["F", from=2-1, to=1-2]
      \arrow["G"{description}, from=1-2, to=2-3]
      \arrow[""{name=0, anchor=center, inner sep=0}, Rightarrow, no head, from=2-1, to=2-3]
      \arrow[""{name=1, anchor=center, inner sep=0}, Rightarrow, no head, from=1-2, to=1-4]
      \arrow["F"', from=2-3, to=1-4]
      \arrow["\eta", shorten <=3pt, Rightarrow, from=0, to=1-2]
      \arrow["\epsilon"', shorten >=3pt, Rightarrow, from=2-3, to=1]
    \end{tikzcd}
  \end{equation*}
\end{definition}

That is, functors $F:\mathbb{C}\to\mathbb D:G$ form an equivalence
if they are inverses up to isomorphism. Our functors $I$ and $J$
satisfy this condition:

\begin{proposition}
  There is an equivalence of categories $\textbf{Mon}\cong\textbf{1Cat}$.
  \begin{proof}
    It is straightforward to verify
    $JI=\text{Id}:\textbf{Mon}\to\textbf{Mon}$. We need
    to define a natural isormophism $\epsilon:IJ\to\text{Id}$.
    Consider $X\in\mathbb C\in\textbf{1Cat}$. Define
    $\epsilon_{\mathbb{C}}(*)=X$ and $\epsilon_{\mathbb{C}}(x) = x$ for
    $x\in\Hom(X,X)$. This is natural in $\mathbb{C}$ as, for all
    $F:\mathbb{C}\to\mathbb D$ in $\textbf{1Cat}$,
    $\epsilon_{\mathbb D}(IJF(*))$ and $IJF(\epsilon_{\mathbb{C}})$
    must be the unique object in $\mathbb D$ and on arrows $\varepsilon$
    is the identity so there is nothing to check. Moreover,
    $\epsilon_{\mathbb{C}}$ is a bijection on objects and arrows so
    it is an isomorphism.
  \end{proof}
\end{proposition}

\section{Monads are monoids}

We have seen that categories are one generalisation of monoids. However,
one may go in a different direction. Observe that an element $e\in M$
may be identified with a map $e : 1 \to M$ where $1=\left\lbrace{*}\right\rbrace$ is a singleton set and we abuse notation to write $e(*) = e$.
One may now ask what is special about $\textbf{Set}$? We could define
a monoid to consists of maps $m:M\times M\to M$ and $e:1\to M$ in
any cartesian category. We are going to go one step further and
replace the cartesian structure with an arbitrary monoidal structure:

\begin{definition}
  Let $(\mathcal C,I,\otimes)$ be a monoidal category. A
  $\mathcal C$-monoid $(M,e,m)$ consists of
  \begin{itemize}
    \item an object $M\in\mathcal C$;
    \item a map $e:I\to\mathcal M$;
    \item a map $m:M\otimes M\to M$ in $\mathcal C$
  \end{itemize}
  such that the following commute:
  \begin{equation}\label{eq:monoid_axioms}
    % https://q.uiver.app/#q=WzAsNCxbMCwwLCJJXFxvdGltZXMgTSJdLFsyLDAsIk1cXG90aW1lcyBNIl0sWzIsMiwiTSJdLFs0LDAsIk1cXG90aW1lcyBJIl0sWzAsMSwiZVxcb3RpbWVzIE0iXSxbMSwyLCJtIl0sWzAsMiwiXFxsYW1iZGEiLDJdLFszLDEsIk1cXG90aW1lcyBlIiwyXSxbMywyLCJcXHJobyJdXQ==
    \begin{tikzcd}
      {I\otimes M} && {M\otimes M} && {M\otimes I} \\
      \\
                   && M
                   \arrow["{e\otimes M}", from=1-1, to=1-3]
                   \arrow["m", from=1-3, to=3-3]
                   \arrow["\lambda"', from=1-1, to=3-3]
                   \arrow["{M\otimes e}"', from=1-5, to=1-3]
                   \arrow["\rho", from=1-5, to=3-3]
    \end{tikzcd}, \hspace{1cm}
    % https://q.uiver.app/#q=WzAsNSxbMCwwLCIoTVxcb3RpbWVzIE0pXFxvdGltZXMgTSJdLFswLDEsIk1cXG90aW1lcyBNIl0sWzEsMiwiTSJdLFsyLDEsIk1cXG90aW1lcyBNIl0sWzIsMCwiTVxcb3RpbWVzKE1cXG90aW1lcyBNKSJdLFswLDEsIm1cXG90aW1lcyBNIiwyXSxbMSwyLCJtIiwyXSxbMywyLCJtIl0sWzAsNCwiXFxhbHBoYSJdLFs0LDMsIk1cXG90aW1lcyBtIl1d
    \begin{tikzcd}
      {(M\otimes M)\otimes M} && {M\otimes(M\otimes M)} \\
      {M\otimes M} && {M\otimes M} \\
                   & M
                   \arrow["{m\otimes M}"', from=1-1, to=2-1]
                   \arrow["m"', from=2-1, to=3-2]
                   \arrow["m", from=2-3, to=3-2]
                   \arrow["\alpha", from=1-1, to=1-3]
                   \arrow["{M\otimes m}", from=1-3, to=2-3]
    \end{tikzcd}
  \end{equation}
\end{definition}

Consider the monoidal structure on $\Set$ given by the cartesian structure.
Then a $\Set$-monoid is just a monoid in the usual sense: We have a
set $M$, a map $e:1\to M$, i.e. an element $e\in M$, and a multiplication
$m:M\times M\to M$. The conditions \ref{eq:monoid_axioms} are just
unitality and associativity, respectively.

\begin{definition}
  A \emph{map of $(\mathcal C,I,\otimes)$-monoids}
  $f:(M,e,m)\to(M',e',m')$ is a map $f:M\to M'$ in $\mathcal C$
  such that the following commute:
  \begin{equation}\label{eq:monoid_map_axioms}
    % https://q.uiver.app/#q=WzAsNCxbMCwwLCJNXFxvdGltZXMgTSJdLFsyLDAsIk0nXFxvdGltZXMgTSciXSxbMCwxLCJNIl0sWzIsMSwiTSciXSxbMCwxLCJmXFxvdGltZXMgZiJdLFsxLDMsIm0nIl0sWzAsMiwibSIsMl0sWzIsMywiZiIsMl1d
    \begin{tikzcd}
      {M\otimes M} && {M'\otimes M'} \\
      M && {M'}
      \arrow["{f\otimes f}", from=1-1, to=1-3]
      \arrow["{m'}", from=1-3, to=2-3]
      \arrow["m"', from=1-1, to=2-1]
      \arrow["f"', from=2-1, to=2-3]
    \end{tikzcd}\hspace{1cm}
    % https://q.uiver.app/#q=WzAsMyxbMCwxLCJNIl0sWzIsMSwiTSciXSxbMSwwLCJJIl0sWzIsMCwiZSIsMl0sWzIsMSwiZSciXSxbMCwxLCJmIiwyXV0=
    \begin{tikzcd}
  & I \\
      M && {M'}
      \arrow["e"', from=1-2, to=2-1]
      \arrow["{e'}", from=1-2, to=2-3]
      \arrow["f"', from=2-1, to=2-3]
    \end{tikzcd}
  \end{equation}
\end{definition}

It is easily verified that the composite of monoid maps
\begin{align*}
  (M,e,m)\xlongrightarrow{f}(M',e',m')\xlongrightarrow{g}(M'',e'',m'')
\end{align*}
is itself a monoid map as
\begin{equation*}
  % https://q.uiver.app/#q=WzAsNixbMCwxLCJNIl0sWzIsMSwiTSciXSxbMiwwLCJNJ1xcb3RpbWVzIE0nIl0sWzAsMCwiTVxcb3RpbWVzIE0iXSxbNCwwLCJNJydcXG90aW1lcyBNJyciXSxbNCwxLCJNJyciXSxbMCwxLCJmIiwyXSxbMiwxLCJtJyJdLFszLDIsImZcXG90aW1lcyBmIl0sWzMsMCwibSIsMl0sWzIsNCwiZ1xcb3RpbWVzIGciXSxbMSw1LCJnIiwyXSxbNCw1LCJtJyciLDFdLFszLDQsIihnXFxjaXJjIGYpXFxvdGltZXMgKGdcXGNpcmMgZikiLDAseyJjdXJ2ZSI6LTV9XV0=
  \begin{tikzcd}
    {M\otimes M} && {M'\otimes M'} && {M''\otimes M''} \\
    M && {M'} && {M''}
    \arrow["f"', from=2-1, to=2-3]
    \arrow["{m'}", from=1-3, to=2-3]
    \arrow["{f\otimes f}", from=1-1, to=1-3]
    \arrow["m"', from=1-1, to=2-1]
    \arrow["{g\otimes g}", from=1-3, to=1-5]
    \arrow["g"', from=2-3, to=2-5]
    \arrow["{m''}"{description}, from=1-5, to=2-5]
    \arrow["{(g\circ f)\otimes (g\circ f)}", curve={height=-30pt}, from=1-1, to=1-5]
  \end{tikzcd}
\end{equation*}
commutes by functoriality of $\otimes:\mathcal C\times\mathcal C\to\mathcal C$. If $f=\identity$, $M=M'$, $e=e'$, and $m=m'$ then the conditions
are trivially satisfied. Hence we have a category of $(\mathcal C,I,\otimes)$-monoids $\textbf{Mon}(\mathcal C,I,\otimes)$.

Other than generalisation for the sake of generalisation, it is not
immediately obvious what we have achieved. However, one need only
consider the category of $k$-vector spaces $\textbf{Vect}_k$ with
its monoidal structure given by the tensor product. A
$\textbf{Vect}_k$-monoid is a $k$-algebra! In particular, we do
not require any additional compatibility conditions between the
monoid multiplication and the scalar product of a vector space.

A less obvious example of monoid objects are monads. Recall
the definition:
\begin{definition}
  Let $\mathcal C$ be a category. A \emph{$\mathcal C$-monad} consists of
  \begin{itemize}
    \item a functor $T:\mathcal C\to\mathcal C$;
    \item a natural transformation $\eta:\text{Id}\to T$;
    \item a natural transformation $\mu:T^2\to T$
  \end{itemize}
  such that the following commute:
  \begin{equation}\label{eq:monad_laws}
    % https://q.uiver.app/#q=WzAsNCxbMiwwLCJUXjIiXSxbMiwyLCJUIl0sWzAsMCwiVCJdLFswLDIsIlReMiJdLFswLDEsIlxcbXUiXSxbMiwwLCJUXFxldGEiXSxbMiwxLCIiLDEseyJsZXZlbCI6Miwic3R5bGUiOnsiaGVhZCI6eyJuYW1lIjoibm9uZSJ9fX1dLFsyLDMsIlxcZXRhIFQiLDJdLFszLDEsIlxcbXUiLDJdXQ==
    \begin{tikzcd}
      T && {T^2} \\
      \\
      {T^2} && T
      \arrow["\mu", from=1-3, to=3-3]
      \arrow["T\eta", from=1-1, to=1-3]
      \arrow[Rightarrow, no head, from=1-1, to=3-3]
      \arrow["{\eta T}"', from=1-1, to=3-1]
      \arrow["\mu"', from=3-1, to=3-3]
    \end{tikzcd}\hspace{1cm}
    % https://q.uiver.app/#q=WzAsNCxbMCwwLCJUXjMiXSxbMiwwLCJUXjIiXSxbMCwyLCJUXjIiXSxbMiwyLCJUIl0sWzAsMSwiVFxcbXUiXSxbMCwyLCJcXG11IFQiLDJdLFsyLDMsIlxcbXUiLDJdLFsxLDMsIlxcbXUiXV0=
    \begin{tikzcd}
      {T^3} && {T^2} \\
      \\
      {T^2} && T
      \arrow["T\mu", from=1-1, to=1-3]
      \arrow["{\mu T}"', from=1-1, to=3-1]
      \arrow["\mu"', from=3-1, to=3-3]
      \arrow["\mu", from=1-3, to=3-3]
    \end{tikzcd}
  \end{equation}
\end{definition}

Here the natural transormation $\eta T : T\to T^2$ has
components $(\eta T)_X = \eta_{TX}$ and similarly
for $\mu T$.

We have seen many examples of monads. Hence we
are ready to move on to defining the category of monads.
As usual, a map of monads is a map of underlying objects that plays
well with the additional structure:

\begin{definition}
  A \emph{map of $\mathcal C$-monads} $\phi:(T,\eta,\mu)\to(T',\eta',\mu')$
  is a natural transformation $\phi:T\to T'$ such that
  \begin{equation}\label{eq:monad_map_laws}
    % https://q.uiver.app/#q=WzAsNyxbMCwwLCJUXjIiXSxbMCwxLCJUIl0sWzIsMSwiVCciXSxbMiwwLCIoVCcpXjIiXSxbNSwwLCJcXHRleHR7SWR9Il0sWzQsMSwiVCJdLFs2LDEsIlQnIl0sWzAsMywiXFxwaGleMiJdLFsxLDIsIlxccGhpIl0sWzAsMSwiXFxtdSIsMl0sWzMsMiwiXFxtdSciXSxbNCw1LCJcXGV0YSIsMl0sWzUsNiwiXFxwaGkiLDJdLFs0LDYsIlxcZXRhJyJdXQ==
    \begin{tikzcd}
      {T^2} && {(T')^2} &&& {\text{Id}} \\
      T && {T'} && T && {T'}
      \arrow["{\phi^2}", from=1-1, to=1-3]
      \arrow["\phi", from=2-1, to=2-3]
      \arrow["\mu"', from=1-1, to=2-1]
      \arrow["{\mu'}", from=1-3, to=2-3]
      \arrow["\eta"', from=1-6, to=2-5]
      \arrow["\phi"', from=2-5, to=2-7]
      \arrow["{\eta'}", from=1-6, to=2-7]
    \end{tikzcd}
  \end{equation}
\end{definition}

Here the natural transformation $\phi^2 : T^2 \to (T')^2$ has
components
\begin{align*}
  T^2 X = T(TX)
  \xlongrightarrow{\phi_{TX}} T'(TX)
  \xlongrightarrow{T'\phi_X} T'(T'X) = (T')^2 X.
\end{align*}
It is straightforward to verify that the composition of monad
maps yields a monad map. Hence we have a category $\textbf{Monad}(\mathcal C)$.

Now recall that we have a bicategory $\textbf{CAT}$ of
categories, functors, and natural transformations. Moreover, we
showed that, for each object $X$ in a bicategory $\mathbf{C}$,
there is an induced monoidal structure on $\Hom(X,X)$. So for
each category $\mathcal C\in\textbf{CAT}$ we obtain a monoidal structure
$(\Hom[\mathcal C,\mathcal C],\text{Id},\circ)$.

The monad laws (\ref{eq:monad_laws}) look similar to
the monoid axioms (\ref{eq:monoid_axioms}). The monoidal structure
on $\Hom[\mathcal C,\mathcal C]$ above allows us to regard monads
as monoid objects and thus obtain an equivalence of categories.

\begin{proposition}
  There is an equivalence of categories
  \begin{align*}
    \textbf{Mon}(\Hom[\mathcal C,\mathcal C],\text{Id},\circ)
    \cong \textbf{Monad}(\mathcal C).
  \end{align*}
  \begin{proof}
    Consider a monad $(T,\eta,\mu)$ on $\mathcal C$. Note that no
    modification is required to turn this into a monoid object:
    We have $T\in\Hom[\mathcal C,\mathcal C]$, $\eta : \text{Id}\to T$
    and $\mu : T \circ T \to T$. Moreover, the natural isomorphisms
    in the bicategory $\textbf{CAT}$ are identities, hence the
    monoid axioms (\ref{eq:monoid_axioms}) are precisely the monad
    laws (\ref{eq:monad_laws}). Note that we have not changed
    anything about the triple $(T,\eta,\mu)$, hence we have a bijection
    of objects between $\textbf{Monad}(\mathcal C)$ and
    $\textbf{Mon}(\Hom[\mathcal C,\mathcal C],\text{Id},\circ)$.

    It is now even more straightforward to see straighforward to see
    that (\ref{eq:monoid_map_axioms}) and (\ref{eq:monad_map_laws})
    for $(\Hom[\mathcal C,\mathcal C],\text{Id},\circ)$-monoids
    and $\mathcal C$-monads.
  \end{proof}
\end{proposition}

\section{Categories are monads}


\printbibliography

\end{document}
