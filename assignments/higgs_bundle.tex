\documentclass{article}
\usepackage{assignment}
\addbibresource{higgs.bib}
\begin{document}
\title{Higgs bundle}
\date{\today}
\maketitle

\section{Complex geometry}

\subsection{Holomorphic functions}

\subsection{Holomorphic 1-forms}

\subsection{Complex vector bundles}

\missingdefinition{trivial rank 1 bundle $1_M$}

\begin{theorem}[\cite{mccarthy2020}]\label{thm:dim_rank_vector_bundles}
  Let $E\to M$ be a real vector bundle on a smooth manifold. If $\rank E > \dim M$ then there is a real vector bundle $E'$ of rank $\rank E' = \dim M$
  such that
  \begin{align*}
    E \cong E' \oplus \bigoplus_{\rank E - \dim M} 1_M.
  \end{align*}
\end{theorem}

\subsection{Holomorphic connection}

\subsection{Complex manifolds}

A complex vector space is just a real vector space $V$ together with a
$J\in\Aut(V)$ such that $J^2 = -\Identity$. We similarly define
a complex structure on a real manifold to be automorphisms of the
tangent bundle:

\begin{definition}
  Let $X$ be a real manifold. A \emph{complex structure} on $X$
  is an automorphism of real vector bundles $J\in\Aut(TX)$ such
  that $J^2 = -\Identity$. The pair $(X,J)$ is referred to as an
  \emph{almost complex manifold}.
\end{definition}

\begin{itemize}
  \item A complex structure smoothly assign a complex structure to each tangent
    space. One may also think of a complex structure as a $(1,1)$-tensor field
    $J\in\Gamma(TX\otimes T^* X)$.
  \item Any real manifold that admits a complex structure must have even
    dimension. This follows immediately from the analogous observation for real
    vector spaces.
  \item Each complex $n$-manifold $X$ induces a complex structure $J$ on the
    real $2n$-manifold $X$. That is, every complex manifold is an almost
    complex manifold. The converse need not be true. \missingexample
  \item On a complex manifold $X$, the complex structure has eigenvalues
    $\pm i$. This induces a splitting
    \begin{align*}
      T_x X = T^{1,0}_x X \oplus T^{0,1}_x X
    \end{align*}
    for each $x\in X$ where $T^{1,0}_x X$ and $T^{0,1}_x X$ correspond
    to the eigenvalues $i$ and $-i$, respectively. There are obvious
    vector bundles $T^{1,0} X$ and $T^{0,1} X$. \question{Is it
    necessarily the case that $\rank(T^{1,0} X) = \rank(TX)/2$?}
  \item By choosing coordinates around $x$ and writing $(\partial / \partial
    z_j)_x = (\partial / \partial x_j - i\partial / \partial y_j)_x / 2$ we
    have bases
    \begin{align*}
      \left\lbrace{\left(\frac{\partial}{\partial z_j}\right)_x}\right\rbrace_j, \hspace{1cm}
      \left\lbrace{\left(\frac{\partial}{\partial \bar z_j}\right)_x}\right\rbrace_j
    \end{align*}
    for $T^{1,0}_x X$ and $T^{0,1}_x X$, respectively.
  \item Analogously, we obtain the splitting
    \begin{align*}
      T^*_x X = \Lambda^{1,0}_x X \oplus \Lambda^{0,1}_x X
    \end{align*}
    and hence bundles $\Lambda^{1,0}X$ and $\Lambda^{0,1}X$. This
    splitting extends to exterior powers of the cotangent bundle:
    \begin{align*}
      \Lambda^k X
      = \bigoplus_{p+q = k} \Lambda^{p,q} X
      = \bigoplus_{p+q=k}\left({\exterior{p}{\Lambda^{1,0} X}}\right)\wedge
      \left({\exterior{q}{\Lambda^{0,1} X}}\right).
    \end{align*}
    We call sections of $\Lambda^{p,q} X$ $(p,q)$-forms and write
    $\Omega^{p,q}(X)=\Gamma(\Lambda^{p,q}X)$.
\end{itemize}

\begin{example}[{\cite[Section 5.2]{neitzke2021}}]
  Let $E\to\Sigma_g$ be a complex rank $k$ vector bundle
  on a Riemannian surface with $d=\deg E$. Then $\End E$ is
  a complex manifold whose structure is induced by the
  homeomorphism $\End E \cong E\otimes E^*$.

  Then $\Omega^{0,1}(\End E)$ is the
\end{example}

\begin{itemize}
  \item Using the projections, the exterior derivative $d:\Omega^k(X)\to\Omega^{k+1}(X)$ splits into maps $d^{0,k+1},\ldots,d^{k+1,0}$ with
    \begin{align*}
      d^{p,q} : \Omega^k(X) \to \Omega^{p,q}(X).
    \end{align*}
\end{itemize}

\section{Moduli spaces in algebraic geometry}

While classifying geometric objects of a certain types, one often
finds that the space of such objects has a rich geometric structure.
For example, the space of all lines in $k^{n+1}$ is
$\mathbb{P}^n_k$. The space of all objects of a certain type is referred
to as a moduli space. \cite{bejleri2020}

\begin{definition}
  Let $F:\op{\Sch}_S\to\Set$ be a presheaf on $S$-schemes.
  A \emph{fine moduli space} of $F$ is an $S$-scheme $M$ that
  represents $F$, i.e. there exists a natural isomorphism
  \begin{align*}
    F \cong \Hom\left({-,M}\right).
  \end{align*}
\end{definition}

\section{Geometric invariant theory}

Consider $X=\Spec R$ and a group $G$ acting on $X$. We want to construct
an affine scheme $X \sslash G$. The usual quotient does not work:
Consider the action of $G_m$ on $\affine{n}{}$.

\begin{definition}
  A \emph{group scheme} $G/S$ is a group object in $\Sch_S$.
\end{definition}

\begin{definition}
  An \emph{algebraic group over $k$} is a smooth separated
  group scheme over $k$.
\end{definition}

\begin{definition}
  A \emph{action} of $G/S$ on $X/S$ is a morphism
  $\rho : G\times X\to X$ satisfying the usual laws with respect
  to $\mu:G\times G\to G$.
\end{definition}

\section{Stable bundles}

We aim to classify vector bundles on a Riemannian surface $\Sigma_g$
of genus $g$. In light of \ref{thm:dim_rank_vector_bundles},
a complex vector bundle $E\to\Sigma_g$ is classified by
$\rank E$ and $\deg E$ where \question{what's $c_1$ here?}
\begin{align*}
  \deg E = c_1(E)[\Sigma_g].
\end{align*}

If $\rank E = 1$, $\deg E = 1$ then the complex vector bundles
$E\to\Sigma_g$ corresponds to the complex torus of complex dimension
$g$, i.e. $\Jac_0(\Sigma_g)=\mathbb{C}^g / \mathbb{Z}^{2g}$.

If $\rank E = 1$ then the complex vector bundles $E\to\Sigma_g$
correspond to the sheaf cohomology \question{what's sheaf cohomology?} group
\begin{align*}
  H^1(\Sigma_g,\mathcal O^*) \cong \mathbb{Z}\times\Jac_0(\Sigma_g).
\end{align*}

If $\rank E>1$ then things get hairy. For some reason \question{What is reallly happening here?}
things stop being discrete (i.e. we don't get a distinct classification
for each combination of rank and dimension). Instead we consider
moduli spaces.

\subsection{Dolbeault operator}

\begin{lemma}
  Let $E\to X$ be a smooth complex vector bundle $E\to X$ on a complex
  manifold. Then
  \begin{align*}
    \Omega^{0,1}(E) = \Omega^{0,1}(X)\otimes\Gamma(E).
  \end{align*}
  \begin{proof}
    Let $\omega\in\Omega^{0,1}(X)$ and $s\in\Gamma(E)$. Define
    $t(s,\omega)\in\Omega^{0,1}=\Gamma({\left({T^{0,1}}\right)^* E})$
    by
    \begin{equation*}
      % https://q.uiver.app/#q=WzAsNCxbMCwwLCJFIl0sWzIsMCwiWCJdLFs0LDAsIihUXnswLDF9KV4qIFgiXSxbNiwwLCIoVF57MCwxfSleKkUiXSxbMCwxLCJcXHBpIl0sWzEsMiwiXFxvbWVnYSJdLFsyLDMsIihUXnswLDF9KV4qIHMiXV0=
      \begin{tikzcd}
        E && X && {(T^{0,1})^* X} && {(T^{0,1})^*E}
        \arrow["\pi", from=1-1, to=1-3]
        \arrow["\omega", from=1-3, to=1-5]
        \arrow["{(T^{0,1})^* s}", from=1-5, to=1-7]
      \end{tikzcd}
    \end{equation*}
    This defines a bilinear map
    \begin{align*}
      t : \Omega^{0,1} X \times \Gamma(E) \to \Omega^{0,1}(E).
    \end{align*}
    yadda yadda
    \missingproof
  \end{proof}
\end{lemma}

\begin{definition}
  A \emph{Dolbeault operator} on a smooth complex vector bundle
  $E\to X$ on a complex manifold is a $\mathbb{C}$-linear operator
  \begin{align*}
    \bar\partial_E : \Gamma(E) \to \Omega^{0,1}(E)
  \end{align*}
  such that
  \begin{enumerate}
    \item $\bar\partial_E^2 = 0$,
    \item $\bar\partial_E(fs) = \bar\partial f \otimes s + f\bar\partial_E s$
      for $s\in\Gamma(E)$ and $f\in\Omega^0(X)$.
  \end{enumerate}
\end{definition}

Two Dolbeault operators $\bar\partial_1$ and $\bar\partial_2$ on
$E\to X$ are equivalent if there exists $\phi\in\Aut(E)$ such that
the following commutes:
\begin{equation*}
  % https://q.uiver.app/#q=WzAsNCxbMCwwLCJcXEdhbW1hKEUpIl0sWzIsMCwiXFxPbWVnYV57MCwxfShFKSJdLFswLDEsIlxcR2FtbWEoRSkiXSxbMiwxLCJcXE9tZWdhXnswLDF9KEUpIl0sWzIsMywiXFxiYXJcXHBhcnRpYWxfMiJdLFswLDEsIlxcYmFyXFxwYXJ0aWFsXzEiXSxbMCwyLCJcXEdhbW1hKFxccGhpKSIsMl0sWzEsMywiXFxPbWVnYV57MCwxfShcXHBoaSkiXV0=
  \begin{tikzcd}
    {\Gamma(E)} && {\Omega^{0,1}(E)} \\
    {\Gamma(E)} && {\Omega^{0,1}(E)}
    \arrow["{\bar\partial_2}", from=2-1, to=2-3]
    \arrow["{\bar\partial_1}", from=1-1, to=1-3]
    \arrow["{\Gamma(\phi)}"', from=1-1, to=2-1]
    \arrow["{\Omega^{0,1}(\phi)}", from=1-3, to=2-3]
  \end{tikzcd}
\end{equation*}
It turns out that there is a one-to-one correspondence between
holomorphic structures on a smooth complex vector bundle $E\to\Sigma_g$
and Dolbeault operators on $E$ up to equivalence.

More precisely, fix a smooth complex vector bundle $E\to\Sigma_G$
and let $\Dol(E)$ denote the set of all Dolbeault operators on $E$.
Now $\Aut(E)$ acts on $\Dol(E)$ by
\begin{equation*}
  % https://q.uiver.app/#q=WzAsNCxbMCwwLCJcXEdhbW1hKEUpIl0sWzIsMCwiXFxPbWVnYV57MCwxfShFKSJdLFswLDEsIlxcR2FtbWEoRSkiXSxbMiwxLCJcXE9tZWdhXnswLDF9KEUpIl0sWzIsMywiXFxiYXJcXHBhcnRpYWwiXSxbMCwxLCJcXHBoaVxcY2RvdFxcYmFyXFxwYXJ0aWFsIl0sWzAsMiwiXFxHYW1tYShcXHBoaSkiLDJdLFszLDEsIlxcT21lZ2FeezAsMX0oXFxwaGleey0xfSkiLDJdXQ==
  \begin{tikzcd}
    {\Gamma(E)} && {\Omega^{0,1}(E)} \\
    {\Gamma(E)} && {\Omega^{0,1}(E)}
    \arrow["\bar\partial", from=2-1, to=2-3]
    \arrow["\phi\cdot\bar\partial", from=1-1, to=1-3]
    \arrow["{\Gamma(\phi)}"', from=1-1, to=2-1]
    \arrow["{\Omega^{0,1}(\phi^{-1})}"', from=2-3, to=1-3]
  \end{tikzcd}
\end{equation*}

Now the holomorphic structures on $E$ up to isomorphism correspond to
elements of \question{what's the map?}
\begin{align*}
  \Dol(E)/\Aut(E).
\end{align*}
Noting that $\Dol(E)$ is an affine space modelled on $\Omega^{0,1}(\End(E))$
\cite{mccarthy2020} \question{what does this mean?} the quotient
is a topological space. However, this space need not be Hausdorff.

Write $\mu(E) = \deg(E)/\rank(E)$. We thus introduce the notion of
stability:

\begin{definition}
  A holomorphic bundle $E\to\Sigma_g$ is \emph{stable} if,
  for all proper non-zero holomorphic subbundles $F\subset E$,
  \begin{align*}
    \mu(F) < \mu(E).
  \end{align*}
\end{definition}

Now let $\Dol(E)^s$ denote the set of Dolbeault operators corresponding
to stable holomorphic structures on $E$. Then $\Dol(E)^s/\Aut(E)$
is Hausdorff. Hence make the following definition:

\begin{definition}
  The moduli space of stable holomorphic vector bundles of rank $n$
  and degree $d$ over a Riemannian surface $\Sigma_g$ is
  \begin{align*}
    \mathcal N^g_{n,d} = \Dol(E)^s / \Aut(E).
  \end{align*}
\end{definition}

\subsection{Quot schemes}

\section{Higgs bundles}

\subsection{Definition}

\begin{definition}
  A \emph{Higgs bundle} consists of
  \begin{itemize}
    \item a holomorphic vector bundle $E\to\Sigma_g$,
    \item
  \end{itemize}
\end{definition}

\subsection{Stability}

A Higgs bundle is said to be stable if its subbundles statisfy certain
properties. Before we can make this precise, we need to define
vector subbundles and their invariance with respect to the Higgs field.

\subsection{$\mathcal M$ vs $\mathcal N$}

\begin{proposition}
  $T^*\mathcal N^g_{n,d}$ is an open dense subset of $\mathcal M^g_{n,d}$.
\end{proposition}

\section{Non-abelian Hodge theory}

\begin{theorem}
  Let $X$ be a compact K\"ahler manifold. Then there is an equivalence
  of categories between irreducible flat vector bundles on $X$
  and stable Higgs bundles
\end{theorem}

\printbibliography

\end{document}
