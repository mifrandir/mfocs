\documentclass{report}
\usepackage{assignment}
\addbibresource{higgs.bib}
\begin{document}
\title{Moduli Spaces of Holomorphic Bundles}
\author{Franz Miltz}
\date{\today}
\maketitle

\tableofcontents
\pagebreak

\chapter*{Open questions}

\begin{enumerate}
  \item Are the analytic spaces $\mathcal N^{an}$
    and $\mathcal N$ isomorphic?
    \begin{itemize}
      \item What is a basis of $\mathcal N$? Maybe a suitable way
        to obtain the complex manifold structure on
        $\mathcal N = \Dol(E)^s / \Aut(E)$ is to observe that
        $\Dol(E)$ is an affine space modelled on
        $\Omega^{0,1}(\End E)$, the same is true for $\Dol(E)^s$,
        and $g\cdot \dol_E = \dol_E + \alpha$ if, and only if,
        $g = e$ and $\alpha = 0$.
      \item Is the map
        \begin{align*}
          f : \mathcal N^{an} &\to \mathcal N \\
          (q : \mathcal O^r_X \surj \mathcal F) &\mapsto \mathbf V(\Ker(q))
        \end{align*}
        continuous?
      \item What values does $f_*\mathcal O_{\mathcal N}^{an}$ take on
        basis elements?
      \item What are the stalks of $\mathcal O_{\mathcal N}$ and
        $\mathcal O_{\mathcal N}^{an}$, respectively?
      \item What is a suitable map
        \begin{align*}
          f^\sharp : \mathcal O_N \to f_*\mathcal O_N^{an}?
        \end{align*}
        Is it a map of locally ringed spaces?
      \item Does the map induce an isomorphism of stalks?
    \end{itemize}
\end{enumerate}

\setcounter{chapter}{-1}

\chapter{Prerequisites}

\section{Complex vector bundles}

\subsection{Complex manifolds}

A complex vector space is just a real vector space $V$ together with a
$J\in\Aut(V)$ such that $J^2 = -\Identity$. We similarly define
a complex structure on a real manifold to be automorphisms of the
tangent bundle:

\begin{definition}
  Let $X$ be a real manifold. A \emph{complex structure} on $X$
  is an automorphism of real vector bundles $J\in\Aut(TX)$ such
  that $J^2 = -\Identity$. The pair $(X,J)$ is referred to as an
  \emph{almost complex manifold}.
\end{definition}

\begin{itemize}
  \item A complex structure smoothly assign a complex structure to each tangent
    space. One may also think of a complex structure as a $(1,1)$-tensor field
    $J\in\Gamma(TX\otimes T^* X)$.
  \item Any real manifold that admits a complex structure must have even
    dimension. This follows immediately from the analogous observation for real
    vector spaces.
  \item Each complex $n$-manifold $X$ induces a complex structure $J$ on the
    real $2n$-manifold $X$. That is, every complex manifold is an almost
    complex manifold. The converse need not be true. \missingexample
  \item On a complex manifold $X$, the complex structure has eigenvalues
    $\pm i$. This induces a splitting
    \begin{align*}
      T_x X = T^{1,0}_x X \oplus T^{0,1}_x X
    \end{align*}
    for each $x\in X$ where $T^{1,0}_x X$ and $T^{0,1}_x X$ correspond
    to the eigenvalues $i$ and $-i$, respectively. There are obvious
    vector bundles $T^{1,0} X$ and $T^{0,1} X$. \question{Is it
    necessarily the case that $\rank(T^{1,0} X) = \rank(TX)/2$?}
  \item By choosing coordinates around $x$ and writing $(\partial / \partial
    z_j)_x = (\partial / \partial x_j - i\partial / \partial y_j)_x / 2$ we
    have bases
    \begin{align*}
      \left\lbrace{\left(\frac{\partial}{\partial z_j}\right)_x}\right\rbrace_j, \hspace{1cm}
      \left\lbrace{\left(\frac{\partial}{\partial \bar z_j}\right)_x}\right\rbrace_j
    \end{align*}
    for $T^{1,0}_x X$ and $T^{0,1}_x X$, respectively.
  \item Analogously, we obtain the splitting
    \begin{align*}
      T^* X = T^*_{1,0} X \oplus T^*_{0,1} X
    \end{align*}
    This splitting extends to exterior powers of the cotangent bundle:
    \begin{align*}
      \Lambda^k T^*X
      = \bigoplus_{p+q = k} T^*_{p,q} X
      = \bigoplus_{p+q=k}\left({\exterior{p}{T^*_{1,0} X}}\right)\wedge
      \left({\exterior{q}{T^*_{0,1} X}}\right).
    \end{align*}
    We call sections of $\Lambda^{p,q} X$ $(p,q)$-forms and write
    $\Omega^{p,q}(X)=\Gamma(T^*_{p,q}X)$.
\end{itemize}

\begin{example}[{\cite[Section 5.2]{neitzke2021}}]
  Let $E\to X$ be a smooth rank $k$ vector bundle
  with $d=\deg E$. Then $\End E$ is a complex manifold whose
  structure is induced by the identification
  $\End E = E\otimes E^*$.
  Moreover, we have a projection map $\End(E)\to X$ induced by
  the same argument.
  If $E$ is holomorphic then so is $\End(E)$.
\end{example}

\subsection{Dolbeault operator}\label{sec:dolbeault_operator}

\begin{definition}
  A \emph{Dolbeault operator} on a smooth vector bundle
  $E$ on a complex manifold $X$ is a $\mathbb{C}$-linear operator
  \begin{align*}
    \dol_E : \Gamma(X,E) \to \Omega^{0,1}(X,E)
  \end{align*}
  such that
  \begin{enumerate}
    \item $\dol_E^2 = 0$,
    \item $\dol_E(fs) = \dol f \otimes s + f\dol_E s$
      for $s\in\Gamma(X,E)$ and $f\in\Omega^0(X)$.
  \end{enumerate}

  A morphism $\dol\to\dol'$ of Dolbeaut operators
  on vector bundles $E$ and $E'$, respectively, is a smooth morphism
  of vector bundles $E\to E'$ such that the following commutes:
  \begin{equation*}
    % https://q.uiver.app/#q=WzAsNCxbMCwwLCJcXEdhbW1hKFgsRSkiXSxbMiwwLCJcXE9tZWdhXnswLDF9KEUpIl0sWzAsMSwiXFxHYW1tYShYLEYpIl0sWzIsMSwiXFxPbWVnYV57MCwxfShGKSJdLFsyLDMsIlxcYmFyXFxwYXJ0aWFsXzIiXSxbMCwxLCJcXGJhclxccGFydGlhbF8xIl0sWzAsMiwiXFxHYW1tYShYLFxccGhpKSIsMl0sWzEsMywiXFxPbWVnYV57MCwxfShcXHBoaSkiXV0=
    \begin{tikzcd}
      {\Gamma(X,E)} && {\Omega^{0,1}(X,E)} \\
      {\Gamma(X,F)} && {\Omega^{0,1}(X,F)}
      \arrow["{\dol_1}", from=1-1, to=1-3]
      \arrow["{\Gamma(X,\varphi)}"', from=1-1, to=2-1]
      \arrow["{\Omega^{0,1}(\varphi)}", from=1-3, to=2-3]
      \arrow["{\dol_2}", from=2-1, to=2-3]
    \end{tikzcd}
  \end{equation*}
  Thus we have a category $\textbf{Dol}(X)$ of smooth vector bundles
  and Dolbeault operators on a complex vector manifold $X$
  with a full subcategory $\textbf{Dol}(E)$ for each smooth vector
  bundle $E$ on $X$.
\end{definition}

\begin{lemma}
  Let $E\to X$ be a smooth vector bundle on a complex manifold $X$.
  Then there is an equivalence of categories
  \begin{align}\label{eq:equivalence_dolbeault_operators}
    \textbf{Dol}(E) \cong \textbf{HolBun}(E)
  \end{align}
  where $\textbf{HolBun}(E)$ is the category of holomorphic vector
  bundles with underlying smooth vector bundle $E$.
  \begin{proof}
    \question{do we need to make assumptions about $X$ here? it is certainly true for $X$ a curve.}
    \missingproof
  \end{proof}
\end{lemma}
Hence we are justified in denoting a holomorphic vector bundle
on $X$ by $(E,\dol)$ where $E$ is a smooth bundle and
$\dol$ is a corresponding Dolbeault operator.

\begin{remark}
  What we want to be true is that the functor
  \begin{align*}
    \textbf{Dol}:\textbf{Bun}\to\textbf{CAT}
  \end{align*}
  is an full and faithful. At least we want it to be the case
  that an equivalence $\textbf{Dol}(E)\cong\textbf{Dol}(F)$
  yields an isomorphism $E\cong F$. But this is not obvious
  and probably not true. For instance, if there is a smooth
  vector bundle $E$ that does not admit a complex structure then
  $\textbf{Dol}(E)$ is empty. It does not seem likely that such
  bundles exist but there is at most one such bundle, up to iso,
  for any given manifold. (This actually smells quite stacky,
  as we are considering the category $\textbf{Dol}(X)$
  with the forgetful functor to $\textbf{Bun}$ whose fibres
  over $E\in\textbf{Bun}$ is exactly $\textbf{Dol}(E)$.)
\end{remark}

Thus, in order to study holomorphic bundles on $X$, it suffices
to fix a real vector bundle $E$ of rank $n$ and degree $d$, and study
the holomorphic structures on it, which is the same as studying
Dolbeault operators on $E$. As we are interested in equivalence
classes of complex structures, we need to consider equivalence
classes in $\textbf{HolBun}(E)$. Such equivalence classes correspond
under (\ref{eq:equivalence_dolbeault_operators}) to equivalence
classes of Dolbeault operators on $E$. That is, we are interested
in the quotient of the set of Dolbeault operators $\text{Dol}(E)$
by the group $\Aut(E)$ acting by
\begin{equation*}
  % https://q.uiver.app/#q=WzAsNCxbMCwwLCJcXEdhbW1hKEUpIl0sWzIsMCwiXFxPbWVnYV57MCwxfShFKSJdLFswLDEsIlxcR2FtbWEoRSkiXSxbMiwxLCJcXE9tZWdhXnswLDF9KEUpIl0sWzIsMywiXFxiYXJcXHBhcnRpYWwiXSxbMCwxLCJcXHBoaVxcY2RvdFxcYmFyXFxwYXJ0aWFsIl0sWzAsMiwiXFxHYW1tYShcXHBoaSkiLDJdLFszLDEsIlxcT21lZ2FeezAsMX0oXFxwaGleey0xfSkiLDJdXQ==
  \begin{tikzcd}
    {\Gamma(X,E)} && {\Omega^{0,1}(X,E)} \\
    {\Gamma(X,E)} && {\Omega^{0,1}(X,E)}
    \arrow["\dol", from=2-1, to=2-3]
    \arrow["\varphi\cdot\dol", from=1-1, to=1-3]
    \arrow["{\Gamma(\varphi)}"', from=1-1, to=2-1]
    \arrow["{\Omega^{0,1}(\varphi^{-1})}"', from=2-3, to=1-3]
  \end{tikzcd}
\end{equation*}

Noting that $\Dol(E)$ is an affine space modelled on
$\Omega^{0,1}(\End(E))$ \cite{mccarthy2020} the quotient
is a topological space. However, this space need not be Hausdorff.


\subsection{Holomorphic tangent bundle}

A complex manifold $X$ of dimension $n$ may be regarded as a
real manifold of dimension $2n$. Hence we have the real tangent bundle
$TX$ on $X$ which we may complexify as $TX\otimes\mathbb{C}$.
The complex structure $J:TX\to TX$ extends to this complexification
by $J(u + iv) = J(u) + iJ(v)$. We have $J^2=-\text{Id}$ and hence
eigenvalues $i$ and $-i$ which yield the decomposition
\begin{align*}
  TX\otimes\mathbb{C} = T^{1,0}X \oplus T^{0,1}X.
\end{align*}
The bundle $T^{1,0}X$ is holomorphic and referred to as the
\emph{holomorphic tangent bundle}. As real vector bundles this
is isomorphic to $TX$ via
\begin{align*}
  TX \longinc TX \otimes C \xlongrightarrow{\pi} T^{1,0}X.
\end{align*}

\subsection{Harmonic forms}

\begin{lemma}
  For every smooth vector bundle $E$ on a complex manifold $X$, there
  is a canonical isomorphism
  \begin{equation}
    (E\otimes\bar E)^* \cong E^* \otimes \bar{E}^*.
  \end{equation}
  \begin{proof}
    \missingproof
  \end{proof}
\end{lemma}

\begin{definition}
  Consider a complex manifold $X$.
  A \emph{hermitian metric} is a section
  \begin{align*}
    h\in\Gamma(X,(T^{1,0}X\otimes T^{0,1})^*)
  \end{align*}
  such that, for every
  $p\in X$ and tangent vectors $u_p,v_p\in T^{1,0}_p X$,
  \begin{itemize}
    \item $h_p (u_p,\bar v_p) = \overline{h_p(v_p,\bar u_p)}$, and
    \item if $v_p\neq 0$ then $h_p(v_p,\bar v_p)>0$.
  \end{itemize}
\end{definition}

That is, there is a positive definite hermitian form
$h_p: T^{1,0}_p\otimes T^{0,1}_p\to\mathbb{C}$ at every point $p\in X$
that varies smoothly. Note that the expression $h_p(e,\bar e) > 0$
makes sense as $h_p(e,\bar e) = \overline{h_p(e,\bar e)}$,
i.e. $h_p(e,\bar e)\in\mathbb{R}$.

We are not going to care about any particular hermitian metric,
but rather about their existence. Fortunately, our underlying
manifold is a compact Riemann surface. This allows us to make use
of the following:

\begin{theorem}
  Every smooth vector bundle on a compact complex manifold admits
  a hermitian metric. In particular, every compact complex manifold admits
  a hermitian metric.
\end{theorem}

Recall that the exterior derivative
\begin{align*}
  d : \Omega^k(X) \to \Omega^{k+1}(X)
\end{align*}
splits into maps
\begin{align*}
  \partial : \Omega^{p,q}(X) \to \Omega^{p+1,q}(X),\hspace{1cm}
  \dol : \Omega^{p,q} \to \Omega^{p,q+1}(X).
\end{align*}
A hermitian metric on $X$ defines a pairing \question{how is this
defined and what is its type?} on $\Omega^*(X)$. With respect to this
pairing we have the adjoints
\begin{align*}
  \partial^* : \Omega^{p,q}(X) \to \Omega^{p-1,q}(X),\hspace{1cm}
  \dol^* : \Omega^{p,q} \to \Omega^{p,q-1}(X).
\end{align*}

This lets us define harmonic forms, i.e. forms that vanish
under the Laplacian:

\begin{definition}
  Let $X$ be a compact manifold with a choice of hermitian metric.
  The \emph{harmonic} $(p,q)$-forms are \question{do we need a
  K\"ahler structure for this to make sense?}
  \begin{align*}
    \mathcal H^{p,q}(X) = \ker(\dol) \cap\ker(\dol^*) \subseteq \Omega^{p,q}(X).
  \end{align*}
\end{definition}

One may think of the spaces of harmonic forms as cohomology groups.
For example, we have the following:

\begin{theorem}[Poincar\'e duality]
  Let $X$ be a K\"ahler manifold of dimension $n$. There is an
  isomorphism
  \begin{align*}
    \mathcal H^{p,q}(X) \cong \mathcal H^{n-q,n-p}(X)^*
  \end{align*}
  given by the pairing
  \begin{align*}
    (\alpha,\beta)\mapsto \int_X \alpha\wedge\beta.
  \end{align*}
\end{theorem}

Moreover, it is possible to use Dolbeault operators to define complexes
and corresponding cohomologies:
\begin{definition}
  Let $X$ be a K\"ahler manifold. Then define the \emph{$(p,q)$-Dolbeault
  cohomology group} by
  \begin{align*}
    H^{p,q}_{\dol}(X) := \frac{\ker(\dol:\Omega^{p,q}(X)\to\Omega^{p,q+1}(X))}{\im(\dol:\Omega^{p,q-1}(X)\to\Omega^{p,q}(X))}
  \end{align*}
  More generally, if $E$ is a holomorphic vector bundle then we have
  cohomology with $E$-coefficients:
  \begin{align*}
    H^{p,q}(X,E) := \frac{\ker(\dol_E:\Omega^{p,q}(E)\to\Omega^{p,q+1}(E))}{\im(\dol_E:\Omega^{p,q-1}(E)\to\Omega^{p,q}(E))}.
  \end{align*}
\end{definition}

This definition agrees with sheaf cohomology:

\begin{theorem}[Dolbeault]
  For $X$ K\"ahler and $E$ holomorphic,
  \begin{align*}
    H^{p,q}(X,E) \cong H^q(X,\Omega^p\otimes E).
  \end{align*}
\end{theorem}

Moreover, this allows us to make precise the sense in which harmonic
forms and cohomology are related:

\begin{theorem}[Hodge]
  Let $X$ be a compact K\"ahler manifold.
  For any $[\alpha]\in H^{p,q}_{\dol}(X)$, there exists a unique
  harmonic representative $\alpha'\in[\alpha]\cap\mathcal H^{p,q}(X)$.
\end{theorem}

\subsection{Chern connections}

Recall a connection on $E$ is an $\mathbb{R}$-linear map
\begin{align*}
  \nabla : \Gamma(X,E)\to \Gamma(X,E\otimes T^* X)
\end{align*}
such that
\begin{align*}
  \nabla(fs) = df \otimes s + f\nabla s
\end{align*}
for all $f\in C^\infty(X)$ and $s\in\Gamma(X,E)$. If we have a
holomorphic structure $\dol_E$ then we may require that connections
respect this structure.

\begin{definition}
  A connection $\nabla$ in $E$ is \emph{compatible with a
  holomorphic structure $\dol_E$} if the following commutes:
  \begin{equation*}
    % https://q.uiver.app/#q=WzAsNCxbMCwwLCJcXEdhbW1hKFgsRSkiXSxbMCwxLCJcXEdhbW1hKFgsRVxcb3RpbWVzIFReKlgpIl0sWzIsMCwiXFxHYW1tYShYLEUpXFxvdGltZXNcXE9tZWdhXnswLDF9KFgpIl0sWzIsMSwiXFxHYW1tYShYLEUpXFxvdGltZXNcXE9tZWdhXjEoWCkiXSxbMCwxLCJcXG5hYmxhIiwyXSxbMCwyLCJcXGJhclxccGFydGlhbF9FIl0sWzEsMywiXFxjb25nIl0sWzMsMiwiXFxHYW1tYShYLEUpXFxvdGltZXNcXHBpIiwyXV0=
    \begin{tikzcd}
      {\Gamma(X,E)} && {\Gamma(X,E)\otimes\Omega^{0,1}(X)} \\
      {\Gamma(X,E\otimes T^*X)} && {\Gamma(X,E)\otimes\Omega^1(X)}
      \arrow["{\bar\partial_E}", from=1-1, to=1-3]
      \arrow["\nabla"', from=1-1, to=2-1]
      \arrow["\cong", from=2-1, to=2-3]
      \arrow["{\Gamma(X,E)\otimes\pi}"', from=2-3, to=1-3]
    \end{tikzcd}
  \end{equation*}
\end{definition}

Moreover, if $E$ is hermitian then we may reasonably require
connections to respect this structure too. To make this precise,
recall that a connection is equivalently a map
\begin{align*}
  \nabla : \Gamma(X,E)\otimes\Gamma(X,TX)\to\Gamma(X,E)
\end{align*}
which we may write as $(s,v)\mapsto \nabla_v s$.


\begin{definition}
  A connection $\nabla$ in $E$ is \emph{compatible with
  a hermitian metric $h$ on $E$} if, for all
  $s,t\in\Gamma(X,E)$ and $v\in\Gamma(X,TX)$,
  \begin{align*}
    v(h(s,\bar t)) = h(\nabla_v s,\bar t) + h(s,\overline{\nabla_v t}).
  \end{align*}
\end{definition}

It turns out that fixing both a hermitian metric as well as a
holomorphic structure determines the connection:

\begin{proposition}
  Let $h$ be a hermitian metric on a holomorphic bundle $(E,\dol_E)$.
  Then there exists a unique connection $F_h$ in $E$ that is
  compatible with both the hermitian and the holomorphic structure.
  It is called the \emph{Chern connection}.
\end{proposition}

\section{Affine GIT}

Consider $X=\Spec R$ and a group $G$ acting on $X$. We want to construct
an affine scheme $X \sslash G$. The usual quotient does not work:
Consider the action of $\mathbb{G}_m$ on $\affine{n}{}$.

\todo{story}

\subsection{Group schemes}

\begin{definition}
  A \emph{group scheme} $G/S$ is a group object in $\Sch_S$.
\end{definition}

In particular, a group scheme comes equipped with three maps:
\begin{align*}
  e:S\to G,\hspace{1cm} i:G\to G,\hspace{1cm} m:G\times G\to G.
\end{align*}

\begin{example}\label{ex:affine_algebraic_groups}
  In the affine case $G = \Spec R$ we may specify the multiplication
  and inverse maps in terms of $k$-algebra homomorphisms
  $m^* : R \to R\otimes R$ and $i^* : R\to R$.
  There are several affine group schemes that we are already familiar
  with:
  \begin{enumerate}
    \item $\mathbb{G}_a = \Spec k[t]$ with comulitiplication
      $t\mapsto t\otimes 1 + 1\otimes t$,
    \item $\mathbb{G}_m = \Spec k[t^\pm] = \affine{}{1} \setminus \left\lbrace{0}\right\rbrace$ with comultiplication
      $t\mapsto t\otimes t$,
    \item $GL_n = \Spec k[x_{ij} : 1\leq i,j\leq n][1/\det((x_ij))]$
      (see e.g. \cite[\href{https://stacks.math.columbia.edu/tag/022W}{Tag 022W}]{stacks-project} for details) with
      comultiplication
      \begin{align*}
        x_{ij} \mapsto \sum_{k=1}^n x_{ik}\otimes x_{kj}.
      \end{align*}
  \end{enumerate}
\end{example}

\todo{maybe we should say a bit more about $GL_n$...}

Note that the functor of points $G=\Hom(-,G):\op{\Sch}_S\to\Set$ induces a
group structure on $G(T)$ for all schemes $T/S$. The induced multiplication
sends $x,y\in G(T)$ to
\begin{align}\label{eq:induced_multiplication_of_points}
  T \xrightarrow{(x,y)}
  G\times G\xrightarrow{m}
  G.
\end{align}
Similarly, we have the identity and inverse of $x\in G(T)$,
\begin{align*}
  T \rightarrow S \xrightarrow{e} G,\hspace{1cm}
  T \xrightarrow{x} G \xrightarrow{i} G,
\end{align*}
respectively.

\begin{example}
  The induced grous of the affine group schemes behave as expected:
  For a $k$-algebra $R$, $\mathbb{G}_a(R) = (R,+)$,
  $\mathbb{G}_m(R) = (R^\times,\times)$, and
  $GL_n(R)$ is the usual group of invertible $n\times n$ matrices
  with coefficients in $R$.

  For example, consider $x\in\mathbb{G}_a(R)$ where $R$ is
  a $k$-algebra. Such an $x$ is given by algebra homomorphsims
  $x^\sharp:k[t] \to R$. That is, we may identify $\mathbb{G}_a(R)$
  with $R$ using the map $x \mapsto x^\sharp(t)$. Now, for
  $x,y\in\mathbb{G}_a(R)$ and $f\in k[t]$, we have the induced group
  multiplication given by
  \begin{align*}
    (xy)^\sharp(f)
    = x^\sharp(f) \cdot 1 + 1 \cdot y^\sharp(f)
    = x^\sharp(f) + y^\sharp(f).
  \end{align*}
  Thus the induced group structure on $\mathbb{G}_a(R)$ is just $(R,+)$.
\end{example}

The converse is also true: If the induced multiplication
(\ref{eq:induced_multiplication_of_points}) defines a group structure
on $G(T)$ for every $T/S$ then $G$ is a group scheme. This may be
shown by constructing a group structure on $G(-)$ in
$[\op{\Sch}_S,\Set]$ and then lifitng it to $G$ in $\Sch_S$ using
Yoneda's lemma.

Similarly, base change preserves group structures.
In particular, if we have a map $T\to S$ then $G_T = G\times T$
so $(G\times G)_T = G_T \times_T G_T$. Thus we obtain a group
scheme $G_T/T$.

\begin{definition}[\cite{lei2020}]
  An \emph{algebraic group} is a smooth separated
  group scheme over a field $k$.
\end{definition}

\begin{example}
  $\mathbb{G}_a$, $\mathbb{G}_m$, and $GL_n$ are all examples of algebraic
  groups. \missingproof
\end{example}

\subsection{Actions}

\begin{definition}
  A \emph{action} of $G/S$ on $X/S$ is a morphism
  $\sigma : G\times X\to X$ satisfying the usual laws with respect
  to $\mu:G\times G\to G$.

\end{definition}

Note that an action $\sigma : G\times X\to X$ induces an action of
$G(T)$ on $X(T)$ by
\begin{equation*}
  T \xlongrightarrow{(g,x)} G\times X \xlongrightarrow{\sigma} X.
\end{equation*}
Similarly, $G_T$ acts on $X_T$ by
\begin{align*}
  G_T \times_T X_T = (G \times X)_T \xlongrightarrow{\sigma_T} X_T.
\end{align*}

Thus, given a $T$-scheme $T'$, we obtain an action of
$G_T(T')$ on $X_T(T')$. We may then note that we may pull back
any $x\in X(T)$ along $T'\to T$ to obtain a $T'$-point of $X$.
This allows us to define stabilisers of group actions:

\begin{definition}
  Let $X/S$ and $T/S$ be schemes and $x\in X(T)$. The
  \emph{stabilizer of $x$} is the subgroup scheme of $G_T$
  that represents the functor $\op{\Sch}_T \to \Set$ given by \todo{op?}\question{how is this the pullback of $X\to X\times X \leftarrow G\times X$?}
  \begin{align*}
    T' \mapsto \left\lbrace{ g \in G_T(T') : g\cdot x = x}\right\rbrace.
  \end{align*}
\end{definition}

Here the action of $G_T(T')$ on $X(T)$ arises as follows:
Firstly, by previous considerations, $G_T(T')$ acts on $X_T(T')$.
Now we may pull back $x\in X(T)$ along $T'\to T$ to obtain a
$T'$-point of $X$. Finally, by universality of the pullback
$X_T$ the maps $x:T'\to X$ and $T'\to T$ induce a unique map
$T'\to X_T$.

\begin{definition}
  Let $G/S$ be a group scheme that acts on $X,Y/S$, respectively.
  A morphism $f:X\to Y$ is \emph{$G$-equivariant} if it commutes
  with the actions of $G$ on $X$ and $Y$ respectively, i.e.
  the following commutes:
  \begin{equation*}
    % https://q.uiver.app/#q=WzAsNCxbMCwwLCJHXFx0aW1lcyBYIl0sWzIsMSwiWSJdLFswLDEsIlgiXSxbMiwwLCJHXFx0aW1lcyBZIl0sWzIsMSwiZiIsMV0sWzAsMl0sWzAsMywiXFx0ZXh0e2lkfVxcdGltZXMgZiIsMV0sWzMsMV1d
    \begin{tikzcd}
      {G\times X} && {G\times Y} \\
      X && Y
      \arrow["{\text{id}\times f}"{description}, from=1-1, to=1-3]
      \arrow[from=1-1, to=2-1]
      \arrow[from=1-3, to=2-3]
      \arrow["f"{description}, from=2-1, to=2-3]
    \end{tikzcd}
  \end{equation*}
  A morphism $f:X\to Y$ is \emph{$G$-invariant} if the following
  commutes:
  \begin{equation*}
    % https://q.uiver.app/#q=WzAsMyxbMCwwLCJHXFx0aW1lcyBYIl0sWzIsMCwiWCJdLFs0LDAsIlkiXSxbMCwxLCJcXHJobyIsMSx7ImN1cnZlIjotMn1dLFswLDEsIlxccGkiLDEseyJjdXJ2ZSI6Mn1dLFsxLDIsImYiLDFdXQ==
    \begin{tikzcd}
      {G\times X} && X && Y
      \arrow["\sigma"{description}, curve={height=-12pt}, from=1-1, to=1-3]
      \arrow["\pi"{description}, curve={height=12pt}, from=1-1, to=1-3]
      \arrow["f"{description}, from=1-3, to=1-5]
    \end{tikzcd}
  \end{equation*}
\end{definition}

\subsection{Invariants}

\begin{definition}
  Let $\sigma : G\times X \to X$ be an action of a group scheme
  $G/S$ on a scheme $X/S$. The \emph{ring of invariants} is the
  subring of $\mathcal O(X)$ given by
  \begin{align*}
    \mathcal O(X)^G :=
    \left\lbrace{f \in \mathcal O(X) : \sigma^\sharp(f) = 1 \otimes f}\right\rbrace.
  \end{align*}
\end{definition}

If $\varphi : X\to Y$ is a $G$-invariant morphism then
the image of $\varphi^\sharp : \mathcal O(Y) \to \mathcal O(X)$ is
contained in $\mathcal O(X)^G$. We would like it to be the
case that this ring of invariants is finitely generated. However,
this may fail even under the best conditions:

\begin{lemma}
  There exists an affine algebraic group $G$ and a $k$-algebra $A$
  such that $G$ acts on $\Spec A$ but $A^G$ is not finitely generated.
\end{lemma}

\subsection{Reductive groups}

Fix a $k$-vector space $V$ and an algebraic group $G$ over $k$,
i.e. a group scheme $G/\Spec k$. Then the algebraic group
$GL(V)$ then the corresponding functor of points sends a $k$-algebra
$R$ to $GL(V)(R) = \Aut_R(V\otimes_k R)$.

\begin{definition}
  A \emph{linear representation} of $G$ on $V$ is a homomorphism
  of group-valued functors $G\to GL(V)$.
\end{definition}

If $V$ is finite-dimensional then $GL(V)$ is
affine~\cite[Example 3.3 (4)]{hoskins} and a linear
representation is equivalent to a homomorphism of algebraic groups
$G\to GL(V)$.

If $G$ is affine, then the identity in $G(\mathcal O(G))$
corresponds to a $\mathcal O(G)$-linear automorphism
of $V\otimes_k\mathcal O(G)$. Note that such an endomorphism
is uniquely determined by its restriction to $V$, i.e. by
a $\mathcal O(G)$-linear morphism
\begin{align*}
  \sigma^* : V\to V\otimes_k \mathcal O(G)
\end{align*}
which is referred to as a \emph{co-module structure on $V$}. In
fact such co-module structures are equivalent to linear
representations of affine algebraic
groups.~\cite[Section 3.1]{hoskins}

\begin{definition}
  Let $\sigma : G\to GL(V)$ be a linear representation. A
  subspace $V'\subseteq V$ is \emph{$G$-invariant} if
  $\sigma^*(V') \subseteq V'\otimes_k\mathcal O(G)$
  and a vector $v\in V$ is \emph{$G$-invariant} if
  $\sigma^*(v) = v\otimes 1$.

  Denote by $V^G\subseteq V$ the subspace of $G$-invariant vectors.
\end{definition}

\begin{definition}
  An affine algebraic group is \emph{unipotent} if every non-trivial
  linear representation $\sigma:G\to GL(V)$ has a non-zero $G$-invariant
  vector.
\end{definition}

\begin{definition}
  An affine algebraic group $G$ is
  \begin{enumerate}
    \item \emph{linearly reductive} if every finite dimensional
      linear representation $G\to GL(V)$ is completely reducible,
      i.e. decomposes into a sum of irreducibles;
    \item \emph{geometrically reductive} if for every finite
      dimensional linear representation $G\to GL(V)$ and every
      non-zero $G$-invariant $v\in V$, there is a $G$-invariant
      non-constant homogeneous $f\in\mathcal O(V)$ such that
      $f(v)\neq 0$;
    \item \emph{reductive} if it is smooth and there are no
      non-trivial smooth normal unipotent algebraic subgroups.
  \end{enumerate}
\end{definition}

That is, an affine algebraic group $G$ is reductive if, for every
non-trivial smooth normal algebraic subgroup $H\triangleleft G$
and every linear representation $\sigma : H\to GL(V)$,
$\sigma^*(v) = v\otimes 1$ if, and only if, $v=0$.

Given that we are interested in $\mathbb{C}$-schemes, the following
will come in handy:

\begin{proposition}
  For smooth algebraic groups over $\mathbb{C}$, the three
  notions of reductivity coincide.
\end{proposition}

\begin{theorem}[Nagata]\label{thm:nagata}
  Let $G$ be a geometrically reductive group and $A$ a finitely
  generated $k$-algebra such that $G$ acts rationally on $\Spec A$.
  The $A^G$ is finitely generated.
\end{theorem}

\begin{theorem}[Hilbert, Munford]
  Let $G$ be a linearly reductive group and $A$ a finitely
  generated $k$-algebra such that $G$ acts rationally on $\Spec A$.
  The $A^G$ is finitely generated.
\end{theorem}

\subsection{Quotients}

\begin{definition}
  A \emph{categorical quotient} of a group $G$ acting on $X$
  is
\end{definition}

\begin{definition}
  Let $G$ be a reductive group acting on an affine scheme $X$.
  The \emph{affine GIT quotient} is the morphism
  \begin{align*}
    \varphi : X \to X \sslash G := \Spec\mathcal O(X)^G
  \end{align*}
  arising from the inclusion $\mathcal O(X)^G \inc \mathcal O(X)$.
\end{definition}

\section{Projective GIT}

We have seen how it is possible to quotient an affine scheme
by a group action. In general, the schemes we are going to be working
with are not going to be affine. However, it will be sufficient to
consider the slightly more general case of projective schemes.
\question{does GIT generalise to arbitrary (non-projective) schemes?}
In order to construct a corresponding GIT quotient, we need to
require a linear structure on the group action. It turns
out that this is not going to be a restriction as we will also
describe how to linearise the group actions that we care about.

Fix a reductive affine algebraic group $G$ that acts on a projective
scheme $X$ via $\sigma : G \times X \to X$.

\subsection{Linear actions}

Let us begin by making what it is that we want from the $G$-action
in order to construct a well-behaved quotient. Observe that a
representation $\rho : G\to GL(V)$ defines an action
$G\times V\to V$ that is linear in $V$ via $g\cdot v := \rho(g)(v)$.

More generally, if we have an embedding $X\inc\projective{n}{}$
and the natural action of $GL_{n+1}$ on $\projective{n}{}$ fixes
$X$, then we may think of an action of a group $G$ on $X$ as linear
if it arises from the action of $GL_{n+1}$ on $\projective{n}{}$.

\begin{definition}[{\cite[Definition 5.1]{hoskins}}]
  A \emph{linear action} of $G$ on $X$ consists of
  an action $\sigma : G\times X\to X$, a group homomorphism
  $G \to GL_{n+1}$, and a $G$-equivariant embedding
  $X\inc\projective{n}{S}$, i.e. we have the following
  diagram:
  \begin{equation*}
    % https://q.uiver.app/#q=WzAsNSxbMCwxLCJYIl0sWzQsMSwiXFxtYXRoYmYgUF5uIl0sWzAsMCwiR1xcdGltZXMgWCJdLFsyLDAsIkdcXHRpbWVzXFxtYXRoYmYgUF5uIl0sWzQsMCwiR0xfblxcdGltZXNcXG1hdGhiZiBQXm4iXSxbMCwxLCIiLDEseyJzdHlsZSI6eyJ0YWlsIjp7Im5hbWUiOiJob29rIiwic2lkZSI6InRvcCJ9fX1dLFsyLDBdLFsyLDMsIiIsMSx7InN0eWxlIjp7InRhaWwiOnsibmFtZSI6Imhvb2siLCJzaWRlIjoidG9wIn19fV0sWzMsNF0sWzQsMV1d
    \begin{tikzcd}
      {G\times X} && {G\times\mathbf P^n_S} && {GL_{n+1}\times\mathbf P^n_S} \\
      X &&&& {\mathbf P^n}
      \arrow[hook, from=1-1, to=1-3]
      \arrow[from=1-1, to=2-1]
      \arrow[from=1-3, to=1-5]
      \arrow[from=1-5, to=2-5]
      \arrow[hook, from=2-1, to=2-5]
    \end{tikzcd}
  \end{equation*}
\end{definition}

\todo{make precise the action of $GL_{n+1}$ on $\mathbf P^n$}

\subsection{Quotient with resepct to linear action}

Fix a linear action $\sigma : G\times X\to X$ of a reductive group $G$
on a projective scheme $X\subseteq\projective{n}{}$. Write
$X=\Proj R(X)$ for some graded $k$-algebra $R(X)$. Then
linearity of the action ensures that the algebra of invariants
inherits a grading:
\begin{align*}
  R(X)^G = \bigoplus_{d\geq 0} R(X)^G_d.
\end{align*}
Now we have the homogeneous ideal $R(X)^G_+\subseteq R(X)^G$.

\begin{definition}
  The \emph{semistable set} of $X$ is the open subscheme
  \begin{align*}
    X^{ss} := X - \Proj R(X)/R(X)_+^G.
  \end{align*}
  The \emph{projective GIT quotient} is the map
  \begin{align*}
    \pi : X^{ss} \to \Proj R(X)^G =: X^{ss}\sslash G
  \end{align*}
  given by the inclusion of finitely generated $k$-algebras
  $R(X)^G \inc R(X)$.
\end{definition}

\begin{theorem}[{\cite[Theorem 5.3]{hoskins}}]
  The projective GIT quotient is a good quotient for the action
  on the subset of semistable points and, moreover,
  $X^{ss}\sslash G$ is a projective scheme.
\end{theorem}

\begin{definition}
  The \emph{stable set} of $X$ is the subscheme $X^s$ of points
  $x\in X$ such that $\dim G_x = 0$ and there is a $G$-invariant
  homogeneous $f\in R(X)^G_+$ such that $x\in X_f$ and the
  action of $G$ on $X_f$ is closed.
\end{definition}

\begin{lemma}[{\cite[Lemma 5.5]{hoskins}}]
  $X^s$ and $X^{ss}$ are open in $X$.
\end{lemma}

\begin{theorem}
  The good quotient $X^{ss}\to X^{ss}\sslash G$ restricts to a
  geometric quotient $X^s \to X^s\sslash G$ such that
  $X^s\sslash G$ is open in $X^{ss}\sslash G$.
\end{theorem}

\begin{proposition}
  There is a bijection between $k$-points of $X^{ss}\sslash G$
  and points in $X^{ss}$ with closed orbit.
\end{proposition}

\subsection{Linearisation}

The projective GIT quotient is well-defined for linear action
on projective schemes. Let us see how we can linearise more general
group actions in order to obtain a quotient.

\begin{definition}
  A \emph{linearisation} of the action of $G$ on $X$ on a line
  bundle $\mathcal L$ is an isormorphism
  \begin{align*}
    \Phi : \sigma^* \mathcal L  \to \pi^*_X\mathcal L
  \end{align*}
  satisfying the cocycle condition
  \question{maybe it would be good to talk about this condition a
    bit more? \cite{hoskins} does it using geometric bundles but
    i think sheaves are nicer; the question is how to present
  things in terms of sheaves in the nicest possible way?}
  \begin{equation}\label{eq:linearisation_cocyle}
    % https://q.uiver.app/#q=WzAsNixbMiwwLCIoXFx0ZXh0e2lkfVxcdGltZXNcXHJobyleKlxccGleKl9YXFxtYXRoY2FsIEwiXSxbMCwwLCIoXFx0ZXh0e2lkfVxcdGltZXNcXHJobyleKlxccmhvXipcXG1hdGhjYWwgTCJdLFs0LDEsIlxccGleKl97MjN9XFxwaV9YXipcXG1hdGhjYWwgTCJdLFsyLDEsIlxccGlfezIzfV4qXFxyaG9eKlxcbWF0aGNhbCBMIl0sWzAsMiwiKFxcbXVcXHRpbWVzXFx0ZXh0e2lkfSleKlxccmhvXipcXG1hdGhjYWwgTCJdLFs0LDIsIihcXG11XFx0aW1lc1xcdGV4dHtpZH0pXipcXHBpX1heKlxcbWF0aGNhbCBMIl0sWzEsMCwiKFxcdGV4dHtpZH1cXHRpbWVzXFxyaG8pXipcXFBoaSJdLFszLDIsIlxccGleKl97MjN9XFxQaGkiXSxbNCwxLCIiLDAseyJsZXZlbCI6Miwic3R5bGUiOnsiaGVhZCI6eyJuYW1lIjoibm9uZSJ9fX1dLFszLDAsIiIsMix7ImxldmVsIjoyLCJzdHlsZSI6eyJoZWFkIjp7Im5hbWUiOiJub25lIn19fV0sWzQsNSwiKFxcbXVcXHRpbWVzXFx0ZXh0e2lkfSleKlxcUGhpIiwyXSxbNSwyLCIiLDEseyJsZXZlbCI6Miwic3R5bGUiOnsiaGVhZCI6eyJuYW1lIjoibm9uZSJ9fX1dXQ==
    \begin{tikzcd}
      {(\text{id}\times\sigma)^*\sigma^*\mathcal L} && {(\text{id}\times\sigma)^*\pi^*_X\mathcal L} \\
                                                    && {\pi_{23}^*\sigma^*\mathcal L} && {\pi^*_{23}\pi_X^*\mathcal L} \\
      {(\mu\times\text{id})^*\sigma^*\mathcal L} &&&& {(\mu\times\text{id})^*\pi_X^*\mathcal L}
      \arrow["{(\text{id}\times\sigma)^*\Phi}", from=1-1, to=1-3]
      \arrow[Rightarrow, no head, from=2-3, to=1-3]
      \arrow["{\pi^*_{23}\Phi}", from=2-3, to=2-5]
      \arrow[Rightarrow, no head, from=3-1, to=1-1]
      \arrow["{(\mu\times\text{id})^*\Phi}"', from=3-1, to=3-5]
      \arrow[Rightarrow, no head, from=3-5, to=2-5]
    \end{tikzcd}
  \end{equation}
\end{definition}

In terms of geometric bundles, this means that a linearisation
is a line bundle $L\to X$ with an isomorphism of bundles
\begin{align*}
  \varphi : G\times L \to \sigma^* L
\end{align*}
such that the map
\begin{align*}
  G\times L \xlongrightarrow{\varphi} \sigma^* L \longrightarrow L
\end{align*}
is an action.

If we have an affine algebraic group $G$ and an action
$\sigma : G\times X\to X$ then the map
\begin{align*}
  H^0(X,\mathcal L)
  \xlongrightarrow{\sigma^*}
  H^0(G\times X,\sigma^*\mathcal L)
  \cong
  H^0(G\times X,\pi_X^*\mathcal L)
  \cong
  H^0(G,\mathcal O_G)
  \otimes
  H^0(X,\mathcal L)
\end{align*}
defines a co-module structure
\begin{align*}
  H^0(X,\mathcal L)\to H^0(X,\mathcal L)\otimes \mathcal O(G).
\end{align*}
That is, we have a natural linear representation
$G\to GL(H^0(X,\mathcal L))$ induced by the linearisation.
\cite[Lemma 5.19]{hoskins}

If $\mathcal L$ is very ample then the evaluation map
\begin{align}\label{eq:evaluation_map}
  H^0(X,\mathcal L) \otimes_k \mathcal O_X \to \mathcal L
\end{align}
which, for each open $U\subseteq X$, is given by given by
$f\otimes_k \lambda \mapsto \lambda\restrict{f}{U}$,
is $G$-equivariant \cite[Remark 5.20]{hoskins}.
I.e. the following commutes
\question{what are the actions here, exactly?}
\begin{equation*}
  % https://q.uiver.app/#q=WzAsNCxbMCwwLCJHXFx0aW1lc1xcbWF0aGJmIFYoSF4wKFgsXFxtYXRoY2FsIEwpXFxvdGltZXNfa1xcbWF0aGNhbCBPX1gpIl0sWzIsMCwiR1xcdGltZXNcXG1hdGhiZiBWKFxcbWF0aGNhbCBMKSJdLFswLDEsIlxcbWF0aGJmIFYoSF4wKFgsXFxtYXRoY2FsIEwpXFxvdGltZXNfa1xcbWF0aGNhbCBPX1gpIl0sWzIsMSwiXFxtYXRoYmYgVihcXG1hdGhjYWwgTCkiXSxbMCwxXSxbMiwzXSxbMCwyXSxbMSwzXV0=
  \begin{tikzcd}
    {G\times\mathbf V(H^0(X,\mathcal L)\otimes_k\mathcal O_X)} && {G\times\mathbf V(\mathcal L)} \\
    {\mathbf V(H^0(X,\mathcal L)\otimes_k\mathcal O_X)} && {\mathbf V(\mathcal L)}
    \arrow[from=1-1, to=1-3]
    \arrow[from=1-1, to=2-1]
    \arrow[from=1-3, to=2-3]
    \arrow[from=2-1, to=2-3]
  \end{tikzcd}
\end{equation*}
Here $\mathbf V = \Spec(\Sym(-))$ is the contravariant functor
$\QCoh(X)^{\text{op}}\to\Sch/X$ which yields an equivalence
between locally free $\mathcal O_X$-modules of rank $n$ and
geometric vector bundles of rank $n$ over $X$.
\cite[Proposition 11.7]{gortz2010}

Consider a linearisation $\mathcal L$ and write $V:=H^0(X,\mathcal L)^*$. Then there is an induced closed embedding
$X\inc\projective{}{}(H^0(X,\mathcal L)^*)$ such that the action of
$G$ on $X$ arises from a linear action of $G$ on $V$. In this sense
a linearisation is a direct generalisation of linear actions
in the sense we have discussed so far.

\subsection{Quotient with respect to linearisation}

Fix an ample linearisation $\mathcal L = \Gamma(-,L)$ of the action
of $G$ on $X$. Consider the graded finitely generated $k$-algebra
\begin{align*}
  R(X,\mathcal L) := \bigoplus_{m\geq 0} H^0(X,\mathcal L^m).
\end{align*}
Now $G$ acts on $H^0(X,\mathcal L^m)$ for each $m$ so we may
consider the invariants
\begin{align*}
  R(X,\mathcal L)^G := \bigoplus_{m\geq 0} H^0(X,\mathcal L^m)^G.
\end{align*}
This is finitely generated.

\begin{definition}
  A point $x\in X$ is \emph{semistable} with respect to $\mathcal L$
  if there is a homoegeneous $s\in R(X,\mathcal L)^G_+$ such that
  $s(x) \neq 0$. Denote by $X^{ss}(\mathcal L)$ the set of all such
  semistable points.

  A point $x\in X$ is \emph{stable} with respect to $\mathcal L$
  if $\dim Gx = \dim G$ and there is a homoegeneous
  $s\in R(X,\mathcal L)^G_+$ such that
  \begin{align*}
    x \in X_\sigma := \left\lbrace{x'\in X : \sigma(x') \neq 0}\right\rbrace
  \end{align*}
  and the action of $G$ on $X_\sigma$ is closed. Denote by
  $X^s(\mathcal L)$ the set of all such stable points.
\end{definition}

\begin{definition}
  The \emph{projective GIT quotient} with respect to $\mathcal L$
  is the morphism
  \begin{align*}
    X^{ss}(\mathcal L) \to \Proj R(X,\mathcal L)^G
    =: X^{ss}(\mathcal L)\sslash G
  \end{align*}
  arising from the inclusion $R(X,\mathcal L)^G \inc R(X,\mathcal L)$.
\end{definition}

\begin{theorem}
  The projective GIT quotient with respect to $\mathcal L$ is a good
  quotient and $X^{ss}(\mathcal L)\sslash G$ is a projective scheme.
  Moreover, the projective GIT quotient restricts to a geometric
  quotient $X^s(\mathcal L)\to X^s(\mathcal L)\sslash G$ where
  $X^s(\mathcal L)\sslash G$ is open in $X^{ss}(\mathcal L)\sslash G$
  and hence quasi-projective.
\end{theorem}

\begin{proposition}
  There is a bijection between $X^{ss}(\mathcal L)\sslash G$
  and closed orbits in $X^{ss}(\mathcal L)$.
\end{proposition}

\subsection{Points of the projective quotient}

\section{Analytification}

\subsection{Complex projective space}

Consider projective space
$\projective{n}{} := \Proj \mathbb{C}[X_0,\ldots,X_n]$.
The corresponding set of $\mathbb{C}$-points is projective space
in the classical sense: $\projective{n}{}(\mathbb{C}) \cong \mathbf{CP}^n$. It is well-known that complex projective space not only
has the structure of a variety, but also of a manifold. In particular,every affine chart $\affine{n}{}\subseteq\projective{n}{}$
gives rise to a chart
$\mathbb{C}^n\cong\affine{n}{}(\mathbb{C})\subseteq\projective{n}{}(\mathbb{C})$.

\chapter{Analytic moduli space}

\section{Holomorphic bundles}

Fix a smooth vector bundle $E$ on a compact Riemann surface $X$
of genus $g\geq 2$. We would now like to define a space that has
as points holomorphic bundles $(E,\dol_E)$ whose underlying smooth
structure agrees with $E$. While we could consider the set of all
such holomorphic structures and possibly even equip it with a
topology, the result is not going to be satisfying. Fortunately,
it is possible to restrict our attention to the reasonable subset
of stable holomorphic bundles which yields a moduli space that
has the structure of a complex manifold.

\subsection{Degree}

If two holomorphic bundles are isomorphic then so are their
underlying smooth bundles. Moreover, if two smooth bundles
are isomorphic then there are certainly of the same rank
but also of the same degree. Let us define the latter.

For every smooth bundle $E$ on $X$ of rank $n$ there are elements
$c_j(E) \in H^{2j}(X,\mathbb{Z})$ for $j\geq 0$. These are
called the Chern classes. There are various ways to construct these
classes. One way is the Chern-Weil approach using connections

and curvature \cite{fine2013}. Another involves the exact sequence
of sheaves
\begin{equation*}
  % https://q.uiver.app/#q=WzAsNSxbMCwwLCIwIl0sWzEsMCwiXFxaIl0sWzIsMCwiXFxtYXRoY2FsIE9fWCJdLFszLDAsIlxcbWF0aGNhbCBPXipfWCJdLFs0LDAsIjAiXSxbMCwxXSxbMSwyXSxbMiwzLCJcXHRleHR7ZXhwfSJdLFszLDRdXQ==
  \begin{tikzcd}
    0 & \mathbb{Z} & {\mathcal O_X} & {\mathcal O^*_X} & 0
    \arrow[from=1-1, to=1-2]
    \arrow[from=1-2, to=1-3]
    \arrow["{\text{exp}}", from=1-3, to=1-4]
    \arrow[from=1-4, to=1-5]
  \end{tikzcd}
\end{equation*}
which gives rise to a map
\begin{align*}
  c_1 : \Pic(X) \cong H^1(X,\mathcal O^*_X)\longrightarrow H^2(X,\mathbb{Z})
\end{align*}
which allows us to define the first Chern class of any rank $r$
bundle by
\begin{align*}
  c_1(E) := c_1(\exterior{r}{E}) \in H^2(X,\mathbb{Z}).
\end{align*}
See \cite{griffiths1994}. We are not going to be interested in how
Chern classes are constructed but rather about what they tell us
about the classification of holomorphic bundles. To this end,
we note the following properties of chern classes:

\begin{lemma}
  For each smooth rank $r$ vector bundle $E$ on $X$,
  $c_j(E) = 0$ whenever $j<1$ or $j>r$.
\end{lemma}

\begin{theorem}\label{thm:chern_exact_sequence}
  If
  \begin{align*}
    0\to E\to E'\to E''\to 0
  \end{align*}
  is an exact sequence of vector bundles then
  \begin{align*}
    c(E') = c(E)c(E'')
  \end{align*}
  where
  \begin{align*}
    c(E) := \sum_j c_j(E) \in H^*(X,\mathbb{Z})
  \end{align*}
  is the total Chern class.
\end{theorem}

A very useful consequence of (\ref{thm:chern_exact_sequence})
is the fact that, for all smooth vector bundles $E$ and $F$,
\begin{align}\label{eq:chern_sum}
  c_1(E\oplus F) = c_1(E) + c_1(F)
\end{align}
in $H^2(X,\mathbb{Z})$.

While Chern classes are useful in their own right, we will
only use the case $j=1$ to define the degree of a vector bundle.

\begin{definition}
  The \emph{degree} of a smooth bundle $E$ on a compact connected
  orientable complex manifold $X$ of dimension $n$ is the integer
  \begin{align*}
    \deg E := c_1(E) [X]
  \end{align*}
  where $[X]\in H^n(X,\mathbb{Z})\cong\mathbb{Z}$.
\end{definition}

In particular, (\ref{eq:chern_sum}) implies
\begin{align}\label{eq:degree_sum}
  \deg (E\oplus F) = \deg E + \deg F
\end{align}
for smooth bundles $E$ and $F$ on $X$.

\subsection{Stability}

The naive moduli space of holomorphic bundles, constructed
as the set of holomorphic bundles up to isomorphism,
i.e. as a quotient of $\Dol E/\Aut E$ as in \ref{sec:dolbeault_operator}, is not Hausdorff and hence not a manifold.
Geometric invariant theory tells us that one way to rectify
this is by identifying a certain subset of \emph{stable}
holomorphic structures and discard the others. This allows us
to obtain a more well-behaved group action and hence a more
desirable quotient.

While it is possible to construct a moduli space of holomorphic
bundles using GIT and the corresponding notion of stability directly,
we are going to use a more analytical, yet equivalent formulation.
Consider the following invariant of a smooth bundle:

\begin{definition}
  The \emph{slope} of a smooth bundle $E$ on $X$ is
  \begin{align*}
    \mu (E) := \deg(E) / \rank(E).
  \end{align*}
\end{definition}

Note that smooth bundles are classified by their rank and
degree. Hence the slope is indeed a fixed invariant of the underlying
smooth structure. How does this help us classify holomorphic bundles?
By looking at the underlying smooth structures of holomorphic
subbundles. In this way we do indeed obtain an invariant of
equivalence classes of holomorphic bundles, as equivalent holomorphic
bundles admit equivalent subbundles. Here is the condition:

\begin{definition}
  A holomorphic bundle $(E,\dol_E)$ is \emph{slope stable} if,
  for all proper non-zero holomorphic subbundles
  $(F,\dol_F)\subset(E,\dol_E)$,
  \begin{align*}
    \mu(F) < \mu(E).
  \end{align*}
  Denote by $\Dol(E)^s$ the subset of $\Dol(E)$ containing only
  stable bundles.
\end{definition}

\subsection{Complex manifold structure on $\Dol(E)$}

We claim that $\Dol(E)^s$ naturally has a complex sturcture,
induced by the complex structure on $\Dol(E)$. Let us
recall the complex struture on the latter first. Consider an
operator $\dol_E\in\Dol(E)$ and another complex linear map
\begin{align*}
  \alpha : \Gamma(X,E) \to \Omega^{0,1}(X)\otimes\Gamma(X,E),
\end{align*}
not necessarily in $\Dol(E)$. We then obtain another map
\begin{align*}
  \dol_E + \alpha : \Gamma(X,E) \to \Omega^{0,1}(X)\otimes \Gamma(X,E).
\end{align*}
Moreover,
\begin{align*}
  (\dol_E + \alpha)(fs) = f\dol_E(s) + f\alpha(s) + \dol f \otimes s
  = f(\dol_E+\alpha)(s) + \dol f \otimes s
\end{align*}
so $\dol_E + \alpha \in \Dol(E)$. Thus $\Dol(E)$ has the structure
of a complex affine space modelled on the infinite dimensional
complex vector space
\begin{align*}
  \Omega^{0,1}(X,\End E) = \Omega^{0,1}(X) \otimes \Gamma(X,E^*) \otimes \Gamma(X,E).
\end{align*}
Therefore, we have a homeomorphism of topological spaces
\begin{align*}
  \Dol(E) \simeq \Omega^{0,1}(X,\End E).
\end{align*}
As we believe in the axiom of choice, we are happy to denote by
$A$ a chosen basis of $\Omega^{0,1}(X,\End E)$.
In particular, we have a basis of $\Dol(E)$ whose elements are
of the form
\begin{align*}
  B(\dol_E,\lambda)
  = \left\lbrace{\dol_E + \sum_{\alpha\in A} \lambda_\alpha \alpha
  : \lambda = \sum_\alpha |\lambda_\alpha|, \#\left\lbrace{\alpha\in A : \lambda_\alpha\neq 0}\right\rbrace < \infty }\right\rbrace
\end{align*}
for $\dol_E\in\Dol(E)$ and $0\neq\lambda\in\mathbb{C}$. That is,
we define a norm on $\Omega^{0,1}(X,\End E)$ by choosing a basis
and then take the opens of $\Dol(E)$ to be generated by open
balls around each point.

\subsection{Complex manifold structure on $\Dol(E)^s$}

Let us make sure that the structure on $\Dol(E)$ induces a
similar structure on $\Dol(E)^s$. That is, we would like
it to be the case that $\Dol(E)^s$ is also an affine space modelled
on $\Omega^{0,1}(X,\End E)$. However, quick observation shows
that this is unlikely: For any two $\dol_E,\dol_E'\in\Dol(E)$
are elements of $\Omega^{0,1}(X,\End E)$ and hence their difference
is too. This means that there exists $\alpha\in\Omega^{0,1}(X,\End E)$
such that $\dol_E' = \dol_E + \alpha$. Now if $\dol_E$ is stable
and $\dol_E'$ is not then this shows that $\Dol(E)^s$ cannot
be modelled on $\Omega^{0,1}(X,\End E)$.

A smooth subbundule $F\subseteq E$ affects slope stability is
relevant for slope stability of $(E,\dol_E)$ if, and only if,
there is a holomorphic structure $\dol_F$ that agrees with
$\dol_E$, i.e. such that the following commutes:
\begin{equation*}
  % https://q.uiver.app/#q=WzAsNCxbMCwxLCJcXEdhbW1hKFgsRSkiXSxbMCwwLCJcXEdhbW1hKFgsRikiXSxbMiwxLCJcXE9tZWdhXnswLDF9KEUpIl0sWzIsMCwiXFxPbWVnYV57MCwxfShGKSJdLFszLDIsIiIsMix7InN0eWxlIjp7InRhaWwiOnsibmFtZSI6Imhvb2siLCJzaWRlIjoidG9wIn19fV0sWzEsMywiXFxiYXJcXHBhcnRpYWxfRiJdLFsxLDAsIiIsMCx7InN0eWxlIjp7InRhaWwiOnsibmFtZSI6Imhvb2siLCJzaWRlIjoidG9wIn19fV0sWzAsMiwiXFxiYXJcXHBhcnRpYWxfRSIsMl1d
  \begin{tikzcd}
    {\Gamma(X,F)} && {\Omega^{0,1}(X,F)} \\
    {\Gamma(X,E)} && {\Omega^{0,1}(X,E)}
    \arrow["{\bar\partial_F}", from=1-1, to=1-3]
    \arrow[hook, from=1-1, to=2-1]
    \arrow[hook, from=1-3, to=2-3]
    \arrow["{\bar\partial_E}"', from=2-1, to=2-3]
  \end{tikzcd}
\end{equation*}
This is equivalent to saying that the restrction of $\dol_E$ to
$F$ induces a holomorphic structure on $F$, i.e. its image is
contained in $\Omega^{0,1}(X,F)$. In order for $\Dol(E)^s$ to be
an affine subspace of $\Dol(E)$, we require there to be a
subspace $V\subseteq\Omega^{0,1}(X,\End E)$ such that
$\dol_E$ restricts to $F$ if, and only if, $\dol_E + \alpha$
restricts to $F$, for all $\alpha \in V$.

Of course, for a fixed $F\subseteq E$, this is the same as saying
that $\alpha$ restricts to a map $\Gamma(X,F)\to\Omega^{0,1}(X,F)$.
For general $F$, this certainly works if the
$\Gamma(X,E)\to\Gamma(X,E)$ component of $\alpha$ respects all
subspaces, i.e. is scalar multiplication. That is, if
$\alpha_0\in\Omega^{0,1}(X)$ and $(E,\dol_E)$ is stable then
$(E,\dol_E + \alpha)$ is
stable too, where we write $\alpha(s) := \alpha_0 \otimes s$.
In this sense, $\Dol(E)^s$ is an affine space modelled on
$\Omega^{0,1}(X)$.

It remains to contemplate the other direction. Is it the case that,
if $(E,\dol_E)$ and $(E,\dol_E + \alpha)$ are stable then
$\alpha$ is of the form $\alpha_0\otimes\identity$ for some
$(0,1)$-form $\alpha_0$? For now, this seems unlikely. In particular,
if two holomorphic structures are related by a non-trivial
automorphism of $E$ then we do not expect their difference to
be trivial along the $\Gamma(X,E)\to\Gamma(X,E)$ component.

\todo{complex structure on $\Dol(E)^s\subseteq\Dol(E)$ as a
submanifold}

\subsection{Quotient}

Now let $\Dol(E)^s$ denote the set of Dolbeault operators corresponding
to stable holomorphic structures on $E$. Then $\Dol(E)^s/\Aut(E)$
is Hausdorff. Hence make the following definition:

\begin{definition}
  The moduli space of stable holomorphic bundles of rank $n$
  and degree $d$ over a Riemannian surface $X$ is
  \begin{align*}
    \mathcal{N}^s(X,E) = \Dol(E)^s / \Aut(E).
  \end{align*}
\end{definition}

\question{what is the complex structure on the group? the quotient?}

It is not hard to see that, for any isomorphic smooth bundles
$E\cong F$ on $X$, the moduli spaces $\mathcal N^s(X,E)$
and $\mathcal N^s(X,F)$ are biholomorphic. \question{is this true?}
Hence, up to biholomorphism, the following is well-defined:
\begin{align*}
  \mathcal N^s_{r,d}(X) := \mathcal N^s(X,E)
\end{align*}
where $E$ is any smooth bundle of rank $r$ and degree $d$ on $X$.

\section{Higgs bundles}

\subsection{Definition}

\begin{definition}
  A \emph{Higgs bundle $(E,\dol_E,\varphi)$} on a Riemann surface
  $X$ consists of a holomorphic vector bundle $(E,\dol_E)$ on $X$
  and a holomorphic bundle map $\varphi : E \to E \otimes K$ where
  $K:=\Lambda^n T^*_{0,1} X$ is the canonical line bundle on $X$.
\end{definition}

Note that the holomorphic bundle map is the same as a section
in $H^0(X,\End E \otimes K)$. To see this, note that
$\End E = E^* \otimes E$ so $\varphi$ induces a bundle map
\begin{align*}
  \mathbb{C}\times X \to \End E \otimes K.
\end{align*}
This induces a map of $\mathbb{C}$-algebras
\begin{align*}
  H^0(X,\mathbb{C}\times X) \to H^0(X,\End(E)\otimes K)
\end{align*}
which is uniquely determined by the image of the constant function
$x\mapsto 1$.

\subsection{Stability}

\begin{definition}
  A Higgs bundle $(E,\dol_E,\varphi)$ is \emph{slope stable}
  if, and only if, for every proper Higgs subbundle
  $(F,\dol_F,\varphi) \subset (E,\dol_E,\varphi)$, $\mu(F) < \mu(E)$.
\end{definition}

\begin{example}
  Every holomorphic bundle $(E,\dol_E)$ gives rise to a trivial
  Higgs bundle $(E,\dol_E,\varphi)$ where $\varphi = 0$ is the zero
  Higgs field. Such a Higgs bundle is stable if, and only if,
  $(E,\dol_E)$ is a stable holomorphic bundle.
\end{example}

It is clear that every Higgs subbundle is a holomorphic subbundle
and hence every Higgs bundle whose underlying holomorphic bundle
is stable must itself be stable. The converse is not true: It
need not be the case that the holomorphic bundle underlying a
stable Higgs bundle is stable.

\subsection{Higgs bundles as cotangent vectors of holomorphic bundles}

Fix a smooth bundle $E$ on a compact Riemann surface $X$. Write
$\mathcal N = \mathcal N^s(X,E)$ and consider a holomorphic structure
$\dol_E\in\mathcal N$. Then the cotangent space of $\mathcal N$ at
$\dol_E$ is
\begin{align}\label{eq:tangent_space_of_n}
  T^*_{\dol_E} \mathcal N = I_{\dol} / I^2_{\dol}
\end{align}
where $I_{\dol}\subseteq \mathcal O(\mathcal N)$ is the ideal
of holomorphic functions on $\mathcal N$ vanishing at $\dol$.
Note that we have a map \todo{actually we don't really... the RHS is
\emph{linear} maps}
\begin{align*}
  I_{\dol} \to (E^*(X)\otimes E(X)\otimes\Omega^{0,1}(X))^*.
\end{align*}
Via the isomorphism of bundles
\begin{align*}
  (E^* \otimes E \otimes \Omega^{0,1})^*
  \cong E^* \otimes E \otimes \Omega^{1,0}
  \cong \End E \otimes K
\end{align*}
this yields a map
\begin{align}
  I_{\dol} \to H^0(X,\End E\otimes K).
\end{align}
Note that $I_{\dol}$ has a complex vector space structure
with respect to which this map is $\mathbb{C}$-linear.
Let us verify that this gives a well-defined map on the quotient
(\ref{eq:tangent_space_of_n}).

\begin{lemma}
  \begin{align*}
    T_{\dol}^*\mathcal N = H^0(X,\End E\otimes K)
  \end{align*}
  \begin{proof}
    \missingproof
  \end{proof}
\end{lemma}

\begin{lemma}\label{lem:analytic_tangent_open_at_point}
  $T^*_{\dol}\mathcal N$ is open in $\mathcal M$.
\end{lemma}

\begin{itemize}
  \item compute $\overline{T^*_{\dol}\mathcal N}$
\end{itemize}

\begin{theorem}
  $T^*\mathcal N \subseteq\mathcal M$ is open and dense.
  \begin{proof}
    The open embeddings $T_{\dol}^*\mathcal N\inc\mathcal M$
    from (\ref{lem:analytic_tangent_open_at_point}) lets us
    define an embedding
    \begin{align*}
      T^*\mathcal N = \bigsqcup_{\dol} T_{\dol}^*\mathcal N
      \longinc \mathcal M
    \end{align*}
    which, moreover, is open as its image is a union of opens
    in $\mathcal M$.
    \missingproof
  \end{proof}
\end{theorem}

\chapter{Algebraic moduli space}

\section{Moduli problems}

While classifying geometric objects of a certain types, one often
finds that the space of such objects has a rich geometric structure.
For example, the space of all lines in $k^{n+1}$ is
$\mathbb{P}^n_k$. The space of all objects of a certain type is
referred to as a moduli space. \cite{bejleri2020}

More precisely, a moduli problem on $S$-schemes is a presheaf
\begin{align*}
  \mathcal M : {(\Sch / S)}^{\text{op}} \to \Set.
\end{align*}
There are various objects that one may consider to be a moduli
space for such a moduli problem.

\subsection{Fine moduli spaces}

The best case scenario arises when a moduli problem $\mathcal M$
is representable. In this case, the moduli problem $\mathcal M$
may itself be thought of as a scheme.

\begin{definition}
  Consider a moduli problem $\mathcal M$ on $\Sch/S$.
  A \emph{fine moduli space} of $\mathcal M$ is a scheme $M\in\Sch/S$
  together with a natural isomorphism
  \begin{align*}
    \eta : \mathcal M \cong \Hom\left({-,M}\right).
  \end{align*}
\end{definition}
It is standard to write $M(T) := \Hom(T,M)$ and hence identify
$M$ with its functor of points. We may then surpress the natural
isomorphism to identify $\mathcal M$ with its fine moduli space $M$.

Before we adopt this abuse of notation, let us consider the
isomorphism $\eta$ a little further. Note we may think of
$F\in \mathcal M(T)$ as the morphism $f : T \to M$.
Moreover, we obtain a pullback map $f^* : \Hom(M,M) \to \Hom(T,M)$
which induces a pullback map $f^* : \mathcal M(M)\to\mathcal M(T)$.
We then observe that $U := {\eta}^{-1}_M(\identity)$
is universal in the sense that every $F$ is obtained
by pulling back $U$ along $f$.

\subsection{Coarse moduli spaces}

\subsection{Grassmannians}

While we are unlikely to actually need Grasmannians, they provide
us with a lot of intuition going forward. Moreover, they were used
by Grothendieck to prove the representability of the Quot scheme
and hence are worth knowing about. Let us recall the elementary
notion first:

\begin{definition}
  Let $V$ be a finite dimensional $k$-vector space. Then
  $G(m,V)$ denotes the set of $n$-dimensional subspaces of $V$.
\end{definition}

For now, this is just a set. However, for suitable choices of
$k$, it has the structure of a differentiable manifold as well as
that of an algebraic variety.
Now note that subspaces $W\in G(m,V)$ give rise to quotients
$V/W$ and we may recover $W=\ker(V\surj V/W)$. Thus subspaces
of $V$ may be identified with isomorphism classes of quotients.
In particular, any isomorphism $V/W \cong V/W$ will make the following
commute:
\begin{equation*}
  % https://q.uiver.app/#q=WzAsMyxbMCwwLCJWIl0sWzIsMCwiVi9XIl0sWzIsMSwiVi9XIl0sWzAsMiwiIiwyLHsic3R5bGUiOnsiaGVhZCI6eyJuYW1lIjoiZXBpIn19fV0sWzAsMSwiIiwwLHsic3R5bGUiOnsiaGVhZCI6eyJuYW1lIjoiZXBpIn19fV0sWzEsMiwiXFxjb25nIl1d
  \begin{tikzcd}
    V && {V/W} \\
      && {V/W}
      \arrow[two heads, from=1-1, to=1-3]
      \arrow[two heads, from=1-1, to=2-3]
      \arrow["\cong", from=1-3, to=2-3]
  \end{tikzcd}
\end{equation*}
Thus we may identify $G(m,V)$ with the set of isomorphism
classes of $k$-linear surjections $V\surj W$ with $\dim V-\dim W=m$.
This allows us to generalise from vector spaces to free sheaves
of modules.~\cite[\href{https://stacks.math.columbia.edu/tag/089R}{Tag 089R}]{stacks-project}. \todo{maybe we want to think and talk about
quotients of sheaves of modules a little more ...}

\begin{definition}
  The functor $G(m,n) : \Sch \to \Set$ associates to
  each scheme $T\in\Sch$ the set $G(m,n)(T)$ of isomorphism
  classes of surjections
  \begin{align*}
    q : \mathcal O^n_T \surj \mathcal F
  \end{align*}
  where $\mathcal F$ is a finite, locally free $\mathcal O_T$-module
  of rank $n-m$ and for each morphism $f:T\to T'$ the map
  $G(m,n)(f)$ sends the isomorphism class of $q$ to that
  of $f^*q$.
\end{definition}

We begin by noting that this does indeed generalise the previous
notion. In particular, there is a canonical bijection
\begin{align*}
  G(m,n)(\Spec k) \cong G(m,k^n)
\end{align*}
as quotients of $\mathcal O^{\oplus n}_{\Spec k}$ are just quotients
of of $k^n$. \question{what does canonical mean here? also, we would want this to be at least an isomorphism of varieties in some sense but i haven't seen anyone talk about this...}

This is a functor and, crucially, representable. Thus we
have a scheme $\mathbf G(m,n)$ with
\begin{align*}
  \mathbf G(m,n)(T) := \Hom(T,\mathbf G(m,n)) = G(m,n)(T).
\end{align*}
Base change gives us the usual $\mathbf G(m,n)_S\in\Sch_S$ and
$\mathbf G(m,n)_R\in\Sch_R$.
Moreover, we are justified in thinking of $\mathbf G(m,n)_S$ as
parametrising $m$-dimensional subspaces of $\affine{n}{S}$ in the
sense that we think of projective $n$-space as parametrising
$n$-dimensional subspaces of $(n+1)$-dimensional affine space.

\begin{proposition}
  Let $n\geq 1$ and $S\in\Sch$. Then there is a canonical isomorphism
  \begin{align*}
    \mathbf G(n,n+1)_S \cong \projective{n}{S}.
  \end{align*}
  \begin{proof}
    It suffices to consider $S=\Spec\mathbb{Z}$.
    See \cite[\href{https://stacks.math.columbia.edu/tag/089V}{Tag 089V}]{stacks-project}.
  \end{proof}
\end{proposition}

\subsection{Quot schemes}

Fix a Noetherian base scheme $S$, a scheme $X/S$ of finite type,
a coherent sheaf $\mathcal E$ on $X$, and another scheme $T/S$.

\begin{definition}
  A \emph{family of quotients of $\mathcal E$ parameterised by $T$} consists of
  \begin{enumerate}
    \item a coherent sheaf $\mathcal F$ on $X_T$
      such that the schematic support of $\mathcal F$ is proper
      over $T$ and $\mathcal F$ is flat over $T$, and
    \item a surjective $\mathcal O_{X_T}$-linear
      homorphism of sheaves $q:\mathcal E_T\to \mathcal F$.
  \end{enumerate}
  A morphism $f:(\mathcal F,q)\to(\mathcal F',q')$ of such families
  is a morphism $f:\mathcal F\to\mathcal F'$ that makes the following
  commute:
  \begin{equation*}
    % https://q.uiver.app/#q=WzAsMyxbMiwwLCJcXG1hdGhjYWwgRiJdLFsyLDIsIlxcbWF0aGNhbCBGJyJdLFswLDAsIkVfVCJdLFsyLDAsInEiXSxbMiwxLCJxJyIsMl0sWzAsMSwiZiJdXQ==
    \begin{tikzcd}
      {\mathcal E_T} && {\mathcal F} \\
      \\
                     && {\mathcal F'}
                     \arrow["q", from=1-1, to=1-3]
                     \arrow["{q'}"', from=1-1, to=3-3]
                     \arrow["f", from=1-3, to=3-3]
    \end{tikzcd}
  \end{equation*}
\end{definition}

\begin{example}
  If $S=\Spec\mathbb C$, $X=\Sigma_g$, and
  $\mathcal E=\mathcal O^{\oplus k}_X$.
  Then a family of quotients $q:\mathcal E_T\surj\mathcal F$
  is a quotient
  \begin{align*}
    (\mathcal O^{\oplus k}_X)_T \cong \mathcal O^{\oplus k}_{X\times_S T} \surj \mathcal F
  \end{align*}
  in $\Coh(X\times_S T)$.
  The kernel of such a quotient turns out to be locally-free
  and of finite-rank \cite[Lemma 4.21]{bertram1993}, i.e. a
  vector bundle on $X\times_S T$.
\end{example}

\begin{lemma}
  If $\mathcal F\in\Coh(X_T)$ is flat over $T$ and
  $q:\mathcal O_{X_T}^{\oplus k} \surj\mathcal F$ is surjective and
  $\mathcal O_{X_T}$-linear then $\Ker(q)\in\Coh(X_T)$ is
  finite locally free.
  \begin{proof}
    We want to show that there is

    Recall that flatness of $\mathcal F$ over $T$ means that,
    for all $x\in X_T$, the stalk $\mathcal F_x$ is a flat
    $\mathcal O_{T,\pi(x)}$-module where
    $\pi : X_T \to T$.
    \missingproof
  \end{proof}
\end{lemma}

\begin{lemma}
  Families $(\mathcal F,q)$ and $(\mathcal F',q')$ are isomorphic if, and only if,
  $\ker(q) = \ker(q')$.
  \begin{proof}
    \missingproof
  \end{proof}
\end{lemma}

Write $\langle\mathcal F,q\rangle$ for equivalence classes
of such families and define the corresponding quot functor
as follows: \todo{the construction with respect to arbitrary
base schemes $S$ may be a bit flawed here, but in the case where $S=\Spec\mathbb{C}$ everything is going to work out so we are not particularly worried.}
\begin{definition}
  The quot functor $\Sch_S^{\text{op}} \to \Set$ is given by
  \begin{align*}
    \mathfrak{Quot}_{\mathcal E,X/S} (T)
    := \left\lbrace{\text{all $\langle\mathcal F,q\rangle$
    parametrised by $T$}}\right\rbrace
  \end{align*}
\end{definition}

The map on morphsims $f:T'\to T$ is as follows.
We may pull back $\text{Coh}(X_T)$ to $\text{Coh}(X_{T'})$
along $\identity\times_S f:X_{T'}\to X_T$. In particular,
note that the following commutes:

\begin{equation*}
  % https://q.uiver.app/#q=WzAsMyxbMiwwLCJcXHRleHR7Q29ofShYX1QpIl0sWzIsMSwiXFx0ZXh0e0NvaH0oWF97VCd9KSJdLFswLDAsIlxcdGV4dHtDb2h9KFgpIl0sWzAsMSwiKFxcdGV4dHtpZH1cXHRpbWVzIGYpXioiXSxbMiwwLCJcXHBpXioiXSxbMiwxLCJcXHBpXioiLDJdXQ==
  \begin{tikzcd}
    {\text{Coh}(X)} && {\text{Coh}(X_T)} \\
                    && {\text{Coh}(X_{T'})}
                    \arrow["{(\text{id}\times f)^*}", from=1-3, to=2-3]
                    \arrow["{\pi^*}", from=1-1, to=1-3]
                    \arrow["{\pi^*}"', from=1-1, to=2-3]
  \end{tikzcd}
\end{equation*}
by functoriality of $(-)^*$. Hence, given a family
$\langle\mathcal F,q\rangle$, we obtain a diagram
\begin{equation*}
  % https://q.uiver.app/#q=WzAsMyxbMCwwLCJFX3tUJ30iXSxbMSwwLCIoXFx0ZXh0e2lkfVxcdGltZXNfU2YpXiogXFxtYXRoY2FsIEVfVCJdLFszLDAsIihcXHRleHR7aWR9XFx0aW1lc19TZileKlxcbWF0aGNhbCBGIl0sWzEsMiwiKFxcdGV4dHtpZH1cXHRpbWVzX1NmKV4qcSJdLFswLDEsIiIsMCx7ImxldmVsIjoyLCJzdHlsZSI6eyJoZWFkIjp7Im5hbWUiOiJub25lIn19fV1d
  \begin{tikzcd}
    {\mathcal E_{T'}} & {(\text{id}\times_Sf)^* \mathcal E_T} && {(\text{id}\times_Sf)^*\mathcal F}
    \arrow["{(\text{id}\times_Sf)^*q}", from=1-2, to=1-4]
    \arrow[Rightarrow, no head, from=1-1, to=1-2]
  \end{tikzcd}
\end{equation*}
in $\Coh(X_{T'})$, i.e. a family of quotients parametrised by
$T'$. This respects isomorphisms of such families and
hence we have defined a map
\begin{align*}
  \mathfrak{Quot}_{\mathcal E,X/S}(T) \to \mathfrak{Quot}_{\mathcal E,X/S}(T').
\end{align*}

\begin{theorem}
  There is a scheme
  \begin{align*}
    \Quot_{X/S}(\mathcal E) \in \Sch_S
  \end{align*}
  that represents $\mathfrak{Quot}_{\mathcal E,X/S}$.
\end{theorem}

\subsection{Splitting of the Quot scheme}

Fix a coherent sheaf $\mathcal E$ and an ample line bundle
$\mathcal L=\Gamma(-,L)$ on a complex projective scheme $X$.

\begin{definition}
  The \emph{Euler characteristic} of $X$ with respect to a
  coherent sheaf $\mathcal E$ is
  \begin{align*}
    \chi(X,\mathcal E) := \sum_{j\geq 0} (-1)^j \dim_k H^j(X,\mathcal E).
  \end{align*}
\end{definition}

\begin{example}\label{ex:euler_char}
  Consider the case where $X$ is a smooth projective curve of
  genus $g$ and $\mathcal E$ is a vector bundle of rank $r$ and
  degree $d$. If $\mathcal L$ has degree $1$ then, for $m$
  sufficiently large,
  \begin{align*}
    \chi(X,\mathcal E\otimes\mathcal L^m) = d + mr + r(1-g).
  \end{align*}
\end{example}

\begin{proposition}
  There exists a polynomial $P(\mathcal E,\mathcal L)\in\mathbb{Q}[t]$
  such that, for $m\in\mathbb{N}$ sufficiently large,
  \begin{align*}
    P(\mathcal E,\mathcal L)(m) = \chi(X,\mathcal E\otimes\mathcal L^m).
  \end{align*}
  This is called the \emph{Hilbert polynomial} of $\mathcal E$
  with respect to $\mathcal L$.
\end{proposition}

\begin{example}\label{ex:hilbert_polynomial}
  In the setting of \ref{ex:euler_char}, we have
  \begin{align*}
    P(\mathcal E,\mathcal L)(t) = rt + d + r(1-g).
  \end{align*}
\end{example}

In particular, using Serre's vanishing theorem, for $m$ sufficiently
large,
\begin{align*}
  P(\mathcal E,\mathcal L)(m) = \dim_k H^0(X,\mathcal E\otimes\mathcal L^{m}).
\end{align*}

\begin{proposition}
  The Quot scheme of $\mathcal E$ decomposes into subschemes of quotients with fixed Hilbert polynomial. That is,
  \begin{align*}
    \Quot_{X}(\mathcal E) = \bigsqcup_{P\in\mathbb{Q}[t]} \Quot_X^{P,\mathcal L}(\mathcal E)
  \end{align*}
  where
  $\Quot_X^{P,\mathcal L}$ is the subscheme of quotients
  $q:\mathcal E_T \surj F$ such that $P(\mathcal F,\mathcal L)=P$.
\end{proposition}

We established that, in the setting of \ref{ex:euler_char},
two quotients $\mathcal F$ and $\mathcal F'$ have the same
Hilbert polynomial with respect to $\mathcal L$ if, and only if,
they have the same rank and degree. Noticing that such an
$\mathcal L$ always exists, we obtain a decomposition
\begin{align*}
  \Quot_{X}(\mathcal E) = \bigsqcup_{r,d} \Quot_X^{r,d}(\mathcal E)
\end{align*}
which does not depend on $\mathcal L$.

\section{Holomorphic bundles}

\subsection{Degree}

Holomorphic bundles in the analytic sense are grouped into classes
of equal rank and degree. It is clear what it means for a locally
free sheaf to have a certain rank but it is not obvious how we
ought to think about rank. We follow \cite[\href{https://stacks.math.columbia.edu/tag/0AYQ}{Tag 0AYQ}]{stacks-project}.

\todo{figure out what this is for a line bundle}

This allows us to define the degree of a finite rank locally free
sheaf in the case where $X$ is a curve:
\begin{definition}
  Let $X$ be a proper scheme over $k$ with $\dim X = 1$
  and let $\mathcal E$ be a locally free sheaf of rank $n$.
  The \emph{degree of $\mathcal E$} is the integer
  \begin{align*}
    \deg(\mathcal E) := \chi(X,\mathcal E)-n\chi(X,\mathcal O_X).
  \end{align*}
\end{definition}

\begin{example}
  $\deg \mathcal O_X = 0$.
\end{example}

\question{what about $\mathcal O_X^N$?}

\subsection{The moduli space as a scheme}

We aim to define the moduli space of (semi)stable holomorphic
bundles on a suitable curve. We will do so without justification.
For details see \cite[section 8.8]{hoskins}.

Fix a connected smooth projective curve $C$ of genus $g\geq 2$,
$n\geq 1$, and $d > n(2g - 1)$. Write $N:=d+n(1-g)$ and
\begin{align*}
  Q := \Quot^{n,d}_X(\mathcal O^N_X).
\end{align*}

\begin{remark}
  We claim that this captures holomorphic bundles on $X$. Let us see
  how. In particular, how does an isomorphism class of quotients
  \begin{align*}
    q : \mathcal O^N_{X\times T} \to \mathcal F
  \end{align*}
  in $\Coh(X\times T)$ constitute a bundle on $X$? Firstly,
  we observed that the kernel of such a quotient is a locally
  free sheaf of rank $N$, \todo{actually this is more involved because we used $n$ and $d$} hence a geometric vector bundle on
  $X\times T$. Secondly, note if we choose $T=\Spec\mathbb{C}$ then
  $Q^s(T)$ does indeed consist of holomorphic bundles on
  $X\times\Spec\mathbb{C}=X$. Thus $\mathbb{C}$-points of
  $Q^s$ are holomorphic bundles on $X$.
\end{remark}

As $Q$ is a fine moduli space, we have a universal quotient
$q:\mathcal O_{Q\times X}^N\surj \mathcal U$. A group action
\begin{align*}
  \sigma : GL_N \times Q \to Q
\end{align*}
is an element of $\Hom(GL_N\times Q,Q)$ and hence corresponds to a
quotient of $\mathcal O_{GL_N\times Q\times X}$ which we construct
as:
\begin{align}
  \label{eq:stable_quot_gln_action}
  \mathcal O^N_{GL_N\times Q\times X}
  \xlongrightarrow{\pi^*_{GL_N}\tau}
  \mathcal O^N_{GL_N\times Q\times X}
  \xlongrightarrow{\pi^*_{Q\times X} q}
  \pi^*_{Q\times X}\mathcal U
\end{align}
where $\pi_{GL_N}$ and $\pi_{Q\times X}$ are the obvious projections
and the sheaf isomorphism
\begin{align*}
  \tau : \mathcal O_{GL_N}^N \to \mathcal O_{GL_N}^N
\end{align*}
arises from the inverse map $i : GL_N \to GL_N$. On $k$-points this
action is given by $g\cdot q := q\circ {g}^{-1}$.

We now linearise this action. We define
\begin{align*}
  \mathcal L := \det(\pi_{Q*}(\mathcal U\otimes \pi^*_X\mathcal O_X(m)))
\end{align*}
\begin{lemma}
  For $m$ sufficiently large, $\mathcal L$ is an ample line bundle on $Q$.
  \begin{proof}
    \missingproof
  \end{proof}
\end{lemma}

We are looking to define an
isomorphism
\begin{align}\label{eq:linearisation_iso}
  \Phi : \sigma^*\mathcal L \to \pi_Q^*\mathcal L
\end{align}
that satisfies the cocyle condition. We begin by considering the
families
\begin{align*}
  k^N \otimes \mathcal O_{GL_N\times Q\times X}
  \xlongrightarrow{(\sigma\times\identity)^* q_Q}
  (\sigma\times\identity)^*\mathcal U
\end{align*}
and
\begin{align*}
  k^N \otimes \mathcal O_{GL_N\times Q\times X}
  \xlongrightarrow{\pi_{GL_N}^*\tau}
  k^N \otimes \mathcal O_{GL_N\times Q\times X}
  \xlongrightarrow{\pi^*_{Q\times X} q_Q}
  \pi^*_{Q\times X} \mathcal U
\end{align*}
of quotients on $GL_N\times Q\times X$. By construction of the
action of $GL_N$ on $Q$, these are equivalent. Hence we have
an isomorphism
\begin{align}\label{eq:universal_linearisation}
  \Phi : (\sigma \times \identity)^*\mathcal U
  \to \pi^*_{Q\times X}\mathcal U
\end{align}
\begin{lemma}
  The isomorphism (\ref{eq:universal_linearisation}) satisfies
  the cocyle condition (\ref{eq:linearisation_cocyle}).
  \begin{proof}
    \missingproof
  \end{proof}
\end{lemma}
Hence we have a linearisation of the $GL_N$-action on $Q\times X$
via $\mathcal U$. This extends to a linearisation of the action
on $Q$ via $\mathcal L$ for $m$ sufficiently large. \todo{make precise}



\begin{definition}
  \begin{align*}
    \pi^s : R^s = Q^s(\mathcal L_m) \to Q\sslash_{\mathcal L_m}^s SL_N =: M^s(n,d).
  \end{align*}
\end{definition}

\todo{say in what sense this actually is a moduli space}

\subsection{Analytification}

We will primarily be concerned with the $\mathbb{C}$-points of our
moduli space which, by construction, correspond to holomorphic
bundles on $X$. Hence it will be fruitful to understand what it
means for a $\mathbb{C}$-point to be stable. It turns out that
the necessary and sufficient condition is quite reminscent of
slope stability in the analytic case.

\begin{proposition}
  For $m$ sufficiently large and
  $q:\mathcal O_X^N\surj\mathcal F\in Q(k)$,
  the following are equivalent:
  \begin{enumerate}
    \item $q\in Q^s(\mathcal L)$;
    \item for all subsheaves $\mathcal F'\subseteq \mathcal F$ with
      $V' := H^0(X,q)^{-1}H^0(X,\mathcal F') \neq 0$,
      $\rank\mathcal F'>0$ and
      \begin{align*}
        \frac{\dim V'}{\rank \mathcal F'}
        <\frac{\dim k^N}{\rank\mathcal F}.
      \end{align*}
  \end{enumerate}
\end{proposition}
Here the picture one has in mind is the following:
\begin{equation*}
  % https://q.uiver.app/#q=WzAsNCxbMCwwLCJIXjAoWCxrXk5cXG90aW1lc1xcbWF0aGNhbCBPX1gpIl0sWzIsMCwiSF4wKFgsXFxtYXRoY2FsIEYpIl0sWzIsMSwiSF4wKFgsXFxtYXRoY2FsIEYnKSJdLFswLDEsIkheMChxKV57LTF9SF4wKFgsXFxtYXRoY2FsIEYnKSJdLFswLDEsIkheMChYLHEpIiwwLHsic3R5bGUiOnsiaGVhZCI6eyJuYW1lIjoiZXBpIn19fV0sWzIsMSwiIiwyLHsic3R5bGUiOnsidGFpbCI6eyJuYW1lIjoiaG9vayIsInNpZGUiOiJ0b3AifX19XSxbMywyLCJIXjAoWCxxKSIsMix7InN0eWxlIjp7ImhlYWQiOnsibmFtZSI6ImVwaSJ9fX1dLFszLDAsIiIsMCx7InN0eWxlIjp7InRhaWwiOnsibmFtZSI6Imhvb2siLCJzaWRlIjoidG9wIn19fV1d
  \begin{tikzcd}
    {H^0(X,k^N\otimes\mathcal O_X)} && {H^0(X,\mathcal F)} \\
    {H^0(X,q)^{-1}H^0(X,\mathcal F')} && {H^0(X,\mathcal F')}
    \arrow["{H^0(X,q)}", two heads, from=1-1, to=1-3]
    \arrow[hook, from=2-1, to=1-1]
    \arrow["{H^0(X,q)}"', two heads, from=2-1, to=2-3]
    \arrow[hook, from=2-3, to=1-3]
  \end{tikzcd}
\end{equation*}
We know that we have a bijection
\begin{align*}
  Q^s(\mathcal L) \sslash GL_N(\mathbb{C}) \cong Q^s(\mathcal L)(\mathbb{C}) / GL_N(\mathbb{C}).
\end{align*}
Moreover, we know that the action of $g\in GL_N(\mathbb{C})$ on
$q : \mathcal O_X^N \surj \mathcal F \in Q(\mathbb{C})$
is given by
\begin{align*}
  g\cdot q : \mathbb{C}^N \otimes \mathcal O_X
  \xlongrightarrow{{g}^{-1}}
  \mathbb{C}^N\otimes\mathcal O_X
  \xlongrightarrow{q}
  \mathcal F.
\end{align*}
Now we observe the following:
\begin{lemma}
  Let $q,q':\mathcal O^N_X\surj\mathcal F\in Q^s(\mathbb{C})$.
  Then $q' = g \cdot q$ for some $g$ if, and only if,
  $\Ker(q)\cong\Ker(q')$.
  \begin{proof}
    By contemplating the following diagram:
    \begin{equation*}
      % https://q.uiver.app/#q=WzAsMTAsWzUsMCwiXFxtYXRoY2FsIEYiXSxbMywwLCJcXG1hdGhjYWwgT15OX1giXSxbMSwwLCJcXHRleHR7S2VyfShxKSJdLFszLDEsIlxcbWF0aGNhbCBPX1heTiJdLFsxLDEsIlxcdGV4dHtLZXJ9KHEnKSJdLFs1LDEsIlxcbWF0aGNhbCBGIl0sWzAsMCwiMCJdLFswLDEsIjAiXSxbNiwwLCIwIl0sWzYsMSwiMCJdLFsyLDFdLFsxLDAsInEiLDAseyJzdHlsZSI6eyJoZWFkIjp7Im5hbWUiOiJlcGkifX19XSxbNCwzXSxbMiw0LCJcXGNvbmciLDJdLFszLDUsInEnIiwyLHsic3R5bGUiOnsiaGVhZCI6eyJuYW1lIjoiZXBpIn19fV0sWzAsNSwiIiwwLHsibGV2ZWwiOjIsInN0eWxlIjp7ImhlYWQiOnsibmFtZSI6Im5vbmUifX19XSxbNiwyXSxbNyw0XSxbMCw4XSxbNSw5XSxbMSwzLCJnIl1d
      \begin{tikzcd}
        0 & {\text{Ker}(q)} && {\mathcal O^N_X} && {\mathcal F} & 0 \\
        0 & {\text{Ker}(q')} && {\mathcal O_X^N} && {\mathcal F} & 0
        \arrow[from=1-1, to=1-2]
        \arrow[from=1-2, to=1-4]
        \arrow["\cong"', from=1-2, to=2-2]
        \arrow["q", two heads, from=1-4, to=1-6]
        \arrow["g", from=1-4, to=2-4]
        \arrow[from=1-6, to=1-7]
        \arrow[Rightarrow, no head, from=1-6, to=2-6]
        \arrow[from=2-1, to=2-2]
        \arrow[from=2-2, to=2-4]
        \arrow["{q'}"', two heads, from=2-4, to=2-6]
        \arrow[from=2-6, to=2-7]
      \end{tikzcd}
    \end{equation*}
  \end{proof}
\end{lemma}
Hence the $\mathbb{C}$-points of $M^s(n,d)$ are exactly
isomorphism classes of holomorphic bundles. Moreover, $M^s(n,d)$
is quasi-projective and hence there is an inclusion
\begin{align*}
  M^s(n,d)(\mathbb{C}) \longinc \projective{}{}
\end{align*}

\subsection{Tangent space}

\begin{theorem}

\end{theorem}

\begin{theorem}\label{thm:holomorphic_zariski_tangent}
  \begin{align*}
    T_{\mathcal E} M^s(n,d) = H^1(X,\Hom(\mathcal E,\mathcal E))
  \end{align*}
  \begin{proof}
    In \cite[Remark 8.66]{hoskins} it is claimed that
    \begin{align*}
      T_{\mathcal E} M^s(n,d) = \Ext^1(\mathcal E,\mathcal E).
    \end{align*}
    Now the claim follows from
    $\Ext^1(\mathcal E,\mathcal F) = H^1(X,\Hom(\mathcal E,\mathcal F))$.
    \missingproof
  \end{proof}
\end{theorem}

Note elements of $\Ext^1$ are equivalence classes of extensions
\begin{align*}
  0 \longrightarrow \mathcal E \longrightarrow \mathcal F\longrightarrow \mathcal E\longrightarrow 0.
\end{align*}
We are looking to construct an $\mathcal O_X$-linear map
\begin{align*}
  \phi : \mathcal E \to \mathcal E \otimes_{\mathcal O_X} \Omega^1_X.
\end{align*}
Have
\begin{align*}
  \Ext^1(\mathcal E,\mathcal E) = R^1\Hom(\mathcal E,-)(\mathcal E)
\end{align*}
want
\begin{align*}
  \Hom(\mathcal E,\mathcal E\otimes_{\mathcal O_X}\Omega^1_X).
\end{align*}

\section{Higgs bundles}

In complete analogy to the analytic case, Higgs bundles have a
sheaf theoretic definition:

\begin{definition}
  A \emph{Higgs field} on a locally free sheaf $\mathcal E$
  on a complex scheme $X$ is an $\mathcal O_X$-linear map
  \begin{align*}
    \phi : \mathcal E \to \mathcal E \otimes \mathcal K_X
  \end{align*}
  where $\mathcal K_X := \Omega^1_X = \det T^* X$ is the canonical locally free sheaf on $X$.
  Hence a \emph{Higgs sheaf} is a locally free sheaf equipped
  with a Higgs field.
\end{definition}

\subsection{As cotangent vectors of holomorphic bundles}

Recall the following property of locally free sheaves:

% https://math.stackexchange.com/questions/3912257/global-section-of-a-locally-free-sheaf
\begin{lemma}[{\cite[Appendix A.3.C6]{hartshorne1977}}]\label{lem:global_sections_of_locally_free_sheaves}
  Consider a locally free sheaf $\mathcal E$ on $X$. There is an
  $\mathcal O(X)$-linear bijection
  \begin{align*}
    H^0(X,\mathcal E) \cong \Hom(\mathcal O_X,\mathcal E).
  \end{align*}
\end{lemma}

Hence the space of Higgs fields of a fixed locally free sheaf
$\mathcal E$ is
\begin{align*}
  \Hom(\mathcal E,\mathcal E\otimes\mathcal K_X)
  &\cong \Hom(\mathcal O_X,\mathcal E^*\otimes\mathcal E\otimes\mathcal K_X) \\
  &\cong H^0(X,\mathcal E^*\otimes\mathcal E\otimes\mathcal K_X)
  &\text{(Lemma \ref{lem:global_sections_of_locally_free_sheaves})} \\
  &\cong H^0(X,\Hom(\mathcal E^*\otimes\mathcal E,\mathcal K_X)) \\
  &\cong H^1(X,\mathcal E^*\otimes\mathcal E)^* &\text{(Serre duality)} \\
  &\cong T^*_{\mathcal E}\mathcal N^s(n,d)
  &\text{(Theorem \ref{thm:holomorphic_zariski_tangent})}
\end{align*}
where $\mathcal E^* := \Hom(\mathcal E,\mathcal O_X)$
is the dual $\mathcal O_X$-module.

\chapter{Comparing the analytic and algebraic spaces}


\printbibliography

\end{document}
