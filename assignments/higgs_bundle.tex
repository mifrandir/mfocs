\documentclass{article}
\usepackage{assignment}
\addbibresource{higgs.bib}
\begin{document}
\title{Moduli Spaces of Holomorphic Bundles}
\author{Franz Miltz}
\date{\today}
\maketitle

\tableofcontents
\pagebreak

\section{Complex vector bundles}

\subsection{Complex manifolds}

A complex vector space is just a real vector space $V$ together with a
$J\in\Aut(V)$ such that $J^2 = -\Identity$. We similarly define
a complex structure on a real manifold to be automorphisms of the
tangent bundle:

\begin{definition}
  Let $X$ be a real manifold. A \emph{complex structure} on $X$
  is an automorphism of real vector bundles $J\in\Aut(TX)$ such
  that $J^2 = -\Identity$. The pair $(X,J)$ is referred to as an
  \emph{almost complex manifold}.
\end{definition}

\begin{itemize}
  \item A complex structure smoothly assign a complex structure to each tangent
    space. One may also think of a complex structure as a $(1,1)$-tensor field
    $J\in\Gamma(TX\otimes T^* X)$.
  \item Any real manifold that admits a complex structure must have even
    dimension. This follows immediately from the analogous observation for real
    vector spaces.
  \item Each complex $n$-manifold $X$ induces a complex structure $J$ on the
    real $2n$-manifold $X$. That is, every complex manifold is an almost
    complex manifold. The converse need not be true. \missingexample
  \item On a complex manifold $X$, the complex structure has eigenvalues
    $\pm i$. This induces a splitting
    \begin{align*}
      T_x X = T^{1,0}_x X \oplus T^{0,1}_x X
    \end{align*}
    for each $x\in X$ where $T^{1,0}_x X$ and $T^{0,1}_x X$ correspond
    to the eigenvalues $i$ and $-i$, respectively. There are obvious
    vector bundles $T^{1,0} X$ and $T^{0,1} X$. \question{Is it
    necessarily the case that $\rank(T^{1,0} X) = \rank(TX)/2$?}
  \item By choosing coordinates around $x$ and writing $(\partial / \partial
    z_j)_x = (\partial / \partial x_j - i\partial / \partial y_j)_x / 2$ we
    have bases
    \begin{align*}
      \left\lbrace{\left(\frac{\partial}{\partial z_j}\right)_x}\right\rbrace_j, \hspace{1cm}
      \left\lbrace{\left(\frac{\partial}{\partial \bar z_j}\right)_x}\right\rbrace_j
    \end{align*}
    for $T^{1,0}_x X$ and $T^{0,1}_x X$, respectively.
  \item Analogously, we obtain the splitting
    \begin{align*}
      T^* X = T^*_{1,0} X \oplus T^*_{0,1} X
    \end{align*}
    This splitting extends to exterior powers of the cotangent bundle:
    \begin{align*}
      \Lambda^k T^*X
      = \bigoplus_{p+q = k} T^*_{p,q} X
      = \bigoplus_{p+q=k}\left({\exterior{p}{T^*_{1,0} X}}\right)\wedge
      \left({\exterior{q}{T^*_{0,1} X}}\right).
    \end{align*}
    We call sections of $\Lambda^{p,q} X$ $(p,q)$-forms and write
    $\Omega^{p,q}(X)=\Gamma(T^*_{p,q}X)$.
\end{itemize}

\begin{example}[{\cite[Section 5.2]{neitzke2021}}]
  Let $E\to X$ be a smooth rank $k$ vector bundle
  with $d=\deg E$. Then $\End E$ is a complex manifold whose
  structure is induced by the identification
  $\End E = E\otimes E^*$.
  Moreover, we have a projection map $\End(E)\to X$ induced by
  the same argument.
  If $E$ is holomorphic then so is $\End(E)$.
\end{example}

\subsection{Dolbeault operator}

\begin{definition}
  A \emph{Dolbeault operator} on a smooth vector bundle
  $E$ on a complex manifold $X$ is a $\mathbb{C}$-linear operator
  \begin{align*}
    \dol_E : \Gamma(X,E) \to \Omega^{0,1}(E)
  \end{align*}
  such that
  \begin{enumerate}
    \item $\dol_E^2 = 0$,
    \item $\dol_E(fs) = \dol f \otimes s + f\dol_E s$
      for $s\in\Gamma(X,E)$ and $f\in\Omega^0(X)$.
  \end{enumerate}

  A morphism $\dol\to\dol'$ of Dolbeaut operators
  on vector bundles $E$ and $E'$, respectively, is a smooth morphism
  of vector bundles $E\to E'$ such that the following commutes:
  \begin{equation*}
    % https://q.uiver.app/#q=WzAsNCxbMCwwLCJcXEdhbW1hKFgsRSkiXSxbMiwwLCJcXE9tZWdhXnswLDF9KEUpIl0sWzAsMSwiXFxHYW1tYShYLEYpIl0sWzIsMSwiXFxPbWVnYV57MCwxfShGKSJdLFsyLDMsIlxcYmFyXFxwYXJ0aWFsXzIiXSxbMCwxLCJcXGJhclxccGFydGlhbF8xIl0sWzAsMiwiXFxHYW1tYShYLFxccGhpKSIsMl0sWzEsMywiXFxPbWVnYV57MCwxfShcXHBoaSkiXV0=
    \begin{tikzcd}
      {\Gamma(X,E)} && {\Omega^{0,1}(E)} \\
      {\Gamma(X,F)} && {\Omega^{0,1}(F)}
      \arrow["{\dol_1}", from=1-1, to=1-3]
      \arrow["{\Gamma(X,\phi)}"', from=1-1, to=2-1]
      \arrow["{\Omega^{0,1}(\phi)}", from=1-3, to=2-3]
      \arrow["{\dol_2}", from=2-1, to=2-3]
    \end{tikzcd}
  \end{equation*}
  Thus we have a category $\textbf{Dol}(X)$ of smooth vector bundles
  and Dolbeault operators on a complex vector manifold $X$
  with a full subcategory $\textbf{Dol}(E)$ for each smooth vector
  bundle $E$ on $X$.
\end{definition}

\begin{lemma}
  Let $E\to X$ be a smooth vector bundle on a complex manifold $X$.
  Then there is an equivalence of categories
  \begin{align}\label{eq:equivalence_dolbeault_operators}
    \textbf{Dol}(E) \cong \textbf{HolBun}(E)
  \end{align}
  where $\textbf{HolBun}(E)$ is the category of holomorphic vector
  bundles with underlying smooth vector bundle $E$.
  \begin{proof}
    \question{do we need to make assumptions about $X$ here? it is certainly true for $X$ a curve.}
    \missingproof
  \end{proof}
\end{lemma}
Hence we are justified in denoting a holomorphic vector bundle
on $X$ by $(E,\dol)$ where $E$ is a smooth bundle and
$\dol$ is a corresponding Dolbeault operator.

\begin{remark}
  What we want to be true is that the functor
  \begin{align*}
    \textbf{Dol}:\textbf{Bun}\to\textbf{CAT}
  \end{align*}
  is an full and faithful. At least we want it to be the case
  that an equivalence $\textbf{Dol}(E)\cong\textbf{Dol}(F)$
  yields an isomorphism $E\cong F$. But this is not obvious
  and probably not true. For instance, if there is a smooth
  vector bundle $E$ that does not admit a complex structure then
  $\textbf{Dol}(E)$ is empty. It does not seem likely that such
  bundles exist but there is at most one such bundle, up to iso,
  for any given manifold. (This actually smells quite stacky,
  as we are considering the category $\textbf{Dol}(X)$
  with the forgetful functor to $\textbf{Bun}$ whose fibres
  over $E\in\textbf{Bun}$ is exactly $\textbf{Dol}(E)$.)
\end{remark}

Thus, in order to study holomorphic bundles on $X$, it suffices
to fix a real vector bundle $E$ of rank $n$ and degree $d$, and study
the holomorphic structures on it, which is the same as studying
Dolbeault operators on $E$. As we are interested in equivalence
classes of complex structures, we need to consider equivalence
classes in $\textbf{HolBun}(E)$. Such equivalence classes correspond
under (\ref{eq:equivalence_dolbeault_operators}) to equivalence
classes of Dolbeault operators on $E$. That is, we are interested
in the quotient of the set of Dolbeault operators $\text{Dol}(E)$
by the group $\Aut(E)$ acting by
\begin{equation*}
  % https://q.uiver.app/#q=WzAsNCxbMCwwLCJcXEdhbW1hKEUpIl0sWzIsMCwiXFxPbWVnYV57MCwxfShFKSJdLFswLDEsIlxcR2FtbWEoRSkiXSxbMiwxLCJcXE9tZWdhXnswLDF9KEUpIl0sWzIsMywiXFxiYXJcXHBhcnRpYWwiXSxbMCwxLCJcXHBoaVxcY2RvdFxcYmFyXFxwYXJ0aWFsIl0sWzAsMiwiXFxHYW1tYShcXHBoaSkiLDJdLFszLDEsIlxcT21lZ2FeezAsMX0oXFxwaGleey0xfSkiLDJdXQ==
  \begin{tikzcd}
    {\Gamma(X,E)} && {\Omega^{0,1}(E)} \\
    {\Gamma(X,E)} && {\Omega^{0,1}(E)}
    \arrow["\dol", from=2-1, to=2-3]
    \arrow["\phi\cdot\dol", from=1-1, to=1-3]
    \arrow["{\Gamma(\phi)}"', from=1-1, to=2-1]
    \arrow["{\Omega^{0,1}(\phi^{-1})}"', from=2-3, to=1-3]
  \end{tikzcd}
\end{equation*}

Noting that $\Dol(E)$ is an affine space modelled on
$\Omega^{0,1}(\End(E))$ \cite{mccarthy2020} the quotient
is a topological space. However, this space need not be Hausdorff.

Write $\mu(E) = \deg(E)/\rank(E)$. We thus introduce the notion of
stability:

\begin{definition}
  A holomorphic bundle $E\to X$ is \emph{stable} if,
  for all proper non-zero holomorphic subbundles $F\subset E$,
  \begin{align*}
    \mu(F) < \mu(E).
  \end{align*}
\end{definition}

Now let $\Dol(E)^s$ denote the set of Dolbeault operators corresponding
to stable holomorphic structures on $E$. Then $\Dol(E)^s/\Aut(E)$
is Hausdorff. Hence make the following definition:

\begin{definition}
  The moduli space of stable holomorphic vector bundles of rank $n$
  and degree $d$ over a Riemannian surface $X$ is
  \begin{align*}
    \mathcal{N}^s_{n,d}(X,E) = \Dol(E)^s / \Aut(E).
  \end{align*}
\end{definition}

\subsection{Holomorphic tangent bundle}

A complex manifold $X$ of dimension $n$ may be regarded as a
real manifold of dimension $2n$. Hence we have the real tangent bundle
$TX$ on $X$ which we may complexify as $TX\otimes\mathbb{C}$.
The complex structure $J:TX\to TX$ extends to this complexification
by $J(u + iv) = J(u) + iJ(v)$. We have $J^2=-\text{Id}$ and hence
eigenvalues $i$ and $-i$ which yield the decomposition
\begin{align*}
  TX\otimes\mathbb{C} = T^{1,0}X \oplus T^{0,1}X.
\end{align*}
The bundle $T^{1,0}X$ is holomorphic and referred to as the
\emph{holomorphic tangent bundle}. As real vector bundles this
is isomorphic to $TX$ via
\begin{align*}
  TX \longinc TX \otimes C \xlongrightarrow{\pi} T^{1,0}X.
\end{align*}

\subsection{Harmonic forms}

\begin{lemma}
  For every smooth vector bundle $E$ on a complex manifold $X$, there
  is a canonical isomorphism
  \begin{equation}
    (E\otimes\bar E)^* \cong E^* \otimes \bar{E}^*.
  \end{equation}
  \begin{proof}
    \missingproof
  \end{proof}
\end{lemma}

\begin{definition}
  Let $E$ be a smooth vector bundle on a complex manifold $X$.
  A \emph{hermitian metric} $h$ is a section of the bundle
  $(E\otimes\bar E)^*$ such that, for every $p\in X$ and
  $e,f\in E_p$,
  \begin{itemize}
    \item $h_p (e,\bar f) = \overline{h_p(f,\bar e)}$, and
    \item if $e\neq 0$ then $h_p(e,\bar e)>0$.
  \end{itemize}
\end{definition}

That is, there is a positive definite hermitian form
$h_p: E_p\otimes \bar{E_p}\to\mathbb{C}$ at every point $p\in X$
that varies smoothly. Note that the expression $h_p(e,\bar e) > 0$
makes sense as $h_p(e,\bar e) = \overline{h_p(e,\bar e)}$,
i.e. $h_p(e,\bar e)\in\mathbb{R}$.

\begin{definition}
  A \emph{hermitian manifold} $(X,h)$ is a complex manifold $X$
  with a hermitian metric $h$ on the holomorphic tangent bundle.
\end{definition}

We are not going to care about any particular hermitian metric,
but rather about their existence. Fortunately, our underlying
manifold is a compact Riemann surface. This allows us to make use
of the following:

\begin{theorem}
  Every smooth vector bundle on a compact complex manifold admits
  a hermitian metric. In particular, every compact complex manifold admits
  a hermitian metric.
\end{theorem}

Recall that the exterior derivative
\begin{align*}
  d : \Omega^k(X) \to \Omega^{k+1}(X)
\end{align*}
splits into maps
\begin{align*}
  \partial : \Omega^{p,q}(X) \to \Omega^{p+1,q}(X),\hspace{1cm}
  \dol : \Omega^{p,q} \to \Omega^{p,q+1}(X).
\end{align*}
A hermitian metric on $X$ defines a pairing \question{how is this
defined and what is its type?} on $\Omega^*(X)$. With respect to this
pairing we have the adjoints
\begin{align*}
  \partial^* : \Omega^{p,q}(X) \to \Omega^{p-1,q}(X),\hspace{1cm}
  \dol^* : \Omega^{p,q} \to \Omega^{p,q-1}(X).
\end{align*}

This lets us define harmonic forms, i.e. forms that vanish
under the Laplacian:

\begin{definition}
  Let $X$ be a compact manifold with a choice of hermitian metric.
  The \emph{harmonic} $(p,q)$-forms are \question{do we need a
  K\"ahler structure for this to make sense?}
  \begin{align*}
    \mathcal H^{p,q}(X) = \ker(\dol) \cap\ker(\dol^*) \subseteq \Omega^{p,q}(X).
  \end{align*}
\end{definition}

One may think of the spaces of harmonic forms as cohomology groups.
For example, we have the following:

\begin{theorem}[Poincar\'e duality]
  Let $X$ be a K\"ahler manifold of dimension $n$. There is an
  isomorphism
  \begin{align*}
    \mathcal H^{p,q}(X) \cong \mathcal H^{n-q,n-p}(X)^*
  \end{align*}
  given by the pairing
  \begin{align*}
    (\alpha,\beta)\mapsto \int_X \alpha\wedge\beta.
  \end{align*}
\end{theorem}

Moreover, it is possible to use Dolbeault operators to define complexes
and corresponding cohomologies:
\begin{definition}
  Let $X$ be a K\"ahler manifold. Then define the \emph{$(p,q)$-Dolbeault
  cohomology group} by
  \begin{align*}
    H^{p,q}_{\dol}(X) := \frac{\ker(\dol:\Omega^{p,q}(X)\to\Omega^{p,q+1}(X))}{\im(\dol:\Omega^{p,q-1}(X)\to\Omega^{p,q}(X))}
  \end{align*}
  More generally, if $E$ is a holomorphic vector bundle then we have
  cohomology with $E$-coefficients:
  \begin{align*}
    H^{p,q}(X,E) := \frac{\ker(\dol_E:\Omega^{p,q}(E)\to\Omega^{p,q+1}(E))}{\im(\dol_E:\Omega^{p,q-1}(E)\to\Omega^{p,q}(E))}.
  \end{align*}
\end{definition}

This definition agrees with sheaf cohomology:

\begin{theorem}[Dolbeault]
  For $X$ K\"ahler and $E$ holomorphic,
  \begin{align*}
    H^{p,q}(X,E) \cong H^q(X,\Omega^p\otimes E).
  \end{align*}
\end{theorem}

Moreover, this allows us to make precise the sense in which harmonic
forms and cohomology are related:

\begin{theorem}[Hodge]
  Let $X$ be a compact K\"ahler manifold.
  For any $[\alpha]\in H^{p,q}_{\dol}(X)$, there exists a unique
  harmonic representative $\alpha'\in[\alpha]\cap\mathcal H^{p,q}(X)$.
\end{theorem}


\subsection{Classification}

We aim to classify vector bundles on a Riemannian surface $\Sigma_g$
of genus $g$. In light of \ref{thm:dim_rank_vector_bundles},
a complex vector bundle $E\to\Sigma_g$ is classified by
$\rank E$ and $\deg E$ where \question{what's $c_1$ here?}
\begin{align*}
  \deg E = c_1(E)[\Sigma_g].
\end{align*}

If $\rank E = 1$, $\deg E = 1$ then the complex vector bundles
$E\to\Sigma_g$ corresponds to the complex torus of complex dimension
$g$, i.e. $\Jac_0(\Sigma_g)=\mathbb{C}^g / \mathbb{Z}^{2g}$.

If $\rank E = 1$ then the complex vector bundles $E\to\Sigma_g$
correspond to the sheaf cohomology \question{what's sheaf cohomology?} group
\begin{align*}
  H^1(\Sigma_g,\mathcal O^*) \cong \mathbb{Z}\times\Jac_0(\Sigma_g).
\end{align*}

If $\rank E>1$ then things get hairy. For some reason \question{What is reallly happening here?}
things stop being discrete (i.e. we don't get a distinct classification
for each combination of rank and dimension). Instead we consider
moduli spaces.

\subsection{Bundles vs sections}

\begin{lemma}
  Let $X$ be a real manifold and $E$ and $F$ vector bundles on $X$.
  Then bundle maps $E\to F$ are in one-to-one correspondence with
  linear maps on global sections. That is, the map
  \begin{align*}
    \Hom(E,F) &\to \Hom(\Gamma(-,E),\Gamma(-,F))\\
    f &\mapsto \Gamma(-,f)
  \end{align*}
  is a bijection.
  \begin{proof}
    \cite[Theorem 2.1]{conrad}
  \end{proof}
\end{lemma}

\begin{lemma}
  Let $X$ be a complex manifold and $E$ and $F$ be complex vector
  bundles on $X$. Then there is a $\mathbb{C}$-linear isomorphism
  \begin{align*}
    \phi : \Gamma(X,E)\otimes\Gamma(X,F)\to \Gamma(X,E\otimes F).
  \end{align*}
  \begin{proof}
    We may define $\phi$ by
    \begin{align*}
      \phi(s\otimes t)(x) = s(x)\otimes t(x).
    \end{align*}
    \missingproof
  \end{proof}
\end{lemma}

\section{Geometric invariant theory}

Consider $X=\Spec R$ and a group $G$ acting on $X$. We want to construct
an affine scheme $X \sslash G$. The usual quotient does not work:
Consider the action of $\mathbb{G}_m$ on $\affine{n}{}$.

\todo{story}

\subsection{Group schemes}

\begin{definition}
  A \emph{group scheme} $G/S$ is a group object in $\Sch_S$.
\end{definition}

In particular, a group scheme comes equipped with three maps:
\begin{align*}
  e:S\to G,\hspace{1cm} i:G\to G,\hspace{1cm} m:G\times G\to G.
\end{align*}

\begin{example}\label{ex:affine_algebraic_groups}
  In the affine case $G = \Spec R$ we may specify the multiplication
  and inverse maps in terms of $k$-algebra homomorphisms
  $m^* : R \to R\otimes R$ and $i^* : R\to R$.
  There are several affine group schemes that we are already familiar
  with:
  \begin{enumerate}
    \item $\mathbb{G}_a = \Spec k[t]$ with comulitiplication
      $t\mapsto t\otimes 1 + 1\otimes t$,
    \item $\mathbb{G}_m = \Spec k[t^\pm] = \affine{}{1} \setminus \left\lbrace{0}\right\rbrace$ with comultiplication
      $t\mapsto t\otimes t$,
    \item $GL_n = \Spec k[x_{ij} : 1\leq i,j\leq n][1/\det((x_ij))]$
      (see e.g. \cite[\href{https://stacks.math.columbia.edu/tag/022W}{Tag 022W}]{stacks-project} for details) with
      comultiplication
      \begin{align*}
        x_{ij} \mapsto \sum_{k=1}^n x_{ik}\otimes x_{kj}.
      \end{align*}
  \end{enumerate}
\end{example}

\todo{maybe we should say a bit more about $GL_n$...}

Note that the functor of points $G=\Hom(-,G):\op{\Sch}_S\to\Set$ induces a
group structure on $G(T)$ for all schemes $T/S$. The induced multiplication
sends $x,y\in G(T)$ to
\begin{align}\label{eq:induced_multiplication_of_points}
  T \xrightarrow{(x,y)}
  G\times G\xrightarrow{m}
  G.
\end{align}
Similarly, we have the identity and inverse of $x\in G(T)$,
\begin{align*}
  T \rightarrow S \xrightarrow{e} G,\hspace{1cm}
  T \xrightarrow{x} G \xrightarrow{i} G,
\end{align*}
respectively.

\begin{example}
  The induced grous of the affine group schemes behave as expected:
  For a $k$-algebra $R$, $\mathbb{G}_a(R) = (R,+)$,
  $\mathbb{G}_m(R) = (R^\times,\times)$, and
  $GL_n(R)$ is the usual group of invertible $n\times n$ matrices
  with coefficients in $R$.

  For example, consider $x\in\mathbb{G}_a(R)$ where $R$ is
  a $k$-algebra. Such an $x$ is given by algebra homomorphsims
  $x^\sharp:k[t] \to R$. That is, we may identify $\mathbb{G}_a(R)$
  with $R$ using the map $x \mapsto x^\sharp(t)$. Now, for
  $x,y\in\mathbb{G}_a(R)$ and $f\in k[t]$, we have the induced group
  multiplication given by
  \begin{align*}
    (xy)^\sharp(f)
    = x^\sharp(f) \cdot 1 + 1 \cdot y^\sharp(f)
    = x^\sharp(f) + y^\sharp(f).
  \end{align*}
  Thus the induced group structure on $\mathbb{G}_a(R)$ is just $(R,+)$.
\end{example}

The converse is also true: If the induced multiplication
(\ref{eq:induced_multiplication_of_points}) defines a group structure
on $G(T)$ for every $T/S$ then $G$ is a group scheme. This may be
shown by constructing a group structure on $G(-)$ in
$[\op{\Sch}_S,\Set]$ and then lifitng it to $G$ in $\Sch_S$ using
Yoneda's lemma.

Similarly, base change preserves group structures.
In particular, if we have a map $T\to S$ then $G_T = G\times T$
so $(G\times G)_T = G_T \times_T G_T$. Thus we obtain a group
scheme $G_T/T$.

\begin{definition}[\cite{lei2020}]
  An \emph{algebraic group} is a smooth separated
  group scheme over a field $k$.
\end{definition}

\begin{example}
  $\mathbb{G}_a$, $\mathbb{G}_m$, and $GL_n$ are all examples of algebraic
  groups. \missingproof
\end{example}

\subsection{Actions}

\begin{definition}
  A \emph{action} of $G/S$ on $X/S$ is a morphism
  $\rho : G\times X\to X$ satisfying the usual laws with respect
  to $\mu:G\times G\to G$.

\end{definition}

Note that an action $\rho : G\times X\to X$ induces an action of
$G(T)$ on $X(T)$ by
\begin{equation*}
  T \xlongrightarrow{(g,x)} G\times X \xlongrightarrow{\rho} X.
\end{equation*}
Similarly, $G_T$ acts on $X_T$ by
\begin{align*}
  G_T \times_T X_T = (G \times X)_T \xlongrightarrow{\rho_T} X_T.
\end{align*}

Thus, given a $T$-scheme $T'$, we obtain an action of
$G_T(T')$ on $X_T(T')$. We may then note that we may pull back
any $x\in X(T)$ along $T'\to T$ to obtain a $T'$-point of $X$.
This allows us to define stabilisers of group actions:

\begin{definition}
  Let $X/S$ and $T/S$ be schemes and $x\in X(T)$. The
  \emph{stabilizer of $x$} is the subgroup scheme of $G_T$
  that represents the functor $\op{\Sch}_T \to \Set$ given by \todo{op?}\question{how is this the pullback of $X\to X\times X \leftarrow G\times X$?}
  \begin{align*}
    T' \mapsto \left\lbrace{ g \in G_T(T') : g\cdot x = x}\right\rbrace.
  \end{align*}
\end{definition}

Here the action of $G_T(T')$ on $X(T)$ arises as follows:
Firstly, by previous considerations, $G_T(T')$ acts on $X_T(T')$.
Now we may pull back $x\in X(T)$ along $T'\to T$ to obtain a
$T'$-point of $X$. Finally, by universality of the pullback
$X_T$ the maps $x:T'\to X$ and $T'\to T$ induce a unique map
$T'\to X_T$.

\begin{definition}
  Let $G/S$ be a group scheme that acts on $X,Y/S$, respectively.
  A morphism $f:X\to Y$ is \emph{$G$-equivariant} if it commutes
  with the actions of $G$ on $X$ and $Y$ respectively, i.e.
  the following commutes:
  \begin{equation*}
    % https://q.uiver.app/#q=WzAsNCxbMCwwLCJHXFx0aW1lcyBYIl0sWzIsMSwiWSJdLFswLDEsIlgiXSxbMiwwLCJHXFx0aW1lcyBZIl0sWzIsMSwiZiIsMV0sWzAsMl0sWzAsMywiXFx0ZXh0e2lkfVxcdGltZXMgZiIsMV0sWzMsMV1d
    \begin{tikzcd}
      {G\times X} && {G\times Y} \\
      X && Y
      \arrow["{\text{id}\times f}"{description}, from=1-1, to=1-3]
      \arrow[from=1-1, to=2-1]
      \arrow[from=1-3, to=2-3]
      \arrow["f"{description}, from=2-1, to=2-3]
    \end{tikzcd}
  \end{equation*}
  A morphism $f:X\to Y$ is \emph{$G$-invariant} if the following
  commutes:
  \begin{equation*}
    % https://q.uiver.app/#q=WzAsMyxbMCwwLCJHXFx0aW1lcyBYIl0sWzIsMCwiWCJdLFs0LDAsIlkiXSxbMCwxLCJcXHJobyIsMSx7ImN1cnZlIjotMn1dLFswLDEsIlxccGkiLDEseyJjdXJ2ZSI6Mn1dLFsxLDIsImYiLDFdXQ==
    \begin{tikzcd}
      {G\times X} && X && Y
      \arrow["\rho"{description}, curve={height=-12pt}, from=1-1, to=1-3]
      \arrow["\pi"{description}, curve={height=12pt}, from=1-1, to=1-3]
      \arrow["f"{description}, from=1-3, to=1-5]
    \end{tikzcd}
  \end{equation*}
\end{definition}

\subsection{Quotients}

\begin{definition}
  A \emph{categorical quotient} of a group $G$ acting on $X$
  is
\end{definition}

\subsection{Linear representation}

Fix a $k$-vector space $V$ and an algebraic group $G$ over $k$,
i.e. a group scheme $G/\Spec k$. Then the algebraic group
$GL(V)$ then the corresponding functor of points sends a $k$-algebra
$R$ to $GL(V)(R) = \Aut_R(V\otimes_k R)$.

\begin{definition}
  A \emph{linear representation} of $G$ on $V$ is a homomorphism
  of group-valued functors $G\to GL(V)$.
\end{definition}

If $V$ is finite-dimensional then $GL(V)$ is
affine~\cite[Example 3.3 (4)]{hoskins} and a linear
representation is equivalent to a homomorphism of algebraic groups
$G\to GL(V)$.

If $G$ is affine, then the identity in $G(\mathcal O(G))$
corresponds to a $\mathcal O(G)$-linear automorphism
of $V\otimes_k\mathcal O(G)$. Note that such an endomorphism
is uniquely determined by its restriction to $V$, i.e. by
a $\mathcal O(G)$-linear morphism
\begin{align*}
  \rho^* : V\to V\otimes_k \mathcal O(G)
\end{align*}
which is referred to as a \emph{co-module structure on $V$}. In
fact such co-module structures are equivalent to linear
representations of affine algebraic
groups.~\cite[Section 3.1]{hoskins}

\begin{definition}
  Let $\rho : G\to GL(V)$ be a linear representation. A
  subspace $V'\subseteq V$ is \emph{$G$-invariant} if
  $\rho^*(V') \subseteq V'\otimes_k\mathcal O(G)$
  and a vector $v\in V$ is \emph{$G$-invariant} if
  $\rho^*(v) = v\otimes 1$.

  Denote by $V^G\subseteq V$ the subspace of $G$-invariant vectors.
\end{definition}

\subsection{Reductive groups}

\begin{definition}
  An affine algebraic group is \emph{unipotent} if every non-trivial
  linear representation $\rho:G\to GL(V)$ has a non-zero $G$-invariant
  vector.
\end{definition}

\begin{definition}
  An affine algebraic group is is \emph{reductive} if it is smooth
  and there are no non-trivial smooth normal unipotent algebraic
  subgroups.
\end{definition}

That is, an affine algebraic group $G$ is reductive if, for every
non-trivial smooth normal algebraic subgroup $H\triangleleft G$
and every linear representation $\rho : H\to GL(V)$,
$\rho^*(v) = v\otimes 1$ if, and only if, $v=0$.

\subsection{Projective quotients}

\begin{definition}
  Let $G$ be an affine algebraic group acting on a scheme $X$
  via $\rho : G\times X\to X$. A \emph{linearisation} of the
  action of $G$ on $X$ on a line bundle $\mathcal L$ is
  an isormorphism
  \begin{align*}
    \Phi : \rho^* \mathcal L  \to \pi^*_X\mathcal L
  \end{align*}
  satisfying the cocycle condition
  \question{maybe it would be good to talk about this condition a
    bit more? \cite{hoskins} does it using geometric bundles but
    i think sheaves are nicer; the question is how to present
  things in terms of sheaves in the nicest possible way?}
  \begin{equation*}
    % https://q.uiver.app/#q=WzAsNixbMiwwLCIoXFx0ZXh0e2lkfVxcdGltZXNcXHJobyleKlxccGleKl9YXFxtYXRoY2FsIEwiXSxbMCwwLCIoXFx0ZXh0e2lkfVxcdGltZXNcXHJobyleKlxccmhvXipcXG1hdGhjYWwgTCJdLFs0LDEsIlxccGleKl97MjN9XFxwaV9YXipcXG1hdGhjYWwgTCJdLFsyLDEsIlxccGlfezIzfV4qXFxyaG9eKlxcbWF0aGNhbCBMIl0sWzAsMiwiKFxcbXVcXHRpbWVzXFx0ZXh0e2lkfSleKlxccmhvXipcXG1hdGhjYWwgTCJdLFs0LDIsIihcXG11XFx0aW1lc1xcdGV4dHtpZH0pXipcXHBpX1heKlxcbWF0aGNhbCBMIl0sWzEsMCwiKFxcdGV4dHtpZH1cXHRpbWVzXFxyaG8pXipcXFBoaSJdLFszLDIsIlxccGleKl97MjN9XFxQaGkiXSxbNCwxLCIiLDAseyJsZXZlbCI6Miwic3R5bGUiOnsiaGVhZCI6eyJuYW1lIjoibm9uZSJ9fX1dLFszLDAsIiIsMix7ImxldmVsIjoyLCJzdHlsZSI6eyJoZWFkIjp7Im5hbWUiOiJub25lIn19fV0sWzQsNSwiKFxcbXVcXHRpbWVzXFx0ZXh0e2lkfSleKlxcUGhpIiwyXSxbNSwyLCIiLDEseyJsZXZlbCI6Miwic3R5bGUiOnsiaGVhZCI6eyJuYW1lIjoibm9uZSJ9fX1dXQ==
    \begin{tikzcd}
      {(\text{id}\times\rho)^*\rho^*\mathcal L} && {(\text{id}\times\rho)^*\pi^*_X\mathcal L} \\
                                                && {\pi_{23}^*\rho^*\mathcal L} && {\pi^*_{23}\pi_X^*\mathcal L} \\
      {(\mu\times\text{id})^*\rho^*\mathcal L} &&&& {(\mu\times\text{id})^*\pi_X^*\mathcal L}
      \arrow["{(\text{id}\times\rho)^*\Phi}", from=1-1, to=1-3]
      \arrow[Rightarrow, no head, from=2-3, to=1-3]
      \arrow["{\pi^*_{23}\Phi}", from=2-3, to=2-5]
      \arrow[Rightarrow, no head, from=3-1, to=1-1]
      \arrow["{(\mu\times\text{id})^*\Phi}"', from=3-1, to=3-5]
      \arrow[Rightarrow, no head, from=3-5, to=2-5]
    \end{tikzcd}
  \end{equation*}
\end{definition}

If we have an affine algebraic group $G$ and an action
$\rho : G\times X\to X$ then the map
\question{what's sheaf cohomology}
\begin{align*}
  H^0(X,\mathcal L)
  \xlongrightarrow{\rho^*}
  H^0(G\times X,\rho^*\mathcal L)
  \cong
  H^0(G\times X,\pi_X^*\mathcal L)
  \cong
  H^0(G,\mathcal O_G)
  \otimes
  H^0(X,\mathcal L)
\end{align*}
defines a co-module structure
\begin{align*}
  H^0(X,\mathcal L)\to H^0(X,\mathcal L)\otimes \mathcal O(G).
\end{align*}
That is, we have a natural linear representation
$G\to GL(H^0(X,\mathcal L))$ induced by the linearisation.
\cite[Lemma 5.19]{hoskins}

Now there is an evaluation map
\begin{align}\label{eq:evaluation_map}
  H^0(X,\mathcal L) \otimes_k \mathcal O_X \to \mathcal L
\end{align}
which, for each open $U\subseteq X$, is given by given by
$f\otimes_k \lambda \mapsto \lambda\restrict{f}{U}$.
This is $G$-equivariant \cite[Remark 5.20]{hoskins},
which I understand to mean that the following commutes:
\question{what are the actions here, exactly?}
\begin{equation*}
  % https://q.uiver.app/#q=WzAsNCxbMCwwLCJHXFx0aW1lc1xcbWF0aGJmIFYoSF4wKFgsXFxtYXRoY2FsIEwpXFxvdGltZXNfa1xcbWF0aGNhbCBPX1gpIl0sWzIsMCwiR1xcdGltZXNcXG1hdGhiZiBWKFxcbWF0aGNhbCBMKSJdLFswLDEsIlxcbWF0aGJmIFYoSF4wKFgsXFxtYXRoY2FsIEwpXFxvdGltZXNfa1xcbWF0aGNhbCBPX1gpIl0sWzIsMSwiXFxtYXRoYmYgVihcXG1hdGhjYWwgTCkiXSxbMCwxXSxbMiwzXSxbMCwyXSxbMSwzXV0=
  \begin{tikzcd}
    {G\times\mathbf V(H^0(X,\mathcal L)\otimes_k\mathcal O_X)} && {G\times\mathbf V(\mathcal L)} \\
    {\mathbf V(H^0(X,\mathcal L)\otimes_k\mathcal O_X)} && {\mathbf V(\mathcal L)}
    \arrow[from=1-1, to=1-3]
    \arrow[from=1-1, to=2-1]
    \arrow[from=1-3, to=2-3]
    \arrow[from=2-1, to=2-3]
  \end{tikzcd}
\end{equation*}
Here $\mathbf V = \Spec(\Sym(-))$ is the contravariant functor
$\QCoh(X)^{\text{op}}\to\Sch/X$ which yields an equivalence
between locally free $\mathcal O_X$-modules of rank $n$ and
geometric vector bundles of rank $n$ over $X$.
\cite[Proposition 11.7]{gortz2010}

\begin{definition}[{\cite[Definition 5.1]{hoskins}}]
  Let $G$ be a reductive \cite[{Definition 4.8}]{hoskins}
  algebraic group that acts on a projective
  scheme $X$. A \emph{linear $G$-equivariant projective
  embedding of $X$} consists of a group homomorphism
  $G \to GL_{n+1}$ and a $G$-equivariant embedding
  $X\inc\projective{n}{S}$, i.e. we have the following
  diagram:
  \begin{equation*}
    % https://q.uiver.app/#q=WzAsNSxbMCwxLCJYIl0sWzQsMSwiXFxtYXRoYmYgUF5uIl0sWzAsMCwiR1xcdGltZXMgWCJdLFsyLDAsIkdcXHRpbWVzXFxtYXRoYmYgUF5uIl0sWzQsMCwiR0xfblxcdGltZXNcXG1hdGhiZiBQXm4iXSxbMCwxLCIiLDEseyJzdHlsZSI6eyJ0YWlsIjp7Im5hbWUiOiJob29rIiwic2lkZSI6InRvcCJ9fX1dLFsyLDBdLFsyLDMsIiIsMSx7InN0eWxlIjp7InRhaWwiOnsibmFtZSI6Imhvb2siLCJzaWRlIjoidG9wIn19fV0sWzMsNF0sWzQsMV1d
    \begin{tikzcd}
      {G\times X} && {G\times\mathbf P^n_S} && {GL_{n+1}\times\mathbf P^n_S} \\
      X &&&& {\mathbf P^n}
      \arrow[hook, from=1-1, to=1-3]
      \arrow[from=1-1, to=2-1]
      \arrow[from=1-3, to=1-5]
      \arrow[from=1-5, to=2-5]
      \arrow[hook, from=2-1, to=2-5]
    \end{tikzcd}
  \end{equation*}
\end{definition}

\todo{make precise the action of $GL_n$ on $\mathbf P^n$}



\section{Moduli problem}

While classifying geometric objects of a certain types, one often
finds that the space of such objects has a rich geometric structure.
For example, the space of all lines in $k^{n+1}$ is
$\mathbb{P}^n_k$. The space of all objects of a certain type is
referred to as a moduli space. \cite{bejleri2020}

More precisely, a moduli problem on $S$-schemes is a presheaf
\begin{align*}
  \mathcal M : {(\Sch / S)}^{\text{op}} \to \Set.
\end{align*}
There are various objects that one may consider to be a moduli
space for such a moduli problem.

\subsection{Fine moduli spaces}

The best case scenario arises when a moduli problem $\mathcal M$
is representable. In this case, the moduli problem $\mathcal M$
may itself be thought of as a scheme.

\begin{definition}
  Consider a moduli problem $\mathcal M$ on $\Sch/S$.
  A \emph{fine moduli space} of $\mathcal M$ is a scheme $M\in\Sch/S$
  together with a natural isomorphism
  \begin{align*}
    \eta : \mathcal M \cong \Hom\left({-,M}\right).
  \end{align*}
\end{definition}
It is standard to write $M(T) := \Hom(T,M)$ and hence identify
$M$ with its functor of points. We may then surpress the natural
isomorphism to identify $\mathcal M$ with its fine moduli space $M$.

Before we adopt this abuse of notation, let us consider the
isomorphism $\eta$ a little further. Note we may think of
$F\in \mathcal M(T)$ as the morphism $f : T \to M$.
Moreover, we obtain a pullback map $f^* : \Hom(M,M) \to \Hom(T,M)$
which induces a pullback map $f^* : \mathcal M(M)\to\mathcal M(T)$.
We then observe that $U := {\eta}^{-1}_M(\identity)$
is universal in the sense that every $F$ is obtained
by pulling back $U$ along $f$.

\subsection{Grassmannians}

While we are unlikely to actually need Grasmannians, they provide
us with a lot of intuition going forward. Moreover, they were used
by Grothendieck to prove the representability of the Quot scheme
and hence are worth knowing about. Let us recall the elementary
notion first:

\begin{definition}
  Let $V$ be a finite dimensional $k$-vector space. Then
  $G(m,V)$ denotes the set of $n$-dimensional subspaces of $V$.
\end{definition}

For now, this is just a set. However, for suitable choices of
$k$, it has the structure of a differentiable manifold as well as
that of an algebraic variety.
Now note that subspaces $W\in G(m,V)$ give rise to quotients
$V/W$ and we may recover $W=\ker(V\surj V/W)$. Thus subspaces
of $V$ may be identified with isomorphism classes of quotients.
In particular, any isomorphism $V/W \cong V/W$ will make the following
commute:
\begin{equation*}
  % https://q.uiver.app/#q=WzAsMyxbMCwwLCJWIl0sWzIsMCwiVi9XIl0sWzIsMSwiVi9XIl0sWzAsMiwiIiwyLHsic3R5bGUiOnsiaGVhZCI6eyJuYW1lIjoiZXBpIn19fV0sWzAsMSwiIiwwLHsic3R5bGUiOnsiaGVhZCI6eyJuYW1lIjoiZXBpIn19fV0sWzEsMiwiXFxjb25nIl1d
  \begin{tikzcd}
    V && {V/W} \\
      && {V/W}
      \arrow[two heads, from=1-1, to=1-3]
      \arrow[two heads, from=1-1, to=2-3]
      \arrow["\cong", from=1-3, to=2-3]
  \end{tikzcd}
\end{equation*}
Thus we may identify $G(m,V)$ with the set of isomorphism
classes of $k$-linear surjections $V\surj W$ with $\dim V-\dim W=m$.
This allows us to generalise from vector spaces to free sheaves
of modules.~\cite[\href{https://stacks.math.columbia.edu/tag/089R}{Tag 089R}]{stacks-project}. \todo{maybe we want to think and talk about
quotients of sheaves of modules a little more ...}

\begin{definition}
  The functor $G(m,n) : \Sch \to \Set$ associates to
  each scheme $T\in\Sch$ the set $G(m,n)(T)$ of isomorphism
  classes of surjections
  \begin{align*}
    q : \mathcal O^{\oplus n}_T \surj \mathcal F
  \end{align*}
  where $\mathcal F$ is a finite, locally free $\mathcal O_T$-module
  of rank $n-m$ and for each morphism $f:T\to T'$ the map
  $G(m,n)(f)$ sends the isomorphism class of $q$ to that
  of $f^*q$.
\end{definition}

We begin by noting that this does indeed generalise the previous
notion. In particular, there is a canonical bijection
\begin{align*}
  G(m,n)(\Spec k) \cong G(m,k^n)
\end{align*}
as quotients of $\mathcal O^{\oplus n}_{\Spec k}$ are just quotients
of of $k^n$. \question{what does canonical mean here? also, we would want this to be at least an isomorphism of varieties in some sense but i haven't seen anyone talk about this...}

This is a functor and, crucially, representable. Thus we
have a scheme $\mathbf G(m,n)$ with
\begin{align*}
  \mathbf G(m,n)(T) := \Hom(T,\mathbf G(m,n)) = G(m,n)(T).
\end{align*}
Base change gives us the usual $\mathbf G(m,n)_S\in\Sch_S$ and
$\mathbf G(m,n)_R\in\Sch_R$.
Moreover, we are justified in thinking of $\mathbf G(m,n)_S$ as
parametrising $m$-dimensional subspaces of $\affine{n}{S}$ in the
sense that we think of projective $n$-space as parametrising
$n$-dimensional subspaces of $(n+1)$-dimensional affine space.

\begin{proposition}
  Let $n\geq 1$ and $S\in\Sch$. Then there is a canonical isomorphism
  \begin{align*}
    \mathbf G(n,n+1)_S \cong \projective{n}{S}.
  \end{align*}
  \begin{proof}
    It suffices to consider $S=\Spec\mathbb{Z}$.
    See \cite[\href{https://stacks.math.columbia.edu/tag/089V}{Tag 089V}]{stacks-project}.
  \end{proof}
\end{proposition}

\subsection{Quot schemes}

Fix a Noetherian base scheme $S$, a scheme $X/S$ of finite type,
a coherent sheaf $\mathcal E$ on $X$, and another scheme $T/S$.

\begin{definition}
  A \emph{family of quotients of $\mathcal E$ parameterised by $T$} consists of
  \begin{enumerate}
    \item a coherent sheaf $\mathcal F$ on $X_T$
      such that the schematic support of $\mathcal F$ is proper
      over $T$ and $\mathcal F$ is flat over $T$, and
    \item a surjective $\mathcal O_{X_T}$-linear
      homorphism of sheaves $q:\mathcal E_T\to \mathcal F$.
  \end{enumerate}
  A morphism $f:(\mathcal F,q)\to(\mathcal F',q')$ of such families
  is a morphism $f:\mathcal F\to\mathcal F'$ that makes the following
  commute:
  \begin{equation*}
    % https://q.uiver.app/#q=WzAsMyxbMiwwLCJcXG1hdGhjYWwgRiJdLFsyLDIsIlxcbWF0aGNhbCBGJyJdLFswLDAsIkVfVCJdLFsyLDAsInEiXSxbMiwxLCJxJyIsMl0sWzAsMSwiZiJdXQ==
    \begin{tikzcd}
      {\mathcal E_T} && {\mathcal F} \\
      \\
                     && {\mathcal F'}
                     \arrow["q", from=1-1, to=1-3]
                     \arrow["{q'}"', from=1-1, to=3-3]
                     \arrow["f", from=1-3, to=3-3]
    \end{tikzcd}
  \end{equation*}
\end{definition}

\begin{example}
  If $S=\Spec\mathbb C$, $X=\Sigma_g$, and
  $\mathcal E=\mathcal O^{\oplus k}_X$.
  Then a family of quotients $q:\mathcal E_T\surj\mathcal F$
  is a quotient
  \begin{align*}
    (\mathcal O^{\oplus k}_X)_T \cong \mathcal O^{\oplus k}_{X\times_S T} \surj \mathcal F
  \end{align*}
  in $\Coh(X\times_S T)$.
  The kernel of such a quotient turns out to be locally-free
  and of finite-rank \cite[Lemma 4.21]{bertram1993}, i.e. a
  vector bundle on $X\times_S T$.
\end{example}

\begin{lemma}
  If $\mathcal F\in\Coh(X_T)$ is flat over $T$ and
  $q:\mathcal O_{X_T}^{\oplus k} \surj\mathcal F$ is surjective and
  $\mathcal O_{X_T}$-linear then $\Ker(q)\in\Coh(X_T)$ is
  finite locally free.
  \begin{proof}
    We want to show that there is

    Recall that flatness of $\mathcal F$ over $T$ means that,
    for all $x\in X_T$, the stalk $\mathcal F_x$ is a flat
    $\mathcal O_{T,\pi(x)}$-module where
    $\pi : X_T \to T$.
    \missingproof
  \end{proof}
\end{lemma}

\begin{lemma}
  Families $(\mathcal F,q)$ and $(\mathcal F',q')$ are isomorphic if, and only if,
  $\ker(q) = \ker(q')$.
  \begin{proof}
    \missingproof
  \end{proof}
\end{lemma}

Write $\langle\mathcal F,q\rangle$ for equivalence classes
of such families and define the corresponding quot functor
as follows:
\begin{definition}
  The quot functor $\Sch_S^{\text{op}} \to \Set$ is given by
  \begin{align*}
    \mathfrak{Quot}_{\mathcal E,X/S} (T)
    := \left\lbrace{\text{all $\langle\mathcal F,q\rangle$
    parametrised by $T$}}\right\rbrace
  \end{align*}
\end{definition}

The map on morphsims $f:T'\to T$ is as follows.
We may pull back $\text{Coh}(X_T)$ to $\text{Coh}(X_{T'})$
along $\identity\times_S f:X_{T'}\to X_T$. In particular,
note that the following commutes:

\begin{equation*}
  % https://q.uiver.app/#q=WzAsMyxbMiwwLCJcXHRleHR7Q29ofShYX1QpIl0sWzIsMSwiXFx0ZXh0e0NvaH0oWF97VCd9KSJdLFswLDAsIlxcdGV4dHtDb2h9KFgpIl0sWzAsMSwiKFxcdGV4dHtpZH1cXHRpbWVzIGYpXioiXSxbMiwwLCJcXHBpXioiXSxbMiwxLCJcXHBpXioiLDJdXQ==
  \begin{tikzcd}
    {\text{Coh}(X)} && {\text{Coh}(X_T)} \\
                    && {\text{Coh}(X_{T'})}
                    \arrow["{(\text{id}\times f)^*}", from=1-3, to=2-3]
                    \arrow["{\pi^*}", from=1-1, to=1-3]
                    \arrow["{\pi^*}"', from=1-1, to=2-3]
  \end{tikzcd}
\end{equation*}
by functoriality of $(-)^*$. Hence, given a family
$\langle\mathcal F,q\rangle$, we obtain a diagram
\begin{equation*}
  % https://q.uiver.app/#q=WzAsMyxbMCwwLCJFX3tUJ30iXSxbMSwwLCIoXFx0ZXh0e2lkfVxcdGltZXNfU2YpXiogXFxtYXRoY2FsIEVfVCJdLFszLDAsIihcXHRleHR7aWR9XFx0aW1lc19TZileKlxcbWF0aGNhbCBGIl0sWzEsMiwiKFxcdGV4dHtpZH1cXHRpbWVzX1NmKV4qcSJdLFswLDEsIiIsMCx7ImxldmVsIjoyLCJzdHlsZSI6eyJoZWFkIjp7Im5hbWUiOiJub25lIn19fV1d
  \begin{tikzcd}
    {\mathcal E_{T'}} & {(\text{id}\times_Sf)^* \mathcal E_T} && {(\text{id}\times_Sf)^*\mathcal F}
    \arrow["{(\text{id}\times_Sf)^*q}", from=1-2, to=1-4]
    \arrow[Rightarrow, no head, from=1-1, to=1-2]
  \end{tikzcd}
\end{equation*}
in $\Coh(X_{T'})$, i.e. a family of quotients parametrised by
$T'$. This respects isomorphisms of such families and
hence we have defined a map
\begin{align*}
  \mathfrak{Quot}_{\mathcal E,X/S}(T) \to \mathfrak{Quot}_{\mathcal E,X/S}(T').
\end{align*}

\begin{theorem}
  There is a scheme
  \begin{align*}
    \text{Quot}_{\mathcal E,X/S} \in \Sch_S
  \end{align*}
  that represents $\mathfrak{Quot}_{\mathcal E,X/S}$.
\end{theorem}

\subsection{Moduli space of stable bundles}

We aim to define the moduli space of (semi)stable bundles on a
suitable curve. We will do so without justification. For details
see \cite[section 8.8]{hoskins}.

Fix a connected smooth projective curve $C$ of genus $g\geq 2$,
$n\geq 1$, and $d > n(2g - 1)$. Define
$N:=d+n(1-g)$ and denote by
\begin{align*}
  Q^s \subset Q := \Quot_C^{n,d}(\mathcal O_C^N).
\end{align*}
the open subscheme of quotients $q:\mathcal O^N_X\surj\mathcal F$
such that $\mathcal F$ is (semi)stable and locally free and
$H^0(q)$ is an isomorphism. \question{why do we need this condition?}

As $Q$ is a fine moduli space, we have a universal quotient
$q:\mathcal O_{Q\times X}^N\surj \mathcal U$
and its restriction to $Q$:
\begin{align*}
  q^s : \mathcal O_{Q^s\times X}^N\surj \restrict{\mathcal U}{Q^s} =: \mathcal U^s.
\end{align*}
A group action
\begin{align*}
  \rho : GL_N \times Q \to Q
\end{align*}
is an element of $\Hom(GL_N\times Q,Q)$ and hence corresponds to a
quotient of $\mathcal O_{GL_N\times Q\times X}$ which we construct
as:
\begin{align}
  \label{eq:stable_quot_gln_action}
  \mathcal O^n_{GL_N\times Q\times X}
  \xlongrightarrow{\pi^*_{GL_N}\tau}
  \mathcal O^N_{GL_N\times Q\times X}
  \xlongrightarrow{\pi^*_{Q\times X} q}
  \pi^*_{Q\times X}\mathcal U
\end{align}
where $\pi_{GL_N}$ and $\pi_{Q\times X}$ are the obvious projections
and the sheaf isomorphism
\begin{align*}
  \tau : \mathcal O_{GL_N}^N \to \mathcal O_{GL_N}^N
\end{align*}
arises from the inverse map $i : GL_N \to GL_N$. On $k$-points this
action is given by $g\cdot q := q\circ {g}^{-1}$.

\begin{lemma}[{\cite[Lemma 8.49]{hoskins}}]
  The orbits in $R^s\subseteq Q$ under the action \ref{eq:stable_quot_gln_action} are in bijective correspondence with isomorphism
  classes of (semi)stable locally free sheaves on $X$ with invariants
  $(n,d)$.
\end{lemma}

\begin{itemize}
  \item restrict to $SL_N$ action
  \item linearise $SL_N$ action
  \item construct coarse moduli space
  \item write down conditions for when it is fine moduli space and
    non-existence result otherwise
\end{itemize}

\begin{remark}
  It is worth investigating how we have captured holomorphic bundles
  in general. In particular, how does an isomorphism class of
  quotients
  \begin{align*}
    q : \mathcal O^N_{X\times T} \to \mathcal F
  \end{align*}
  in $\Coh(X\times T)$ constitute a bundle on $X$? Firstly,
  we observed that the kernel of such a quotient is a locally
  free sheaf of rank $N$, \todo{actually this is more involved because we used $n$ and $d$} hence a geometric vector bundle on
  $X\times T$. Secondly, note if we choose $T=\Spec\mathbb{C}$ then
  $Q^s(T)$ does indeed consist of holomorphic bundles on
  $X\times\Spec\mathbb{C}=X$. Thus $\mathbb{C}$-points of
  $Q^s$ are holomorphic bundles on $X$.
\end{remark}

\section{Higgs bundles}

\subsection{Definition}

\begin{definition}
  A \emph{Higgs bundle $(E,\phi)$} on a Riemann surface
  $X$ consists of a holomorphic vector bundle $E$ on $X$
  and a holomorphic bundle map $\phi : E \to E \otimes K$ where
  $K:=\Lambda^n T^*_{0,1} X$ is the canonical line bundle on $X$.
\end{definition}

Note that the holomorphic bundle map is the same as a section
in $\Gamma(X,\End E \otimes K)$. To see this, note that
$\End E = E^* \otimes E$ so $\phi$ induces a bundle map
\begin{align*}
  \mathbb{C}\times X \to \End E \otimes K.
\end{align*}
This induces a map of $\mathbb{C}$-algebras
\begin{align*}
  \Gamma(X,\mathbb{C}\times X) \to \Gamma(X,\End(E)\otimes K)
\end{align*}
which is uniquely determined by the image of the constant function
$x\mapsto 1$.

\subsection{Higgs bundles as cotangent vectors of holomorphic bundles}

Fix a smooth bundle $E$ on a compact Riemann surface $X$. Write
$\mathcal N = \mathcal N^s(X,E)$ and consider a holomorphic structure
$\dol\in\mathcal N$. Then the cotangent space of $\mathcal N$ at
$\dol$ is
\begin{align}\label{eq:tangent_space_of_n}
  T^*_{\dol} \mathcal N = I_{\dol} / I^2_{\dol}
\end{align}
where $I_{\dol}\subseteq \mathcal O(\mathcal N)$ is the ideal
of holomorphic functions on $\mathcal N$ vanishing at $\dol$.
Note that we have a map \todo{actually we don't really... the RHS is
\emph{linear} maps}
\begin{align*}
  I_{\dol} \to (E^*(X)\otimes E(X)\otimes\Omega^{0,1}(X))^*.
\end{align*}
Via the isomorphism of bundles
\begin{align*}
  (E^* \otimes E \otimes \Omega^{0,1})^*
  \cong E^* \otimes E \otimes \Omega^{1,0}
  \cong \End E \otimes K
\end{align*}
this yields a map
\begin{align}
  I_{\dol} \to H^0(X,\End E\otimes K).
\end{align}
Note that $I_{\dol}$ has a complex vector space structure
with respect to which this map is $\mathbb{C}$-linear.
Let us verify that this gives a well-defined map on the quotient
(\ref{eq:tangent_space_of_n}).


\section{Non-abelian Hodge theory}

\begin{theorem}
  Let $X$ be a compact K\"ahler manifold. Then there is an equivalence
  of categories between irreducible flat vector bundles on $X$
  and stable Higgs bundles
\end{theorem}

\printbibliography

\end{document}
