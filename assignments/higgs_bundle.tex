\documentclass{article}
\usepackage{assignment}
\addbibresource{higgs.bib}
\begin{document}
\title{Higgs bundle}
\date{\today}
\maketitle

\section{Complex geometry}

\subsection{Holomorphic functions}

\subsection{Holomorphic 1-forms}

\subsection{Complex vector bundles}

\missingdefinition{trivial rank 1 bundle $1_M$}

\begin{theorem}[\cite{mccarthy2020}]\label{thm:dim_rank_vector_bundles}
  Let $E\to M$ be a real vector bundle on a smooth manifold. If $\rank E > \dim M$ then there is a real vector bundle $E'$ of rank $\rank E' = \dim M$
  such that
  \begin{align*}
    E \cong E' \oplus \bigoplus_{\rank E - \dim M} 1_M.
  \end{align*}
\end{theorem}

\subsection{Holomorphic connection}

\subsection{Complex manifolds}

A complex vector space is just a real vector space $V$ together with a
$J\in\Aut(V)$ such that $J^2 = -\Identity$. We similarly define
a complex structure on a real manifold to be automorphisms of the
tangent bundle:

\begin{definition}
  Let $X$ be a real manifold. A \emph{complex structure} on $X$
  is an automorphism of real vector bundles $J\in\Aut(TX)$ such
  that $J^2 = -\Identity$. The pair $(X,J)$ is referred to as an
  \emph{almost complex manifold}.
\end{definition}

\begin{itemize}
  \item A complex structure smoothly assign a complex structure to each tangent
    space. One may also think of a complex structure as a $(1,1)$-tensor field
    $J\in\Gamma(TX\otimes T^* X)$.
  \item Any real manifold that admits a complex structure must have even
    dimension. This follows immediately from the analogous observation for real
    vector spaces.
  \item Each complex $n$-manifold $X$ induces a complex structure $J$ on the
    real $2n$-manifold $X$. That is, every complex manifold is an almost
    complex manifold. The converse need not be true. \missingexample
  \item On a complex manifold $X$, the complex structure has eigenvalues
    $\pm i$. This induces a splitting
    \begin{align*}
      T_x X = T^{1,0}_x X \oplus T^{0,1}_x X
    \end{align*}
    for each $x\in X$ where $T^{1,0}_x X$ and $T^{0,1}_x X$ correspond
    to the eigenvalues $i$ and $-i$, respectively. There are obvious
    vector bundles $T^{1,0} X$ and $T^{0,1} X$. \question{Is it
    necessarily the case that $\rank(T^{1,0} X) = \rank(TX)/2$?}
  \item By choosing coordinates around $x$ and writing $(\partial / \partial
    z_j)_x = (\partial / \partial x_j - i\partial / \partial y_j)_x / 2$ we
    have bases
    \begin{align*}
      \left\lbrace{\left(\frac{\partial}{\partial z_j}\right)_x}\right\rbrace_j, \hspace{1cm}
      \left\lbrace{\left(\frac{\partial}{\partial \bar z_j}\right)_x}\right\rbrace_j
    \end{align*}
    for $T^{1,0}_x X$ and $T^{0,1}_x X$, respectively.
  \item Analogously, we obtain the splitting
    \begin{align*}
      T^*_x X = \Lambda^{1,0}_x X \oplus \Lambda^{0,1}_x X
    \end{align*}
    and hence bundles $\Lambda^{1,0}X$ and $\Lambda^{0,1}X$. This
    splitting extends to exterior powers of the cotangent bundle:
    \begin{align*}
      \Lambda^k X
      = \bigoplus_{p+q = k} \Lambda^{p,q} X
      = \bigoplus_{p+q=k}\left({\exterior{p}{\Lambda^{1,0} X}}\right)\wedge
      \left({\exterior{q}{\Lambda^{0,1} X}}\right).
    \end{align*}
    We call sections of $\Lambda^{p,q} X$ $(p,q)$-forms and write
    $\Omega^{p,q}(X)=\Gamma(\Lambda^{p,q}X)$.
\end{itemize}

\begin{example}[{\cite[Section 5.2]{neitzke2021}}]
  Let $E\to\Sigma_g$ be a complex rank $k$ vector bundle
  on a Riemannian surface with $d=\deg E$. Then $\End E$ is
  a complex manifold whose structure is induced by the
  homeomorphism $\End E \cong E\otimes E^*$.

  Then $\Omega^{0,1}(\End E)$ is the
\end{example}

\begin{itemize}
  \item Using the projections, the exterior derivative $d:\Omega^k(X)\to\Omega^{k+1}(X)$ splits into maps $d^{0,k+1},\ldots,d^{k+1,0}$ with
    \begin{align*}
      d^{p,q} : \Omega^k(X) \to \Omega^{p,q}(X).
    \end{align*}
\end{itemize}


\section{Geometric invariant theory}

Consider $X=\Spec R$ and a group $G$ acting on $X$. We want to construct
an affine scheme $X \sslash G$. The usual quotient does not work:
Consider the action of $\mathbb{G}_m$ on $\affine{n}{}$.

\todo{story}

\subsection{Group schemes and actions}

\begin{definition}
  A \emph{group scheme} $G/S$ is a group object in $\Sch_S$.
\end{definition}

In particular, a group scheme comes equipped with three maps:
\begin{align*}
  e:S\to G,\hspace{1cm} i:G\to G,\hspace{1cm} m:G\times G\to G.
\end{align*}

\begin{example}
  In the affine case $G = \Spec R$ we may specify the multiplication
  and inverse maps in terms of $k$-algebra homomorphisms
  $m^* : R \to R\otimes R$ and $i^* : R\to R$.
  There are several affine group schemes that we are already familiar
  with:
  \begin{enumerate}
    \item $\mathbb{G}_a = \Spec k[t]$ with comulitiplication
      $t\mapsto t\otimes 1 + 1\otimes t$,
    \item $\mathbb{G}_m = \Spec k[t^\pm] = \affine{}{1} \setminus \left\lbrace{0}\right\rbrace$ with comultiplication
      $t\mapsto t\otimes t$,
    \item $GL_n = \Spec k[x_{ij} : 1\leq i,j\leq n][1/\det((x_ij))]$
      (see e.g. \cite[\href{https://stacks.math.columbia.edu/tag/022W}{Tag 022W}]{stacks-project} for details) with
      comultiplication
      \begin{align*}
        x_{ij} \mapsto \sum_{k=1}^n x_{ik}\otimes x_{kj}.
      \end{align*}
  \end{enumerate}
\end{example}

Note that the functor of points $G=\Hom(-,G):\op{\Sch}_S\to\Set$ induces a
group structure on $G(T)$ for all schemes $T/S$. The induced multiplication
sends $x,y\in G(T)$ to
\begin{align}\label{eq:induced_multiplication_of_points}
  T \xrightarrow{(x,y)}
  G\times G\xrightarrow{m}
  G.
\end{align}
Similarly, we have the identity and inverse of $x\in G(T)$,
\begin{align*}
  T \rightarrow S \xrightarrow{e} G,\hspace{1cm}
  T \xrightarrow{x} G \xrightarrow{i} G,
\end{align*}
respectively.

\begin{example}
  The induced grous of the affine group schemes behave as expected:
  For a $k$-algebra $R$, $\mathbb{G}_a(R) = (R,+)$,
  $\mathbb{G}_m(R) = (R^\times,\times)$, and
  $GL_n(R)$ is the usual group of invertible $n\times n$ matrices
  with coefficients in $R$.

  For example, consider $x\in\mathbb{G}_a(R)$ where $R$ is
  a $k$-algebra. Such an $x$ is given by algebra homomorphsims
  $x^\sharp:k[t] \to R$. That is, we may identify $\mathbb{G}_a(R)$
  with $R$ using the map $x \mapsto x^\sharp(t)$. Now, for
  $x,y\in\mathbb{G}_a(R)$ and $f\in k[t]$, we have the induced group
  multiplication given by
  \begin{align*}
    (xy)^\sharp(f)
    = x^\sharp(f) \cdot 1 + 1 \cdot y^\sharp(f)
    = x^\sharp(f) + y^\sharp(f).
  \end{align*}
  Thus the induced group structure on $\mathbb{G}_a(R)$ is just $(R,+)$.
\end{example}

The converse is also true: If the induced multiplication
(\ref{eq:induced_multiplication_of_points}) defines a group structure
on $G(T)$ for every $T/S$ then $G$ is a group scheme. This may be
shown by constructing a group structure on $G(-)$ in
$[\op{\Sch}_S,\Set]$ and then lifitng it to $G$ in $\Sch_S$ using
Yoneda's lemma.

Similarly, base change preserves group structures.
In particular, if we have a map $T\to S$ then $G_T = G\times T$
so $(G\times G)_T = G_T \times_T G_T$. Thus we obtain a group
scheme $G_T/T$.

\begin{definition}[\cite{lei2020}]
  An \emph{algebraic group} is a smooth separated
  group scheme over a field $k$.
\end{definition}

\begin{example}
  $\mathbb{G}_a$, $\mathbb{G}_m$, and $GL_n$ are all examples of algebraic
  groups. \missingproof
\end{example}

\begin{definition}
  A \emph{action} of $G/S$ on $X/S$ is a morphism
  $\rho : G\times X\to X$ satisfying the usual laws with respect
  to $\mu:G\times G\to G$.

\end{definition}

Note that an action $\rho : G\times X\to X$ induces an action of
$G(T)$ on $X(T)$ by
\begin{equation*}
  T \xlongrightarrow{(g,x)} G\times X \xlongrightarrow{\rho} X.
\end{equation*}
Similarly, $G_T$ acts on $X_T$ by
\begin{align*}
  G_T \times_T X_T = (G \times X)_T \xlongrightarrow{\rho_T} X_T.
\end{align*}

Thus, given a $T$-scheme $T'$, we obtain an action of
$G_T(T')$ on $X_T(T')$. We may then note that we may pull back
any $x\in X(T)$ along $T'\to T$ to obtain a $T'$-point of $X$.
This allows us to define stabilisers of group actions:

\begin{definition}
  Let $X/S$ and $T/S$ be schemes and $x\in X(T)$. The
  \emph{stabilizer of $x$} is the subgroup scheme of $G_T$
  that represents the functor $\op{\Sch}_T \to \Set$ given by \todo{op?}\question{how is this the pullback of $X\to X\times X \leftarrow G\times X$?}
  \begin{align*}
    T' \mapsto \left\lbrace{ g \in G_T(T') : g\cdot x = x}\right\rbrace.
  \end{align*}
\end{definition}

Here the action of $G_T(T')$ on $X(T)$ arises as follows:
Firstly, by previous considerations, $G_T(T')$ acts on $X_T(T')$.
Now we may pull back $x\in X(T)$ along $T'\to T$ to obtain a
$T'$-point of $X$. Finally, by universality of the pullback
$X_T$ the maps $x:T'\to X$ and $T'\to T$ induce a unique map
$T'\to X_T$.

\subsection{Quotients}

\begin{definition}
  A \emph{categorical quotient} of a group $G$ acting on $X$
  is
\end{definition}

\section{Symplectic quotients}

We aim to classify vector bundles on a Riemannian surface $\Sigma_g$
of genus $g$. In light of \ref{thm:dim_rank_vector_bundles},
a complex vector bundle $E\to\Sigma_g$ is classified by
$\rank E$ and $\deg E$ where \question{what's $c_1$ here?}
\begin{align*}
  \deg E = c_1(E)[\Sigma_g].
\end{align*}

If $\rank E = 1$, $\deg E = 1$ then the complex vector bundles
$E\to\Sigma_g$ corresponds to the complex torus of complex dimension
$g$, i.e. $\Jac_0(\Sigma_g)=\mathbb{C}^g / \mathbb{Z}^{2g}$.

If $\rank E = 1$ then the complex vector bundles $E\to\Sigma_g$
correspond to the sheaf cohomology \question{what's sheaf cohomology?} group
\begin{align*}
  H^1(\Sigma_g,\mathcal O^*) \cong \mathbb{Z}\times\Jac_0(\Sigma_g).
\end{align*}

If $\rank E>1$ then things get hairy. For some reason \question{What is reallly happening here?}
things stop being discrete (i.e. we don't get a distinct classification
for each combination of rank and dimension). Instead we consider
moduli spaces.

\subsection{Dolbeault operator}

\begin{lemma}
  Let $E\to X$ be a smooth complex vector bundle $E\to X$ on a complex
  manifold. Then
  \begin{align*}
    \Omega^{0,1}(E) = \Omega^{0,1}(X)\otimes\Gamma(E).
  \end{align*}
  \begin{proof}
    Let $\omega\in\Omega^{0,1}(X)$ and $s\in\Gamma(E)$. Define
    $t(s,\omega)\in\Omega^{0,1}=\Gamma({\left({T^{0,1}}\right)^* E})$
    by
    \begin{equation*}
      % https://q.uiver.app/#q=WzAsNCxbMCwwLCJFIl0sWzIsMCwiWCJdLFs0LDAsIihUXnswLDF9KV4qIFgiXSxbNiwwLCIoVF57MCwxfSleKkUiXSxbMCwxLCJcXHBpIl0sWzEsMiwiXFxvbWVnYSJdLFsyLDMsIihUXnswLDF9KV4qIHMiXV0=
      \begin{tikzcd}
        E && X && {(T^{0,1})^* X} && {(T^{0,1})^*E}
        \arrow["\pi", from=1-1, to=1-3]
        \arrow["\omega", from=1-3, to=1-5]
        \arrow["{(T^{0,1})^* s}", from=1-5, to=1-7]
      \end{tikzcd}
    \end{equation*}
    This defines a bilinear map
    \begin{align*}
      t : \Omega^{0,1} X \times \Gamma(E) \to \Omega^{0,1}(E).
    \end{align*}
    yadda yadda
    \missingproof
  \end{proof}
\end{lemma}

\begin{definition}
  A \emph{Dolbeault operator} on a smooth complex vector bundle
  $E\to X$ on a complex manifold is a $\mathbb{C}$-linear operator
  \begin{align*}
    \bar\partial_E : \Gamma(E) \to \Omega^{0,1}(E)
  \end{align*}
  such that
  \begin{enumerate}
    \item $\bar\partial_E^2 = 0$,
    \item $\bar\partial_E(fs) = \bar\partial f \otimes s + f\bar\partial_E s$
      for $s\in\Gamma(E)$ and $f\in\Omega^0(X)$.
  \end{enumerate}
\end{definition}

Two Dolbeault operators $\bar\partial_1$ and $\bar\partial_2$ on
$E\to X$ are equivalent if there exists $\phi\in\Aut(E)$ such that
the following commutes:
\begin{equation*}
  % https://q.uiver.app/#q=WzAsNCxbMCwwLCJcXEdhbW1hKEUpIl0sWzIsMCwiXFxPbWVnYV57MCwxfShFKSJdLFswLDEsIlxcR2FtbWEoRSkiXSxbMiwxLCJcXE9tZWdhXnswLDF9KEUpIl0sWzIsMywiXFxiYXJcXHBhcnRpYWxfMiJdLFswLDEsIlxcYmFyXFxwYXJ0aWFsXzEiXSxbMCwyLCJcXEdhbW1hKFxccGhpKSIsMl0sWzEsMywiXFxPbWVnYV57MCwxfShcXHBoaSkiXV0=
  \begin{tikzcd}
    {\Gamma(E)} && {\Omega^{0,1}(E)} \\
    {\Gamma(E)} && {\Omega^{0,1}(E)}
    \arrow["{\bar\partial_2}", from=2-1, to=2-3]
    \arrow["{\bar\partial_1}", from=1-1, to=1-3]
    \arrow["{\Gamma(\phi)}"', from=1-1, to=2-1]
    \arrow["{\Omega^{0,1}(\phi)}", from=1-3, to=2-3]
  \end{tikzcd}
\end{equation*}
It turns out that there is a one-to-one correspondence between
holomorphic structures on a smooth complex vector bundle $E\to\Sigma_g$
and Dolbeault operators on $E$ up to equivalence.

More precisely, fix a smooth complex vector bundle $E\to\Sigma_G$
and let $\Dol(E)$ denote the set of all Dolbeault operators on $E$.
Now $\Aut(E)$ acts on $\Dol(E)$ by
\begin{equation*}
  % https://q.uiver.app/#q=WzAsNCxbMCwwLCJcXEdhbW1hKEUpIl0sWzIsMCwiXFxPbWVnYV57MCwxfShFKSJdLFswLDEsIlxcR2FtbWEoRSkiXSxbMiwxLCJcXE9tZWdhXnswLDF9KEUpIl0sWzIsMywiXFxiYXJcXHBhcnRpYWwiXSxbMCwxLCJcXHBoaVxcY2RvdFxcYmFyXFxwYXJ0aWFsIl0sWzAsMiwiXFxHYW1tYShcXHBoaSkiLDJdLFszLDEsIlxcT21lZ2FeezAsMX0oXFxwaGleey0xfSkiLDJdXQ==
  \begin{tikzcd}
    {\Gamma(E)} && {\Omega^{0,1}(E)} \\
    {\Gamma(E)} && {\Omega^{0,1}(E)}
    \arrow["\bar\partial", from=2-1, to=2-3]
    \arrow["\phi\cdot\bar\partial", from=1-1, to=1-3]
    \arrow["{\Gamma(\phi)}"', from=1-1, to=2-1]
    \arrow["{\Omega^{0,1}(\phi^{-1})}"', from=2-3, to=1-3]
  \end{tikzcd}
\end{equation*}

Now the holomorphic structures on $E$ up to isomorphism correspond to
elements of \question{what's the map?}
\begin{align*}
  \Dol(E)/\Aut(E).
\end{align*}
Noting that $\Dol(E)$ is an affine space modelled on $\Omega^{0,1}(\End(E))$
\cite{mccarthy2020} \question{what does this mean?} the quotient
is a topological space. However, this space need not be Hausdorff.

Write $\mu(E) = \deg(E)/\rank(E)$. We thus introduce the notion of
stability:

\begin{definition}
  A holomorphic bundle $E\to\Sigma_g$ is \emph{stable} if,
  for all proper non-zero holomorphic subbundles $F\subset E$,
  \begin{align*}
    \mu(F) < \mu(E).
  \end{align*}
\end{definition}

Now let $\Dol(E)^s$ denote the set of Dolbeault operators corresponding
to stable holomorphic structures on $E$. Then $\Dol(E)^s/\Aut(E)$
is Hausdorff. Hence make the following definition:

\begin{definition}
  The moduli space of stable holomorphic vector bundles of rank $n$
  and degree $d$ over a Riemannian surface $\Sigma_g$ is
  \begin{align*}
    \mathcal N^g_{n,d} = \Dol(E)^s / \Aut(E).
  \end{align*}
\end{definition}

\section{Moduli problem}

While classifying geometric objects of a certain types, one often
finds that the space of such objects has a rich geometric structure.
For example, the space of all lines in $k^{n+1}$ is
$\mathbb{P}^n_k$. The space of all objects of a certain type is
referred to as a moduli space. \cite{bejleri2020}

More precisely, a moduli problem on $S$-schemes is a presheaf
\begin{align*}
  \mathcal M : {(\Sch / S)}^{\text{op}} \to \Set.
\end{align*}
There are various objects that one may consider to be a moduli
space for such a moduli problem.

\subsection{Fine moduli spaces}

The best case scenario arises when a moduli problem $\mathcal M$
is representable. In this case, the moduli problem $\mathcal M$
may itself be thought of as a scheme.

\begin{definition}
  Consider a moduli problem $\mathcal M$ on $\Sch/S$.
  A \emph{fine moduli space} of $\mathcal M$ is a scheme $M\in\Sch/S$
  together with a natural isomorphism
  \begin{align*}
    \eta : \mathcal M \cong \Hom\left({-,M}\right).
  \end{align*}
\end{definition}
It is standard to write $M(T) := \Hom(T,M)$ and hence identify
$M$ with its functor of points. We may then surpress the natural
isomorphism to identify $\mathcal M$ with its fine moduli space $M$.

Before we adopt this abuse of notation, let us consider the
isomorphism $\eta$ a little further. Note we may think of
$F\in \mathcal M(T)$ as the morphism $f : T \to M$.
Moreover, we obtain a pullback map $f^* : \Hom(M,M) \to \Hom(T,M)$
which induces a pullback map $f^* : \mathcal M(M)\to\mathcal M(T)$.
We then observe that $U := {\eta}^{-1}_M(\identity)$
is universal in the sense that every $F$ is obtained
by pulling back $U$ along $f$.

\subsection{Grassmannians}

While we are unlikely to actually need Grasmannians, they provide
us with a lot of intuition going forward. Moreover, they were used
by Grothendieck to prove the representability of the Quot scheme
and hence are worth knowing about. Let us recall the elementary
notion first:

\begin{definition}
  Let $V$ be a finite dimensional $k$-vector space. Then
  $G(m,V)$ denotes the set of $n$-dimensional subspaces of $V$.
\end{definition}

For now, this is just a set. However, for suitable choices of
$k$, it has the structure of a differentiable manifold as well as
that of an algebraic variety.
Now note that subspaces $W\in G(m,V)$ give rise to quotients
$V/W$ and we may recover $W=\ker(V\surj V/W)$. Thus subspaces
of $V$ may be identified with isomorphism classes of quotients.
In particular, any isomorphism $V/W \cong V/W$ will make the following
commute:
\begin{equation*}
  % https://q.uiver.app/#q=WzAsMyxbMCwwLCJWIl0sWzIsMCwiVi9XIl0sWzIsMSwiVi9XIl0sWzAsMiwiIiwyLHsic3R5bGUiOnsiaGVhZCI6eyJuYW1lIjoiZXBpIn19fV0sWzAsMSwiIiwwLHsic3R5bGUiOnsiaGVhZCI6eyJuYW1lIjoiZXBpIn19fV0sWzEsMiwiXFxjb25nIl1d
  \begin{tikzcd}
    V && {V/W} \\
      && {V/W}
      \arrow[two heads, from=1-1, to=1-3]
      \arrow[two heads, from=1-1, to=2-3]
      \arrow["\cong", from=1-3, to=2-3]
  \end{tikzcd}
\end{equation*}
Thus we may identify $G(m,V)$ with the set of isomorphism
classes of $k$-linear surjections $V\surj W$ with $\dim V-\dim W=m$.
This allows us to generalise from vector spaces to free sheaves
of modules.~\cite[\href{https://stacks.math.columbia.edu/tag/089R}{Tag 089R}]{stacks-project}. \todo{maybe we want to think and talk about
quotients of sheaves of modules a little more ...}

\begin{definition}
  The functor $G(m,n) : \Sch \to \Set$ associates to
  each scheme $T\in\Sch$ the set $G(m,n)(T)$ of isomorphism
  classes of surjections
  \begin{align*}
    q : \mathcal O^{\oplus n}_T \surj \mathcal F
  \end{align*}
  where $\mathcal F$ is a finite, locally free $\mathcal O_T$-module
  of rank $n-m$ and for each morphism $f:T\to T'$ the map
  $G(m,n)(f)$ sends the isomorphism class of $q$ to that
  of $f^*q$.
\end{definition}

We begin by noting that this does indeed generalise the previous
notion. In particular, there is a canonical bijection
\begin{align*}
  G(m,n)(\Spec k) \cong G(m,k^n)
\end{align*}
as quotients of $\mathcal O^{\oplus n}_{\Spec k}$ are just quotients
of of $k^n$. \question{what does canonical mean here? also, we would want this to be at least an isomorphism of varieties in some sense but i haven't seen anyone talk about this...}

This is a functor and, crucially, representable. Thus we
have a scheme $\mathbf G(m,n)$ with
\begin{align*}
  \mathbf G(m,n)(T) := \Hom(T,\mathbf G(m,n)) = G(m,n)(T).
\end{align*}
Base change gives us the usual $\mathbf G(m,n)_S\in\Sch_S$ and
$\mathbf G(m,n)_R\in\Sch_R$.
Moreover, we are justified in thinking of $\mathbf G(m,n)_S$ as
parametrising $m$-dimensional subspaces of $\affine{n}{S}$ in the
sense that we think of projective $n$-space as parametrising
$n$-dimensional subspaces of $(n+1)$-dimensional affine space.

\begin{proposition}
  Let $n\geq 1$ and $S\in\Sch$. Then there is a canonical isomorphism
  \begin{align*}
    \mathbf G(n,n+1)_S \cong \projective{n}{S}.
  \end{align*}
  \begin{proof}
    It suffices to consider $S=\Spec\mathbb{Z}$.
    See \cite[\href{https://stacks.math.columbia.edu/tag/089V}{Tag 089V}]{stacks-project}.
  \end{proof}
\end{proposition}

\subsection{Quot schemes}

Fix a Noetherian base scheme $S$, a scheme $X/S$ of finite type,
a coherent sheaf $\mathcal E$ on $X$, and another scheme $T/S$.

\begin{definition}
  A \emph{family of quotients of $\mathcal E$ parameterised by $T$} consists of
  \begin{enumerate}
    \item a coherent sheaf $\mathcal F$ on $X_T$
      such that the schematic support of $\mathcal F$ is proper
      over $T$ and $\mathcal F$ is flat over $T$, and
    \item a surjective $\mathcal O_{X_T}$-linear
      homorphism of sheaves $q:\mathcal E_T\to \mathcal F$.
  \end{enumerate}
  A morphism $f:(\mathcal F,q)\to(\mathcal F',q')$ of such families
  is a morphism $f:\mathcal F\to\mathcal F'$ that makes the following
  commute:
  \begin{equation*}
    % https://q.uiver.app/#q=WzAsMyxbMiwwLCJcXG1hdGhjYWwgRiJdLFsyLDIsIlxcbWF0aGNhbCBGJyJdLFswLDAsIkVfVCJdLFsyLDAsInEiXSxbMiwxLCJxJyIsMl0sWzAsMSwiZiJdXQ==
    \begin{tikzcd}
      {\mathcal E_T} && {\mathcal F} \\
      \\
                     && {\mathcal F'}
                     \arrow["q", from=1-1, to=1-3]
                     \arrow["{q'}"', from=1-1, to=3-3]
                     \arrow["f", from=1-3, to=3-3]
    \end{tikzcd}
  \end{equation*}
\end{definition}

\begin{example}
  If $S=\Spec\mathbb C$, $X=\Sigma_g$, and
  $\mathcal E=\mathcal O^{\oplus k}_X$.
  Then a family of quotients $q:\mathcal E_T\surj\mathcal F$
  is a quotient
  \begin{align*}
    (\mathcal O^{\oplus k}_X)_T \cong \mathcal O^{\oplus k}_{X\times_S T} \surj \mathcal F
  \end{align*}
  in $\Coh(X\times_S T)$.
  The kernel of such a quotient turns out to be locally-free
  and of finite-rank \cite[Lemma 4.21]{bertram1993}, i.e. a
  vector bundle on $X\times_S T$.
\end{example}

\begin{lemma}
  If $\mathcal F\in\Coh(X_T)$ is flat over $T$ and
  $q:\mathcal O_{X_T}^{\oplus k} \surj F$ is surjective and
  $\mathcal O_{X_T}$-linear then $\Ker(q)\in\Coh(X_T)$ is
  finite locally free.
  \begin{proof}
    We want to show that there is

    Recall that flatness of $\mathcal F$ over $T$ means that,
    for all $x\in X_T$, the stalk $\mathcal F_x$ is a flat
    $\mathcal O_{T,\pi(x)}$-module where
    $\pi : X_T \to T$.
    \missingproof
  \end{proof}
\end{lemma}

\begin{lemma}
  Families $(\mathcal F,q)$ and $(\mathcal F',q')$ are isomorphic if, and only if,
  $\ker(q) = \ker(q')$.
  \begin{proof}
    \missingproof
  \end{proof}
\end{lemma}

Write $\langle\mathcal F,q\rangle$ for equivalence classes
of such families and define the corresponding quot functor
as follows:
\begin{definition}
  The quot functor $\Sch_S^{\text{op}} \to \Set$ is given by
  \begin{align*}
    \mathfrak{Quot}_{\mathcal E,X/S} (T)
    := \left\lbrace{\text{all $\langle\mathcal F,q\rangle$
    parametrised by $T$}}\right\rbrace
  \end{align*}
\end{definition}

The map on morphsims $f:T'\to T$ is as follows.
We may pull back $\text{Coh}(X_T)$ to $\text{Coh}(X_{T'})$
along $\identity\times_S f:X_{T'}\to X_T$. In particular,
note that the following commutes:

\begin{equation*}
  % https://q.uiver.app/#q=WzAsMyxbMiwwLCJcXHRleHR7Q29ofShYX1QpIl0sWzIsMSwiXFx0ZXh0e0NvaH0oWF97VCd9KSJdLFswLDAsIlxcdGV4dHtDb2h9KFgpIl0sWzAsMSwiKFxcdGV4dHtpZH1cXHRpbWVzIGYpXioiXSxbMiwwLCJcXHBpXioiXSxbMiwxLCJcXHBpXioiLDJdXQ==
  \begin{tikzcd}
    {\text{Coh}(X)} && {\text{Coh}(X_T)} \\
                    && {\text{Coh}(X_{T'})}
                    \arrow["{(\text{id}\times f)^*}", from=1-3, to=2-3]
                    \arrow["{\pi^*}", from=1-1, to=1-3]
                    \arrow["{\pi^*}"', from=1-1, to=2-3]
  \end{tikzcd}
\end{equation*}
by functoriality of $(-)^*$. Hence, given a family
$\langle\mathcal F,q\rangle$, we obtain a diagram
\begin{equation*}
  % https://q.uiver.app/#q=WzAsMyxbMCwwLCJFX3tUJ30iXSxbMSwwLCIoXFx0ZXh0e2lkfVxcdGltZXNfU2YpXiogXFxtYXRoY2FsIEVfVCJdLFszLDAsIihcXHRleHR7aWR9XFx0aW1lc19TZileKlxcbWF0aGNhbCBGIl0sWzEsMiwiKFxcdGV4dHtpZH1cXHRpbWVzX1NmKV4qcSJdLFswLDEsIiIsMCx7ImxldmVsIjoyLCJzdHlsZSI6eyJoZWFkIjp7Im5hbWUiOiJub25lIn19fV1d
  \begin{tikzcd}
    {\mathcal E_{T'}} & {(\text{id}\times_Sf)^* \mathcal \mathcal E_T} && {(\text{id}\times_Sf)^*\mathcal F}
    \arrow["{(\text{id}\times_Sf)^*q}", from=1-2, to=1-4]
    \arrow[Rightarrow, no head, from=1-1, to=1-2]
  \end{tikzcd}
\end{equation*}
in $\Coh(X_{T'})$, i.e. a family of quotients parametrised by
$T'$. This respects isomorphisms of such families and
hence we have defined a map
\begin{align*}
  \mathfrak{Quot}_{\mathcal E,X/S}(T) \to \mathfrak{Quot}_{\mathcal E,X/S}(T').
\end{align*}

\begin{theorem}
  There is a scheme
  \begin{align*}
    \text{Quot}_{\mathcal E,X/S} \in \Sch_S
  \end{align*}
  that represents $\mathfrak{Quot}_{\mathcal E,X/S}$.
\end{theorem}

\subsection{Moduli space of stable bundles}

We aim to define the moduli space of (semi)stable bundles on a
suitable curve. We will do so without justification. For details
see \cite[section 8.8]{hoskins}.

Fix a connected smooth projective curve $C$ of $g\geq 2$,
$n\geq 1$, and $d > n(2g - 1)$. Define
$N:=d+n(1-g)$ and denote by
\begin{align*}
  Q^s \subset Q := \Quot_C^{n,d}(\mathcal O_C^N).
\end{align*}
the open subscheme of quotients $q:\mathcal O^N_X\surj\mathcal F$
such that $\mathcal F$ is (semi)stable and locally free and
$H^0(q)$ is an isomorphism.

As $Q$ is a fine moduli space, we have a universal quotient
$q:\mathcal O_{Q\times X}^N\surj \mathcal U$
and its restriction to $Q$:
\begin{align*}
  q^s : \mathcal O_{Q^s\times X}^N\surj \restrict{\mathcal U}{Q^s} =: \mathcal U^s.
\end{align*}

Now define the action
\begin{align*}
  \rho : GL_N \times Q \to Q
\end{align*}
under the isomorphism $Q(GL_N\times Q) \cong \Hom(GL_N\times Q,Q)$
by the quotient family
\begin{align*}
  \mathcal O^N_{GL_N\times Q\times X}
  = k^N \otimes \mathcal O_{GL_N\times Q\times X}
  \xlongrightarrow{\pi^*_{GL_N}\tau}
  k^N \otimes \mathcal O_{GL_N\times Q\times X}
  \xlongrightarrow{\pi^*_{Q\times X} q}
  \pi^*_{Q\times X}\mathcal U
\end{align*}

\section{Higgs bundles}

\subsection{Definition}

\begin{definition}
  A \emph{Higgs bundle} consists of
  \begin{itemize}
    \item a holomorphic vector bundle $E\to\Sigma_g$,
    \item
  \end{itemize}
\end{definition}

\subsection{Stability}

A Higgs bundle is said to be stable if its subbundles statisfy certain
properties. Before we can make this precise, we need to define
vector subbundles and their invariance with respect to the Higgs field.

\subsection{$\mathcal M$ vs $\mathcal N$}

\begin{proposition}
  $T^*\mathcal N^g_{n,d}$ is an open dense subset of $\mathcal M^g_{n,d}$.
\end{proposition}

\section{Non-abelian Hodge theory}

\begin{theorem}
  Let $X$ be a compact K\"ahler manifold. Then there is an equivalence
  of categories between irreducible flat vector bundles on $X$
  and stable Higgs bundles
\end{theorem}

\printbibliography

\end{document}
