\documentclass{article}
\usepackage{assignment}
\begin{document}
\title{The Complexity of Computing Mixed Nash Equilibria}
\author{Candidate 1082394}
\date{\today}
\maketitle

We review the complexity of computing mixed Nash equilibria. We begin by
recalling the definition and important properties of Nash equilibria. We
proceed by defining the search problem \texttt{Nash} of finding approximate
Nash equilibria and establishing membership of \textbf{TFNP}. We then introduce
\textbf{PPAD}, a subclass of \textbf{TFNP} of which \texttt{Nash} is a complete
member. This is proven by considering the problem of finding fixed points of
certain well-behaved problems, which turns out to be a suitable
\textbf{PPAD}-complete problem to reduce \texttt{Nash} from and to. We outline
the procedure in both directions.
We finish our review by considering the bigger picture of complexity
theory.

\section{Mixed strategy NE}

A Nash equilibrium is a strategy configuration with no beneficial
deviation for any player, i.e. no player has a strategy that would have
lead to a higher pay-off.

For a strategy profile $\mu$ and a strategy $\mu_i'$ write
$\mu_i' | \mu$ to mean the strategy profile with
\begin{align*}
  (\mu_i' | \mu)_j = \begin{cases}
    \mu_j & i\neq j \\
    \mu_i' & i = j
  \end{cases}
\end{align*}

\begin{definition}
  A (mixed strategy) \emph{Nash equilibrium} a mixed strategy profile $\mu$
  such that, for all $1\leq i\leq k$ and all mixed strategies $\mu_i'$,
  $EU_i(\mu_i' | \mu) \leq EU_i(\mu)$.
\end{definition}

Note that we may write any strategy $\mu_i$ as a linear
combination of pure strategies:
\begin{align*}
  \mu_i = \sum_{\sigma_i\in\Sigma_i} \prc{\sigma_i}{\mu_i} \sigma_i.
\end{align*}
The support of a mixed strategy $\mu_i$ is
\begin{align*}
  \supp(\mu_i) = \left\lbrace{\sigma_i \in \Sigma_i : \prc{\sigma_i}{\mu_i} > 0}\right\rbrace.
\end{align*}
Pure strategies are important because they allow us to characterise
Nash equilibria:

\begin{theorem}\label{thm:indifference}
  A strategy configuration $\mu$ is a
  Nash equilibrium if, and only if, for all $1\leq i\leq k$,
  and all $\sigma_i\in\supp(\mu_i)$ and $\sigma_i'\in\Sigma_i$,
  \begin{align}\label{eq:indifference}
    EU_i(\sigma_i | \mu) \geq EU_i(\sigma_i' | \mu).
  \end{align}
  \begin{proof}
    Firstly, note that the condition implies $EU_i(\sigma_i|\mu)=EU_i(\sigma_i'|\mu)$ whenever $\sigma_i,\sigma_i'\in\supp(\mu_i)$. By the
    indifference principle, this holds whenever $\mu$ is a Nash
    equilibrium. Moreover, we must have $EU_i(\sigma_i|\mu)\geq EU_i(\sigma'_i|\mu)$ whenever $\sigma'_i\not\in\supp(\mu_i)$ as otherwise
    $\sigma'_i$ would provide player $i$ with a beneficial deviation
    from $\mu$.

    Now suppose $\mu$ is a strategy profile that satisfies (\ref{eq:indifference}) and fix a player $1\leq i\leq k$.
    Then, for any strategy $\mu_i'$ and distinguished
    $\sigma_i^*\in\supp(\mu_i)$ and $\sigma_i'\in\Sigma_i$, we use linearity
    of expectation to find
    \begin{align*}
      EU_i(\mu_i' | \mu)
      = \sum_{\sigma_i\in\Sigma_i} EU_i (\sigma_i|\mu)\prc{\sigma_i}{\mu_i'}
      = EU_i (\sigma_i|\mu) \sum_{\sigma_i\in\supp(\mu'_i)} \prc{\sigma_i}{\mu_i'}
      = EU_i (\sigma_i'|\mu)
      \leq EU_i (\sigma_i^*|\mu).
    \end{align*}
    Hence $EU_i(\mu_i' | \mu) \leq EU_i(\sigma^*_i | \mu) = EU_i(\mu^*_i | \mu) =
    EU_i(\mu)$. This shows that $\mu$ is a Nash equilibrium.
  \end{proof}
\end{theorem}

The theorem gives us a way to verify that a given strategy configuration is in fact a Nash equilibrium. This is similar to problems in \textbf{NP}
in that, given a witness, we may verify in polynomial time that the
output is correct.

However, in contrast to problems in \textbf{NP}, computing Nash
equilibria is not truly a decision problem: by Nash's theorem, every
game has a mixed strategy equilibrium. Hence we need to phrase
things in terms of search problems.

\section{Search problems}

We are going to make precise the idea that, in order to solve
a problem, one is not only required to decide whether there exists
a solution, but rather to exhibit a witness. That is, we require a
machine that, given a problem instance, produces an output satisfying
a certain binary relation.

\begin{definition}
  A \emph{search problem} is given by a set of instances $X$,
  a set of candidates $Y$, and a relation $R\subseteq X\times Y$.
  An algorithm is said to \emph{solve} a search problem if
  it computes a function $f:X\to Y\sqcup\left\lbrace{\bot}\right\rbrace$
  such that $f(x) = \bot$ if, and only if, $\res{R}{x} = \emptyset$
  and $f(x)\in\res{R}{x}$ otherwise where
  $\res{R}{x} = \left\lbrace{y\in Y: xRy}\right\rbrace$.
\end{definition}

There are plenty of search problems which we are already familiar with.
For example, one may consider boolean expressions and satisfying assignments
(wich leads to the search problem \texttt{FSAT}),
graphs and Hamiltonian paths, weighted graphs and Hamiltonian paths shorter
than a given value, and similar adaptations of decision problems. Note that the
typical \textbf{NP}-complete decision problems have associated search problems
of similar difficulty. For example, given an oracle that decides whether a
given boolean expression has a satisfying assignment, one may find a satisfying
assignment in $2n$ queries to the oracle by sequentially setting variables to
true or false and checking whether the remaining expression remains
satisfiable. \cite{goldberg2022}

The search problem of interest to us is computing Nash equilibria. However, for
general $k$, it need not be the case that the probabilities involved are
rational. For this, an other reasons which will become apparent later on, we
introduce the notion of an approximate Nash equilibrium. Usually, this is done
by saying that a point is an $\varepsilon$-Nash equilibrium if all players
stand to improve their pay-offs by at most $\varepsilon$ by deviating. We adapt
change this notion slightly by forcing $\varepsilon$ to be $2^{-d}$ for some
integer $d$.

\begin{definition}
  Let $d\in\mathbb{N}$. A $d$-Nash equilibrium is a strategy profile
  $\mu$ such that, for all $1\leq i\leq k$ and all mixed strategies
  $\mu_i'$, $\left\vert{EU_i(\mu_i' | \mu) - EU_i(\mu)}\right\vert\leq 2^{-d}$.
\end{definition}

One may think of such approximate Nash equilibria as being Nash equilibria
up to the $d$-th bit. Note however, that approximate Nash equilibria
need not be close to true Nash equilibria whatsoever. Hence, finding
approximate Nash equilibria is indeed a strictly easier problem, not
just for reasons to do with representations of games and comutability
of functions on uncountable domains.

This allows us to define a search problem which one may hope to
solve:

\begin{definition}\label{def:nash}
  Denote by $k\texttt{-Nash}$ the problem defined as folows:
  \begin{itemize}
    \item instances consist of an integer $d$ and a $k$-player game $G$,
    \item candidates are all possible strategy configurations, and
    \item solutions are $d$-Nash equilibria of $G$.
  \end{itemize}
\end{definition}

While the typical $\textbf{NP}$ decision problems have associated search
problems of similar difficulty, the same cannot be said for $k$\texttt{-Nash}
as the corresponding decision problem is trivial. We are thus motivated
to introduce a new complexity class:

\begin{definition}
  The class \textbf{FNP} contains all search problems $R\subseteq X\times Y$
  for which the characteristic function $\chi_R : X\times
  Y\to\left\lbrace{0,1}\right\rbrace$ is computable in polynomial time.
\end{definition}

Before we can establish $k\texttt{-Nash}$ as a member of this class,
we need to make precise what we mean by the size of a game.
As we will be concerned with whether or not and in what sense Nash
equilibria are computable in polynomial time, we do not need to be
particularly careful. Define the size of a $k$-player game $G$ to be
$|G|=\prod_{i=1}^k|\Sigma_i|$.

\begin{corollary}
  $d\texttt{-Nash}\in\textbf{FNP}$.
  \begin{proof}
    Let $G$ be a game of size $n$. Suppose we are given a strategy
    configuration $\mu$. By \ref{thm:indifference}, it suffices to check that
    $EU_i(\mu) \geq EU_i(\sigma_i|\mu)$ for all players $i$ and all pure
    strategies $\sigma_i\in\Sigma_i$. Hence we may decide whether $\mu$ is a
    Nash equilibrium in $n$ checks.
  \end{proof}
\end{corollary}


We have already noted that \textbf{FNP} contains some problems,
such as \texttt{FSAT}, for which there may not be any solutions
as well as others, such as \texttt{Nash}, for which there is
guaranteed to be a solution.

To capture this difference, we have the following subclass
of \textbf{FNP}:

\begin{definition}
  The class \textbf{TFNP} contains all search problems in \textbf{FNP}
  that are guaranteed to have a solution. That is
  \begin{align*}
    \textbf{TFNP} = \left\lbrace{(X,Y,R) \in \textbf{FNP} : \forall x\in X. \res{R}{x}\neq\emptyset}\right\rbrace
  \end{align*}
\end{definition}

In order to compare the complexities of different search problems,
we need to extend the notion of reductions from decision problems to
search problems:

\begin{definition}
  Let $P=(X,Y,R)$ and $P'=(X',Y',R')$ be search problems. We say
  $P$ reduces to $P'$ (in polynomial time) and write $P\leq_P P'$
  if there exist polynomial time computable functions $f:X\to X'$ and
  $g:X\times Y'\to Y$ such that
  \begin{align*}
    (f(x),y)\in R' \text{ if, and only if, } (x,g(x,y))\in R.
  \end{align*}
\end{definition}

We do not expect \textbf{TFNP} to have any complete problems, i.e.
problems which all other problems may be reduced to. This is in part
because we have yet to find one and in part because there are similarities
between \textbf{TFNP} and other complexity classes which have been
proven to not have complete problems. \cite{goldberg2017}
This observation leads us to focus on subclasses of \textbf{TFNP} that
do in fact have complete problems.

\section{The End-of-Line problem and PPAD}

The surest way to ensure that a comlexity class has complete problems
is to define it in terms of a problem which is taken to be complete.
To this end, consider the following:

Let $V$ be a set of size $2^n$. We may use functions
$p,s:V\to V$ to define a directed acyclic graph $G(p,s)=(V,E)$ by
saying there is an edge $(u,v)\in E$ if, and only if, $p(v)=u$
and $s(u)=v$. A sink then is a vertex $u\in V$ such that,
for all $v\in V$, $u\neq p(v)$ or $v\neq s(u)$. Similarly,
a source is a vertex $v\in V$ such that, for all $u\in V$,
$u\neq p(v)$ or $v\neq s(u)$.

Moreover, if we have $V=\left\lbrace{0,1}\right\rbrace^n$,
any such $p$ and $s$ may be thought of a Boolean circuits.

\begin{definition}
  Let $P$ and $S$ be algorithms computing functions
  $\left\lbrace{0,1}\right\rbrace^n\to\left\lbrace{0,1}\right\rbrace^n$.
  Then the \emph{induced graph $G(P,S)$} has vertices $\left\lbrace{0,1}\right\rbrace^n$
  and edges
  \begin{align*}
    E = \left\lbrace{(u,v) : u,v\in V, u=P(v), v=S(u)}\right\rbrace.
  \end{align*}
\end{definition}

\begin{definition}
  \texttt{EndOfLine} is the search problem with
  \begin{itemize}
    \item instances consisting of polynomial time algorithms $P,S$
      computing functions $\left\lbrace{0,1}\right\rbrace^n\to\left\lbrace{0,1}\right\rbrace^n$ such that
      $G(P,S)$ is a directed acyclic graph, and a
      $t\in\left\lbrace{0,1}\right\rbrace^n$ which is a source in $G(P,S)$,
    \item candidates all binary strings, i.e. elements of
      $\left\lbrace{0,1}\right\rbrace^* = \bigcup_{n=1}^\infty \left\lbrace{0,1}\right\rbrace^n$, and
    \item solutions strings different from $t$ which are either a sink or
      a source.
  \end{itemize}
\end{definition}

\begin{definition}
  The class \textbf{PPAD} consists of all problems in \textbf{FNP}
  that reduce to \texttt{EndOfLine}.
\end{definition}

We observe $\texttt{EndOfLine}\in\textbf{TFNP}$ so
$\textbf{PPAD}\subseteq\textbf{TFNP}\subseteq\textbf{FNP}$.
Over the next two chapters we will establish a few more members
of \textbf{PPAD} and eventually show that \texttt{Nash} is
\textbf{PPAD}-complete.

\section{Brouwer's fixed point theorem}

We are going to break both the reductions
$k\texttt{-Nash}\leq_P\texttt{EndOfLine}$ and
$\texttt{EndOfLine}\leq_P k\texttt{-Nash}$, into two steps by
establishing an intermediate problem $\texttt{Brouwer}$ and show
\begin{align*}
  k\texttt{-Nash}\leq_P\texttt{Brouwer}\leq_P\texttt{EndOfLine},
  \hspace{1cm}
  \texttt{EndOfLine}\leq_P \texttt{Brouwer}\leq_P k\texttt{-Nash}.
\end{align*}

Let us now introduce $\texttt{Brouwer}$.
From topology, we have the following very important result:

\begin{theorem}[Brouwer]
  Every continuous function $[0,1]^n\to[0,1]^n$ has a fixed point.
\end{theorem}

This leads to the natural problem of computing such fixed points.
However, once again there are some problems we need to address.
The fixed points may be irrational. Hence we define approximate
fixed points in analogy to approximate Nash equilibria:

\begin{definition}
  Let $f:\left[{0,1}\right]^k \to \left[{0,1}\right]^k$ be continuous and
  $d\geq 0$. A $d$-fixed point is an $x\in\left[{0,1}\right]^k$ such that
  $|f(x)-x|\leq 2^{-d}$.
\end{definition}

Once again, approximate fixed points need not be close to actual fixed
points. We will also require our functions to be reasonably
well-behaved. In particular, we will want to be able to bound the change
in the output by the change of the input.

\begin{definition}
  Let $K\geq 0$. A function $f:\left[{0,1}\right]^k\to\left[{0,1}\right]^k$ is \emph{$K$-Lipschitz} if, for all $x,x'\in\left[{0,1}\right]^k$,
  \begin{align*}
    |f(x)-f(x')| \leq K|x-x'|.
  \end{align*}
\end{definition}

This rules out functions such as the topologists sine curve $x\mapsto\sin(1/x)$ which may be extended to a continuous non-Lipschitz function $\left[{0,1}\right]\to\left[{0,1}\right]$.
We now show that Brouwer fixed points of Lipschitz functions may always be
approximated by multiples of $2^{-d}$.

\begin{lemma}
  Let $f:\left[{0,1}\right]^k\to\left[{0,1}\right]^k$ be a
  $K$-Lipschitz function and $d\geq 0$. Then there is a $\tilde x \in [0,1]^k$ such that $\tilde x$ is a multiple of $2^{-d'}$ and $\tilde x$
  is an $d$-fixed point of $f$ and $d'-d=O(\log(K))$.
  \begin{proof}
    Fix $d,K,f$. Let $x$ be a Brouwer fixed point of $f$.
    Choose
    \begin{align}\label{eq:d}
      d' = \lceil \log_2\left({kK}\right) + d + 1\rceil
    \end{align}
    and $\tilde x = \lfloor{2^{d'} x}\rfloor2^{-d'}$.
    Then
    \begin{align*}
      |\tilde x - x| \leq \sqrt{k} 2^{-d'} \leq 2^{\log_2 k-d'}.
    \end{align*}
    Using that $x$ is a fixed point and the triangle inequality,
    \begin{align*}
      |f(\tilde x)-\tilde x| \leq |f(\tilde x)-f(x)|+|\tilde x - x|
      \leq (K+1)2^{\log_2 k-d'} \leq K2^{\log_2 k+1-d'}\leq 2^{-d}.
    \end{align*}
    Thus $\tilde x$ is a $d$-fixed point.
  \end{proof}
\end{lemma}

The second issue we encounter when turning Brouwer's theorem into
a search problem is the representation of a continuous function
$\left[{0,1}\right]^k\to\left[{0,1}\right]^k$. In light of the result
above, it will be sufficient to consider functions which are
efficiently computable for multiples of $2^{-d}$ corresponding
to $\varepsilon$ and $K$. We will consider arithmetic circuits
involving gates for (1) addition, (2) multiplication, (3) division by 2,
and (4) comparisons with binary outputs.

\begin{definition}
  Let $f:\left[{0,1}\right]^k\to\left[{0,1}\right]^k$ be a
  $K$-Lipschitz function. A family of circuits $\{C_j\}$ \emph{efficiently computes} $f$ if
  \begin{enumerate}

    \item for all integer multiples $\tilde x\in\left[{0,1}\right]^k$ of $2^{-j}$, $|C_j(\tilde x) - f(\tilde x)|\leq 2^{-j}$, and
    \item each $C_j$ may be constructed in time polynomial in $j$.
  \end{enumerate}
\end{definition}

Note that our definition immediately implies that such $C_d$ contain
at most $O(\text{poly}(d))$ gates and hence, given the algorithm to
construct $C_d$, computing $f(\tilde x)$ up to an error $2^{-d}$ is
$O(\text{poly}(d))$ as a whole.

\begin{definition}
  Denote by $\texttt{Brouwer}$ the search problem with
  \begin{itemize}
    \item instances consisting of integers $K,d$ and a family of
      circuits $\left\lbrace{C_j}\right\rbrace$ that
      efficiently computes a $K$-Lipschitz function $f:\left[{0,1}\right]^k\to\left[{0,1}\right]^k$,
    \item candidate all integer multiples $\tilde x\in\left[{0,1}\right]^k$ of $2^{-j}$ for any positive integers $j$ and $k$, and
    \item solutions multiples of $2^{-d}$ that are $d$-fixed points of $f$.
  \end{itemize}
\end{definition}

\section{\texttt{Brouwer} is PPAD complete}

It is now time to establish \texttt{Brouwer} as a suitable intermediate
problem to establish that \texttt{Nash} is \textbf{PPAD}-complete.
While we will not dive deep into the technical details, we give an overview
of the proof given in \cite{daskalakis2008}.

\begin{theorem}
  $\texttt{Brouwer}\leq_P\texttt{EndOfLine}$.
  \begin{proof}
    We prove the case of $k=2$ but the proof may be generalised to
    work for all $k$.

    Fix an instance $(d,K,\left\lbrace{C_j}\right\rbrace)$ of $\texttt{Brouwer}$.
    Divide $[0,1]^2$ into $2^{2d}$ squares of side length $2^{-d}$ and
    further subdivide each square into two triangles. Thus we obtain
    a triangulation where each triangle is contained in a disk of
    diameter $\leq 2^{-d}$.
    Colour each vertex $x$ in the triangulation based on the direction of
    $(f(x)-x)/|f(x)-x|$ by assigning green, blue, and red to $[0,\pi/2)$, $[\pi/2,5\pi/4)$, and $[5\pi/4,2\pi)$, respectively.
    Should the angle be undefined because $f(x)-x=0$ then we have
    found a fixed point and hence a $d$-fixed point.
    We extend the triangulation by adding a column of red vertices on the
    left where we colour the bottom most vertex green and all the others
    red.

    We then construct a directed graph $G$ by taking the vertices to be the triangles
    and draw an edge between adjacent triangles such that there is
    a red vertex on the left and a green vertex on the right.
    It is
    easy to see that the resulting graph contains only cycles
    and paths from source to target. Note that the triangle $t_0$ where
    the unique change from red to green occurs on the left is not
    trichromatic. This is the case because points on the left
    boundary may not be moved further to the left, hence they cannot
    be coloured blue. Moreover, there is an outgoing edge of $t_0$
    which cannot be part of a cycle. Thus it must lead to
    a sink, i.e. a trichromatic triangle.

    By construction, we have a sink $t_0$. Further, given a triangle one
    may efficiently check the colours of the nodes by using the $C_j$
    and then find the neighbouring triangles by reflecting the
    remaining vertex.
    Hence we have polynomial time algorithms to compute the neighbour(s)
    of any triangle. This gives us an instance of \texttt{EndOfLine} with the solution yielding
    a trichromatic triangle.
    It is now straightforward to argue
    that, because the distances between the nodes of the trichromatic
    triangle are small and because each node is mapped in a different
    direction, each node must be an approximate fixed point of $f$.

    It remains to make sure that (a) all the operations described
    may be performed in polynomial time and (b) the various degrees
    of approximation do indeed lead to a solution with respect to the
    given $d$. The latter may be resolved by choosing a suitably
    fine triangulation and hence will not be an issue.
  \end{proof}
\end{theorem}

\begin{theorem}
  $\texttt{EndOfLine}\leq_P\texttt{Brouwer}$.
  \begin{proof}
    The proof of this is rather technical. It is achieved by constructing
    a 4-colouring of a grid of $2^{3d}$ points from a line graph with
    $2^d$ vertices. This is done by representing edges as paths
    of three colours in the grid, connecting points associated to the
    original vertices. This is done in a way that ensures
    that all four colours only appear in proximity at sinks. The grid
    is then converted to a function $[0,1]^3\to[0,1]^3$ by assigning
    to each colour a direction and defining the displacement at each point to be a linear interpolation of the colours that the closest grid points.

    Thus approximate fixed points only appear when all four colours
    appear close to each other, i.e. at sinks. For all the technical
    details see \cite{daskalakis2008}.
  \end{proof}
\end{theorem}


\section{\texttt{Nash} is PPAD-complete}

Now that we have established that \texttt{Brouwer} is PPAD complete,
we need only show that \texttt{Nash} is equivalent to \texttt{Bouwer}.
This indirect way of showing that \texttt{Nash} is equivalent to
\texttt{EndOfLine} is an obvious thing to do because the classical
proof of Nash's theorem uses Bouwer's fixed point theorem and hence
almost gives us the reduction for free:

\begin{theorem}
  $k\texttt{-Nash}\leq_P\texttt{Brouwer}$.
  \begin{proof}[Sketch of the proof]
    When using Brouwer's theorem to prove Nash's theorem one
    constructs a function $b$ on strategy configurations that
    moves each players strategy towards the best response.
    (see e.g. \cite{karsten2017})
    If one identifies the space of mixed strategies with a suitable
    $\left[{0,1}\right]^n$ then $b$ is continuous and hence has a
    fixed points. Fixed points correspond to Nash equilibria.

    While this proves Nash's theorem, some work is required to
    yield the reduction:
    \begin{enumerate}
      \item check that $b:\left[{0,1}\right]^n\to\left[{0,1}\right]^n$
        is Lipschitz,
      \item check that $b$ is computable in polynomial time,
      \item check that $d$-fixed points correspond to
        $d$-Nash equilibria,
      \item check that the \texttt{Brouwer} has size polynomial
        in the size of the original \texttt{Nash} instance.
    \end{enumerate}
    These checks turn out to be routine. Depending on the exact
    definition of the function $b$ one may be requried to do some
    scaling to be able to use the same parameter $d$ for both
    instances.
  \end{proof}
\end{theorem}

The other direction is far less obvious. We follow \cite{daskalakis2008}
and \cite{daskalakis2009}. Our goal now is to
construct a game whose Nash equilibria correspond to fixed points of
a function $f:\left[{0,1}\right]^k\to\left[{0,1}\right]^k$.
More precisely, we are given a family of circuits $\left\lbrace{C_d}\right\rbrace$ that efficiently compute $f$ and we want to construct a
family of games $\left\lbrace{G_d}\right\rbrace$ such that
the $d$-fixed points of $C_d$ correspond to the $d$-Nash equilibria of $G_d$.

We begin by constructing such games for each of the elementary gates
in $C_d$:
\begin{itemize}
  \item $G^\times$ has four players $x,y,z,w$, each of whom have two pure strategies
    $S$ and $G$.
    Define the following utilities:
    \begin{align*}
      U_w(\sigma_x,\sigma_y,\sigma_z, S_w) = \begin{cases}
        1 & \text{if } \sigma_x=\sigma_y= G \\
        0 & \text{otherwise}
      \end{cases}
    \end{align*}
  \item $G^+$ has four players $x,y,z,w$, each of whom have two pure strategies
    $S$ and $G$.
    Define the following utilities:
    \begin{align*}
      U_w(\sigma_x,\sigma_y,\sigma_z, S_w) = |\left\lbrace{p\in\left\lbrace{x,y}\right\rbrace : \sigma_p = G}\right\rbrace|,
    \end{align*}
  \item $G^{/2}$ has three players $x,z,w$, each of whom have two pure strategies
    $S$ and $G$.
    Define the following utilities:
    \begin{align*}
      U_w(\sigma_x,\sigma_z, S_w) = \begin{cases}
        2 & \text{if } \sigma_x= G \\
        1 & \text{otherwise}
      \end{cases}
    \end{align*}
  \item $G^<$ has four players $x,y,z,w$ each of whom have two pure strategies $S$ and $G$. Define the following utilities:
    \begin{align*}
      U_w(\sigma_x,\sigma_y,\sigma_z,S_w) = \begin{cases}
        1 & \text{if }\sigma_x = S,\sigma_y = G\\
        0 & \text{otherwise}
      \end{cases}, \hspace{1cm}
      U_w(\sigma_x,\sigma_y,\sigma_z,G_w) = \begin{cases}
        1 & \text{if }\sigma_x = G,\sigma_y = S\\
        0 & \text{otherwise}
      \end{cases}
    \end{align*}
\end{itemize}
Moreover, for each of the games we define
\begin{align*}
  U_w(\sigma_x,\sigma_y,\sigma_z, G_w) = \begin{cases}
    1 & \text{if } \sigma_z= G \\
    0 & \text{otherwise}
  \end{cases}
  , \hspace{1cm}
  U_z(\sigma_x,\sigma_y,\sigma_z,\sigma_w) = \begin{cases}
    1 & \text{if }\sigma_z\neq\sigma_w\\
    0 & \text{otherwise}
  \end{cases}
\end{align*}

It is now straightfoward to see that in the Nash
equilibria $\mu$ of $G^\times$, $G^+$ and $G^{/2}$ we must have $\mu_z =
\mu_x\mu_y$, $\mu_z = \mu_x + \mu_y$, and $\mu_z = \mu_x / 2$, respectively,
where we write $\mu_p = \prc{G_p}{\mu}$. Note also that if we have
players $x,t,y$ where we pay $t$ to play the opposite of $x$ and
$y$ to play the opposite of $t$, then $\mu_x=\mu_y$. Using this
construction we may compose games to achieve more complicated
computations.

However, there is a problem: while for $G^<$ we have
$\mu_z = 0$ if $\mu_x<\mu_y$ and $\mu_z = 1$ if $\mu_x>\mu_y$,
$\mu_z$ could take any value in the case where $\mu_x=\mu_y$.
Moreover, there is the following result that says that constructing
a better game to handle comparisons is hopeless:

\begin{lemma}
  Consider a $k$ player game where each player has exactly
  two pure strategies $S$ and $G$. Then there is no triple $x,y,z$ of
  distinct players such that
  \begin{enumerate}
    \item for all $p,q\in[0,1]$, there is a Nash equilibrium
      $\mu$ with $\mu_x = p$ and $\mu_y = q$, and
    \item every Nash equilibrium $\mu$ satisfies
      \begin{align}\label{eq:good_comparator}
        \mu_z = \begin{cases}
          0 &\text{if }\mu_x \geq \mu_y, \\
          1 & \text{if }\mu_x < \mu_y.
        \end{cases}
      \end{align}
  \end{enumerate}
  \begin{proof}
    Suppose there was such a game. We extend it as follows:
    \begin{enumerate}
      \item add a new player $t$ who receives a pay-off if, and only if,
        she plays the same as $x$,
      \item player $x$ receives a pay-off if, and only if, she plays
        opposite to $t$,
      \item player $y$ receives a pay-off if, and only if, she plays
        opposite to $z$.
    \end{enumerate}
    Note that we did not change the pay-offs for $z$. Hence any
    Nash equilibrium must satisfy (\ref{eq:good_comparator}).
    However, our extensions force $\mu_x = \mu_t = 1/2$
    and $\mu_y = 1-\mu_z$. Thus we obtain the unsatisfiable condition
    \begin{align*}
      \mu_z = \begin{cases}
        0 & \text{if } \mu_z \geq 1/2,\\
        1 & \text{if } \mu_z < 1/2
      \end{cases}.
    \end{align*}
    This means there cannot be a Nash equilibrium in this new game,
    contradicting Nash's theorem.
  \end{proof}
\end{lemma}

Once again we are saved by approximate Nash equilibria. One may
modify the circuit so as to evaluate the original circuit at many
points close to the point of interest, depending on the desired
accuracy. This will result in a computation that is correct up to
an additive error. Hence we may combine the above games using
appropriate pay-offs to construct a game that has approximate
Nash equilibria corresponding to approximate fixed points of the
input function.

However, the resulting game will have a number of players linear in $n$,
the number of gates in the input circuit. Now each player has two strategies
meaning our game has size $\Omega(2^n)$. To resolve this, regard
each game as a graph by drawing an edge from player $i$ to $j$ if
the strategy chosen by $i$ affects player $j$. It is then possible
to modify the games $G^+,\ldots,G^<$ and the way in which they are connected
to ensure that the resulting game may be 3-coloured in such a way as
to make sure that no two nodes with the same colour are connected by a
path of length two or shorter.

Using this colouring, we may create a three player game where each
player represents a colour. Call each the players in the new game a
super player. Each of these players actions will correspond
to one action by one player in the previous game which we constructed
out of the elementary gates. Thus each of these
super players has a set of actions linear in the number of players.
The pay-offs are awarded to the super player whenever they represent
an original player all of whose dependents have chosen a strategy,
i.e. if the other super players have chosen strategies corresponding
to the dependent players.

To minimise the incentives towards super players of choosing to play
for original players that generate higher pay-offs, we award further
pay-offs corresponding to generalised rock-paper-scissors games to
each of the super players. Choosing sufficiently large pay-offs for these
games will result in the super players choosing a strategy from each
player with approximately equal probability. Thus approximate Nash
equilibria in the super game will encode all the information required
to construct approximate Nash equilibria for the original game.

Even though we have left out some key technical details, this outlines
the proof of the following:

\begin{theorem}
  $\texttt{Brouwer}\leq_P 3\texttt{-Nash}$.
\end{theorem}

Using the obvious reduction $k\texttt{-Nash}\leq_P k'\texttt{-Nash}$
for $k\leq k'$ we thus obtain \textbf{PPAD}-completeness of $k\texttt{-Nash}$
for $k\geq 3$.
The case $k=2$ may be resolved in various ways, see e.g.
\cite{chen2007}. Thus $k\texttt{-Nash}$ is \textbf{PPAD}-complete.

\section{Consequences}

It is believed that the inclusion $\textbf{FP}\subseteq\textbf{PPAD}$ is strict.
While proving this remains an open problem, it is intuitively clear why this
ought to be the case: if \texttt{EndOfLine} were in $\textbf{FP}$ then one
would be able to perform a search on a graph with exponential size in
polynomial time. Knowing that computing approximate Nash equilibria is
\textbf{PPAD}-complete thus tells us that we should not expect to solve
this problem in polynomial time. This further motivates two fields of
research: Finding classes of games that may be solved efficiently as
well as other methods of solving games in a practical setting, such as
randomisation and machine learning.

From a complexity point of view, the result does not add a lot to the
bigger picture. The question remains of which of the inclusions
\begin{align*}
  \textbf{FP} \subseteq \textbf{PPAD} \subseteq \textbf{TFNP} \subseteq
  \textbf{FNP},
\end{align*}
if any, are strict. The two inclusions on the right are interesting
because it is not clear which way they ought to be resolved:
$\textbf{PPAD}=\textbf{TFNP}$ would imply that \textbf{TFNP} does
indeed have complete problems. Moreover, it would collapse
several intermediate complexity classes such as \textbf{PPA}
and \textbf{PPP} into one. On the other hand, if
$\textbf{TFNP}=\textbf{FNP}$ then $\textbf{NP}=\textbf{coNP}$.
\cite{goldberg2018} Of course, it may be the case that all these
inclusions are indeed strict in which case \textbf{PPAD} problems
remain intractable but in an abstract sense less so than \texttt{SAT}.

Similarly, it is well-known by the time hierarchy theorem that
one of the inclusions
\begin{align*}
  \textbf{P}\subseteq\textbf{NP}\subseteq\textbf{PSPACE}\subseteq\textbf{EXPTIME}
\end{align*}
is strict, but even showing that one is strict does not rule out
any other. Thus proving that computing Nash equilibria
is intractable would prove $\textbf{P}\neq\textbf{NP}$ but the
other two questions would remain.

\printbibliography

\end{document}
