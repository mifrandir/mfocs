\documentclass{article}
\usepackage{assignment}
\addbibresource{references.bib}
\begin{document}
\title{The Complexity of Computing Mixed Nash Equilibria}
\author{Franz Miltz}
\date{\today}
\maketitle

\section{Mixed strategy NE}

A Nash equilibrium is a strategy configuration with no beneficial
deviation for any player, i.e. no player has a strategy that would have
lead to a higher pay-off.

For a strategy profile $\mu$ and a strategy $\mu_i'$ write
$\mu_i' | \mu$ to mean the strategy profile with
\begin{align*}
  (\mu_i' | \mu)_j = \begin{cases}
    \mu_j & i\neq j \\
    \mu_i' & i = j
  \end{cases}
\end{align*}

\begin{definition}
  A (mixed strategy) \emph{Nash equilibrium} a mixed strategy profile $\mu$
  such that, for all $1\leq i\leq k$ and all mixed strategies $\mu_i'$,
  $EU_i(\mu_i' | \mu) \leq EU_i(\mu)$.
\end{definition}

Note that we may write any strategy $\mu_i$ as a linear
combination of pure strategies:
\begin{align*}
  \mu_i = \sum_{\sigma_i\in\Sigma_i} \prc{\sigma_i}{\mu_i} \sigma_i.
\end{align*}
The support of a mixed strategy $\mu_i$ is
\begin{align*}
  \supp(\mu_i) = \left\lbrace{\sigma_i \in \Sigma_i : \prc{\sigma_i}{\mu_i} > 0}\right\rbrace.
\end{align*}
Pure strategies are important because they allow us to characterise
Nash equilibria:

\begin{theorem}\label{thm:indifference}
  A strategy configuration $\mu$ is a
  Nash equilibrium if, and only if, for all $1\leq i\leq k$,
  and all $\sigma_i\in\supp(\mu_i)$ and $\sigma_i'\in\Sigma_i$,
  \begin{align}\label{eq:indifference}
    EU_i(\sigma_i | \mu) \geq EU_i(\sigma_i' | \mu).
  \end{align}
  \begin{proof}
    Firstly, note that the condition implies $EU_i(\sigma_i|\mu)=EU_i(\sigma_i'|\mu)$ whenever $\sigma_i,\sigma_i'\in\supp(\mu_i)$. By the
    indifference principle, this holds whenever $\mu$ is a Nash
    equilibrium. Moreover, we must have $EU_i(\sigma_i|\mu)\geq EU_i(\sigma'_i|\mu)$ whenever $\sigma'_i\not\in\supp(\mu_i)$ as otherwise
    $\sigma'_i$ would provide player $i$ with a beneficial deviation
    from $\mu$.

    Now suppose $\mu$ is a strategy profile that satisfies (\ref{eq:indifference}) and fix a player $1\leq i\leq k$.
    Then, for any strategy $\mu_i'$ and distinguished
    $\sigma_i^*\in\supp(\mu_i)$ and $\sigma_i'\in\Sigma_i$, we use linearity
    of expectation to find
    \begin{align*}
      EU_i(\mu_i' | \mu)
      = \sum_{\sigma_i\in\Sigma_i} EU_i (\sigma_i|\mu)\prc{\sigma_i}{\mu_i'}
      = EU_i (\sigma_i|\mu) \sum_{\sigma_i\in\supp(\mu'_i)} \prc{\sigma_i}{\mu_i'}
      = EU_i (\sigma_i'|\mu)
      \leq EU_i (\sigma_i^*|\mu).
    \end{align*}
    Hence $EU_i(\mu_i' | \mu) \leq EU_i(\sigma^*_i | \mu) = EU_i(\mu^*_i | \mu) =
    EU_i(\mu)$. This shows that $\mu$ is a Nash equilibrium.
  \end{proof}
\end{theorem}

The theorem gives us a way to verify that a given strategy configuration is in fact a Nash equilibrium. This is similar to problems in \textbf{NP}
in that, given a witness, we may verify in polynomial time that the
output is correct.

However, in contrast to problems in \textbf{NP}, computing Nash
equilibria is not truly a decision problem: by Nash's theorem, every
game has a mixed strategy equilibrium. Hence we need to phrase
things in terms of search problems.

\section{Search problems}

We are going to make precise the idea that, in order to solve
a problem, one is not only required to decide whether there exists
a solution, but rather to exhibit a witness. That is, we require a
machine that, given a problem instance, produces an output satisfying
a certain binary relation.

\begin{definition}
  A \emph{search problem} is given by a set of instances $X$,
  a set of candidates $Y$, and a relation $R\subseteq X\times Y$.
\end{definition}

One search problem is computing Nash equilibria:

\begin{definition}\label{def:nash}
  The problem of computing a Nash equilibrium of a $k$-player game,
  $k$\texttt{-Nash}, has as its instances $X$ the set of all normal
  form games, as its candidates $Y$ all strategy configurations, and
  as its defining relation
  \begin{align*}
    R = \left\lbrace{ (G,\mu) : \text{$\mu$ is a Nash equilibrium in $G$}}\right\rbrace\subseteq X\times Y.
  \end{align*}
\end{definition}

While the typical \textbf{NP}-complete decision problems have
associated search problems of similar difficulty, \todo{reference lectures}
we have seen that the converse does not hold for $k$\texttt{-Nash}. Hence
we require new complexity classes to study search problems:

\begin{definition}
  The class \textbf{FNP} contains all search problems $R\subseteq X\times Y$
  for which the characteristic function $\chi_R : X\times
  Y\to\left\lbrace{0,1}\right\rbrace$ is computable in polynomial time.
\end{definition}

Note that every game has a Nash equilibrium. However, for general $k$,
it need not be the case that the probabilities involved are rational.
While the complexity of finding such irrational equilibria may be
phrased in terms of approximation, we avoid such problems by focusing
on the two player case. Henceforth we write $\texttt{Nash}=2\texttt{-Nash}$.
\todo{introduce approximate nash equilibria}

Finally, we need to make precise what we mean by the size of a game.
As we will be concerned with whether or not and in what sense Nash
equilibria are computable in polynomial time, we do not need to be
particularly careful. Define the size of a $k$-player game $G$ to be
$|G|=\max_i|\Sigma_i|\leq n$.

\begin{corollary}
  $\texttt{Nash}\in\textbf{FNP}$.
  \begin{proof}
    Fix $G$. Suppose we are given a strategy configuration $\mu$.
    By \ref{thm:indifference}, it suffices to check that
    $EU_i(\mu) \geq EU_i(\sigma_i|\mu)$ for all players $i$ and all
    pure strategies $\sigma_i\in\Sigma_i$. Hence we may decide whether
    $\mu$ is a Nash equilibrium in $kn=O(n)$ checks.
  \end{proof}
\end{corollary}

In order to compare the complexities of different search problems,
we need to extend the notion of reductions from decision problems to
search problems:

\begin{definition}
  Let $P=(X,Y,R)$ and $P'=(X',Y',R')$ be search problems. We say
  $P$ reduces to $P'$ (in polynomial time) and write $P\leq_P P'$
  if there exist polynomial time computable functions $f:X\to X'$ and
  $g:X\times Y'\to Y$ such that
  \begin{align*}
    (f(x),y)\in R' \text{ if, and only if, } (x,g(x,y))\in R.
  \end{align*}
\end{definition}

\todo{explain why it is unlikely that nash is in P}

We have already noted that \textbf{FNP} contains some problems,
such as \texttt{FSAT} \todo{define this}, for which there may not be any solutions
as well as others, such as \texttt{Nash}, for which there is
guaranteed to be a solution.

To capture this difference, we have the following subclass
of \textbf{FNP}:

\begin{definition}
  The class \textbf{TFNP} contains all search problems in \textbf{FNP}
  that are guaranteed to have a solution. That is
  \begin{align*}
    \textbf{TFNP} = \left\lbrace{R \in \textbf{FNP} : \forall x\in X. \exists y\in Y. xRy}\right\rbrace
  \end{align*}
\end{definition}

\todo{introduce some \textbf{TFNP} problems that reduce to end of line}

\section{The End-of-Line problem and PPAD}

Let $V$ be a set of size $2^n$. We may use functions
$p,s:V\to V$ to define a directed acyclic graph $G(p,s)=(V,E)$ by
saying there is an edge $(u,v)\in E$ if, and only if, $p(v)=u$
and $s(u)=v$. A sink then is a vertex $u\in V$ such that,
for all $v\in V$, $u\neq p(v)$ or $v\neq s(u)$. Similarly,
a source is a vertex $v\in V$ such that, for all $u\in V$,
$u\neq p(v)$ or $v\neq s(u)$.

Moreover, if we have $V=\left\lbrace{0,1}\right\rbrace^n$,
any such $p$ and $s$ may be thought of a Boolean circuits.

\begin{definition}
  Let $P$ and $S$ be algorithms computing functions
  $\left\lbrace{0,1}\right\rbrace^n\to\left\lbrace{0,1}\right\rbrace^n$.
  Then the \emph{induced graph $G(P,S)$} has vertices $\left\lbrace{0,1}\right\rbrace^n$
  and edges
  \begin{align*}
    E = \left\lbrace{(u,v) : u,v\in V, u=P(v), v=S(u)}\right\rbrace.
  \end{align*}
\end{definition}

\begin{definition}\todo{make more readable}
  \texttt{EndOfLine} is the search problem $(X,Y,R)$ with
  \begin{itemize}
    \item $(P,S,t)\in X$ if, and only if,
      \begin{itemize}
        \item $P,S$ are polynomial time algorithms computing
          functions $\left\lbrace{0,1}\right\rbrace^n\to\left\lbrace{0,1}\right\rbrace^n$,
        \item $G(P,S)$ is a directed acyclic graph, and
        \item $t$ is a source in $G(P,S)$.
      \end{itemize}
    \item $Y = \left\lbrace{0,1}\right\rbrace^* = \bigcup_{n=1}^\infty \left\lbrace{0,1}\right\rbrace^n$, and
    \item $((P,S,s),y)\in R$ if, and only if, at least one of the followoing holds:
      \begin{itemize}
        \item $y$ is a source in $G(P,S)$ and $y\neq s$, or
        \item $y$ is a sink in $G(P,S)$.
      \end{itemize}
  \end{itemize}
\end{definition}

\begin{definition}
  The class \textbf{PPAD} consists of all problems in \textbf{FNP}
  that reduce to \texttt{EndOfLine}.
\end{definition}

We observe $\texttt{EndOfLine}\in\textbf{TFNP}$ so
$\textbf{PPAD}\subseteq\textbf{TFNP}\subseteq\textbf{FNP}$.
Over the next two chapters we will establish a few more members
of \textbf{PPAD} and eventually show that \texttt{Nash} is
\textbf{PPAD}-complete.

\section{Brouwer's fixed point theorem}

From topology, we have the following very important result:

\begin{theorem}[Brouwer]
  Every continuous function $[0,1]^n\to[0,1]^n$ has a fixed point.
\end{theorem}

This leads to the natural problem of computing such fixed points.
However, once again there are some problems we need to address.
The fixed points may be irrational. Hence we make the following
definition:

\begin{definition}
  Let $f:\left[{0,1}\right]^k \to \left[{0,1}\right]^k$ be continuous and
  $d\geq 0$. A $d$-fixed point is an $x\in\left[{0,1}\right]^k$ such that
  $|f(x)-x|\leq 2^{-d}$.
\end{definition}

Intuitively, $x$ is a $d$-fixed point of $f$ if $x$ and $f(x)$ agree on the
first $d$ bits. We will also require our functions to be reasonably
well-behaved.

\begin{definition}
  Let $K\geq 0$. A function $f:\left[{0,1}\right]^k\to\left[{0,1}\right]^k$ is \emph{$K$-Lipschitz} if, for all $x,x'\in\left[{0,1}\right]^k$,
  \begin{align*}
    |f(x)-f(x')| \leq K|x-x'|.
  \end{align*}
\end{definition}

We now show that Brouwer fixed points of Lipschitz functions may always be
approximated by multiples of $2^{-d}$.

\begin{lemma}
  Let $f:\left[{0,1}\right]^k\to\left[{0,1}\right]^k$ be a
  $K$-Lipschitz function and $d\geq 0$. Then there is a $\tilde x \in [0,1]^k$ such that $\tilde x$ is a multiple of $2^{-d'}$ and $\tilde x$
  is an $d$-fixed point of $f$ and $d'-d=O(\log(K))$.
  \begin{proof}
    Fix $d,K,f$. Let $x$ be a Brouwer fixed point of $f$.
    Choose
    \begin{align}\label{eq:d}
      d' = \lceil \log_2\left({kK}\right) + d + 1\rceil
    \end{align}
    and $\tilde x = \lfloor{2^{d'} x}\rfloor2^{-d'}$.
    Then
    \begin{align*}
      |\tilde x - x| \leq \sqrt{k} 2^{-d'} \leq 2^{\log_2 k-d'}.
    \end{align*}
    Using that $x$ is a fixed point and the triangle inequality,
    \begin{align*}
      |f(\tilde x)-\tilde x| \leq |f(\tilde x)-f(x)|+|\tilde x - x|
      \leq (K+1)2^{\log_2 k-d'} \leq K2^{\log_2 k+1-d'}\leq 2^{-d}.
    \end{align*}
    Thus $\tilde x$ is a $d$-fixed point.
  \end{proof}
\end{lemma}

The second issue we encounter when turning Brouwer's theorem into
a search problem is the representation of a continuous function
$\left[{0,1}\right]^k\to\left[{0,1}\right]^k$. In light of the result
above, it will be sufficient to consider functions which are
efficiently computable for multiples of $2^{-d}$ corresponding
to $\varepsilon$ and $K$.

Hence we may assume, without contradicting
any of the previous considerations, that $f$ is given using a
circuit of classical circuit of the following elementary gates:

\begin{itemize}
  \item $(x,y)\mapsto xy$,
  \item $(x,y)\mapsto x + y$,
  \item $x\mapsto x/2$,
  \item $(x,y)\mapsto \begin{cases}
      0 & \text{if }x>y \\
      1 & \text{if }x\leq y.
    \end{cases}$
\end{itemize}

\begin{definition}
  Let $f:\left[{0,1}\right]^k\to\left[{0,1}\right]^k$ be a
  $K$-Lipschitz function. A family of circuits $\{C_d\}$ \emph{efficiently computes} $f$ if
  \begin{enumerate}

    \item for all integer multiples $\tilde x\in\left[{0,1}\right]^k$ of $2^{-d}$, $|C_d(\tilde x) - f(\tilde x)|\leq 2^{-d}$,
    \item if each $C_d$ may be constructed in time
      polynomial in $d$.
  \end{enumerate}
\end{definition}

Note that our definition immediately implies that such $C_d$ contain
at most $O(\text{poly}(d))$ gates and hence, given the algorithm to
construct $C_d$, computing $f(\tilde x)$ up to an error $2^{-d}$ is
$O(\text{poly}(d))$ as a whole.

\begin{definition}\todo{polish this, too many $d$'s}
  Denote by $\texttt{Brouwer}$ the search problem defined as follows:
  \begin{itemize}
    \item an instance consists of integers $K,d$ and a family of
      circuits $\left\lbrace{C_d}\right\rbrace$ that
      efficiently computes a $K$-Lipschitz function $f:\left[{0,1}\right]^k\to\left[{0,1}\right]^k$,
    \item a candidate is any integer multiple $\tilde x\in\left[{0,1}\right]^k$ of $2^{-d}$ for any positive integer $d$ and $k$,
    \item a candidate is a solution if, and only if, it is a
      $d$-fixed point of $f$. \todo{this is not quite right}
  \end{itemize}
\end{definition}

\section{\texttt{Brouwer} is PPAD complete}

\begin{theorem} \todo{generalise to $k$}
  $\texttt{Brouwer}\leq_P\texttt{EndOfLine}$.
  \begin{proof}
    Fix an instance $(\varepsilon,K,A_f)$ of $\texttt{Brouwer}$.
    Triangulate $[0,1]^2$ in such a way that each triangle is
    contained in a disk of diameter $\leq\varepsilon$. This is
    possible by dividing $[0,1]^2$ into $O(K/\varepsilon)=O(2^n)$ squares
    and then splitting each square into two triangles.
    Colour each
    vertex $x$ in the triangulation based on the direction of
    $(f(x)-x)/|f(x)-x|$ by assigning green, blue, and red to $[0,\pi/2)$, $[\pi/2,5\pi/4)$, and $[5\pi/4,2\pi)$, respectively.
    Should the angle be undefined because $f(x)-x=0$ then we have
    found a fixed point.
    We extend the triangulation by adding a column of red vertices on the
    left where we colour the bottom most vertex green and all the others
    red.

    We then construct a directed graph $G$ by taking the vertices to be the triangles
    and draw an edge between adjacent triangles such that there is
    a red vertex on the left and a green vertex on the right.
    It is
    easy to see that the resulting graph contains only cycles
    and paths from source to target. Note that the triangle $t_0$ where
    the unique change from red to green occurs on the left is not
    trichromatic. This is the case because points on the left
    boundary may not be moved further to the left, hence they cannot
    be coloured blue. Moreover, there is an outgoing edge of $t_0$
    which cannot be part of a cycle. Thus it must lead to
    a sink, i.e. a trichromatic triangle.

    By construction, we have a sink $t_0$. Further, given a triangle one
    may efficiently check the colours of the nodes by using $A_f$
    and then find the neighbouring triangles by reflecting the
    remaining vertex.
    Hence we have polynomial time algorithms to compute the neighbour(s)
    of any triangle. This gives us an instance of \texttt{EndOfLine} with the solution yielding
    a trichromatic triangle.
    It is now straightforward to argue
    that, because the distances between the nodes of the trichromatic
    triangle are small and because each node is mapped in a different
    direction, each node must be an approximate fixed point of $f$.
    This finishes the reduction.
  \end{proof}
\end{theorem}

\begin{theorem}
  $\texttt{EndOfLine}\leq_P\texttt{Brouwer}$.
  \begin{proof}
    \missingproof
  \end{proof}
\end{theorem}


\section{\texttt{Nash} is PPAD-complete}

Now that we have established that \texttt{Brouwer} is PPAD complete,
we need only show that \texttt{Nash} is equivalent to \texttt{Bouwer}.
This indirect way of showing that \texttt{Nash} is equivalent to
\texttt{EndOfLine} is an obvious thing to do because the classical
proof of Nash's theorem uses Bouwer's fixed point theorem and hence
almost gives us the reduction for free:

\begin{theorem}
  $\texttt{Nash}\leq_P\texttt{Brouwer}$.
  \begin{proof}[Sketch of the proof]
    When using Brouwer's theorem to prove Nash's theorem one
    constructs a function $b$ on strategy configurations that
    moves each players strategy towards the best response.
    (see e.g. \cite{karsten2017})
    If one identifies the space of mixed strategies with a suitable
    $\left[{0,1}\right]^n$ then $b$ is continuous and hence has a
    fixed points. Fixed points correspond to Nash equilibria.

    While this proves Nash's theorem, some work is required to
    yield the reduction:
    \begin{enumerate}
      \item check that $b:\left[{0,1}\right]^n\to\left[{0,1}\right]^n$
        is Lipschitz,
      \item check that $b$ is computable in polynomial time,
      \item check that $\varepsilon$-fixed points correspond to
        $\varepsilon$-Nash equilibria,
      \item check that the \texttt{Brouwer} has size polynomial
        in the size of the original \texttt{Nash} instance.
    \end{enumerate}
    These checks turn out to be routine. Depending on the exact
    definition of the function $b$ one may be requried to do some
    scaling to be able to use the same $\varepsilon$ for both
    instances.
  \end{proof}
\end{theorem}

The other direction is far less obvious. Our goal now is to
construct a game whose Nash equilibria correspond to fixed points of
a function $f:\left[{0,1}\right]^k\to\left[{0,1}\right]^k$.
More precisely, we are given a family of circuits $\left\lbrace{C_d}\right\rbrace$ that efficiently compute $f$ and we want to construct a
family of games $\left\lbrace{G_d}\right\rbrace$ such that
the $d$-fixed points of $C_d$ correspond to the $d$-Nash equilibria of $G_d$.

We begin by constructing such games for each of the elementary gates
in $C_d$:
\begin{itemize}
  \item $G^\times$ has four players $x,y,z,w$, each of whom have two pure strategies
    $S$ and $G$.
    Define the following utilities:
    \begin{align*}
      U_w(\sigma_x,\sigma_y,\sigma_z, S_w) = \begin{cases}
        1 & \text{if } \sigma_x=\sigma_y= G \\
        0 & \text{otherwise}
      \end{cases}
    \end{align*}
  \item $G^+$ has four players $x,y,z,w$, each of whom have two pure strategies
    $S$ and $G$.
    Define the following utilities:
    \begin{align*}
      U_w(\sigma_x,\sigma_y,\sigma_z, S_w) = |\left\lbrace{p\in\left\lbrace{x,y}\right\rbrace : \sigma_p = G}\right\rbrace|,
    \end{align*}
  \item $G^{/2}$ has three players $x,z,w$, each of whom have two pure strategies
    $S$ and $G$.
    Define the following utilities:
    \begin{align*}
      U_w(\sigma_x,\sigma_z, S_w) = \begin{cases}
        2 & \text{if } \sigma_x= G \\
        1 & \text{otherwise}
      \end{cases}
    \end{align*}
  \item $G^<$ has four players $x,y,z,w$ each of whom have two pure strategies $S$ and $G$. Define the following utilities:
    \begin{align*}
      U_w(\sigma_x,\sigma_y,\sigma_z,S_w) = \begin{cases}
        1 & \text{if }\sigma_x = S,\sigma_y = G\\
        0 & \text{otherwise}
      \end{cases}, \hspace{1cm}
      U_w(\sigma_x,\sigma_y,\sigma_z,G_w) = \begin{cases}
        1 & \text{if }\sigma_x = G,\sigma_y = S\\
        0 & \text{otherwise}
      \end{cases}
    \end{align*}
\end{itemize}
Moreover, for each of the games we define
\begin{align*}
  U_w(\sigma_x,\sigma_y,\sigma_z, G_w) = \begin{cases}
    1 & \text{if } \sigma_z= G \\
    0 & \text{otherwise}
  \end{cases}
  , \hspace{1cm}
  U_z(\sigma_x,\sigma_y,\sigma_z,\sigma_w) = \begin{cases}
    1 & \text{if }\sigma_z\neq\sigma_w\\
    0 & \text{otherwise}
  \end{cases}
\end{align*}

It is now straightfoward to see that in the Nash
equilibria $\mu$ of $G^\times$, $G^+$ and $G^{/2}$ we must have $\mu_z =
\mu_x\mu_y$, $\mu_z = \mu_x + \mu_y$, and $\mu_z = \mu_x / 2$, respectively,
where we write $\mu_p = \prc{G_p}{\mu}$. Note also that if we have
players $x,t,y$ where we pay $t$ to play the opposite of $x$ and
$y$ to play the opposite of $t$, then $\mu_x=\mu_y$. Using this
construction we may compose games to achieve more complicated
computations.

However, there is a problem: while for $G^<$ we have
$\mu_z = 0$ if $\mu_x<\mu_y$ and $\mu_z = 1$ if $\mu_x>\mu_y$,
$\mu_z$ could take any value in the case where $\mu_x=\mu_y$.
Moreover, there is the following result that says that constructing
a better game to handle comparisons is hopeless:

\begin{lemma}
  Consider a $k$ player game where each player has exactly
  two pure strategies $S$ and $G$. Then there is no triple $x,y,z$ of
  distinct players such that
  \begin{enumerate}
    \item for all $p,q\in[0,1]$, there is a Nash equilibrium
      $\mu$ with $\mu_x = p$ and $\mu_y = q$, and
    \item every Nash equilibrium $\mu$ satisfies
      \begin{align}\label{eq:good_comparator}
        \mu_z = \begin{cases}
          1 & \text{if }\mu_x < \mu_y \\
          0 & \text{otherwise}
        \end{cases}.
      \end{align}
  \end{enumerate}
  \begin{proof}
    Suppose there was such a game. We extend it as follows:
    \begin{enumerate}
      \item add a new player $t$ who receives a pay-off if, and only if,
        she plays opposite to $x$,
      \item player $x$ receives a pay-off if, and only if, she plays
        opposite to $t$,
      \item player $y$ receives a pay-off if, and only if, she plays
        opposite to $z$.
    \end{enumerate}
    Note that we did not change the pay-offs for $z$. Hence any
    Nash equilibrium must satisfy (\ref{eq:good_comparator}).
    However, our extensions force $\mu_x = \mu_t = 1/2$
    and $\mu_y = 1-\mu_z$. Thus we obtain the unsatisfiable condition
    \begin{align*}
      \mu_z = \begin{cases}
        1 & \text{if }1/2 > \mu_z \\
        0 & \text{otherwise}
      \end{cases}.
    \end{align*}
    This means there cannot be a Nash equilibrium in this new game,
    contradicting Nash's theorem.
  \end{proof}
\end{lemma}

We are going to solve this problem by evaluating the circuit
for several points close to our point of interest and averaging the
result. The Lipschitz property of the function will then allow
us to bound the error. Hence we obtain an approximate Nash equilibrium.

\begin{theorem}
  $\texttt{Brouwer}\leq_P\texttt{Nash}$.
  \begin{proof}
    \missingproof
  \end{proof}
\end{theorem}

\printbibliography

\end{document}
