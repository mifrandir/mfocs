\documentclass[12pt]{ociamthesis}  % default square logo
%\documentclass[12pt,beltcrest]{ociamthesis} % use old belt crest logo
%\documentclass[12pt,shieldcrest]{ociamthesis} % use older shield crest logo

\usepackage{dissertation}

%input macros (i.e. write your own macros file called mymacros.tex
%and uncomment the next line)
%\include{mymacros}

\title{Moduli Spaces of\\[1ex]Holomorphic Bundles}   %note \\[1ex] is a line break in the title

\author{Franz Miltz}             %your name
\college{Lady Margaret Hall}  %your college

%\renewcommand{\submittedtext}{change the default text here if needed}
\degree{Master of Science}     %the degree
\degreedate{2024}         %the degree date

\addbibresource{references.bib}

%end the preamble and start the document
\begin{document}

%this baselineskip gives sufficient line spacing for an examiner to easily
%markup the thesis with comments
\baselineskip=18pt plus1pt

%set the number of sectioning levels that get number and appear in the contents
\setcounter{secnumdepth}{3}
\setcounter{tocdepth}{3}


\maketitle                  % create a title page from the preamble info
\begin{dedication}
  To somebody
\end{dedication}
\begin{acknowledgements}

\end{acknowledgements}
\begin{abstract}

\end{abstract}

\begin{romanpages}          % start roman page numbering
  \tableofcontents            % generate and include a table of contents
\end{romanpages}            % end roman page numbering

\chapter{Introduction}

\begin{itemize}
  \item Riemann surfaces
  \item complex geometry at the intersection between the analytic and algebraic settings
\end{itemize}

\chapter{As Manifolds}

Consider the set of holomorphic bundles on a compact Riemann surface,
up to isomorphism. What can we say about this set? What additional
structure does it have? How does this structure change if we equip
the bundles with additional data?
In this chapter we are going to construct complex manifolds whose
points correspond to equivalence classes of holomorphic bundles
and Higgs bundles, a type of decorated holomorphic bundle. These
manifolds are referred to as moduli spaces. While it
is possible to construct topological spaces representing all such bundles,
the result is non-Hausdorff and hence does not provide us with a
satisfying answer. The majority of the chapter is going to be spent
on rectifying this issue.

The key observation will be to ignore certain `unstable' bundles. We
are going to find that doing so does not lose much information but
allows for the construction of the moduli spaces. Although unstable
bundles are the main obstacle, removing them does not trivialise the
problem. To obtain the moduli spaces, we have to consider quotients
of infinite dimensional manifolds by group actions. Such considerations
require us to introduce the theory of Banach manifolds.

The first part of this chapter is going to be spent on the kinds of
objects that we aim to classify, i.e. holomorphic bundles and Higgs
bundles on compact Riemann surfaces. We recall some facts and establish
notation. The next to sections are going to correspond to the two
main steps in obtaining a complex manifold structure on the topological
space of equivalence classes of holomorphic bundles and Higgs bundles,
respectively. Firstly, we construct a Banach manifolds of stable
bundles. Secondly, we take a quotient with respect to the appropriate
group action. Towards the end of the chapter we establish some
further facts about the moduli spaces and how they relate to each other.

\paragraph*{Notation}

Before we begin, let us establish some notation. Unless otherwise
indicated,

\begin{itemize}
  \item all vector spaces are complex, maps of vector spaces are
        $\mathbb{C}$-linear, and tensor products of vector spaces are
        over $\mathbb{C}$;
  \item all manifolds are smooth;
  \item dimensions are taken over $\mathbb{C}$;
  \item the word `bundle' refers to a vector bundle.
\end{itemize}

\section{Holomorphic Bundles on Compact Riemann Surfaces}

Holomorphic maps are smooth. Hence, holomorphic vector bundles
are smooth vector bundles. In contrast to maps, in the case of vector
bundles the word `holomorphic' does not refer to a property, but
a structure. In particular, there may be many holomorphic bundles with
the same underlying smooth bundle.

\subsection{Complex Vector Bundles}

Recall the definition of a smooth bundle on a complex
manifold:

\begin{definition}
  Let $X$ be a complex manifold of dimension $n$. A
  \emph{smooth vector bundle} $E$ of rank $r$ on $X$ is a smooth map
  $\pi : E\to X$ such that there
  is an open cover $X = \bigcup_i U_i$ and diffeomorphisms
  \begin{equation}\label{eq:smooth_trivialisation}
    \phi_i : {\pi}^{-1}U_i \cong \mathbb{C}^r \times U_i
  \end{equation}
  such that, for every chart $U\subseteq U_i\cap U_j$, the
  map $\phi_i \circ {\phi_j}^{-1}$ restricts to a linear
  automorphism of $\mathbb{C}^r\times U \cong \mathbb{C}^{r+n}$.
\end{definition}

\begin{example}
  A complex manifold $X$ of dimension $n$ may be regarded as a
  real manifold of dimension $2n$. Any smooth real vector bundle
  $E$ on $X$ corresponds to a \emph{complexified} smooth bundle
  $E_{\mathbb{C}} := E\otimes_{\mathbb{R}} \mathbb{C}$ of equal rank.
  In particular, we may complexify the real vector bundles
  $TX$ and $T^*X$ to obtain smooth vector bundles
  $T_{\mathbb{C}} X$ and $T_{\mathbb{C}}^*X$
  on $X$ of rank
  \begin{align*}
    \rank(T_\mathbb{C} X)
    = \rank_{\mathbb{R}} (TX) = 2n.
  \end{align*}
\end{example}

Now a holomorphic bundle is a smooth bundle where the trivialisations
(\ref{eq:smooth_trivialisation}) are biholomorphic. As holomorphic
maps are smooth, each holomorphic bundle has an underlying smooth
structure.

\begin{example}
  smooth bundle with multiple holomorphic structures
  \missingexample
\end{example}

\begin{example}
  Recall that a complex manifold $X$ of dimension $n$ induces a
  complex structure on the real smooth rank $2n$ vector bundle $TX$.
  That is, there is a map of smooth real vector bundles
  $J : TX \to TX$ such that $J^2 = -1$. Note that this extends to
  an automorphism of the complexification $T_{\mathbb{C}}X$.
  This defines a splitting into subbundles
  \begin{align}\label{eq:tangent_decomposition}
    T_{\mathbb{C}} X = T_{1,0}X \oplus T_{0,1}X
  \end{align}
  whose fibres are eigenspaces of $i$ and $-i$, respectively. The
  complex bundle $T_{1,0} X$ of rank $n$ is called the
  \emph{holomorphic tangent bundle} of $X$. Indeed, if we regard $TX$
  as a complex vector bundle via $J$ then the map
  \begin{align}\label{eq:holomorphic_tangent_bundle}
    TX
    \longrightarrow T_{\mathbb{C}} X
    = T_{1,0}X \oplus T_{0,1}X
    \longrightarrow T_{1,0}X
  \end{align}
  is a $\mathbb{C}$-linear isomorphism and hence induces a holomorphic
  structure on $TX$, justifying the name.
\end{example}

One is usually interested in what kinds of sections a vector bundle
admits. Indeed, in the next chapter we are going to make extensive use
of the fact that a vector bundle is uniquely determined by its sections.
In the complex case, there are a few definitions:

\begin{definition}
  \begin{itemize}
    \item If $E\to X$ is a smooth vector bundle then $\Gamma^\infty(U,E)$
          denotes the vector space of smooth sections on $U$.
    \item If $\mathcal E\to X$ is holomorphic then $\Gamma(U,\mathcal E)$
          denotes the vector space of holomorphic sections on $U$.
  \end{itemize}
\end{definition}

Note that if $\mathcal E$ is holomoprhic then it is also smooth.
However, there are some important differences between
$\Gamma^\infty(U,\mathcal E)$ and $\Gamma(U,\mathcal E)$. While both are
complex vector spaces $\Gamma^\infty(U,E)$ is typically infinite
dimensional while $\Gamma(U,E)$ is always finite
dimensional.~\cite[Theorem 1.4.1]{ma2007}

\subsection{Dolbeault Cohomology}
\missingcitation

Consider a complex manifold $X$ of dimension $n$. Recall the de Rahm
complex
\begin{align*}
  \cdots \xlongrightarrow{d}
  \Omega_{\mathbb{R}}^{k-1}(X)\xlongrightarrow{d}
  \Omega_{\mathbb{R}}^{k}(X)\xlongrightarrow{d}
  \Omega_{\mathbb{R}}^{k+1}(X)\xlongrightarrow{d}
  \cdots
\end{align*}
of the underlying real manifold where
$\Omega^0_{\mathbb{R}}(X) = C^\infty(X,\mathbb{R})$ and,
for $k > 0$,
\begin{align*}
  \Omega^k_{\mathbb{R}}(X) := \Gamma^\infty(X,\Lambda^k T^*X).
\end{align*}
We may complexify this to obtain
$\Omega^k(X) := \Omega^k(X)\otimes_{\mathbb{R}} \mathbb{C}$.
The splitting $T^*_\mathbb{C} X = T^*_{1,0}X \oplus T^*_{0,1}X$
leads us to define
\begin{align*}
  \Omega^{p,q}(X)
  := \Gamma^\infty(X,\Lambda^p T^*_{1,0}X \wedge\Lambda^q T^*_{0,1}X)
\end{align*}
and hence we have
\begin{align*}
  \Omega^j(X) = \bigoplus_{p+q=k} \Omega^{p,q}(X).
\end{align*}
Moreover, the operator $d : \Omega^{p,q}(X) \to \Omega^{p+q+1}(X)$
splits into $d = \partial + \dol$ where
\begin{align*}
  \partial : \Omega^{p,q}(X) \to \Omega^{p+1,q}(X),\hspace{1cm}
  \dol : \Omega^{p,q}(X) \to \Omega^{p,q+1}(X).
\end{align*}
Locally, these are given by
\begin{align*}
  \partial (f dz_I \wedge d\bar z_J) = \sum_{i=1}^n \frac{\partial f}{\partial z_i} dz_i\wedge dz_I \wedge d\bar z_J, \\
  \dol (f dz_I \wedge d\bar z_J) = \sum_{j=1}^n \frac{\partial f}{\partial \bar z_j} d\bar z_j\wedge dz_I \wedge d\bar z_J.
\end{align*}
In particular, we now have a family of complexes
\begin{align}\label{eq:dolbeault_complex}
  \cdots \xlongrightarrow{\dol}
  \Omega^{p,q-1}(X)\xlongrightarrow{\dol}
  \Omega^{p,q}(X)\xlongrightarrow{\dol}
  \Omega^{p,q+1}(X)\xlongrightarrow{\dol}
  \cdots
\end{align}
\begin{definition}
  The cohomology of the complex \ref{eq:dolbeault_complex} is
  called \emph{Dolbeault cohomology}. To be precise, the
  $(p,q)$ Dolbeault cohomology group is
  \begin{align*}
    H^{p,q}(X) := \frac{
      \ker(\dol : \Omega^{p,q}(X) \to \Omega^{p,q+1}(X))
    }{
      \im(\dol : \Omega^{p,q-1}(X) \to \Omega^{p,q}(X))
    }
  \end{align*}
\end{definition}

\subsection{Dolbeault Cohomology of Holomorphic Bundles}

\todo{explain how holomorphic bundles give rise to $\dol$-operators}
For a smooth bundle $E$ of rank $r$, consider the spaces of
$E$-valued $k$-forms and $(p,q)$-forms, respectively,
\begin{align*}
  \Omega^k(X,E) := \Omega^k(X)\otimes\Gamma^\infty(X,E),\hspace{1cm}
  \Omega^{p,q}(X,E) := \Omega^{p,q}(X)\otimes\Gamma^\infty(X,E).
\end{align*}
If we are given a holomorphic bundle $\mathcal E$ with underlying
smooth bundle $E$
it is straightforward to verify that the operator $\dol$ extends
to an operator
\begin{align}\label{eq:general_dolbeault_operator}
  \dol_E : \Omega^{p,q}(X,E) \to \Omega^{p,q+1}(X,E)
\end{align}
by considering local trivialisations where we have
\begin{align*}
  \dol_E \left({\sum_{i=1}^k \xi_i \otimes s_i}\right)
  = \sum_{i=1}^k \dol \xi_i \otimes s_i.
\end{align*}
Moreover, the operator $\dol_E$ satisfies
\begin{align*}
  \dol_E(\xi \otimes s) = \dol\xi \otimes s + (-1)^{p+q} \xi \wedge \dol_E s.
\end{align*}
Hence it is uniquely determined by the component
$\Omega^0(X,\mathcal E) \to \Omega^{0,1}(X,\mathcal E)$. Indeed,
any linear map $\alpha : \Omega^0(X,E) \to \Omega^{0,1}(X,E)$ may be
extended to a map $\Omega^{p,q}(X,E) \to \Omega^{p,q+1}(X,E)$
via
\begin{align*}
  \alpha(\xi \otimes s) = \dol\xi \otimes s + (-1)^{p+q}\xi\wedge \alpha(s).
\end{align*}
This leads us to the following definition:

\begin{definition}
  A \emph{Dolbeault operator} on a smooth bundle $E$ on a complex
  manifold $X$ is a linear operator
  \begin{align*}
    \dol_E : \Omega^0(X,E) \to \Omega^{0,1}(X,E)
  \end{align*}
  such that $\dol_E^2 = 0$ and, for all $s\in\Omega^0(X,E)$ and
  $f\in \Omega^0(X)$,
  \begin{align*}
    \dol_E (fs) = \dol f\otimes s + f \dol_E s.
  \end{align*}
\end{definition}

The process outlined above is reversible. That is, a holomorphic
bundle is just a smooth bundle with a Dolbeault operator. To make
this precise, we need to say what morphisms of such operators are.
Consider a map $f : E \to F$ of smooth bundles on a fixed complex
manifold $X$. This induces a map
$f_*:\Gamma^\infty(X,E)\to\Gamma^\infty(X,F)$
given by $s \mapsto f\circ s$. Hence $f$ is a map of Dolbeault operators
if, and only if, it commutes with the operators:

\begin{definition}
  A \emph{morphism of Dolbeault operators} $f : \dol_E \to \dol_F$
  is a map of smooth bundles $f : E\to F$ such that the following commutes:
  \begin{equation*}
    % https://q.uiver.app/#q=WzAsNCxbMCwwLCJcXE9tZWdhXjAoWCxFKSJdLFswLDEsIlxcT21lZ2FeMChYLEYpIl0sWzIsMSwiXFxPbWVnYV57MCwxfShYLEYpIl0sWzIsMCwiXFxPbWVnYV57MCwxfShYLEUpIl0sWzAsMywiXFxiYXJcXHBhcnRpYWxfRSJdLFsxLDIsIlxcYmFyXFxwYXJ0aWFsX0YiXSxbMCwxLCJmXyoiLDJdLFszLDIsImZfKiIsMl1d
    \begin{tikzcd}
      {\Omega^0(X,E)} && {\Omega^{0,1}(X,E)} \\
      {\Omega^0(X,F)} && {\Omega^{0,1}(X,F)}
      \arrow["{\bar\partial_E}", from=1-1, to=1-3]
      \arrow["{f_*}"', from=1-1, to=2-1]
      \arrow["{f_*}"', from=1-3, to=2-3]
      \arrow["{\bar\partial_F}", from=2-1, to=2-3]
    \end{tikzcd}
  \end{equation*}
\end{definition}

\begin{theorem}
  Fix a complex manifold $X$. There is an equivalence of categories
  between holomorphic bundles on $X$ and smooth bundles with corresponding
  Dolbeault operators.
  \begin{proof}
    This is essentially \cite[Theorem 3.2]{moroianu2004}. One only
    needs to verify that a map of smooth bundles is a map of
    holomorphic bundles if, and only if, it respects the holomorphic
    structure. \todo{maybe we should do this proof?}
  \end{proof}
\end{theorem}

Thus we are justified in referring to a pair $(E,\dol_E)$ as a
holomorphic vector bundle.
Note that the Dolbeault operator induced by a holomorphic bundle is
unique only up to isomorphism. Fortunately, taking cohomologies
is functorial and thus we obtain the Dolbeault cohomology of
holomorphic vector bundles:

\begin{definition}
  Let $(E,\dol_E)$ be a holomorphic vector bundle on $X$. The
  \emph{$(p,q)$ Dolbeault cohomology group of $X$ with coefficients in
    $E$} is
  \begin{align*}
    H^{p,q}(X,(E,\dol_E)) := \frac{
      \ker(\dol_E : \Omega^{p,q}(X,E) \to \Omega^{p,q+1}(X,E))
    }{
      \im(\dol_E : \Omega^{p,q-1}(X,E) \to \Omega^{p,q}(X,E))
    }.
  \end{align*}
\end{definition}

\subsection{Higgs Bundles}

\todo{write about why Higgs bundles exist, what they're good for, yadda
  yadda}

\begin{definition}
  The \emph{canonical bundle} $\mathcal K_X$ on a complex manifold $X$
  of dimension $n$ is the holomorphic vector bundle defined as
  \begin{align*}
    \mathcal K_X := \det T^*_{1,0} X := \wedge^n T^*_{1,0} X.
  \end{align*}
\end{definition}

Note that the canonical bundle is a line bundle on every $X$.
\todo{explain / cite}

\begin{definition}
  Let $(E,\dol_E)$ be a holomorphic bundle on a complex manifold $X$.
  A \emph{Higgs field} on $(E,\dol_E)$ is a holomorphic section
  $\phi \in \Omega^{1,0}(X,\End E)$ satisfying $\phi\wedge\phi = 0$.
  A \emph{Higgs bundle} is a holomorphic bundle equipped with
  a Higgs field.
\end{definition}
We immediately observe that, in the case where $X=C$ is a compact
Riemann surface, the condition $\phi\wedge\phi=0$ is trivially
satisfied. However, if one is to consider moduli spaces of Higgs bundles
over more general base spaces then this condition is quite important.

Consider Higgs bundles $(E,\dol_E,\phi_E)$ and $(F,\dol_F,\phi_F)$
and a map of smooth bundles $f:E\to F$. We say $f$ is a map of Higgs
bundles if, and only if, it is a map of holomorphic bundles
$f:(E,\dol_E)\to (F,\dol_F)$ and $\phi_E = f^*\phi_F$.
This means in particular that, if
$(E,\dol_E,\phi_E)\cong(F,\dol_F,\phi_F)$
are isomorphic Higgs bundles then we have isomorphic holomorphic bundles
$(E,\dol_E)\cong(F,\dol_F)$ and hence isomorphic smooth bundles
$E\cong F$.

\section{Spaces of Holomorphic Bundles}

After a brief introduction to holomorphic bundles on Riemann surfaces,
we are now ready to contemplate the spaces that such bundles might
form. The first observation we are going to make is that the case of
smooth bundles is very straightforward. To do this, we are going to
introduce some topological invariants of smooth bundles.

This allows us to construct a topological space of holomorphic structures
on a fixed smooth bundle. Unfortunately, after taking the quotient
of said space by isomorphism of holomorphic structures, we obtain a
non-Hausdorff topological space.

\subsection{Chern Classes}

We noticed that in order for two holomorphic bundles to be isomorphic
they must have the same underlying smooth bundles. Hence it is worth
investigating when smooth bundles are isomorphic. Fix a complex
manifold $X$.

The first observation we make is that, if we have isomorphic smooth
bundles $E\cong E'$ then $\rank E = \rank E'$. However, the converse
is not true. To see this it suffices to consider any non-trivial
bundle and compare it to the trivial bundle of equal rank. E.g.
$TX$ on $S^2$ will do.

This leads us to consider another invariant: Chern classes. For
every smooth bundles $E$ on $X$ of rank $r$ there are elements
$c_j(E)\in H^{2j}(X,\mathbb{Z})$ for $j\geq 0$ called the Chern
classes of $E$. There are different ways of constructing these classes,
e.g. \cite{fine2013} and \cite{griffiths1994}, for our purposes
it is sufficient to know a few key properties:

\begin{lemma}\missingcitation
  Let $E$ be a smooth bundle on $X$ of rank $r$. Then
  \begin{enumerate}
    \item $c_j(E) = 0$ whenever $j<1$ or $j>r$.
    \item If $F$ is a smooth bundle on $X$ such that $E\cong F$
          then $c_i(E) = c_i(F)$ for all $i$.
    \item For every other smooth bundle $F$ on $X$,
          \begin{align*}
            c_i(E\oplus F) = \bigoplus_{j=0}^{i} c_j(E)\cup c_{i-j}(F).
          \end{align*}
  \end{enumerate}
\end{lemma}

\subsection{Degree}

Our goal is to classify holomorphic bundles on a compact Riemann surface,
i.e. a compact connected complex manifold of dimension 1. Hence we
ought to consider this case more closely. To this end, fix a compact
Riemann surface $C$.

\todo{genus}

Note that the real manifold of dimension $2$ underlying $C$ is orientable. \missingcitation
Hence the cohomology group $H^2(X,\mathbb{Z})$ is generated by the
fundamental class $[C]$.

\begin{definition}
  The \emph{degree} of a smooth bundle $E$ on $C$ is the integer
  \begin{align*}
    \deg E = c_1(E)\cdot[C].
  \end{align*}
\end{definition}

\begin{theorem}
  Smooth bundles on $C$ are isomorphic if, and only if, they have
  equal rank and degree.
\end{theorem}

\missingsection



\subsection{The Naive Quotients}

Having established holomorphic bundles on compact Riemann surfaces,
we are ready to contemplate the spaces that they form. While it is
quite straightforward to obtain topological spaces of equivalence
classes of holomorphic bundles, it turns out that neither is Hausdorff.

In light of previous results, fix a smooth bundle $E$ of rank $r$
and degree $d$ on a compact Riemann surface $C$.
Recall that a holomorphic bundle with underlying smooth structure $E$
is given by a Dolbeault operator $\dol_E$. Hence we have a set of
\begin{align*}
  A^{\dol} (E) := \left\lbrace{
    \text{Dolbeault operators $\dol_E : \Omega^0(C,E) \to \Omega^{0,1}(C,E)$}
  }\right\rbrace.
\end{align*}
This has an obvious topology given by the following observations:
\begin{lemma}\label{lem:affine_space_of_dolbeault_operators}
  Consider linear maps
  \begin{align*}
    \dol_E,\alpha : \Omega^0(C,E) \to \Omega^{0,1}(C,E).
  \end{align*}
  Then $\dol_E\in A^{\dol} (E)$ if, and only if, $\dol_E+\alpha\in A^{\dol} (E)$.
  \begin{proof}
    We calculate in local coordinates:
    \begin{align*}
      (\dol_E + \alpha)^2(s_i)
       & = \sum_{j,k} (\dol_{ijk} + \alpha_{ijk})(\dol_E + \alpha)( d\bar z_j\otimes s_k)                                                                                     \\
       & = \sum_{j,k} (\dol_{ijk} + \alpha_{ijk})(\dol(d\bar z_j) \otimes s_k -  d\bar z_j \wedge \sum_{j',k'}(\dol_{k,j',k'} + \alpha_{k,j',k'}) d\bar z_{j'}\otimes s_{k'}) \\
       & = - \sum_{j,j',k,k'} (\dol_{ijk} + \alpha_{ijk})(\dol_{k,j',k'} + \alpha_{k,j',k'}) d\bar z_j \wedge d\bar z_{j'}\otimes s_{k'}                                      \\
       & = - \sum_{j,j'}
      d\bar z_j \wedge d\bar z_{j'}\otimes \left(
      \sum_{k,k'} (\dol_{ijk} + \alpha_{ijk})(\dol_{k,j',k'} + \alpha_{k,j',k'}) s_{k}\right)                                                                                 \\
       & = 0
    \end{align*}
    Moreover, for $f\in C^\infty(C)$ and $s\in\Omega^0(C,E)$,
    \begin{align*}
      (\dol_E + \alpha)(fs) = f\dol_E(s) + f\alpha(s) + \dol f \otimes s
      = f(\dol_E+\alpha)(s) + \dol f \otimes s.
    \end{align*}
    Hence $\dol_E + \alpha$ satisfies both properties of Dolbeault
    operators.
  \end{proof}
\end{lemma}

Thus we may write $A^{\dol} (E) = \dol_E + \Omega^{0,1}(C,\End E)$
for some fixed $\dol_E$. This induces a topology on $A^{\dol} (E)$.
In particular, $A^{\dol} (E)$ is an affine space modelled on
$\Omega^{0,1}(C,\End E)$.

This allows us to define the space of all Higgs bundles
\begin{align*}
  A^\phi(E) := \bigsqcup_{\dol_E \in A^{\dol} (E)} \Gamma(C,\End(E,\dol_E)\otimes K_C)
\end{align*}
whose topology is induced by the topologies on
$\Gamma(C,\End(E,\dol_E)\otimes K_C)$ and $A^{\dol} (E)$.
All that is left to do is take a quotient by equivalence of holomorphic
bundles, respectively Higgs bundles. One of the major topics of this
paper is to explore various ways in which we may take quotients by
group actions. Thus we better start phrasing everything in this language.
In this instance we are interested in taking the action of $f\in\Aut(E)$
on $\dol_E$ via
\begin{equation}
  % https://q.uiver.app/#q=WzAsNCxbMCwwLCJcXE9tZWdhXjAoWCxFKSJdLFswLDEsIlxcT21lZ2FeMChYLEUpIl0sWzIsMCwiXFxPbWVnYV57MCwxfShYLEUpIl0sWzIsMSwiXFxPbWVnYV57MCwxfShYLEUpIl0sWzEsMywiXFxiYXJcXHBhcnRpYWxfRSIsMl0sWzAsMSwiZl8qXnstMX0iLDJdLFswLDIsImZcXGNkb3RcXGJhclxccGFydGlhbF9FIl0sWzMsMiwiZl8qIiwyXV0=
  \begin{tikzcd}
    {\Omega^0(X,E)} && {\Omega^{0,1}(X,E)} \\
    {\Omega^0(X,E)} && {\Omega^{0,1}(X,E)}
    \arrow["{f\cdot\bar\partial_E}", from=1-1, to=1-3]
    \arrow["{f_*^{-1}}"', from=1-1, to=2-1]
    \arrow["{\bar\partial_E}"', from=2-1, to=2-3]
    \arrow["{f_*}"', from=2-3, to=1-3]
  \end{tikzcd}
\end{equation}
\begin{lemma}
  The topological space $A^{\dol} (E)/\Aut(E)$ of isomorphism classes of
  Dolbeault operators is not Hausdorff. In particular, it does not
  admit a manifold structure.
  \begin{proof}
    \missingproof
  \end{proof}
\end{lemma}

Note that we have a similar action of $\Aut(E)$ on $A^\phi(E)$ and
indeed many other topological spaces of holomorphic bundles. However,
it is quite clear that any space that arises as a disjoint union
over $A^{\dol} (E)$ must also be non-Hausdorff. Therefore we need
to use more sophisticated techniques to construct any moduli
space of holomorphic bundles.

\section{Gauge Theory}\todo{motivate or improve this name}

To construct a manifolds whose points are equivalence classes of
holomorphic bundles, we need to take quotients of manifolds by
group actions. To be precise, we are going to take a K\"ahler
quotients of a Banach manifolds by actions of Banach Lie groups.

There are a few things to take care of before this can be made
precise. Firstly, we need to define what it means to be an infinite
dimensional manifold and make sure our topological spaces satisfy
the requirements. Secondly, we ought to define and obtain K\"ahler
structures on the manifolds. Finally, we observe how to take quotients
in this setting.

Once again fix a smooth bundle $E$ of rank $r$ on a compact
Riemann surface $C$.

\subsection{Connections}

Connections are going to help us in two ways: Firstly, they are going to
be to correspond to Dolbeault operators and hence lead to an
alternative point of view on spaces of holomorphic bundles.
Secondly, they are going to allow us to define the Sobolev norm and
hence provide us with a suitable Banach manifold structure on
the moduli spaces.

Recall that connections enable us to differentiate sections of
vector bundles:

\begin{definition}
  A \emph{connection} in a smooth complex bundle $E$ on
  a complex manifold $X$ is a linear map $\nabla : \Omega^0(X,E) \to \Omega^1(X,E)$
  such that
  \begin{align*}
    \nabla (fs) = df \otimes s + f\nabla s,
  \end{align*}
  for all $f\in \Omega^0(X)$ and $s\in\Omega^0(X,E)$.
\end{definition}

\begin{example}
  If we are given connections $\nabla$ and $\nabla'$ in bundles $E$ and $E'$ on $X$,
  respectively, then we obtain a connection in the bundle $E\otimes E'$ via
  \begin{align*}
    \nabla \otimes I + I \otimes \nabla'.
  \end{align*}
  Similarly, there is a connection $\nabla^*$ in the dual bundle $E^*$ given by
  \begin{align*}
    d\langle \xi,s\rangle
    = \langle \nabla^*\xi, s \rangle + \langle \xi,\nabla s\rangle
  \end{align*}
  where $\langle\cdot,\cdot\rangle$ denotes the natural pairing $E^*\times E\to\mathbb{C}$.
  In particular, we obtain a connection $\End\nabla$ in $\End E = E^* \otimes E$.
\end{example}

Note that connections look very similar to Dolbeault operators.
Indeed, via projections to the $(1,0)$ and $(0,1)$ parts, respectively,
we obtain a splitting $\nabla = \nabla_{1,0} + \nabla_{0,1}$
where
\begin{align*}
  \nabla_{0,1} : \Omega^0(X,E) \to \Omega^{0,1}(X,E).
\end{align*}
We may ask whether this is the same as a holomorphic structure on $E$:

\begin{definition}
  Let $\dol_E$ be a Dolbeault operator on $E$.
  A connection $\nabla$ in $E$ is \emph{compatible with $\dol_E$}
  if, and only if, $\nabla_{0,1} = \dol_E$. In general, we say
  $\nabla$ is \emph{holomorphic} if it is compatible with some
  $\dol_E$.
\end{definition}

Note that all connections $\nabla$ satisfy the Leibniz rule. It is
straightforward to check that
\begin{align*}
  \nabla_{0,1} (fs) = \dol f\otimes s + f \nabla_{0,1} s.
\end{align*}
Hence $\nabla$ is holomorphic if, and only if, $(\nabla_{0,1})^2 = 0$.

\subsection{Unitary Connections}

Holomorphic connections are very closely related to Dolbeault operators,
but they are not quite the same. To map one to the other, we are going
to require some additional information. The correct geometric
structure to consider is a metric:

\begin{definition}\todo{fix the dual notation}
  A \emph{hermitian metric} on $E$ is a smooth section $h$ of
  $(E\otimes\bar E)^\vee$ that is sesquilinear and positive definite.
  That is, $h(s,\bar t) = \overline{h(t,\bar s)}$ and $h(s,\bar s) > 0$
  if $s \neq 0$.
\end{definition}

While such metrics are interesting in their own right, we have no
intention to study them more than absolutely necessary. Indeed, it will
become apparent that the mere presence of a metric is sufficient.
Fortunately for us, it will always be possible to choose a metric:

\begin{theorem}\label{thm:hermitian_structures_exist}\missingcitation
  Every smooth vector bundle on a compact complex manifold admits a
  hermitian metric.
\end{theorem}

Fix such a metric $h$ on $E$. We are going to restrict our attention to
connections that are compatible with this metric:

\begin{definition}
  A connection $\nabla$ in $E$ is \emph{$h$-unitary}
  if, and only if, for all $s,t\in\Omega^0(X,E)$,
  $dh(s,\bar t) = h(\nabla s,\bar t) + h(s,\overline{\nabla t})$.
\end{definition}

Now consider the set of all connections that respect $h$ and induce
some Dolbeault operator:
\begin{align*}
  A_\nabla(E,h) := \left\lbrace{\text{$h$-unitary holomorphic connections $\nabla$ in $E$}}\right\rbrace.
\end{align*}
Analogous to $A^{\dol} (E)$, this is an affine space. To see what it is
modelled on, consider the principal $GL_r(\mathbb{C})$-bundle whose fibres \todo{this is
not correct!}
are the automorphisms of the fibres of $E$. We may denote this bundle by $GL(E)$.
We then have a principal $U(r)$-bundle $U(E,h)$ of $h$-unitary automorphisms
of $E$. By taking Lie algebras we obtain vector bundles $\mathfrak{gl}(E)$
and $\mathfrak u(E)$. In particular, we have a sequence of inclusions
\begin{align*}
  \Omega^0_{\mathbb{R}}(C,\mathfrak u(E))
  \subseteq \Omega^0_{\mathbb{R}}(C,\mathfrak{gl}(E))
  \subseteq \Omega^0(C,\mathfrak{gl}(E)).
\end{align*}
The reader who is unfamiliar with principal bundles
ought not to be discouraged: $\mathfrak{gl}(E)$ is simply the smooth complex
vector bundle $\End E$ and $\mathfrak{u}(E)$ is the real vector subbundle
of skew-hermitian endomorphisms of $E$.

\begin{lemma}
  $A_\nabla(E,h)$ is an affine space modelled on
  $\Omega^1_{\mathbb{R}}(C,\mathfrak u(E))$.
  \begin{proof}
    The operations are induced by $\Omega^1(C,\mathfrak{gl}(E))$.
    Verifying that general $\nabla + \xi$ for
    $\xi\in\Omega^1_{\mathbb{R}}(C,\mathfrak u(E))$
    satsify the Leibniz rule and the integrability
    condition $(\nabla_{0,1})^2=0$ is analogous to
    \ref{lem:affine_space_of_dolbeault_operators}. Hence $\nabla + \xi$
    is a holomorphic connection.

    To see that $\nabla + \xi$ is $h$-unitary we notice
    \begin{align*}
      h((\nabla + \xi)s, \bar t) + h(s,\overline{(\nabla + \xi)t})
      =  h(\nabla s, \bar t) + h(s,\overline{\nabla t})
    \end{align*}
    as $\xi$ is skew-hermitian.
  \end{proof}
\end{lemma}

Combining what we have seen so far with the next result, the
similarities between $A^{\dol} (E)$ and $A_\nabla(E,h)$ grow even more
striking:

\begin{lemma}\label{lem:real_vector_space_iso}
  As real vector spaces,
  $\Omega^1_{\mathbb{R}}(C,\mathfrak u(E))\cong\Omega^{0,1}(C,\mathfrak{gl}(E))$.
  \begin{proof}
    Now recall the isomorphism (\ref{eq:holomorphic_tangent_bundle}) of
    smooth vector bundles $T^* X \cong T^*_{1,0} X$. Composing with
    complex conjugation yields an isomorphism of real vector bundles
    $T^*X \cong T^*_{0,1}$. Hence we have an $\mathbb{R}$-linear
    isomorphism
    \begin{align*}
      \Omega^1_{\mathbb{R}}(C,\mathfrak u(E))
      \cong \Gamma^\infty(C,T^*_{0,1}C \otimes_{\mathbb{R}}\mathfrak u(E))
    \end{align*}
    Noticing that $T_{0,1}^* C$ is a complex vector bundle
    we find
    \begin{align*}
      \Gamma^\infty(C,T^*_{0,1}C \otimes_{\mathbb{R}}\mathfrak u(E))
      \cong\Gamma^\infty(C,T^*_{0,1}C \otimes\mathfrak u(E)_{\mathbb{C}}).
    \end{align*}
    Now the claim follows from the isomorphism of Lie algebras
    $\mathfrak{gl}_n(\mathbb{C})\cong\mathfrak u(n)_{\mathbb{C}}$
    given by $A \mapsto A-A^\dagger$.
  \end{proof}
\end{lemma}

It is not a coincidence that $A^{\dol} (E)$ and $A_\nabla(E,h)$ both are
affine spaces modelled on the same underlying real vector space.
Indeed, in the presence of $h$, every Dolbeault operator has a unique
compatible connection:

\begin{theorem}\label{thm:chern_connection}
  If $h$ is a hermitian metric on $E$ and $\dol_E$ is a Dolbeault
  operator then there exists a unique connection $\nabla^h(\dol_E)$
  that is compatible with both $h$ and $\dol_E$.
  \begin{proof}
    \missingproof
  \end{proof}
\end{theorem}

In particular, this gives us a bijection
$\nabla^h : A^{\dol} (E) \to A_\nabla(E,h)$ whose inverse is the restriction
$\nabla \mapsto \nabla_{0,1}$. It is not difficult to verify that this
is a homeomorphism of affine spaces which is locally given by
the $\mathbb{R}$-linear isomorphism in \ref{lem:real_vector_space_iso}.

Let us abuse notation slightly and denote by $U(E,h)$ the group of smooth
$h$-unitary sections of $\End E$. That is, $U(E,h) := \Gamma^\infty(C,U(E,h))$.
We have an action of $U(E,h)$ on $A_\nabla(E,h)$ via conjugation. That is,
for $g\in U(E,h)$ and $\nabla\in A_\nabla(E,h)$, $g\cdot\nabla$ is the map
\begin{align*}
  \Omega^0(X,E) \xlongrightarrow{\Omega^0(X,{g}^{-1})}
  \Omega^0(X,E) \xlongrightarrow{\nabla}
  \Omega^1(X,E) \xlongrightarrow{\Omega^1(X,g)}
  \Omega^1(X,E).
\end{align*}
It is not difficult to verify that this is indeed an action.
In fact, $g\cdot\nabla$ is almost immediately a holomorphic connection.
One then sees that $g\cdot\nabla$ is unitary because
$g$ and $\nabla$ are.

\todo{complex structure on the affine space}

\subsection{Doubled Connections}

We have successfully viewed the space of holomorphic bundles $A^{\dol} (E)$
as an affine space modelled on
$\Omega^1_{\mathbb{R}}(C,\mathfrak{u}(E))$. This is eventually going
to help us introduce a symplectic structure on $A^{\dol} (E)$ and hence
take a symplectic quotient. Let us do the same for the space of Higgs
bundles $A^\phi(E)$. We are going to identify $A^{\phi}$ with an affine
space modelled on $\Omega^1_{\mathbb{R}}(C,\mathfrak u(E))^{\oplus 2}$.
Once again fix a hermitian metric $h$ on $E$.

Just as holomorphic bundles correspond to connections, Higgs bundles
correspond to doubled connections.

\begin{definition}
  A \emph{doubled connection} in $E$ is a pair $(\nabla,\Phi)$
  where $\nabla$ is a $h$-unitary connection in $E$ and
  $\Phi\in\Omega^1_{\mathbb{R}}(C,\mathfrak u(E))$.
\end{definition}

Hence we have the space
\begin{align}\label{eq:doubled_connections_space}
  A^{\Phi} := A_\nabla \times \Omega^1_{\mathbb{R}}(C,\mathfrak u(E))
\end{align}
which naturally has the structure of an affine space modelled on
the real vector space
$\Omega^1_{\mathbb{R}}(C,\mathfrak u(E))\oplus\Omega_{\mathbb{R}}^1(C,\mathfrak u(E))$. Using the isomorphism (\ref{lem:real_vector_space_iso})
and the map given by (\ref{thm:chern_connection}) we have an inclusion
of topological spaces
\begin{align*}
  A^\phi(E) \subseteq A^{\dol} (E) \times \Omega^{1,0}(\End E) \cong A^\Phi (E,h).
\end{align*}
Note that in general $A^\phi(E)$ and $A^\Phi (E,h)$ are not isomorphic. In
particular, a doubled connection $(\nabla,\Phi)$ corresponds to
a pair $(\dol_E,\phi)$ where $\dol_E$ is a Dolbeault operator and
$\phi$ is a section in $\Omega^{1,0}(C,\End E)$, not necessarily
holomorphic. We will have to deal with this at a later stage.

For now, observe that the group $G(E,h)$ also acts on $A^\Phi (E,h)$
by conjugation:
\begin{align*}
  G(E,h) \times A^\Phi (E,h) & \to A^\Phi (E,h)                        \\
  (g,\nabla,\Phi)            & \mapsto (g\nabla{g}^{-1},g\Phi{g}^{-1})
\end{align*}


\subsection{Sobolev Spaces}

Our next goal is to equip the affine spaces of connections $A_\nabla(E,h)$
and $A^\Phi (E,h)$ with manifold structures. Of course, both are affine
spaces modelled on infinite dimensional vector spaces. In the next
section we are going to require our infinite dimensional manifolds to
be locally isomorphic to Banach spaces, rather than arbitrary
vector spaces. While we are going to profit from this in the long run,
it means that we have to turn $\Omega^1_{\mathbb{R}}(C,\mathfrak u(E))$
into a Banach space. We are going to do this by using a completion of
the space of smooth sections.

Consider the following, more general setting. Let $X$ be a compact
Riemannian manifold $X$ of dimension $n$, let $E$ be a smooth complex
bundle on $X$ with hermitian metric $h$, let $\nabla$ be a $h$-unitary
connection in $E$, and let $\ell\geq 0$ and $1<p<\infty$ be integers.
Here we are going to think of $\nabla$ as a derivative of sections and
define a space $W_{p,\ell}(X,E)$ which may be thought of as the space of
sections $X\to E$ with $k$ derivatives, all of which in $L^p$. As
is the case with many of the concepts we encounter throughout this paper,
these Sobolev spaces are interesting geometric objects to
contemplate but we are going to use them as a tool. The main
benefit we are going to get from using this particular Banach space
completion is the theory of elliptic regularity. \missingcitation

Note that the metrics on $E$ and $X$ define norms on smooth sections
of $E$ and $T^*X$, respectively, and hence on all
$\Omega^k_{\mathbb{R}}(X,E)$.
Moreover, the metric on $X$ induces a volume form $d\text{vol}_X$.
With this in mind, we are ready to define the Sobolev norm on
$\Gamma^\infty(X,E)$:

\begin{definition}
  Let $\nabla$ be a $h$-unitary connection in $E$.
  The \emph{$(p,k)$-Sobolev norm} with respect to $\nabla$ on
  $\Gamma^\infty(X,E)$ by
  \begin{align*}
    \Vert s\Vert_{p,\ell}^\nabla := \left(
    \sum_{i=0}^\ell \int_X \Vert \nabla^i s \Vert^p d\text{vol}_X
    \right)^{1/p}.
  \end{align*}
\end{definition}
\begin{example}
  Observe that Sobolev norms generalise $L^p$ norms. In particular,
  we have $\Vert\cdot\Vert^\nabla_{p,0} = \Vert\cdot\Vert_{L^p}$.
\end{example}
Note that the norm depends on the choice of connection $\nabla$. Of
course, we do not want moduli spaces to have the same dependency. Hence
we establish the following:
\begin{lemma}
  Let $\nabla$ and $\nabla'$ be $h$-unitary connections in $E$. Then the
  norms $\Vert\cdot\Vert_{p,\ell}^\nabla$ and
  $\Vert\cdot\Vert_{p,\ell}^{\nabla'}$ are equivalent.
  \begin{proof}
    \missingproof
  \end{proof}
\end{lemma}

Hence we are justified in defining a completion that does not depend
on the connection:

\begin{definition}
  The \emph{Sobolev space} $W_{p,k}(X,E)$ is the Banach completion of
  $\Gamma^\infty(X,E)$ with respect to the norm
  $\Vert\cdot\Vert_{p,\ell}^\nabla$ for any $h$-unitary connection
  $\nabla$ in $E$.
\end{definition}

Observe that we immediately have a sequence of inclusions
\begin{align*}
  L^2(X,E) =
  W_{2,0}(X,E) \supseteq
  \cdots \supseteq
  W_{2,\ell}(X,E) \supseteq
  \cdots
  \supseteq
  C^\infty(X,E)
\end{align*}
where the last inclusion intuitively follows from the fact
$C^\infty \subseteq L^2$ and the observation that each derivative
$\nabla^i s$ is $C^\infty$. For a more rigorous argument, see
\cite[Corollary 3.8.3]{bc2009}.
Moreover, $C^\infty$ is dense in $L^2$ so it must be dense in each
$W^{2,\ell}$, too. Thus we will be able to regard $\Gamma^\infty(X,E)$
as a Banach space by expanding it slightly.

As usual, we are going to think of the elements of $W_{2,\ell}(X,E)$ as
sections in $\Gamma^\infty(X,E)$. However, it is worth remembering that
sections that differ on a set of measure zero are identified in the
completion.

\begin{example}
  Note that the metric on $X$ is a metric on $T^*X$. Hence we have
  a metric on $E\otimes (T^*X)^{\wedge k}$ for all $k$. This allows
  us to define
  \begin{align*}
    \Omega_{\mathbb{R}}^k(X,E)_\ell := W_{2,\ell}(X,E\otimes(T^*X)^{\wedge k}).
  \end{align*}
\end{example}

The last feature we require our Banach space completions to have is
that it applies to maps of the underlying spaces, too. Of course,
it will not in general be true that every map
$\Gamma^\infty(X,E)\to\Gamma^\infty(X,F)$ extends to a map on
Sobolev spaces. Fortunately, there are some reasonable assumptions
we can make. Consider the following class of such operators:

\begin{definition}
  Let $E$ and $F$ be smooth bundles on $X$. A \emph{differential
    operator of order $m$} is a $C^\infty(X)$-linear map
  \begin{align*}
    P : \Gamma^\infty(X,E) \to \Gamma^\infty(X,F)
  \end{align*}
  such that, for each $x\in X$, there is a chart $U\ni x$ with
  local coordinates $(x_1,\ldots,x_n)$ trivialising $E$ and $F$ such that
  \begin{align*}
    P(\Gamma^\infty(U,E))\subseteq \Gamma^\infty(U,F),
  \end{align*}
  and
  \begin{align*}
    \restrict{P}{U}(s) = \sum_{|I|\leq m} f_I \partial^I
  \end{align*}
  where, for each multi-index
  $I=\left\lbrace{i_1,\ldots,i_n}\right\rbrace$,
  $f_I \in \Gamma^\infty(U,\Hom(E,F))$ and
\end{definition}

\begin{theorem}
  Let $P:E\to F$ be a map of smooth bundles on $X$ such that, for
  each chart $U\subseteq X$ with trivialisation
  \todo{make this precise}
  See \cite[{Proposition 3.8.4}]{bc2009}.
\end{theorem}

\subsection{Banach Manifolds}

Smooth and complex manifolds are often defined using charts and
atlases. However, once the necessary theory has been established,
it is far easier to say that a complex manifold
of dimension $n$ is a locally ringed space locally isomorphic to
$(\mathbb{C}^n,\mathcal O_{\mathbb{C}^n})$ where
$\mathcal O_{\mathbb{C}^n}$ is the sheaf of holomorphic functions on
$\mathbb{C}^n$.

To define infinite dimensional manifolds, we are going to replace
$\mathbb{C}^n$ with infinite dimensional vector spaces. Of course,
the point of the usual manifold definition is to facilitate analysis
on a wider range of topological spaces. To this end, we are going to
restrict our attention to the case of Banach spaces:

\begin{definition}
  A \emph{Banach manifold} over a Banach space $V$ is a locally ringed
  space locally isomorphic to $(V,C^\infty_V)$ where $C^\infty_V$ is the
  sheaf of smooth functions on $V$.
\end{definition}

Of course Banach manifolds come with all the bells and whistles that
we know and love. In particular, we have smooth maps of Banach manifolds,
vector bundles, tangent and cotangent spaces, derivatives, and
differential forms. \todo{doublecheck which of these we really need
  and which we need to define in more detail}

\begin{example}
  Every smooth manifold is locally isomorphic to
  $(\mathbb{R}^n,C^\infty_{\mathbb{R}^n})$ and hence is a Banach manifold.
  In particular, every complex manifold has an underlying Banach manifold.
\end{example}

\begin{example}
  Let us consider our main applications. We have the affine space
  $A_\nabla(E,h)$ modelled on $\Omega^1_{\mathbb{R}}(C,\mathfrak u(E))$.
  For each $\ell \geq 0$, we have the corresponding Sobolev completion
  \begin{align*}
    A_\nabla(E,h)_\ell := A_\nabla(E,h) + \Omega^1_{\mathbb{R}}(C,\mathfrak u(E))_\ell
  \end{align*}
  which are naturally Banach manifolds.
  In particular, for each $\nabla\in A_\nabla(E,h)_\ell$, we have the
  cotangent space
  \begin{align*}
    T^*_\nabla A_\nabla(E,h)_\ell
    = \Omega^1_{\mathbb{R}}(C,\mathfrak u(E))_\ell.
  \end{align*}
  Noticing that the direct sum of Banach spaces remains a Banach space,
  the affine space (\ref{eq:doubled_connections_space}) of doubled connections also
  extends to a Banach manifold
  \begin{align*}
    A^\Phi(E,h)_\ell := A^\Phi(E,h) + \Omega^1_{\mathbb{R}}(C,\mathfrak u(E))_\ell
    \oplus \Omega^1_{\mathbb{R}}(C,\mathfrak u(E))_\ell
  \end{align*}
  with the obvious cotangent space at $(\nabla,\Phi)$:
  \begin{align*}
    T^*_{\nabla,\Phi} A^\Phi(E,h)_\ell
    = \Omega^1_{\mathbb{R}}(C,\mathfrak u(E))_\ell
    \oplus \Omega^1_{\mathbb{R}}(C,\mathfrak u(E))_\ell
  \end{align*}
\end{example}

\begin{example}
  Consider the group $U(E,h)$ of unitary
  sections of $\End E$. The Sobolev norms on $\Gamma^\infty(C,\End E)$
  restrict to norms on the subset $U(E,h)$ and hence we
  may consider the completions
  \begin{align*}
    U(E,h)_\ell := W_{2,\ell}(C,U(E,h)) \subset W_{2,\ell}(C,\End E).
  \end{align*}
  One may show that this is a Banach manifold by applying the
  analogue of the regular value theorem. In particular,
  the tangent space at $g\in U(E,h)_\ell$
  \begin{align*}
    T_g U(E,h)_\ell = W_{2,\ell}(C,\mathfrak u(E)).
  \end{align*}
\end{example}

\missingsection

\section{Quotients of Banach Manifolds}

Having obtained a manifold structure on the affine spaces and the groups
that act on them, we are in a great position to take quotients to construct
the analytic moduli spaces of holomorphic bundles that we are after. To do this,
one needs to ensure that the group actions are sufficiently well-behaved. Moreover,
we need to ensure that the actions on affine spaces extend to a Banach Lie group
action on Banach manifolds.

\subsection{Proper and Free Group Actions}

Before we extend groups and actions to the realm of infinite dimensional
Banach spaces, let us study the group actions that we have thus
far encountered a little more closely. Later we are going to require our
actions to be free and proper. Let us recall what that means.

Properness is a topological property. That is, it applies only to
topological groups:

\begin{definition}
  An action $\rho : G\times X\to X$ of a topological group $G$
  on a topological space $X$ is \emph{proper} if the map
  \begin{align}\label{eq:proper_map}
    (\rho,\pi) : G\times X \to X\times X
  \end{align}
  is proper, i.e. the preimage of every compact set in $X\times X$
  is compact.
\end{definition}

\begin{example}
  One easily sees if $G$ is compact then every $G$-action is proper.
  In particular, the action of $U(1)$ on $U(E,h)$ is
  proper.
\end{example}

\begin{example}\label{ex:h_proper_action}
  The action of $U(E,h)$ on $A_\nabla(E,h)$ is proper.
  \begin{proof}
    We sketch the proof given in
    \cite[{Proposition 7.1.14}]{kobayashi1987}.
    The goal is to show that, if we have a sequence $(g_j,\nabla_j)$
    such that $\nabla_j\to\nabla$ and $g_j\cdot\nabla_j\to\nabla'$
    then $g_j\to g$ such that $g\cdot\nabla = \nabla'$. To do this
    one moves to the principal $U(n)$-bundle $P$ associated to $(E,h)$
    where each $g_j$ commutes with the action of the structure
    group. The next step is to focus on a single fibre by fixing $x_0\in X$ and
    $p_0\in P_{x_0}$ and considering curves $c$ starting at $x_0$ in $X$.
    One then uses $\nabla$ and $\nabla'$ to obtain lifts $\tilde c$ and
    $\tilde c'$ of $c$ in $P$. Now the limit $g$ of $(g_j)$ must satisfy
    $g(\tilde c) = \tilde c'$. This condition on $g$, together with
    commutativity with the action of the structure group, identifies it
    uniquely.
  \end{proof}
\end{example}

\begin{example}\label{ex:higgs_proper_action}
  the action of $U(E,h)$ on $A^\Phi (E,h)$ is proper.
  \missingexample
\end{example}

Being free is not a topological property, i.e. any action of any group on any
set may be free. Nonetheless, it is particularly important in a geometric
setting as it ensures that the orbits are homogeneous and hence all points in
the quotient are locally indistinguishable, as one expects from manifolds.

\begin{definition}
  An action $\rho : G\times X\to X$ is \emph{free} if, for all
  $g\in G$ and $x\in X$, $g\cdot x = x$ implies $g = e$.
\end{definition}

\begin{example}\label{ex:not_free}
  Consider the action of $U(E,h)$ on $A_\nabla(E)$. This is not a free action
  because scalar multiplication commutes with the connection. That is,
  for all $\lambda\in U(1)$, $g\in U(E)$, and $\nabla\in A_\nabla(E,h)$,
  it holds that $\lambda g\cdot \nabla = \nabla$. This flaw is easily fixed,
  however. In fact, we immediately have an action of
  \begin{align*}
    G(E,h) := U(E,h) / U(1)
  \end{align*}
  on $A_\nabla(E,h)$. Unfortunately, this action is not free either. In particular,
  if there is a non-trivial proper subbundle $E'\subseteq E$ that is preserved
  by $\nabla$ then non-trivial automorphisms of $E'$ act trivially on $\nabla$.
\end{example}

\begin{example}\label{ex:not_free_double}
  The fact that the $G(E,h)$ action on $A_\nabla(E,h)$ is not free does not
  immediately imply that the action on $A^\Phi(E,h)$ is not either. However,
  similar reasoning shows why it cannot be free: If there is a non-trivial
  proper subbundle $E'\subseteq E$ that is preserved by $\nabla$ and $\Phi$,
  respectively, then an automorphism $g\in G(E,h)$ need only be trivial
  outside of $E'$ to act trivially on $(\nabla,\Phi)$.
\end{example}


\subsection{Banach Lie Groups}

In particular, there is a natural notion of a Banach Lie group and
corresponding actions on Banach manifolds.

\begin{example}
  The Banach manifold $U(E,h)_\ell$ is a Banach Lie group with the usual
  multiplication. One need only check that $\Vert\cdot\Vert_{2,\ell}$-Cauchy
  sequences $(g_i)$ and $(h_j)$ in $U(E,h)$ yield a Cauchy sequence
  $(g_ih_i)$.
\end{example}

In complete analogy to the finite dimensional case, the tangent space
at the identity of a Banach Lie group may be equipped with a Lie bracket
and hence is called the Lie algebra of the group.

\begin{example}
  The Lie algebra of $U(E,h)_\ell$ is $W_{2,\ell}(C,\mathfrak u(E))$.
\end{example}

Similarly, a action of a Banach Lie group on a Banach manifold
is a Banach Lie group action if, and only if, it is given by a
smooth map. Here we use the fact that the product of Banach manifolds
has a natural Banach manifold structure.

We observe that Sobolev completions allow us to lift group
actions:

\begin{example}
  Consider the action of $U(E,h)$ on $A_\nabla(E,h)$
  by conjugation. This extends to an action of $U(E,h)_{\ell+1}$
  on $A_\nabla(E,h)_\ell$ via
  \begin{align*}
    (g,\nabla) & \mapsto g\nabla g^{-1}.
  \end{align*}
  Note that, if we fix $\nabla_0\in\Omega^1(C,\mathfrak u(E,h))$ and write
  $\nabla = \nabla_0 + $
  for $s\in\Omega^0_{\mathbb{R}}(C,E)$ we have
  \begin{align*}
    g\nabla g^{-1}(s) = g\nabla(g^{-1}s) = g(g^{-1}(\nabla s) + (\nabla g^{-1})s)
    = \nabla s + g(\nabla g^{-1})s.
  \end{align*}
  Hence we require $g\in U(E,h)_{\ell + 1}$ to make sure that $g\cdot\nabla\in A_\nabla(E,h)_\ell$.
  Analogously we have a Banach Lie group action of $U(E,h)_{\ell+1}$
  on $A^\Phi (E,h)_\ell$.
  \todo{mention that they're free and proper}
\end{example}

Recall the infinitesimal action
$\rho_* : \mathfrak g \to \Gamma^\infty(X,TX)$ given by
\begin{align*}
  \rho_*(\xi)_x := \restrict{\frac{d}{dt}}{t=0} \exp(t\xi)\cdot x
\end{align*}
at $x\in X$.

All of our recent discussions about proper and free actions, Sobolev
completions, and Banach manifolds are motivated by the next theorem.
In particular, it will allow us to construct moduli spaces as, possibly
infinite dimensional, Banach manifolds:

\begin{theorem}\label{thm:banach_quotient}
  Consider a proper free action $\rho : G\times X\to X$ of a Banach
  Lie group $G$ on a Banach manifold $X$. If each tangent space
  $T_x(G\cdot x) \subseteq T_x X$ is closed and complemented then
  $X/G$ is a Banach manifold and $TX/\rho_*(\mathfrak g) \cong T(X/G)$.
  \todo{doublecheck}
  \begin{proof}
    The idea is to construct a slice at each point $x\in X$, i.e. a submanifold
    $x\in S_x\subseteq X$ such that $T_x X = T_xS_x \oplus T_x(G\cdot x)$. A slice is
    what induces the manifold structure of $X/G$ at $x$. More preciesley, if a slice
    exists at $x\in X$ then we may choose a sufficiently small chart $U\ni x$
    such that $\pi : X \to X/G$ gives a homeomorphism onto $\pi(U)$. Doing this
    at every $x\in X$ gives us the manifold structure on $X/G$. Now the tangent space
    to $X/G$ at the equivalence class of $x$ is
    \begin{align}\label{eq:banach_quotient_tangent}
      T_x S_x = T_x X / T_x (G\cdot x) = T_x X/\rho_*(\mathfrak g)_x.
    \end{align}
    The difficult part is making sure that a slice exists at every $x\in X$. One may use
    one of the established theorems such as \cite[Theorem 3.28]{diez2019}
    or \cite[Theorem 5.2.6]{palais1992}. The latter applies particularly well
    as the group action is Fredholm by our assumption that it be free and satisfy
    $T_x(G\cdot x)\subseteq T_x X$.
  \end{proof}
\end{theorem}

\begin{example}
  Observe that the Banach Lie subgroup $U(1)\subseteq U(E,h)$ acts freely and properly
  on $U(E,h)$ via $(\lambda,g)\mapsto \lambda g$. Moreover, each tangent space
  \begin{align*}
    T_g(U(1)g) =
    i\mathbb{R} \subseteq W_{2,\ell}(C,\mathfrak u(E))
  \end{align*}
  is complemented as $W_{2,\ell}(C,\mathfrak u(E,h))$ is in fact a Hilbert space.
  Thus the quotient
  \begin{align*}
    G(E,h)_{\ell} := U(E,h)_\ell / U(1)
  \end{align*}
  is a Banach manifold whose tangent space at $g\in G(E,h)_\ell$ is
  \begin{align*}
    T_g G(E,h)_\ell = W_{2,\ell}(C,\mathfrak u(E,h))/i\mathbb{R}.
  \end{align*}
\end{example}

Unfortunately, we are not yet ready to apply the theorem \ref{thm:banach_quotient} to
the actions of $G(E,h)_{\ell+1}$ on $A_\nabla(E,h)_\ell$ and
$A^\Phi(E,h)_\ell$ as they are not free. Moreover, the tangent spaces fail to
be closed:

\begin{example}
  \missingexample
\end{example}

In the next section we are going to rectify both issues by restricting the spaces
sufficiently. Indeed, our argument in (\ref{ex:not_free}) already partly tells us
what the correct subsets to consider will be.

\subsection{Harmonic Connections}

In some sense, failure of the tangent space space to be closed is a bigger issue than
a non-free group action. Indeed, if the action is not free we simply cannot assume that
the map $g\mapsto g\cdot x$ is a diffeomorphism $G\cong G\cdot x$. This means in particular
that the calculation 
(\ref{eq:banach_quotient_tangent}) fails. However, slices still exist so there are ways to
locally identify the quotient with a slice. In particular, we obtain a quotient that is
locally a manifold, although the tangent spaces may not be uniform.

On the other hand, if the tangent space is not closed then it may not be complemented and
hence there may not be a slice to identify the quotient with. Hence we address this issue first.
There are multiple ways of restricting the affine spaces we are thinking about to
ensure orbits with closed tangent spaces. We do this by introducing harmonicity.

Observe that if we have a compact Hermitian manifold $X$ may refer to the adjoints \todo{hermitian or Hermitian?}
\begin{align*}
  \dol^* : \Omega^{p,q}(X)\to \Omega^{p,q-1}(X)
\end{align*}
of the operator $\dol : \Omega^{p,q}(X)\to\Omega^{p,q+1}(X)$. This allows us to define
what it means for a form on $X$ to be harmonic:

\begin{definition}
  A $(p,q)$-form $\alpha$ on a compact Hermitian manifold $X$ is \emph{harmonic} if
  $\dol \alpha = \dol^* \alpha = 0$.
\end{definition}

\begin{definition}
  A connection $\nabla\in A_\nabla(E,h)$ is \emph{Yang-Mills} if
  there exists a harmonic form $\alpha\in\Omega^2(C)$ such that
  $F_\nabla = i\alpha\otimes I$ in $\Omega^2(C,\End E)$.
\end{definition}

\begin{example}
  Consider the space $A_\nabla^{ss}(E,h)$ of Yang-Mills connections.  
  We observe that gauge transformations map $I\in\End E$ to itself.
  Hence $A_\nabla^{ss}(E,h)\subseteq A_\nabla(E,h)$ is $U(E,h)$-invariant.
  Thus we have a proper free $G(E,h)$-action on $A_\nabla(E,h)$. Moreover,
  that extends to a proper free action of $G(E,h)_{\ell+1}$ on
  $A_\nabla^{ss}(E,h)_\ell$.
\end{example}

\begin{definition}
  A doubled connection $(\nabla,\Phi=\phi-\phi^\dagger)\in A_\Phi(E,h)$ is \emph{Hitchin}
  if $\nabla_{0,1}\phi = 0$ and $[\phi,\phi^\dagger]$ is Einstein.
\end{definition}

Note in particular if $\Phi = 0$ then $\phi = \phi^\dagger$ and hence
$[\phi,\phi^\dagger] = 0$ so $\nabla$ is Einstein.

\subsection{Reducible Connections}

It is crucial that the actions used to construct quotients of Banach manifolds
are proper and free. We are now going to restrict $A_\nabla$ and $A^\Phi$ to
ensure that this is the case. Of course we will have to deal with the loss of
information that this entails at a later stage.

In light of our observations in \ref{ex:not_free} and \ref{ex:not_free_double}
we make the following definitions:

\begin{definition}
  A connection $\nabla\in A_\nabla (E,h)$ is \emph{reducible} if there exists
  a non-trivial proper subbundle $E'\subseteq E$ such that $\nabla$ restricts
  to a connection on $E'$.
\end{definition}

\subsection{On Symplectic Quotients}

\section{Properties of the Analytic Moduli Spaces}

\subsection{Independence of the Choice of Sobolev Completion}

\subsection{Tangent Space and Dimension}

\begin{theorem}
  Let $[\dol_E]\in M^{\dol}(E)$. Then the tangent space at $[\dol_E]$ is
  \begin{align*}
    T_{[\dol_E]}\mathcal M^{\dol}(E) = H^{0,1}(C,\End(E,\dol_E)).
  \end{align*}
  \begin{proof}[Proof sketch]
    A full proof is given in \cite[223-225]{kobayashi1987}. The key idea is to look
    at curves in $M^{\dol}(E)$ to obtain the tangent space at $\nabla$ as a set of
    equivalence classes in $\Omega^1_{\mathbb{R}}(C,\mathfrak u(E))_\ell$. One then
    identifies this space with $H^{0,1}(C,\End(E,\dol_E))$ via a sequence of calculations
    involving harmonic forms.
  \end{proof}
\end{theorem}

It is worth noting that the proof does not rely on $M^{\dol_E}(E)$ being a manifold
at all. Indeed, a similar calculation is done for a moduli space that is not
in general a manifold, yet it makes sense to classify curves passing through a point.

Having calculated the tangent space, we may be relieved and observe that the moduli
space is indeed a complex manifold in the usual sense:

\begin{corollary}
  $M^{\dol}(E)$ is a complex manifold of finite dimension.
  \begin{proof}
    By Dolbeault's theorem we may express the tangent space in terms of holomorphic
    sections as
    \begin{align*}
      T_{[\dol_E]}\mathcal M^{\dol}(E) = \Gamma(C,\End(E,\dol_E) \otimes T_{1,0}^*C).
    \end{align*}
    The space of holomorphic sections of a holomorphic vector bundle is finite dimensional,
    hence so is $M^{\dol}$. Moreover, the tangent space naturally is a complex vector
    space and hence we have a natural complex structure on $M^{\dol}(E)$.
  \end{proof}
\end{corollary}

Knowing the tangent space also allows us to calculate the dimension:

\begin{corollary}
  $\dim_{\mathbb{C}} M^{\dol}(E) = (g-1)r^2 + 1$.
  \begin{proof}
    By Dolbeault's theorem we may express the tangent space in terms of holomorphic
    sections as
    \begin{align*}
      T_{[\dol_E]}\mathcal M^{\dol}(E) = \Gamma(C,\End(E,\dol_E) \otimes T_{1,0}^*C).
    \end{align*}
    Hence, using sheaf cohomology,
    \missingproof
  \end{proof}
\end{corollary}

\section{To Do}

\begin{itemize}
  \item motivate Einstein connections
  \item show that action on Einstein connections yields closed and complemented orbits
  \item show that irreducible Einstein connections form submanifold
  \item show that tangent spaces to orbits are closed
  \item show that the moduli spaces do not depend on $\ell$
\end{itemize}

\missingsection

\chapter{As Schemes}

The moduli space of holomorphic bundles on a compact Riemann
surface maybe constructed as a complex manifold exclusively
using analytic methods. Since this space was first constructed
in \missingcitation
the theory of moduli spaces has come a long way.
The goal of this section is to formally define moduli spaces
as schemes representing functors, explain how such schemes
are constructed, and hence formally define the schemes whose
points correspond to holomorphic bundles and Higgs bundles
on a compact Riemann surface, respectively.

We begin by translating fundmental notions such as compact
Riemann surfaces, holomorphic bundles, and Higgs bundles into
the language of algebraic geometry. This allows us to state
precisely the moduli problems that we aim to solve. The construction
of the related moduli spaces is achieved in multiple steps which
mirror the analytic approach: Firstly, we observe that the
holomorphic bundles we are interested in may be thought of as
points of certain Quot schemes. Secondly, we define group
actions on these schemes whose orbits correspond to the
equivalence classes we are after. Finally, we use geometric
invariant theory (GIT) to take the quotients by these group
actions. While it is going to require a significant amount of
work to motivate and define the appropriate GIT quotients,
we will see that the same machinery can be used to solve
a plethora of similar problems without much extra work.

Once constructed, we will take some time to study the moduli
spaces which we have thus obtained. It turns out that the GIT
quotients have several desriable properties such as the being
quasi-projective varieties. Moreover, we will use deformation
theory to show that holomorphic bundles may still be thought
of as cotangent vectors of holomorphic bundles.

\paragraph*{Notation}

In addtion to the notation used in the previous chapters, unless
otherwise indicated,
\begin{itemize}
  \item all schemes, maps, and products are taken over $\mathbb{C}$,
        i.e.~in the slice category
        $\Sch := \Sch_{\mathbb{Z}}/\Spec\mathbb{C}$;
  \item a projective scheme is a projective scheme over $\mathbb{C}$
        in the sense of \cite{hartshorne1977} - that is, $X$ is projective
        if there exists a closed immersion $X\inc\projective{n}{}$ for
        some $n$;
  \item sheaves on a scheme $X$ are sheaves of $\mathscr O_X$-modules,
        maps of sheaves are $\mathscr O_X$-linear, and tensor products
        of sheaves are taken over $\mathscr O_X$.
\end{itemize}

\section{Holomorphic Bundles as Locally Free Sheaves}

\missingsection

\subsection{Compact Riemann Surfaces as Algebraic Curves}
\label{sec:surfaces_as_curves}

We are going to view moduli spaces of holomorphic bundles
as schemes. The first step towards this is translating the
base space. While it is not in general possible to view
every manifold as a scheme \missingcitation, it is a well
known fact that compact Riemann surfaces correspond to smooth
algebraic curves.
This fact is proven in various places, see \cite[215]{griffiths1994}
for the general idea and \cite[5-16]{harris2011}
for a comprehensive treatment.

It is worth recalling what the algebraic structure on a compact
Riemann surface $C$ of genus $g\geq 2$ is. The easiest way to obtain
this structure is by constructing an embedding into an ambient
scheme. The correct scheme to consider will be the projectivisation
of an $n$-dimensional complex vector space $V$ which is given
by
\begin{align*}
  \projective{}{}(V) := \Proj\left({
        \bigoplus_{d=0}^\infty \Sym^d(V^\vee)
      }\right)
\end{align*}
Complex points in $\projective{}{}(V)$ may be identified
with one-dimensional subspaces of $V$. \missingcitation
Now if $\mathscr L$ is
an invertible sheaf on $C$ then there is a natural map
\begin{align}\label{eq:natural_line_bundle_map}
  H^0(C,\mathscr L)\otimes\mathscr O_C \to \mathscr L
\end{align}
given by $s\otimes f \mapsto f\restrict{s}{U}$. We will be
particularly interested in the case where $\mathscr L$ is a
quotient of the free sheaf $H^0(C,\mathscr L)\otimes\mathscr O_C$.

\begin{definition}
  A sheaf $\mathscr F$ on a ringed space $X$ is
  \emph{globally generated} if the induced map
  $H^0(X,\mathscr F) \otimes \mathscr O_X \to \mathscr F$
  is a surjection.
\end{definition}

\todo{check notation here}
In particular, if $\mathscr L$ is globally generated then,
for every $x\in C$, it induces a surjection on stalks
$H^0(C,\mathscr L) \surj \mathscr L_x$ whose kernel is a
subspace of codimension 1, i.e. a closed point of
$\projective{}{}(H^0(C,\mathscr L)^\vee)$. It turns out that this map is an embedding
whenever $\deg\mathscr L > 2g$. \cite[Proposition 2.14]{harris2011}
All that is left to do is to find invertible sheaves with large
degree. Fortunately, for any invertible sheaf $\mathscr L$,
the tensor powers $\mathscr L^{\otimes m}$ are invertible and
$\deg\mathscr L^{\otimes m} = m\deg\mathscr L$. Hence any invertible
sheaf of positive degree will do. Fortunately, the degree of
the canonical sheaf $\Omega_C$ is $2g-2$. As we have restricted
our attention to the case $g\geq 2$, we are free to choose
$m\geq 2$ and hence $\mathscr L = \Omega_C^{\otimes m}$ to obtain
an embedding $C\inc\projective{}{}(H^0(C,\mathscr L)^\vee)$, as
required.
By Chow's theorem, this makes $C$ an algebraic subvariety
and hence an algebraic curve.

\subsection{Vector Bundles on Schemes}

As we are now able to regard our base space, a compact Riemann
surface, as a scheme, it makes sense to translate vector bundles
into this setting. It is not very difficult to come up with a
sensible definition based on the usual setting of manifolds:

\begin{definition}[{\cite[Definition 11.5]{gortz2010}}]
  \label{def:vector_bundle}
  A \emph{vector bundle} of rank $r$ on a scheme $X$ is
  a map of schemes $\pi : E \to X$ such that there is an open
  cover $X = \bigcup_i U_i$ and isomorphisms
  \begin{align}\label{eq:trivialisation}
    \phi_i : {\pi}^{-1}U_i \cong \affine{r}{}\times U_i
  \end{align}
  such that, for every affine $U = \Spec R \subseteq U_i \cap U_j$,
  the map $\phi_i \circ \phi^{-1}_j$ restricts to a $R$-linear
  automorphism of $\affine{r}{}\times U = \Spec R[T_1,\ldots,T_r]$.
\end{definition}

\begin{example}
  \begin{itemize}
    \item trivial bundle
    \item bundle on curve from previous section, e.g. $T^1$ or
          $T^1 \# T^1$.
  \end{itemize}
\end{example}

\begin{itemize}
  \item show that holomorphic bundles on a compact Riemann surface $C$
        are the same thing as vector bundles on $C$, regarded as a smooth algebraic curve and hence a scheme
\end{itemize}

\missingsection

\subsection{Locally Free Sheaves}\label{sec:locally_free_sheaves}

While it was easy to translate vector bundles from manifolds
to schemes, the resulting notion may not always be the most
useful. It is much more natural to talk about sheaves. We know
that with each vector bundle $E\to X$ comes a sheaf of sections
$\Gamma(-,E)$. This has a natural $\mathscr O_X$-module structure.

\begin{lemma}
  The sheaf of sections of a vector bundle $E$ on a scheme $X$ is
  locally free.
  \begin{proof}
    If $E=\affine{r}{}\times X$ is
    trivial then the sections $s\in\Gamma(U,E)$ are just maps
    $s:U\to \affine{r}{}$. If, moreover, $U$ is affine,
    $\Gamma(U,E)$ corresponds to maps of
    $\mathbb{C}$-algebras
    $\mathbb{C}[T_1,\ldots,T_n]\to\mathscr O_X(U)$ so
    $\Gamma(U,E)\cong \mathscr O^r_X(U)$, i.e. $\Gamma(-,E)$
    is free.
    More generally, for each element $U_i$ of the cover in
    \ref{def:vector_bundle}, we can consider affines $U\subseteq U_i$
    to find $\restrict{\Gamma(-,E)}{U_i}\cong \restrict{\mathscr O_X^r}{U_i}$,
    as required.
  \end{proof}
\end{lemma}

However, much more is true. It turns out that on a scheme $X$
a vector bundle of rank $n$ is the same thing as a locally
free sheaf of rank $n$. That is, there is an equivalence
of categories given by $E \mapsto \Gamma(-,E)^\vee$.~\cite[128-129]{hartshorne1977}
We will say that a sheaf $\mathscr E$ corresponds to a vector bundle
$E$ and write $\mathscr E=\mathcal V(E)$ to mean
$\mathscr E\cong\Gamma(-,E)^\vee$.

\subsection{Higgs Sheaves}

\missingsection

\subsection{Moduli Problems}

We have shown that holomorphic bundles on a compact Riemann surface
correspond to locally free sheaves on a certain smooth algebraic
curve. Now we would like to construct schemes whose points are
holomorphic bundles. By points of a scheme $M$ one usually means the
closed points. In the case where $M$ is locally of finite type,
closed points are $\mathbb{C}$-points, i.e. maps
$\Spec\mathbb{C}\to M$. \cite[Corollary 3.36]{gortz2010} As we
are hoping for our moduli spaces to be geometrically well-behaved,
this is a reasonable approximation to make. We are now justified in
thinking of a moduli problem as sending a scheme $T$ to the
$T$-points of a hypothetical moduli space.

\begin{definition}
  A \emph{moduli problem} is a functor $\mathcal M:\Sch^{op}\to\Set$.
\end{definition}

\begin{example}
  \missingexample
  \begin{itemize}
    \item lines in a vector space
    \item holomorphic bundles up to isomorphism
    \item Higgs bundles up to isomorphism
    \item some coarse example
  \end{itemize}
\end{example}

Now the obvious notion of a moduli space is a scheme whose functor
of points is precisely the moduli problem. This is called a fine
moduli space:

\begin{definition}
  A \emph{fine moduli space} of a moduli problem $\mathcal M$
  is a scheme $M$ with a natural isomorphism
  $\eta : \mathcal M \cong \Hom(-,M)$.
\end{definition}

Slightly abusing notation, we may simply write
$\eta : \mathcal M \cong M$. We are often going to surpress the
isomorphism $\eta$ and call $M$ the fine moduli space.

\begin{example}
  \begin{itemize}
    \item projectivisation/projective space
  \end{itemize}
\end{example}

Beyond the fact that a fine moduli spaces represent the moduli
problem and thus are the best possible solution, there are some
pleasant properties to observe. Consider an element
$F\in\mathcal M(T)$, for some $T$. Observe that under $\eta$,
this corresponds to a map $f:T\to M$ which induces pullback maps
$f^* : \Hom(M,M)\to\Hom(T,M)$ and hence
$F^* : \mathcal M(M)\to \mathcal M(T)$.
Now note that there is an element $U\in\mathcal M(M)$ that
corresponds to the identity on $M$. Thus $F^* U = F$, so every
element $\mathcal M(T)$ is given by pulling back $U$. Hence
we make the following definition:

\begin{definition}
  Let $\mathcal M$ be a moduli problem and $M$ a fine moduli space.
  The \emph{universal element} $U\in\mathcal M(M)$ is
  $U:=\eta^{-1}_M (\identity)$.
\end{definition}

\begin{example}
  \missingexample
\end{example}

However, it turns out that this is often too much to ask. Indeed,
we really care about the $\mathbb{C}$-points and hence
we are otherwise ready accept some deviation so long as it is not
possible to do better:

\begin{definition}\missingcitation
  A \emph{coarse moduli space} of a moduli problem $\mathcal M$
  is a scheme $M$ together with a natural transformation
  $\eta : \mathcal M \to \Hom(-,M)$ such that
  \begin{itemize}
    \item $\eta : \mathcal M(\Spec\mathbb{C})\to M(\Spec\mathbb{C})$ is a bijection;
    \item for every scheme $N$, every natural transformation
          $\nu : M \to \Hom(-,N)$ factors through $\eta$ as in the diagram
          \missingdefinition{universal property}
  \end{itemize}
\end{definition}

\begin{example}
  \missingexample
\end{example}

Unfortunately, even coarse moduli spaces may not exist. Here is
one property that prevents a moduli space from existing. We will
see an example of this later. \todo{add reference to problem of all
  locally free sheaves}

\begin{lemma}\label{lem:no_coarse_condition}
  Let $\mathcal M$ be a moduli problem and
  $x,y\in \affine{1}{}(\mathbb{C})$. If there is an
  $F\in\mathcal M(\affine{1}{})$ such that
  $\mathcal M(x)(F) = \mathcal M(y)(F)$ if, and only if,
  $x,y\neq 0$ or $x=y=0$ then there is no coarse moduli space
  for $\mathcal M$.
  \begin{proof}
    Following \cite[Lemma 2.27]{hoskins2016}.
    \missingproof
  \end{proof}
\end{lemma}

By construction, both kinds of moduli spaces are unique up to
isomorphism and every fine moduli space is a coarse moduli
space. Hence solving moduli problems usually proceeds in
three steps:

\begin{enumerate}
  \item Formulate the problem by defining the desired functor of
        points.
  \item Construct a coarse moduli space.
  \item Check under which conditions the moduli space is fine.
\end{enumerate}

\section{Topology of Sheaves on Schemes}

Vector bundles correspond to locally free sheaves. Hence it is worth
taking some time to study locally free sheaves and, more generally,
sheaves on schemes. This is of course a very wide field
so we are going to highlight only a few key properties.

The main goal of this section is to establish some topological
invariants of coherent sheaves, generalising the grouping of
vector bundles by rank and degree. The obvious tool for studying
the topology of sheaves is cohomology. While we will have little
to do with sheaf cohomology in its raw form, we are going to
observe several properties of sheaves on curves and, more generally,
projective schemes that are going to prove useful.

\subsection{Twisting}

We begin by revisiting one of the most important family of sheaves
on projective schemes. See e.g. \cite{gortz2010} or
\cite{hartshorne1977} for detailed treatments.

Consider a projective scheme $X$. Recall that we may write
$X=\Proj R$ for some $R=\mathbb{C}[T_0,\ldots,T_n]/I$ \cite[II Corollary 5.16]{hartshorne1977}. Such a scheme comes
equipped with a particularly important family of invertible sheaves
called Serre's twisting sheaves. To define them, recall that each
graded $R$-module $M$ defines a unique sheaf $\tilde M$ on $X$
that satisfies $\tilde M (D_+(f)) = M_{(f)}$ where $M_{(f)}$
denotes the homogeneous localisation $M$ at $f\in R$. This
correspondence between modules and sheaves allows the following
definition:

\begin{definition}[{\cite[13.4]{gortz2010}}]
  For $m\in\mathbb{Z}$, define the graded $R$-module $R(m)$ by
  $R(d)_d := R_{m+d}$ for all $d\in\mathbb{Z}$. \emph{Serre's
    twisting sheaf} is
  \begin{align*}
    \mathscr O_X(m) := \widetilde{R(m)}.
  \end{align*}
  More generally, for a sheaf $\mathscr F$ on $X$, define
  $\mathscr F(m) := \mathscr F \otimes \mathscr O_X(m)$.
\end{definition}
The notation $\mathscr O_X(m)$ is justified by the observation
$\mathscr O_X(m) \cong \mathscr O_X \otimes \mathscr O_X(m)$.
Of course, the main case of interest for us is
$R=\mathbb{C}[T_0,\ldots,T_n]$, i.e. $X=\projective{n}{}$.
In this case, the twisting sheaf corresponds to a line bundle,
i.e. is invertible. That is, it has rank 1.

\begin{proposition}[{\cite[Proposition 13.15]{gortz2010}}]
  \label{thm:twisiting_sheaf_invertible}
  If $R$ is finitely generated as a $R_0$-algebra, then each
  $\mathscr O_X(m)$ is an invertible sheaf.
  \begin{proof}
    \missingproof
  \end{proof}
\end{proposition}

Why are the twisting sheaves important to us? They are well-behaved
with respect to the parameters. In particular,
under the assumption of \ref{thm:twisiting_sheaf_invertible},
$(\mathscr F,m)\mapsto \mathscr \mathscr F(m)$ defines an action of
the integers on the group of sheaves on $X$:

\begin{lemma}\label{lem:additivity_twisting_sheaf}
  Let $R$ be finitely generated as a $R_0$-algebra and $m,n\in\mathbb{Z}$.
  Then $\mathscr O_X(m) \otimes \mathscr O_X(n) \cong \mathscr O_X(m + n)$. Hence
  $\mathscr F(m) \otimes \mathscr O_X(n) \cong \mathscr F(m+n)$.
  \begin{proof}
    \missingproof
  \end{proof}
\end{lemma}

We will see that topological properties of sheaves behave well
with respect to taking direct sums and tensor products and hence
with respect to twisting, too. Moreover, for suitable $R$ and
sufficiently large $m$, $\mathscr F(m)$ is globally generated. In this
case, for $m\geq 1$, the sheaves $\mathscr O_X(m)$ are ample.

\subsection{Ample sheaves}

Recall the definition of an ample sheaf:

\begin{definition}
  An invertible sheaf $\mathscr L$ on a quasi-compact quasi-separated
  scheme $X$ is \emph{ample} if, for all quasi-coherent sheaves of
  finite type $\mathscr F$, for $m$ sufficiently large,
  $\mathscr F\otimes \mathscr L^{\otimes m}$ is
  globally generated.
\end{definition}

On general schemes, ample sheaves need not exist. However, on
projective space the twisting sheaves serve as an example:

\begin{proposition}
  Let $R$ be a finitely generated $R_0$-algebra,
  write $X=\Proj R$ and let $m,n\geq 1$. Then
  $\mathscr O_X(m)$ is ample.
  \begin{proof}
    See \cite[Example 13.45]{gortz2010}.
  \end{proof}
\end{proposition}

More generally, any quasi-projective quasi-compact scheme has
an ample sheaf given by pulling back a twisted sheaf:

\begin{proposition}
  Let $X$ be projective, $U\subseteq X$ quasi-compact, and
  $j : U \inc X$ an open immersion then. Then, for some $m\geq 1$,
  $j^*\mathscr O_X(m)$ is ample.
  \begin{proof}
    See \cite[\href{https://stacks.math.columbia.edu/tag/01Q2}{Tag 01Q2}]{stacks-project}.
  \end{proof}
\end{proposition}

\begin{corollary}
  Every quasi-projective scheme has an ample sheaf.
\end{corollary}

\todo{motivate all this a little bit more}

\subsection{Euler Characteristic}

The fundamental topological invariant of sheaves that we are going
to be interested in is the Euler characteristic. This is defined
in the obvious way, replacing regular cohomology with sheaf
cohomology:

\begin{definition}
  The \emph{Euler characteristic} of a coherent sheaf $\mathscr F$
  on a projective scheme $X$ is
  \begin{align*}
    \chi (X,\mathscr F) := \sum_{j=0}^\infty (-1)^j \dim H^j (X,\mathscr F).
  \end{align*}
\end{definition}

Note that, by finite-dimensionality of coherent cohomology
(e.g. \cite[\href{https://stacks.math.columbia.edu/tag/02O6}{Tag 02O6}]{stacks-project}) and Grothendieck's vanishing theorem (e.g.
\cite[III Theorem 2.7]{hartshorne1977}), this is sum is finite for
all $\mathscr F$.

\begin{example}
  Our primary interest will be in the case where
  $\dim X = 1$, so $H^j(X,\mathscr F)=0$ for $j\geq 2$ and
  \begin{align*}
    \chi (X,\mathscr F) = \dim H^0(X,\mathscr F)-\dim H^1(X,\mathscr F).
  \end{align*}
  If we moreover take $\mathscr F=\mathscr O_X$ then
  $\chi (X,\mathscr O_X) = 1 - g$. \todo{justify}
\end{example}

\begin{example}
  Consider the case $X=\projective{n}{}$.
  From \missingcitation we know that
  % https://swc-math.github.io/notes/files/06StillmanNotes.pdf
  % https://achinger.impan.pl/fac/fac.pdf
  \begin{align*}
    \dim H^j(\projective{n}{},\mathscr O_{\projective{n}{}}(m)) =
    \begin{cases}
      (n+1)^m         & \text{if }j = 0 \\
      0               & \text{if }0<j<n \\
      \binom{-n-1}{m} & \text{if }j=n
    \end{cases}
  \end{align*}
  Hence \todo{doublecheck}
  \begin{align*}
    \chi(\projective{n}{},\mathscr O_{\projective{n}{}}(m))
    = (n+1)^m + \binom{-n-1}{m}.
  \end{align*}
\end{example}

\begin{lemma}
  \todo{additivity on short exact sequences}
\end{lemma}

\begin{lemma}
  \todo{invariance under base change}
\end{lemma}

\subsection{Hilbert Polynomials}

While the Euler characteristic certainly is a topological invariant,
it is rather limiting. Rather than focusing on a single integer,
the Euler characteristic,
we are going to keep track of all the Euler characteristics of all
the twists of a sheaf. This turns out to be described by a polynomial.

Consider a projective scheme $X$ with an ample sheaf $\mathscr L$
and a coherent sheaf $\mathscr F$.

\begin{definition}
  The \emph{Hilbert polynomial} of $\mathscr F$ at $m\in\mathbb{Z}$
  is
  \begin{align*}
    P(\mathscr F,\mathscr L)(m) := \chi(X,\mathscr F \otimes \mathscr L^{\otimes m}).
  \end{align*}
\end{definition}

A priori, the Hilbert polynomial is a function taking integers to
integers. The fact that it is described by a polynomial is
surprising and non-trivial.

\begin{lemma}
  There exists a unique polynomial $p\in\mathbb{Q}[t]$ such that,
  for all $m$, $P(\mathscr F,\mathscr L)(m) = p(m)$.
  \begin{proof}
    See \cite[Lemma 1.2.1]{huybrechts2010}.
  \end{proof}
\end{lemma}

\begin{example}
  \missingexample
\end{example}

Serre's vanishing theorem \missingcitation states that,
for $m$ sufficiently large and $j>0$,
$H^j(X,\mathscr F\otimes\mathscr L^{\otimes m})=0$. Hence,
eventually,
$P(\mathscr F,\mathscr L)(m) = \dim H^0(X,\mathscr F\otimes \mathscr L^{\otimes m})$.

\begin{lemma}
  \todo{invariance under base change}
\end{lemma}

\subsection{Degree}

While the Hilbert polynomial of a coherent sheaf is in general the
correct invariant to consider, we are working with locally free
sheaves on a curve. Hence some simplifications are in order.
Recall that for vector bundles on a curve we defined the degree.
A similar definition is possible for locally free sheaves:

\begin{definition}[{\cite[\href{https://stacks.math.columbia.edu/tag/0AYQ}{Tag 0AYQ}]{stacks-project}}]
  The \emph{degree} of a locally free sheaf $\mathscr F$ of rank $r$
  on $C$ is $\deg \mathscr F := \chi (C,\mathscr F) - r\chi(C,\mathscr O_C)$.
\end{definition}

\begin{example}
  As the rank of $\mathscr O_C$ is $1$ its degree must be $0$.
\end{example}

\begin{lemma}\label{lem:degree_of_tensor}
  If $\mathscr E$ and $\mathscr F$ are locally free on $C$ then
  \begin{align*}
    \deg(\mathscr E\otimes\mathscr F) = \rank\mathscr E\deg\mathscr F + \rank\mathscr F\deg\mathscr E.
  \end{align*}
  \begin{proof}
    \cite[Exercise 8.24]{hoskins2016}.
    \missingproof
  \end{proof}
\end{lemma}

\begin{example}
  Using~\ref{lem:degree_of_tensor}, we find \todo{make sure $\mathscr O_X(1)$ has degree 1}
  \begin{align*}
    \deg \mathscr E(m) = \deg\mathscr E + m\rank\mathscr E.
  \end{align*}
\end{example}

\missingsection

\section{Locally Free Sheaves as Points of Quot Schemes}

The first step in constructing the analytic moduli spaces of
holomorphic bundles was to consider the slightly larger space of
all bundles bundles of a certain rank and degree, including repeated
occurences of isomorphic instances. While this preliminary space
did not serve as a true
classification,
it provided us with a good starting point for future analysis. As the
algebraic construction follows a similar procedure, it is now
time to construct an analogous scheme.

We define a much more general moduli problem called
the Quot functor which was shown to have a fine moduli space by
Grothendieck. This is the first piece of machinery which will do a
lot of the work for us and for many other moduli problems of vector
bundles. We will show that there is a Quot scheme whose points
may be thought of as locally free sheaves of a certain rank and
degree.

\subsection{All Locally Free Sheaves}

Our approach to constructing moduli spaces of equivalence classes of
sheaves is to consider a larger space and then quotient by a suitable
group action. In the analytic case, we started with the affine space of
all holomorphic bundles. On the algebraic side, this does not quite work.
Let us see why that is.

The first step is to define the correct moduli problem. The key idea
is to associate to a scheme $T$ a family of locally free sheaves
$\mathscr E_t$. We may view a sheaf $\mathscr E$ on $C_T := C\times T$
as a family of sheaves on $C$ indexed by $T$. If we have a point $t\in T$
then we have the fibre $C_t := \Spec k(t) \times C$ and the sheaf
$\mathscr E_t := \restrict{\mathscr E}{C_t}$. In the case where
$t$ is a $\mathbb{C}$-point, $C_t \cong C$ so $\mathscr E_t$ is indeed
a sheaf on $C$. Moreover, if $\mathscr E$ is flat over $T$ then
each $\mathscr E_t$ has the same Hilbert polynomial.

To truly view locally free sheaves $\mathscr E,\mathscr F$ on
$C_T$ flat over $T$ as a family, we need to adjust our notion of
equivalence. In particular, we require all the fibres to be isomorphic.

\begin{lemma}
  Let $\mathscr E$ and $\mathscr F$ be locally free sheaves on $C_T$
  flat over $T$. Then $\mathscr E_t\cong\mathscr F_t$ for all $t\in T$
  if, and only if, there exists an invertible sheaf $\mathscr L$ on
  $T$ such that $\mathscr E \cong \mathscr F \otimes \pi^*\mathscr L$
  where $\pi : C_T \to C$ is the base change.
  \begin{proof}
    \todo{not sure this is true; if it isn't then we need to find a
      different explanation}
    \missingproof
  \end{proof}
\end{lemma}

Hence we make the following definition:

\begin{definition}
  Define the moduli problem $\mathcal A : \Sch^{\text{op}} \to \Set$
  bysending a scheme $T$ to
  \begin{align*}
    \mathcal A(T) := \left\lbrace{\text{locally free sheaves $\mathscr E$ on $C_T$ flat over $T$}}\right\rbrace/\sim
  \end{align*}
  where $\mathscr E\sim\mathscr F$ if, and only if, there is a line bundle
  $\mathscr L$ on $T$ such that
  $\mathscr E \cong \mathscr F\otimes\pi^* \mathscr L$. Maps
  $f: T'\to T$ get sent to the pullback map
  $\mathscr E\mapsto f^*\mathscr E$.
\end{definition}

However, this fails to have any moduli space at all:

\begin{lemma}\label{lem:no_coarse_moduli_space}
  The moduli problem $\mathcal A$ of locally free sheaves on $C$
  does not admit a coarse moduli space.
  \begin{proof}
    Follwing \cite[Example 2.2]{hoskins2016}. We aim to construct
    an $\mathscr E\in\mathcal A(\affine{1}{})$ that satisfies the
    condition in \ref{lem:no_coarse_condition}.
    \missingproof
  \end{proof}
\end{lemma}

Let us think about the case of Higgs sheaves. Consider the moduli problem
given by \todo{add new equivalence; make sure the canonical sheaf restricts to a canonical sheaf}
\begin{align*}
  \mathcal A_H(T) = \left\lbrace{\text{Higgs sheaves $(\mathscr E,\phi)$
      on $X_T$}}\right\rbrace/\sim.
\end{align*}
where $(\mathscr E,\phi)\sim(\mathscr E',\phi')$ if, and only if,
there is an isomorphism $\mathscr E\cong\mathscr E'$ that makes the
following commute:
\begin{equation*}
  % https://q.uiver.app/#q=WzAsNCxbMCwwLCJcXG1hdGhjYWwgRSJdLFswLDEsIlxcbWF0aGNhbCBFJyJdLFsyLDEsIlxcbWF0aGNhbCBFJ1xcb3RpbWVzXFxPbWVnYV4xX3tDXFx0aW1lcyBUfSJdLFsyLDAsIlxcbWF0aGNhbCBFXFxvdGltZXMgXFxPbWVnYV4xX3tDXFx0aW1lcyBUfSJdLFswLDMsIlxccGhpIl0sWzEsMiwiXFxwaGknIl0sWzAsMSwiXFxjb25nIiwyXSxbMywyLCJcXGNvbmciXV0=
  \begin{tikzcd}
    {\mathscr E} && {\mathscr E\otimes \Omega^1_{C\times T}} \\
    {\mathscr E'} && {\mathscr E'\otimes\Omega^1_{C\times T}}
    \arrow["\phi", from=1-1, to=1-3]
    \arrow["\cong"', from=1-1, to=2-1]
    \arrow["\cong", from=1-3, to=2-3]
    \arrow["{\phi'}", from=2-1, to=2-3]
  \end{tikzcd}
\end{equation*}
Once again, by restricting to fibres over $t\in T(\mathbb{C})$
a family of Higgs sheaves $(\mathscr E,\phi)$ on $X_T$ yields
a Higgs field
\begin{align*}
  \phi_t : \mathscr E_t \to \mathscr E_t \otimes \Omega^1_C.
\end{align*}
Unfortunately, we find ourselves in a similar position.
\begin{corollary}
  The moduli problem $\mathcal A_H$ of Higgs sheaves on $C$ does not admit
  a coarse moduli space.
  \begin{proof}
    Consider the natural transformation $\mathcal A\to\mathcal A_H$
    given by $\mathscr E \mapsto (\mathscr E,0)$. Now the sheaf
    $\mathscr E\in\mathcal A(\affine{1}{})$ from the proof of
    \ref{lem:no_coarse_moduli_space} pushes forward to a Higgs field
    $(\mathscr E,0)\in\mathcal A_H(\affine{1}{})$ which, moreover,
    satisfies the same property.
  \end{proof}
\end{corollary}
Thus it is not possible to construct the moduli spaces we desire
in the naive way. There are essentially two ways to get around this.
One option is to deal with
moduli stacks instead. This has the advantage of solving the problem
without discarding any information. However, it involves dealing with
algebraic stacks which are unwieldy, even more so than schemes.
See \cite{cm2017} for this point of view. We will instead take an
approach that is much in line with the analytic construction by narrowing
our attention to a subclass of locally free sheaves. Indeed, we are
going to end up with those sheaves corresponding to (semi-)stable
bundles.

\subsection{Quotients}

If a sheaf $\mathscr F$ is globally generated and
$\dim H^0(X,\mathscr F) = N$ then $\mathscr F$ may be thought
of as a quotient of the free sheaf $\mathscr O_X^{\oplus N}$
by choosing an isomorphism
\begin{align*}
  H^0(X,\mathscr F) \otimes \mathscr O_X \cong \mathbb{C}^N \otimes \mathscr O_X \cong \mathscr O_X^{\oplus N}.
\end{align*}
It will turn out that being globally generated will be a reasonable
condition in the sense that we are not missing too many vector
bundles. Thus it is sensible to study quotients of coherent sheaves.

\begin{definition}
  Let $\mathscr E$ be a coherent sheaf on a scheme $X$.
  A \emph{family of quotients} of $\mathscr E$ over a scheme $T$
  consists of a sheaf $\mathscr F$ on $X_T$ flat over $T$ and a
  surjection $q:\pi^*\mathscr E\surj\mathscr F$ where
  $\pi : X_T \to X$ is the base change.
\end{definition}

Note the condition that the quotient be flat. Recall that a sheaf
$\mathscr F$ on $X_T$ is flat over $T$ if each of the functors
$M \mapsto \mathscr F_{(x,t)} \otimes_{\mathscr O_{T,t}} M$ is exact.
In light of \cite[III Theorem 9.9]{hartshorne1977} this is a
particularly useful condition for us as it ensures that the fibres
$\mathscr F_t$ all have the same Hilbert polynomial. Hence we may
refer to the Hilbert polynomial $P(q,\mathscr L)$ of the family
$q:\pi^*\mathscr E\surj\mathscr F$.

\begin{example}
  We saw earlier that a globally generated locally free sheaf
  $\mathscr F$ of rank $r$ and degree $d$ on $C$ together with
  a choice of isomorphism $\mathbb{C}^N \cong H^0(C,\mathscr F)$
  give rise to a family of quotients
  \begin{align*}
    H^0(C,\mathscr F)\otimes\mathscr O_C \cong \mathscr O^{\oplus N}_C
    \surj \mathscr F
  \end{align*}
  over $\mathbb{C}$ where $N := \dim H^0(X,\mathscr F)$.
\end{example}

Consider two quotients $\mathscr F$ and $\mathscr F'$ of a
sheaf $\mathscr E$.
Such quotients are equivalent if there is an isomorphism
$\mathscr F\cong\mathscr F'$ such that the following commutes
\begin{equation}\label{eq:quotient_equivalence}
  % https://q.uiver.app/#q=WzAsMyxbMCwwLCJcXG1hdGhjYWwgRSJdLFsyLDAsIlxcbWF0aGNhbCBGIl0sWzIsMSwiXFxtYXRoY2FsIEYnIl0sWzEsMiwiXFxjb25nIl0sWzAsMSwicSIsMCx7InN0eWxlIjp7ImhlYWQiOnsibmFtZSI6ImVwaSJ9fX1dLFswLDIsInEnIiwyLHsic3R5bGUiOnsiaGVhZCI6eyJuYW1lIjoiZXBpIn19fV1d
  \begin{tikzcd}
    {\mathscr E} && {\mathscr F} \\
    && {\mathscr F'}
    \arrow["q", two heads, from=1-1, to=1-3]
    \arrow["{q'}"', two heads, from=1-1, to=2-3]
    \arrow["\cong", from=1-3, to=2-3]
  \end{tikzcd}
\end{equation}

We would now like to consider the moduli problem of equivalence
classes of quotients of a coherent sheaf $\mathscr E$ on $X$.
We note that base change preserves quotients. In particular,
if $q : \mathscr E \surj \mathscr F$ is a quotient and
$f : Y \to X$ is a base change then
$f^*q : f^*\mathscr E \surj f^*\mathscr F$ is a quotient
by \cite[\href{https://stacks.math.columbia.edu/tag/01U9}{Tag 01U9}]{stacks-project}.

\subsection{Quot functors}

Thus we may consider the moduli problem of equivalence classes of
quotients of a given sheaf:

\begin{definition}\missingcitation
  Let $\mathscr E$ be a coherent sheaf on a projective scheme $X$.
  The corresponding \emph{Quot functor} is given by sending each
  scheme $T$ to
  \begin{align*}
    \mathcal Q(\mathscr E)(T) = \left\lbrace{
      \text{quotients $q : \pi^*\mathscr E \surj \mathscr F$
        over $T$}
    }\right\rbrace / \sim
  \end{align*}
  and $f : T' \to T$ to
  $\mathcal Q(\mathscr E)(f) = (\identity \times f)^*$ where
  $\pi^*\mathscr E$ is the pullback of $\mathscr E$ along
  the base change $\pi:X_T\to X$.
\end{definition}

Now note that this functor splits
\begin{align*}
  \mathcal Q(\mathscr E)
  = \bigsqcup_{P\in\mathbb{Q}[t]} \mathcal Q(\mathscr E)^{P,\mathscr L}.
\end{align*}
where each $\mathcal Q(\mathscr E)^{P,\mathscr L}$ contains families
with Hilbert polynomial $P$.

\begin{example}
  Recall on $C$, we have
  \begin{align*}
    P(\mathscr F,\mathscr O_C(1))(t) = rt + d + r(1-g)
  \end{align*}
  for any locally free sheaf $\mathscr F$ of rank $r$ and degree $d$.
  Thus, for any coherent sheaf $\mathscr E$, we have a splitting
  \begin{align*}
    \mathcal Q(\mathscr E) = \bigsqcup_{r,d} \mathcal Q(\mathscr E)^{r,d}.
  \end{align*}
  \missingexample
\end{example}

\subsection{Quot schemes}

One of the reasons the Quot functor is of interest to us is because
Grothendieck constructed a moduli space for it. \missingcitation

\begin{theorem}
  Let $X$ be a projective scheme with an ample sheaf $\mathscr L$,
  let $\mathscr E$ be a coherent sheaf on $X$, and let
  $p\in\mathbb{Q}[t]$. Then the Quot functor
  $\mathcal Q(\mathscr E)^{p,\mathscr L}$ has a fine moduli space
  $Q(\mathscr E)$ called the \emph{Quot scheme}.
  \begin{proof}
    For an extensive overview see \cite{hoskins2016}.
  \end{proof}
\end{theorem}

\begin{example}
  Once again, in the case $X=C$ we have schemes
  $Q(\mathscr E)^{r,d}$.
  \missingexample
\end{example}

\missingsection

\section{Geometric Invariant Theory}

The second major step in constructing moduli spaces of holomorphic
bundles, algebraic or analytic, is to take a quotient by an automorphism
group. In particular, we noticed that different choices of isomorphism
$H^0(X,\mathscr F)\cong\mathbb{C}^N$ give rise to different quotients.
The goal of this section is to eliminate these redundancies.

The first step is defining group actions in the setting of schemes.
This allows us to make precise what we mean by a quotient with
respect to such an action and what properties we would like such
quotients to satisfy. To construct such quotients we are going to
use a very powerful tool called geometric invariant theory (GIT).
In the affine case, the GIT quotients will not be very difficult
to obtain. However, a lot of work will be required to generalise the
construction sufficiently to suit our needs. We will find that
GIT naturally gives rise to a notion of stability which coincides
with the stability of vector bundles.

\subsection{Group Schemes}

\begin{definition}
  A \emph{group scheme} is a group object in $\Sch$.
\end{definition}
In particular, a group scheme consists of a scheme $G$ and three maps
\begin{align}\label{eq:group_maps}
  e:\Spec\mathbb{C}\to G,\hspace{1cm}
  i:G\to G,\hspace{1cm}
  m:G\times G\to G
\end{align}
satisfying the usual conditions of unitality, inverses, and associativity.
As usual with classical groups, we are going to leave the maps
$e$, $i$, and $m$ implicit.

\begin{example}\label{ex:group_schemes}
  In the affine case $G = \Spec R$ we may specify the multiplication
  and inverse maps in terms of $\mathbb{C}$-algebra homomorphisms
  $m^* : R \to R\otimes R$ and $i^* : R\to R$.
  There are several affine group schemes that we are already
  intuitively familiar with:
  \begin{enumerate}
    \item The additive group $\mathbf{G}_a := \Spec k[t]$ with
          comulitiplication $t\mapsto t\otimes 1 + 1\otimes t$,
    \item The multiplicative group $\mathbf{G}_m := \Spec k[t^\pm]$
          with comultiplication $t\mapsto t\otimes t$,
    \item The general linear group
          \begin{align*}
            GL_n := \Spec k[x_{ij} : 1\leq i,j\leq n][1/\det(x_{ij})]
          \end{align*}
          where $(x_{ij})$ is the $n\times n$ matrix with entries $x_{ij}$
          with comultiplication
          \begin{align*}
            x_{ij} \mapsto \sum_{k=1}^n x_{ik}\otimes x_{kj}.
          \end{align*}
          See e.g. \cite[\href{https://stacks.math.columbia.edu/tag/022W}{Tag 022W}]{stacks-project} for details.
    \item The sepcial linear group $SL_N$ analogous to the above by
          quotienting by $\det(x_{ij})^2 - 1$.
  \end{enumerate}
\end{example}
But what do these group schemes have to do with their well-known
counterparts. For example, how does the scheme $GL_n$ relate to the
group $GL_n(\mathbb{C})$? The notation is no coincidence.
Group schemes induce group structures on their points:
\begin{lemma}
  Let $G$ be a scheme with maps (\ref{eq:group_maps}). The following
  are equivalent:
  \begin{enumerate}
    \item $G$ is a group scheme.
    \item For every scheme $T$, $G(T)$ is a group.
  \end{enumerate}
  \begin{proof}

    \missingproof
  \end{proof}
\end{lemma}

\begin{example}
  The induced grous of the affine group schemes behave as expected:
  For a $\mathbb{C}$-algebra $R$, $\mathbf{G}_a(R) = (R,+)$,
  $\mathbf{G}_m(R) = (R^\times,\times)$, and
  $GL_n(R)$ is the usual group of invertible $n\times n$ matrices
  with coefficients in $R$.

  In more detail, consider $x\in\mathbf{G}_a(R)$ where $R$ is
  a $\mathbb{C}$-algebra. Such an $x$ is given by algebra homomorphsims
  $x^\sharp:\mathbb{C}[t] \to R$. That is, we may identify
  $\mathbf{G}_a(R)$ with $R$ using the map $x \mapsto x^\sharp(t)$. Now,
  for $x,y\in\mathbf{G}_a(R)$ and $f\in \mathbb{C}[t]$, we have the
  induced group multiplication given by
  \begin{align*}
    (xy)^\sharp(f)
    = x^\sharp(f) \cdot 1 + 1 \cdot y^\sharp(f)
    = x^\sharp(f) + y^\sharp(f).
  \end{align*}
  Thus the induced group structure on $\mathbf{G}_a(R)$ is just $(R,+)$.
\end{example}

\begin{definition}
  A group scheme is \emph{algebraic} if it is smooth and separated
  over $\mathbb{C}$.
\end{definition}

\begin{example}
  All the group schemes in \ref{ex:group_schemes} are affine algebraic
  groups. (See \cite[IV Theorem 9,3]{milne2012} for smoothness
  and \cite[Remark 3.2]{hoskins2016} for separatedness.) While group
  schemes are worth studying their full generality, the affine algebraic
  case will be sufficient for our purposes. Hence the only examples
  that the reader should have in mind are the ones already presented.
\end{example}

\subsection{Actions}

For this section, fix a scheme $X$ and a group scheme $G$.

\begin{definition}
  An \emph{action} of $G$ on $X$ is a morphism
  $\sigma : G\times X\to X$ satisfying the usual laws with respect
  to $m:G\times G\to G$.
\end{definition}

Note that an action $\sigma : G\times X\to X$ induces an action of
$G(T)$ on $X(T)$ by
\begin{equation*}
  T \xlongrightarrow{(g,x)} G\times X \xlongrightarrow{\sigma} X.
\end{equation*}

\begin{example}
  On locally free sheaves.
  \missingexample
\end{example}

\begin{example}
  On Higgs sheaves.
  \missingexample
\end{example}


\subsection{Invariants}

The points of a quotient ought to correspond to orbits of the group
action. In other words, the quotient map should be invariant.
Fortunately, we are able to define

\begin{definition}
  Let $G$ be a group scheme that acts on $X,Y$, respectively.
  A morphism $f:X\to Y$ is \emph{$G$-invariant} if the following
  commutes:
  \begin{equation*}
    % https://q.uiver.app/#q=WzAsMyxbMCwwLCJHXFx0aW1lcyBYIl0sWzIsMCwiWCJdLFs0LDAsIlkiXSxbMCwxLCJcXHJobyIsMSx7ImN1cnZlIjotMn1dLFswLDEsIlxccGkiLDEseyJjdXJ2ZSI6Mn1dLFsxLDIsImYiLDFdXQ==
    \begin{tikzcd}
      {G\times X} && X && Y
      \arrow["\sigma"{description}, curve={height=-12pt}, from=1-1, to=1-3]
      \arrow["\pi"{description}, curve={height=12pt}, from=1-1, to=1-3]
      \arrow["f"{description}, from=1-3, to=1-5]
    \end{tikzcd}
  \end{equation*}
\end{definition}

The obvious challenge of taking quotients in geometric settings
is that there is not just a topology but also a geometric
structure to take into account. A quotient $X/G$ should satisfy the
property that any $G$-invariant $X\to Z$ uniquely factors through
$X\surj X/G$:
\begin{equation*}
  % https://q.uiver.app/#q=WzAsNCxbMiwwLCJYIl0sWzQsMCwiWC9HIl0sWzQsMSwiWiJdLFswLDAsIkdcXHRpbWVzIFgiXSxbMCwxLCIiLDAseyJzdHlsZSI6eyJoZWFkIjp7Im5hbWUiOiJlcGkifX19XSxbMCwyXSxbMSwyLCJcXGV4aXN0cyEiLDAseyJzdHlsZSI6eyJib2R5Ijp7Im5hbWUiOiJkYXNoZWQifX19XSxbMywwLCJcXHNpZ21hIiwwLHsiY3VydmUiOi0yfV0sWzMsMCwiXFxwaSIsMix7ImN1cnZlIjoyfV1d
  \begin{tikzcd}
    {G\times X} && X && {X/G} \\
    &&&& Z
    \arrow["\sigma", curve={height=-12pt}, from=1-1, to=1-3]
    \arrow["\pi"', curve={height=12pt}, from=1-1, to=1-3]
    \arrow[two heads, from=1-3, to=1-5]
    \arrow[from=1-3, to=2-5]
    \arrow["{\exists!}", dashed, from=1-5, to=2-5]
  \end{tikzcd}
\end{equation*}
Considering structure sheaves, we get the following picture:
\begin{equation*}
  % https://q.uiver.app/#q=WzAsNCxbMiwwLCJcXG1hdGhjYWwgT19YIl0sWzQsMCwiXFxtYXRoY2FsIE9fe1gvR30iXSxbNCwxLCJcXG1hdGhjYWwgT19aIl0sWzAsMCwiXFxtYXRoY2FsIE9fR1xcb3RpbWVzXFxtYXRoY2FsIE9fWCJdLFsxLDAsIiIsMix7InN0eWxlIjp7InRhaWwiOnsibmFtZSI6Imhvb2siLCJzaWRlIjoiYm90dG9tIn19fV0sWzIsMF0sWzIsMSwiIiwyLHsic3R5bGUiOnsiYm9keSI6eyJuYW1lIjoiZGFzaGVkIn19fV0sWzAsMywiXFxzaWdtYV5cXHNoYXJwIiwyLHsiY3VydmUiOjJ9XSxbMCwzLCJcXHBpXlxcc2hhcnAiLDAseyJjdXJ2ZSI6LTJ9XV0=
  \begin{tikzcd}
    {\mathscr O_G\otimes\mathscr O_X} && {\mathscr O_X} && {\mathscr O_{X/G}} \\
    &&&& {\mathscr O_Z}
    \arrow["{\sigma^\sharp}"', curve={height=12pt}, from=1-3, to=1-1]
    \arrow["{\pi^\sharp}", curve={height=-12pt}, from=1-3, to=1-1]
    \arrow[hook', from=1-5, to=1-3]
    \arrow[from=2-5, to=1-3]
    \arrow[dashed, from=2-5, to=1-5]
  \end{tikzcd}
\end{equation*}

Now we have a good idea what the subsheaf
$\mathscr O_{X/G}\inc\mathscr O_X$ should be:
\begin{definition}
  Let $\sigma : G\times X \to X$ be an action of an affine algebraic
  $G$ on a scheme $X$. The \emph{ring of invariants}
  on an affine open $U\subseteq X$ is the subring of $\mathscr O_X(U)$
  given by
  \begin{align*}
    \mathscr O_X^G(U) :=
    \left\lbrace{f \in \mathscr O_X(U) : \sigma^\sharp(f) = 1 \otimes f}\right\rbrace.
  \end{align*}
  Thus we have a sheaf $\mathscr O_X^G$ on $X$.
\end{definition}

\begin{example}
  \missingexample
\end{example}

\subsection{Quotients}

Consider a group scheme $G$ acting on $X$. What do we mean by a
quotient of $X$ by the $G$-action? We are looking for a $G$-invariant
surjective map $f : X\surj Y$ such that $Y$ contains as much of th
geometrical information of $X$ as possible.

\todo{explain better why these conditions are relevant}

\begin{definition}[{\cite[Definition 4.2.2]{huybrechts2010}}]
  A \emph{good quotient} of $X$ by $G$-action is a map
  $f : X\surj Y$ such that the following hold:
  \todo{fix definition}
  \begin{enumerate}
    \item $f$ is affine.
    \item $f$ is surjective and $Y$ has the quotient topology.
    \item The map $f^\sharp : \mathscr O_Y \to \mathscr O_X$
          is an isomorphism $\mathscr O_Y \cong \mathscr O_X^G$.
    \item For every closed $G$-invariant $V\subseteq X$,
          $f(V)$ is closed in $Y$.
    \item For all disjoint closed $G$-invariant $V,V'\subseteq X$,
          $f(V)$ and $f(V')$ are disjoint.
  \end{enumerate}
\end{definition}

\begin{definition}
  geometric quotient
\end{definition}

\subsection{Reductive Groups}

\missingsection

\subsection{Of Affine Schemes}

\missingsection

\subsection{By Linear Actions}

\missingsection

\subsection{By Linearised Actions}

\missingsection

\chapter{Analytification}

\missingsection

\chapter{Future Work}

\missingsection

\pagebreak
\renewcommand{\bibname}{References}
\addcontentsline{toc}{chapter}{References}
\printbibliography

\end{document}
