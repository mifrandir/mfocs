\documentclass[12pt]{ociamthesis}  % default square logo
%\documentclass[12pt,beltcrest]{ociamthesis} % use old belt crest logo
%\documentclass[12pt,shieldcrest]{ociamthesis} % use older shield crest logo

\usepackage{dissertation}

%input macros (i.e. write your own macros file called mymacros.tex
%and uncomment the next line)
%\include{mymacros}

\title{Moduli Spaces of Holomorphic Bundles}   %note \\[1ex] is a line break in the title

\author{Franz Miltz}             %your name
\college{Lady Margaret Hall}  %your college

%\renewcommand{\submittedtext}{change the default text here if needed}
\degree{Master of Science}     %the degree
\degreedate{2024}         %the degree date

%end the preamble and start the document
\begin{document}

%this baselineskip gives sufficient line spacing for an examiner to easily
%markup the thesis with comments
\baselineskip=18pt plus1pt

%set the number of sectioning levels that get number and appear in the contents
\setcounter{secnumdepth}{3}
\setcounter{tocdepth}{3}


\maketitle                  % create a title page from the preamble info
\begin{dedication}
  To somebody
\end{dedication}
\begin{acknowledgements}

\end{acknowledgements}
\begin{abstract}

\end{abstract}

\begin{romanpages}          % start roman page numbering
  \tableofcontents            % generate and include a table of contents
\end{romanpages}            % end roman page numbering

\chapter{Introduction}

\chapter{As Manifolds}

\chapter{As Schemes}

The moduli space of holomorphic bundles on a compact Riemann
surface maybe constructed as a complex manifold exclusively
using analytic methods. Since this space was first constructed
in \missingcitation
the theory of moduli spaces has come a long way.
The goal of this section is to formally define moduli spaces
as schemes representing functors, explain how such schemes
are constructed, and hence formally define the schemes whose
points correspond to holomorphic bundles and Higgs bundles
on a compact Riemann surface, respectively.

We begin by translating fundmental notions such as compact
Riemann surfaces, holomorphic bundles, and Higgs bundles into
the language of algebraic geometry. This allows us to state
precisely the moduli problems that we aim to solve. The construction
of the related moduli spaces is achieved in multiple steps which
mirror the analytic approach: Firstly, we observe that the
holomorphic bundles we are interested in may be thought of as
points of certain Quot schemes. Secondly, we define group
actions on these schemes whose orbits correspond to the
equivalence classes we are after. Finally, we use geometric
invariant theory (GIT) to take the quotients by these group
actions. While it is going to require a significant amount of
work to motivate and define the appropriate GIT quotients,
we will see that the same machinery can be used to solve
a plethora of similar problems without much extra work.

Once constructed, we will take some time to study the moduli
spaces which we have thus obtained. It turns out that the GIT
quotients have several desriable properties such as the being
quasi-projective varieties. Moreover, we will use deformation
theory to show that holomorphic bundles may still be thought
of as cotangent vectors of holomorphic bundles.

\paragraph*{Notation}

Before we begin, let us establish some notation. All schemes
will be over $\mathbb{C}$, i.e. elements of the slice category
$\Sch := \Sch_{\mathbb{Z}}/\Spec\mathbb{C}$. Unless otherwise
indicated, products of schemes are fibre products
$X\times Y := X\times_{\Spec\mathbb{C}} Y$, and
tensor products are taken with respect to $\mathbb{C}$.
By a sheaf on a scheme $X$ we mean a sheaf of $\mathcal O_X$-modules.

\section{Vector Bundles as Points of Quot Schemes}

\missingsection

\subsection{Compact Riemann Surfaces as Algebraic Curves}

We are going to view moduli spaces of holomorphic bundles
as schemes. The first step towards this is translating the
base space. While it is not in general possible to view
every manifold as a scheme \missingcitation, it is a well
known fact that compact Riemann surfaces correspond to smooth
algebraic curves.
This fact is proven in various places, see \cite[215]{griffiths1994}
for the general idea and \cite[5-16]{harris2011}
for a comprehensive treatment.

It is worth recalling what the algebraic structure on a compact
Riemann surface $C$ of genus $g\geq 2$ is. The easiest way to obtain
this structure is by constructing an embedding
$X\inc\projective{N}{}$ for some $N$. To do this, recall
the projectivisation of an $n$-dimensional complex vector
space $V$ is given by
\begin{align*}
  \projective{}{}(V) := \Proj\left({
      \bigoplus_{d=0}^\infty \Sym^d(V^\vee)
  }\right)
\end{align*}
Complex points in $\projective{}{}(V)$ are commonly identified
with one-dimensional subspaces of $V$. Now if $\mathcal L$ is
an invertible sheaf on $C$ then there is a natural map
\begin{align}\label{eq:natural_line_bundle_map}
  H^0(C,\mathcal L)\otimes\mathcal O_C \to \mathcal L
\end{align}
given by $s\otimes f \mapsto f\restrict{s}{U}$. If this map
(\ref{eq:natural_line_bundle_map}) is a surjection then,
for every $x\in C$, it induces a surjection on stalks
$H^0(C,\mathcal L) \surj \mathcal L_x$ whose kernel is a
subspace of codimension 1, i.e. an element of
$\projective{}{}(H^0(C,\mathcal L)^\vee)$. Now one may choose an
isomorphism $\projective{}{}(H^0(C,\mathcal L)^\vee)\cong\projective{N}{}$ where $N = \dim H^0(C,\mathcal L)$ to get a map
$C\to\projective{N}{}$. It turns out that this map is an embedding
whenever $\deg\mathcal L > 2g$. \cite[Proposition 2.14]{harris2011}
All that is left to do is to find invertible sheaves with large
degree. Fortunately, for any invertible sheaf $\mathcal L$,
$\deg\mathcal L^{\otimes m} = m\deg\mathcal L$. Hence any line
bundle of positive degree will do. Fortunately, the degree of
the canonical bundle $\Omega_C$ is $2g-2$. As we have restricted
our attention to the case $g\geq 2$, we can choose
$\mathcal L = \Omega_C^{\otimes m}$ for $m \geq 2$ to obtain
an embedding $C\inc\projective{}{}(H^0(C,\mathcal L))$ as required.
By Chow's theorem, this makes $C$ an algebraic subvariety
and hence an algebraic curve.

We will not be moving in the other direction, but it is worth
mentioning that this is indeed possible. In particular, there
is an embedding $C\inc\projective{3}{}$ and

\missingsection

\subsection{Vector Bundles on Schemes}

As we are now able to regard our base space, a compact Riemann
surface, as a scheme, it makes sense to translate vector bundles
into this setting. It is not very difficult to come up with a
sensible definition based on the usual setting of manifolds:

\begin{definition}[{\cite[Definition 11.5]{gortz2010}}]
  \label{def:vector_bundle}
  A \emph{vector bundle} of rank $r$ on a scheme $X$ is
  a map of schemes $\pi : E \to X$ such that there is an open
  cover $X = \bigcup_i U_i$ and isomorphisms
  \begin{align}\label{eq:trivialisation}
    \phi_i : {\pi}^{-1}U_i \cong \affine{r}{}\times U_i
  \end{align}
  such that, for every affine $U = \Spec R \subseteq U_i \cap U_j$,
  the automorphism $\phi_i \circ \phi^{-1}_j$
  of $\affine{r}{}\times U = \Spec R[T_1,\ldots,T_r]$
  corresponds to an $R$-linear map.
\end{definition}

\begin{example}
  \begin{itemize}
    \item trivial bundle
    \item bundle on curve from previous section, e.g. $T^1$ or
      $T^1 \# T^1$.
  \end{itemize}
\end{example}

\missingsection

\subsection{Locally Free Sheaves}

While it was easy to translate vector bundles from manifolds
to sheaves, the resulting notion may not always be the most
useful. It is much more natural to talk about sheaves. We know
that to each vector bundle $E\to X$ comes a sheaf of sections
$\mathcal E := \Gamma(-,E)$. This has a natural $\mathcal O_X$-module structure.

\begin{lemma}
  The sheaf $\mathcal E$ corresponding to a vector bundle $E\to X$
  is locally free.
  \begin{proof}
    If $E=\affine{r}{}\times X$ is
    trivial then the sections $s\in\mathcal E(U)$ are just maps
    $s:U\to \affine{r}{}$. If, moreover, $U$ is affine,
    $\mathcal E(U)$ corresponds to maps of $\mathbb{C}$-algebras
    $\mathbb{C}[T_1,\ldots,T_n]\to\mathcal O_X(U)$ so
    $\mathcal E(U)\cong \mathcal O^r_X(U)$, i.e. $\mathcal E$
    is free.
    More generally, for each element $U_i$ of the cover in
    \ref{def:vector_bundle}, we have
    $\restrict{E}{U_i}\cong \restrict{\mathcal O_X^r}{U_i}$, as
    required.
  \end{proof}
\end{lemma}

\missingsection

\subsection{The Quot Scheme}

\missingsection

\section{GIT Quotients}

\missingsection

\subsection{Group Schemes}

\missingsection

\subsection{Reductive Groups}

\missingsection

\subsection{Of Affine Schemes}

\missingsection

\subsection{By Linear Actions}

\missingsection

\subsection{By Linearised Actions}

\missingsection

\chapter{Analytification}

\missingsection

\chapter{Future Work}

\missingsection

\addcontentsline{toc}{chapter}{Bibliography}
\renewcommand{\bibname}{References}
\printbibliography

\end{document}
