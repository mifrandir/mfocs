\documentclass[12pt]{ociamthesis}  % default square logo
%\documentclass[12pt,beltcrest]{ociamthesis} % use old belt crest logo
%\documentclass[12pt,shieldcrest]{ociamthesis} % use older shield crest logo

\usepackage{dissertation}

%input macros (i.e. write your own macros file called mymacros.tex
%and uncomment the next line)
%\include{mymacros}

\title{Analytic and Algebraic\\[1ex]Moduli Spaces of Vector Bundles}   %note \\[1ex] is a line break in the title

\author{Franz Miltz}             %your name
\college{Lady Margaret Hall}  %your college

\renewcommand{\submittedtext}{A dissertation submitted for the degree of}
\degree{Master of Science}     %the degree
\degreedate{August 2024}         %the degree date

\addbibresource{references.bib}

%end the preamble and start the document
\begin{document}

%this baselineskip gives sufficient line spacing for an examiner to easily
%markup the thesis with comments
\baselineskip=18pt plus1pt

%set the number of sectioning levels that get number and appear in the contents
\setcounter{secnumdepth}{3}
\setcounter{tocdepth}{3}


\maketitle                  % create a title page from the preamble info
\begin{dedication}
  \todo{dedication}
\end{dedication}

\begin{acknowledgements}
  \todo{acknowledgements}
\end{acknowledgements}

\begin{abstract}
  \todo{abstract}
\end{abstract}

\begin{romanpages}          % start roman page numbering
  \tableofcontents            % generate and include a table of contents
\end{romanpages}            % end roman page numbering

\chapter{Introduction}

\begin{itemize}
  \item Riemann surfaces
  \item complex geometry at the intersection between the analytic and algebraic settings
\end{itemize}

\todo{make sure that every section introduces all the objects we are thinking about}

\chapter{As Manifolds}

Consider the set of holomorphic bundles on a compact Riemann surface,
up to isomorphism. What can we say about this set? What additional
structure does it have? How does this structure change if we equip
the bundles with additional data?
In this chapter we are going to construct complex manifolds whose
points correspond to equivalence classes of holomorphic bundles
and Higgs bundles, a type of decorated holomorphic bundle. These
manifolds are referred to as moduli spaces. While it
is possible to construct topological spaces representing all such bundles,
the result is non-Hausdorff and hence does not provide us with a
satisfying answer. The majority of the chapter is going to be spent
on rectifying this issue.

The key observation will be to ignore certain `unstable' bundles. We
are going to find that doing so does not lose much information but
allows for the construction of the moduli spaces. Although unstable
bundles are the main obstacle, removing them does not trivialise the
problem. To obtain the moduli spaces, we have to consider quotients
of infinite dimensional manifolds by group actions. Such considerations
require us to introduce the theory of Banach manifolds.

The first part of this chapter is going to be spent discussing the kinds of
objects that we aim to classify, i.e. holomorphic bundles and Higgs
bundles on compact Riemann surfaces. We recall some facts and establish
notation. From then on we work towards constructing the moduli spaces.
The first step is to establish how one would do so naively and
see why this approach fails. Secondly, we establish a correspondence
between holomorphic bundles and unitary connections. After establishing the necessary
theory we endow the topological spaces of all unitary connections
with an infinite dimensional manifold structure. Thirdly, we establish
conditions under which we may take quotients of infinite dimensional
manifolds by group actions. This allows us to find submanifolds on which
those conditions are met and thus define the analytic moduli spaces of
unitary actions. Finally we establish some key properties of the
moduli spaces by characterising membership by a more natural
condition and computing tangent spaces.

\paragraph*{Notation}

Before we begin, let us establish some notation. Unless otherwise
indicated,

\begin{itemize}
  \item all vector spaces are complex, maps of vector spaces are
        $\mathbb{C}$-linear, and tensor products of vector spaces are
        over $\mathbb{C}$;
  \item all manifolds are smooth;
  \item dimensions are taken over $\mathbb{C}$;
  \item the word `bundle' refers to a vector bundle.
\end{itemize}

\section{Holomorphic Bundles on Compact Riemann Surfaces}

Holomorphic maps are smooth. Hence, holomorphic vector bundles
are smooth vector bundles. In contrast to maps, in the case of vector
bundles the word `holomorphic' does not refer to a property, but
a structure. In particular, there may be many holomorphic bundles with
the same underlying smooth bundle.

\subsection{Complex Vector Bundles}

Recall the definition of a smooth bundle on a complex
manifold:

\begin{definition}\label{def:complex_bundle}
  Let $X$ be a complex manifold of dimension $n$. A
  \emph{smooth vector bundle} $E$ of rank $r$ on $X$ is a smooth map
  $\pi : E\to X$ such that there
  is an open cover $X = \bigcup_i U_i$ and diffeomorphisms
  \begin{equation}\label{eq:smooth_trivialisation}
    \phi_i : {\pi}^{-1}U_i \cong \mathbb{C}^r \times U_i
  \end{equation}
  such that, for every chart $U\subseteq U_i\cap U_j$, the
  map $\phi_i \circ {\phi_j}^{-1}$ restricts to a linear
  automorphism of $\mathbb{C}^r\times U \cong \mathbb{C}^{r+n}$.
\end{definition}

\begin{example}
  A complex manifold $X$ of dimension $n$ may be regarded as a
  real manifold of dimension $2n$. Any smooth real vector bundle
  $E$ on $X$ corresponds to a \emph{complexified} smooth bundle
  $E_{\mathbb{C}} := E\otimes_{\mathbb{R}} \mathbb{C}$ of equal rank.
  In particular, we may complexify the real vector bundles
  $TX$ and $T^*X$ to obtain smooth vector bundles
  $T_{\mathbb{C}} X$ and $T_{\mathbb{C}}^*X$
  on $X$ of rank
  \begin{align*}
    \rank(T_\mathbb{C} X)
    = \rank_{\mathbb{R}} (TX) = 2n.
  \end{align*}
\end{example}

Now a holomorphic bundle is a smooth bundle where the trivialisations
(\ref{eq:smooth_trivialisation}) are biholomorphic. As holomorphic
maps are smooth, each holomorphic bundle has an underlying smooth
structure.

\begin{example}
  smooth bundle with multiple holomorphic structures
  \missingexample
\end{example}

\begin{example}
  Recall that a complex manifold $X$ of dimension $n$ induces a
  complex structure on the real smooth rank $2n$ vector bundle $TX$.
  That is, there is a map of smooth real vector bundles
  $J : TX \to TX$ such that $J^2 = -1$. Note that this extends to
  an automorphism of the complexification $TX_{\mathbb{C}}$.
  This defines a splitting into subbundles
  \begin{align}\label{eq:tangent_decomposition}
    TX_{\mathbb{C}}  = TX^{1,0} \oplus TX^{0,1}
  \end{align}
  whose fibres are eigenspaces of $i$ and $-i$, respectively. The
  complex bundle $TX^{1,0}$ of rank $n$ is called the
  \emph{holomorphic tangent bundle} of $X$. Indeed, if we regard $TX$
  as a complex vector bundle via $J$ then the map
  \begin{align}\label{eq:holomorphic_tangent_bundle}
    TX
    \longrightarrow TX_{\mathbb C}
    = TX^{1,0} \oplus TX^{0,1}
    \longrightarrow TX^{1,0}
  \end{align}
  is a $\mathbb{C}$-linear isomorphism and hence induces a holomorphic
  structure on $TX$, justifying the name.
\end{example}

One is usually interested in what kinds of sections a vector bundle
admits. Indeed, in the next chapter we are going to make extensive use
of the fact that a vector bundle is uniquely determined by its sheaf
of sections. In the complex case, we need to be careful about the
kind of sections we are referring to:

\begin{itemize}
  \item If $E\to X$ is a smooth vector bundle then $\Gamma^\infty(U,E)$
        denotes the vector space of smooth sections on $U$.
  \item If $\mathcal E\to X$ is holomorphic then $\Gamma(U,\mathcal E)$
        denotes the vector space of holomorphic sections on $U$.
\end{itemize}

Note that if $\mathcal E$ is holomorphic then it is also smooth.
However, there are some important differences between
$\Gamma^\infty(U,\mathcal E)$ and $\Gamma(U,\mathcal E)$. While both are
complex vector spaces $\Gamma^\infty(U,\mathcal E)$ is typically infinite
dimensional while $\Gamma(U,\mathcal E)$ is always finite
dimensional.~\cite[Theorem 1.4.1]{ma2007}

\subsection{Dolbeault Cohomology}
\missingcitation

Consider a complex manifold $X$ of dimension $n$. Recall the de Rahm
complex
\begin{align*}
  \cdots \xlongrightarrow{d}
  \Omega_{\mathbb{R}}^{k-1}(X)\xlongrightarrow{d}
  \Omega_{\mathbb{R}}^{k}(X)\xlongrightarrow{d}
  \Omega_{\mathbb{R}}^{k+1}(X)\xlongrightarrow{d}
  \cdots
\end{align*}
of the underlying real manifold where
$\Omega^0_{\mathbb{R}}(X) = C^\infty(X,\mathbb{R})$ and,
for $k > 0$,
\begin{align*}
  \Omega^k_{\mathbb{R}}(X) := \Gamma^\infty(X,\Lambda^k T^*X).
\end{align*}
We may complexify this to obtain
$\Omega^k_{\mathbb C}(X) := \Omega^k_{\mathbb R}(X)\otimes_{\mathbb{R}} \mathbb{C}$.
The splitting $T^*X_\mathbb{C} = T^*X^{1,0} \oplus T^*X^{0,1}$
leads us to define
\begin{align*}
  \Omega^{p,q}(X)
  := \Gamma^\infty(X,\Lambda^p T^*X^{1,0} \wedge\Lambda^q T^*_{0,1}X)
\end{align*}
and hence we have
\begin{align*}
  \Omega^k_{\mathbb C}(X) = \bigoplus_{p+q=k} \Omega^{p,q}(X).
\end{align*}
Moreover, the operator $d : \Omega^{p,q}(X) \to \Omega^{p+q+1}_{\mathbb C}(X)$
splits into $d = \partial + \dol$ where
\begin{align*}
  \partial : \Omega^{p,q}(X) \to \Omega^{p+1,q}(X),\hspace{1cm}
  \dol : \Omega^{p,q}(X) \to \Omega^{p,q+1}(X).
\end{align*}
Locally, these are given by
\begin{align*}
  \partial (f dz_I \wedge d\bar z_J) = \sum_{i=1}^n \frac{\partial f}{\partial z_i} dz_i\wedge dz_I \wedge d\bar z_J, \\
  \dol (f dz_I \wedge d\bar z_J) = \sum_{j=1}^n \frac{\partial f}{\partial \bar z_j} d\bar z_j\wedge dz_I \wedge d\bar z_J.
\end{align*}
In particular, we now have a family of complexes
\begin{align}\label{eq:dolbeault_complex}
  \cdots \xlongrightarrow\dol
  \Omega^{p,q-1}(X)\xlongrightarrow\dol
  \Omega^{p,q}(X)\xlongrightarrow\dol
  \Omega^{p,q+1}(X)\xlongrightarrow\dol
  \cdots
\end{align}
\begin{definition}
  The cohomology of the complex \ref{eq:dolbeault_complex} is
  called \emph{Dolbeault cohomology}. To be precise, the
  $(p,q)$ Dolbeault cohomology group is
  \begin{align*}
    H^{p,q}(X) := \frac{
      \ker(\dol : \Omega^{p,q}(X) \to \Omega^{p,q+1}(X))
    }{
      \im(\dol : \Omega^{p,q-1}(X) \to \Omega^{p,q}(X))
    }.
  \end{align*}
\end{definition}

\subsection{Dolbeault Cohomology of Holomorphic Bundles}

For a smooth bundle $E$ of rank $r$, consider the spaces of
$E$-valued $k$-forms and $(p,q)$-forms, respectively,
\begin{align*}
  \Omega^k(X,E)     & := \Omega^k_{\mathbb C}(X)\otimes_{C^\infty(X,\mathbb C)}\Gamma^\infty(X,E), \\
  \Omega^{p,q}(X,E) & := \Omega^{p,q}(X)\otimes_{C^\infty(X,\mathbb C)}\Gamma^\infty(X,E).
\end{align*}
If we are given a holomorphic bundle $\mathcal E$ with underlying
smooth bundle $E$
it is straightforward to verify that the operator $\dol$ extends
to an operator
\begin{align}\label{eq:general_dolbeault_operator}
  \dol_E : \Omega^{p,q}(X,E) \to \Omega^{p,q+1}(X,E)
\end{align}
by considering local trivialisations where we have
\begin{align*}
  \dol_E \left({\sum_{i=1}^k \xi_i \otimes s_i}\right)
  = \sum_{i=1}^k \dol \xi_i \otimes s_i.
\end{align*}
Moreover, the operator $\dol_E$ satisfies
\begin{align*}
  \dol_E(\xi \otimes s) = \dol\xi \otimes s + (-1)^{p+q} \xi \wedge \dol_E s.
\end{align*}
Hence it is uniquely determined by the component
$\Omega^0(X,\mathcal E) \to \Omega^{0,1}(X,\mathcal E)$. Indeed,
any linear map $\alpha : \Omega^0(X,E) \to \Omega^{0,1}(X,E)$ may be
extended to a map $\Omega^{p,q}(X,E) \to \Omega^{p,q+1}(X,E)$
via
\begin{align*}
  \alpha(\xi \otimes s) = \dol\xi \otimes s + (-1)^{p+q}\xi\wedge \alpha(s).
\end{align*}
This leads us to the following definition:

\begin{definition}
  A \emph{Dolbeault operator} on a smooth bundle $E$ on a complex
  manifold $X$ is a linear operator
  \begin{align*}
    \dol_E : \Omega^0(X,E) \to \Omega^{0,1}(X,E)
  \end{align*}
  such that $\dol_E^2 = 0$ and, for all $s\in\Omega^0(X,E)$ and
  $f\in \Omega^0_{\mathbb C}(X)$,
  \begin{align*}
    \dol_E (fs) = \dol f\otimes s + f \dol_E s.
  \end{align*}
\end{definition}

The process outlined above is reversible. That is, a holomorphic
bundle is just a smooth bundle with a Dolbeault operator. To make
this precise, we need to say what morphisms of such operators are.
Consider a map $f : E \to F$ of smooth bundles on a fixed complex
manifold $X$. This induces a map
$f_*:\Gamma^\infty(X,E)\to\Gamma^\infty(X,F)$
given by $s \mapsto f\circ s$. Hence $f$ is a map of Dolbeault operators
if, and only if, it commutes with the operators:

\begin{definition}
  A \emph{morphism of Dolbeault operators} $f : \dol_E \to \dol_F$
  is a map of smooth bundles $f : E\to F$ such that the following commutes:
  \begin{equation*}
    % https://q.uiver.app/#q=WzAsNCxbMCwwLCJcXE9tZWdhXjAoWCxFKSJdLFswLDEsIlxcT21lZ2FeMChYLEYpIl0sWzIsMSwiXFxPbWVnYV57MCwxfShYLEYpIl0sWzIsMCwiXFxPbWVnYV57MCwxfShYLEUpIl0sWzAsMywiXFxiYXJcXHBhcnRpYWxfRSJdLFsxLDIsIlxcYmFyXFxwYXJ0aWFsX0YiXSxbMCwxLCJmXyoiLDJdLFszLDIsImZfKiIsMl1d
    \begin{tikzcd}
      {\Omega^0(X,E)} && {\Omega^{0,1}(X,E)} \\
      {\Omega^0(X,F)} && {\Omega^{0,1}(X,F)}
      \arrow["{\bar\partial_E}", from=1-1, to=1-3]
      \arrow["{f_*}"', from=1-1, to=2-1]
      \arrow["{f_*}"', from=1-3, to=2-3]
      \arrow["{\bar\partial_F}", from=2-1, to=2-3]
    \end{tikzcd}
  \end{equation*}
\end{definition}

\begin{theorem}
  Fix a complex manifold $X$. There is an equivalence of categories
  between holomorphic bundles on $X$ and smooth bundles with corresponding
  Dolbeault operators.
  \begin{proof}
    This is essentially \cite[Theorem 3.2]{moroianu2004}. One only
    needs to verify that a map of smooth bundles is a map of
    holomorphic bundles if, and only if, it respects the holomorphic
    structure. \todo{maybe we should do this proof?}
  \end{proof}
\end{theorem}

Thus we are justified in referring to a pair $(E,\dol_E)$ as a
holomorphic vector bundle. In particular, we have the trivial
equality $\Gamma^\infty(X,(E,\dol_E)) = \Gamma^\infty(X,E)$.
Note that the Dolbeault operator induced by a holomorphic bundle is
unique only up to isomorphism. Fortunately, taking cohomologies
is functorial and thus we obtain the Dolbeault cohomology of
holomorphic vector bundles:

\begin{definition}
  Let $(E,\dol_E)$ be a holomorphic vector bundle on $X$. The
  \emph{$(p,q)$ Dolbeault cohomology group of $X$ with coefficients in
    $E$} is
  \begin{align*}
    H^{p,q}(X,(E,\dol_E)) := \frac{
      \ker(\dol_E : \Omega^{p,q}(X,E) \to \Omega^{p,q+1}(X,E))
    }{
      \im(\dol_E : \Omega^{p,q-1}(X,E) \to \Omega^{p,q}(X,E))
    }.
  \end{align*}
\end{definition}

As many cohomology theories, this is agrees with sheaf cohomology
in some sense. In particular, we have Dolbeault's theorem:

\begin{theorem}[Dolbeault]\label{thm:dolbeault}
  Let $(E,\dol_E)$ be a holomorphic vector bundle on a complex manifold
  $X$ of dimension $n$. Then
  \begin{align*}
    H^{p,q}(X,(E,\dol_E)) = H^q(X,(E,\dol_E)\otimes K^{\otimes p})
  \end{align*}
  where $K$ is the \emph{canonical bundle}
  $K_X := \bigwedge^n T_{1,0}^* X$.
\end{theorem}

\subsection{Higgs Bundles}

\todo{write about why Higgs bundles exist, what they're good for, yadda
  yadda}

\begin{definition}
  Let $(E,\dol_E)$ be a holomorphic bundle on a complex manifold $X$.
  A \emph{Higgs field} on $(E,\dol_E)$ is a holomorphic section
  $\varphi \in \Omega^{1,0}(X,\End(E))$ satisfying $\varphi\wedge\varphi = 0$.
  A \emph{Higgs bundle} is a holomorphic bundle equipped with
  a Higgs field.
\end{definition}
We immediately observe that, in the case where $X=C$ is a compact
Riemann surface, the condition $\varphi\wedge\varphi=0$ is trivially
satisfied. However, if one is to consider moduli spaces of Higgs bundles
over more general base spaces then this condition is quite important.

Observe that a Higgs field is a section of $\End(E) \otimes K$.
Recalling the trivial fact that $H^0(X,(E,\dol_E))=\Gamma(X,(E,\dol_E))$
in sheaf cohomology, we use Dolbeault's theorem \ref{thm:dolbeault} to
find that the space of Higgs bundles is
\begin{align*}
  \Gamma(X,\End(E,\dol_E)\otimes K) = H^{1,0}(X,\End(E,\dol_E)\otimes K).
\end{align*}

Consider Higgs bundles $(E,\dol_E,\phi_E)$ and $(F,\dol_F,\phi_F)$
and a map of smooth bundles $f:E\to F$. We say $f$ is a map of Higgs
bundles if, and only if, it is a map of holomorphic bundles
$f:(E,\dol_E)\to (F,\dol_F)$ and $\phi_E = f^*\phi_F$.
This means in particular that, if
$(E,\dol_E,\phi_E)\cong(F,\dol_F,\phi_F)$
are isomorphic Higgs bundles then we have isomorphic holomorphic bundles
$(E,\dol_E)\cong(F,\dol_F)$ and hence isomorphic smooth bundles
$E\cong F$.

\section{Spaces of Holomorphic Bundles}

After a brief introduction to holomorphic bundles on Riemann surfaces,
we are now ready to contemplate the spaces that such bundles might
form. The first observation we are going to make is that the case of
smooth bundles is very straightforward. To do this, we are going to
introduce some topological invariants of smooth bundles.

This allows us to construct a topological space of holomorphic structures
on a fixed smooth bundle. Unfortunately, after taking the quotient
of said space by isomorphism of holomorphic structures, we obtain a
non-Hausdorff topological space.

\subsection{Chern Classes}

We noticed that in order for two holomorphic bundles to be isomorphic
they must have the same underlying smooth bundles. Hence it is worth
investigating when smooth bundles are isomorphic. Fix a complex
manifold $X$.

The first observation we make is that, if we have isomorphic smooth
bundles $E\cong E'$ then $\rank E = \rank E'$. However, the converse
is not true. To see this it suffices to consider any non-trivial
bundle and compare it to the trivial bundle of equal rank. E.g.
$TS^2$ will do.

This leads us to consider another invariant: Chern classes. For
every smooth bundles $E$ on $X$ of rank $r$ there are elements
$c_j(E)\in H^{2j}(X,\mathbb{Z})$ for $j\geq 0$ called the Chern
classes of $E$. There are different ways of constructing these classes,
e.g. \cite{fine2013} and \cite{griffiths1994}, for our purposes
it is sufficient to know a few key properties:

\begin{lemma}\missingcitation
  Let $E$ be a smooth bundle on $X$ of rank $r$. Then
  \begin{enumerate}
    \item $c_j(E) = 0$ whenever $j<1$ or $j>r$.
    \item If $F$ is a smooth bundle on $X$ such that $E\cong F$
          then $c_i(E) = c_i(F)$ for all $i$.
    \item For every other smooth bundle $F$ on $X$,
          \begin{align*}
            c_i(E\oplus F) = \bigoplus_{j=0}^{i} c_j(E)\cup c_{i-j}(F).
          \end{align*}
  \end{enumerate}
\end{lemma}

\subsection{Degree}

Our goal is to classify holomorphic bundles on a compact Riemann surface,
i.e. a compact connected complex manifold of dimension 1. Hence we
ought to consider this case more closely. To this end, fix a compact
Riemann surface $C$.

\todo{genus}

Note that the real manifold of dimension $2$ underlying $C$ is orientable. \missingcitation
Hence the cohomology group $H^2(X,\mathbb{Z})$ is generated by the
fundamental class $[C]$.

\begin{definition}
  The \emph{degree} of a smooth bundle $E$ on $C$ is the integer
  \begin{align*}
    \deg E = c_1(E)\cdot[C].
  \end{align*}
\end{definition}

\begin{theorem}
  Smooth bundles on $C$ are isomorphic if, and only if, they have
  equal rank and degree.
  \begin{proof}
    \missingproof
  \end{proof}
\end{theorem}

\missingsection

\subsection{The Naive Quotients}

Having established holomorphic bundles on compact Riemann surfaces,
we are ready to contemplate the spaces that they form. While it is
quite straightforward to obtain topological spaces of equivalence
classes of holomorphic bundles, the result is not Hausdorff.
This extends to decorated holomorphic bundles such as Higgs bundles.

In light of previous results, fix a smooth bundle $E$ of rank $r$
and degree $d$ on a compact Riemann surface $C$.
Recall that a holomorphic bundle with underlying smooth structure $E$
is given by a Dolbeault operator $\dol_E$. Hence we have a set of
\begin{align*}
  A_\dol (E) := \left\lbrace{
    \text{Dolbeault operators $\dol_E : \Omega^0(C,E) \to \Omega^{0,1}(C,E)$}
  }\right\rbrace.
\end{align*}
This has an obvious topology given by the following observations:
\begin{lemma}\label{lem:affine_space_of_dolbeault_operators}
  Consider any two linear maps \todo{$C^\infty$-linear?}
  \begin{align*}
    \dol_E,\alpha : \Omega^0(C,E) \to \Omega^{0,1}(C,E).
  \end{align*}
  Then $\dol_E\in A_\dol (E)$ if, and only if, $\dol_E+\alpha\in A_\dol (E)$.
  \begin{proof}
    We calculate in local coordinates:
    \begin{align*}
      (\dol_E + \alpha)^2(s_i)
       & = \sum_{j,k} (\dol_{ijk} + \alpha_{ijk})(\dol_E + \alpha)( d\bar z_j\otimes s_k)                                                                                     \\
       & = \sum_{j,k} (\dol_{ijk} + \alpha_{ijk})(\dol(d\bar z_j) \otimes s_k -  d\bar z_j \wedge \sum_{j',k'}(\dol_{k,j',k'} + \alpha_{k,j',k'}) d\bar z_{j'}\otimes s_{k'}) \\
       & = - \sum_{j,j',k,k'} (\dol_{ijk} + \alpha_{ijk})(\dol_{k,j',k'} + \alpha_{k,j',k'}) d\bar z_j \wedge d\bar z_{j'}\otimes s_{k'}                                      \\
       & = - \sum_{j,j'}
      d\bar z_j \wedge d\bar z_{j'}\otimes \left(
      \sum_{k,k'} (\dol_{ijk} + \alpha_{ijk})(\dol_{k,j',k'} + \alpha_{k,j',k'}) s_{k}\right)                                                                                 \\
       & = 0
    \end{align*}
    Moreover, for $f\in C^\infty(C)$ and $s\in\Omega^0(C,E)$,
    \begin{align*}
      (\dol_E + \alpha)(fs) = f\dol_E(s) + f\alpha(s) + \dol f \otimes s
      = f(\dol_E+\alpha)(s) + \dol f \otimes s.
    \end{align*}
    Hence $\dol_E + \alpha$ satisfies both properties of Dolbeault
    operators.
  \end{proof}
\end{lemma}

Thus we may write $A_\dol (E) = \dol_E + \Omega^{0,1}(C,\End(E))$
for some fixed $\dol_E$. This induces a topology on $A_\dol (E)$.
In particular, $A_\dol (E)$ is an affine space modelled on
$\Omega^{0,1}(C,\End(E))$.

This allows us to define the space of all Higgs bundles
\begin{align*}
  A_\varphi(E) := \bigsqcup_{\dol_E \in A_\dol (E)} \Gamma(C,\End(E,\dol_E)\otimes K_C)
\end{align*}
whose topology is induced by the topologies on
$\Gamma(C,\End(E,\dol_E)\otimes K_C)$ and $A_\dol (E)$.
All that is left to do is take a quotient by equivalence of holomorphic
bundles, respectively Higgs bundles. One of the major topics of this
paper is to explore various ways in which we may take quotients by
group actions. Thus we ought to start phrasing things in this language.

The group we are interested in is the group $\Aut(E)$ of automorphisms
of $E$. Note that an automorphism of $E$ is the same thing as
an invertible section of $\End(E)$. Hence consider the group
of everywhere invertible sections of $\End(E)$:
\begin{align*}
  GL(E) := \{A \in \Gamma^\infty(C,\End(E)) : \forall x \in C.\:\det A_x \neq 0\}
\end{align*}
Elements $g\in GL(E)$ are usually called gauge transformations and
$GL(E)$ as a whole is referred to as the gauge group.
The gauge group acts on $A_\dol(E)$ by conjugation.
Unfortunately, the quotient leaves a lot to be desired:
\begin{lemma}
  The topological space $A_\dol (E)/GL(E)$ of isomorphism classes of
  Dolbeault operators is not Hausdorff. In particular, it does not
  admit a manifold structure.
  \begin{proof}
    \missingproof
  \end{proof}
\end{lemma}

Note that we have a similar action of $GL(E)$ on $A_\varphi(E)$ and
indeed many other topological spaces of holomorphic bundles. However,
it is quite clear that any space that arises as such a disjoint union
over $A_\dol (E)$ must also be non-Hausdorff. Therefore we need
to use more sophisticated techniques to construct any moduli
space of holomorphic bundles.

\section{From Bundles to Connections}

The actions of the complex gauge group on the spaces of holomorphic
bundles are badly behaved. We aim to rectify this by temporarily
changing our point of view from holomorphic bundles to unitary
connections. It turns out that the corresponding topological spaces
are diffeomorphic and that the complex gauge group action on holomorphic bundles
may be interpreted as a real gauge group action on unitary connections.

Once again fix a smooth bundle $E$ of rank $r$ on a compact
Riemann surface $C$.

\subsection{Connections}

Connections are going to help us in two ways: Firstly, they are going to
be to correspond to Dolbeault operators and hence lead to an
alternative point of view on spaces of holomorphic bundles.
Secondly, they are going to allow us to define the Sobolev norm and
hence provide us with a suitable Banach manifold structure on
the moduli spaces.

Recall that connections enable us to differentiate sections of
vector bundles:

\begin{definition}
  A \emph{connection} in a smooth complex bundle $E$ on
  a complex manifold $X$ is a $C^\infty(X,\mathbb C)$-linear
  map $\nabla : \Omega^0(X,E) \to \Omega^1(X,E)$
  such that
  \begin{align*}
    \nabla (fs) = df \otimes s + f\nabla s,
  \end{align*}
  for all $f\in \Omega^0_{\mathbb C}(X)$ and $s\in\Omega^0(X,E)$.
\end{definition}

\begin{example}\label{ex:connection_on_end}
  If we are given connections $\nabla$ and $\nabla'$ in bundles $E$ and $E'$ on $X$,
  respectively, then we obtain a connection in the bundle $E\otimes E'$ via
  \begin{align*}
    \nabla \otimes \identity + \identity \otimes \nabla'.
  \end{align*}
  Similarly, there is a connection $\nabla^*$ in the dual bundle $E^*$ given by
  \begin{align*}
    d\langle \xi,s\rangle
    = \langle \nabla^*\xi, s \rangle + \langle \xi,\nabla s\rangle
  \end{align*}
  where $\langle\cdot,\cdot\rangle$ denotes the natural pairing $E^*\times E\to\mathbb{C}$.
  In particular, we obtain a connection $\End\nabla$ in $\End(E) = E^* \otimes E$.
\end{example}

Note that connections look very similar to Dolbeault operators.
Indeed, via projections to the $(1,0)$ and $(0,1)$ parts, respectively,
we obtain a splitting $\nabla = \nabla_{1,0} + \nabla_{0,1}$
where
\begin{align*}
  \nabla_{0,1} : \Omega^0(X,E) \to \Omega^{0,1}(X,E).
\end{align*}
We may ask whether this is the same as a holomorphic structure on $E$:

\begin{definition}
  Let $\dol_E$ be a Dolbeault operator on $E$.
  A connection $\nabla$ in $E$ is \emph{compatible with $\dol_E$}
  if, and only if, $\nabla_{0,1} = \dol_E$. In general, we say
  $\nabla$ is \emph{holomorphic} if it is compatible with some
  $\dol_E$.
\end{definition}

Note that all connections $\nabla$ satisfy the Leibniz rule. It is
straightforward to check that
\begin{align*}
  \nabla_{0,1} (fs) = \dol f\otimes s + f \nabla_{0,1} s.
\end{align*}
Hence $\nabla$ is holomorphic if, and only if, $(\nabla_{0,1})^2 = 0$.

\subsection{Unitary Connections}

Holomorphic connections are very closely related to Dolbeault operators,
but they are not quite the same. To map one to the other, we are going
to require some additional information. The correct geometric
structure to consider is a metric:

\begin{definition}
  A \emph{Hermitian metric} on $E$ is a smooth section $h$ of
  $(E\otimes\bar E)^*$ that is sesquilinear and positive definite.
  That is, $h(s,\bar t) = \overline{h(t,\bar s)}$ and $h(s,\bar s) > 0$
  if $s \neq 0$.
\end{definition}

While such metrics are interesting in their own right, we have no
intention to study them more than absolutely necessary. Indeed, it will
become apparent that the mere presence of a metric is sufficient.
Fortunately for us, it will always be possible to choose a metric:

\begin{theorem}\label{thm:hermitian_structures_exist}\missingcitation
  Every smooth vector bundle on a compact complex manifold admits a
  Hermitian metric.
\end{theorem}

Fix such a metric $h$ on $E$. We are going to restrict our attention to
connections that are compatible with this metric. We call such
connections unitary because they correspond to connections
of the unitary principal bundle corresponding to $(E,h)$.

\begin{definition}
  A connection $\nabla$ in $E$ is \emph{$h$-unitary}
  if, and only if, for all $s,t\in\Omega^0(X,E)$,
  \begin{align}\label{eq:unitary_connection}
    dh(s,\bar t) = h(\nabla s,\bar t) + h(s,\overline{\nabla t}).
  \end{align}
\end{definition}

\begin{lemma}
  Every Hermitian bundle $(E,h)$ admits a $h$-unitary connection.
  \begin{proof}
    Choose any connection $\nabla$ and take
    $\nabla' := \frac{1}{2}(\nabla + \nabla^\dagger)$
    where the adjoint is taken in the sense of
    (\ref{eq:unitary_connection}).
  \end{proof}
\end{lemma}

Now consider the set of all connections that respect $h$ and induce
some Dolbeault operator:
\begin{align*}
  A_\nabla(E,h) := \left\lbrace{\text{$h$-unitary holomorphic connections $\nabla$ in $E$}}\right\rbrace.
\end{align*}
Analogous to $A_\dol (E)$, this is an affine space.
As the connections are required to be unitary, their differences
are given by sections of the real smooth bundle $\End(E,h)$ of
skew-Hermitian endomorphisms. Just as the fibres of $\End(E)$
are Lie algebras of the automorphism group $\Aut(E_x)\cong GL_r(\mathbb C)$,
the fibres of $\End(E,h)$ are Lie algebras of $\Aut(E_x,h)\cong U(r)$.

\begin{lemma}
  $A_\nabla(E,h)$ is an affine space modelled on
  $\Omega^1_{\mathbb{R}}(C,\End(E,h))$.
  \begin{proof}
    The operations are induced by $\Omega^1(C,\End(E))$.
    Verifying that general $\nabla + \xi$ for
    $\xi\in\Omega^1_{\mathbb{R}}(C,\End(E,h))$
    satsify the Leibniz rule and the integrability
    condition $(\nabla_{0,1})^2=0$ is analogous to
    \ref{lem:affine_space_of_dolbeault_operators}. Hence $\nabla + \xi$
    is a holomorphic connection.

    To see that $\nabla + \xi$ is $h$-unitary we notice
    \begin{align*}
      h((\nabla + \xi)s, \bar t) + h(s,\overline{(\nabla + \xi)t})
      =  h(\nabla s, \bar t) + h(s,\overline{\nabla t})
    \end{align*}
    as $\xi$ is skew-Hermitian.
  \end{proof}
\end{lemma}

Combining what we have seen so far with the next result, the
similarities between $A_\dol (E)$ and $A_\nabla(E,h)$ grow even more
striking:

\begin{lemma}\label{lem:real_vector_space_iso}
  As $C^\infty(C,\mathbb R)$-vector spaces,
  $\Omega^1_{\mathbb{R}}(C,\End(E,h))\cong\Omega^{0,1}(C,\End(E))$.
  \begin{proof}
    Now recall the isomorphism (\ref{eq:holomorphic_tangent_bundle}) of
    smooth vector bundles $T^* X \cong T^*X^{1,0}$. Composing with
    complex conjugation yields an isomorphism of real vector bundles
    $T^*X \cong T^*_{0,1}$. Hence we have an $\mathbb{R}$-linear
    isomorphism
    \begin{align*}
      \Omega^1_{\mathbb{R}}(C,\End(E,h))
      \cong \Gamma^\infty(C,T^*_{0,1}C \otimes_{\mathbb{R}}\End(E,h))
    \end{align*}
    Noticing that $T_{0,1}^* C$ is a complex vector bundle
    we find
    \begin{align*}
      \Gamma^\infty(C,T^*_{0,1}C \otimes_{\mathbb{R}}\End(E,h))
      \cong\Gamma^\infty(C,T^*_{0,1}C \otimes\End(E,h)_{\mathbb{C}}).
    \end{align*}
    Now the claim follows from the isomorphism of Lie algebras
    \begin{align*}
      \End(E)
      =\mathfrak{gl}_n(\mathbb{C})
      \cong\mathfrak u(n)_{\mathbb{C}}
      =\End(E,h)_{\mathbb C}
    \end{align*}
    given by $A \mapsto A-A^\dagger$.
  \end{proof}
\end{lemma}

It is not a coincidence that $A_\dol (E)$ and $A_\nabla(E,h)$ both are
affine spaces modelled on the same underlying real vector space.
Indeed, in the presence of $h$, every Dolbeault operator has a unique
compatible connection:

\begin{theorem}\label{thm:chern_connection}
  If $h$ is a Hermitian metric on $E$ and $\dol_E$ is a Dolbeault
  operator then there exists a unique connection $\nabla^h(\dol_E)$
  that is compatible with both $h$ and $\dol_E$.
  \begin{proof}
    \missingproof
  \end{proof}
\end{theorem}

In particular, this gives us a bijection
\begin{align}\label{eq:chern_correspondence}
  \nabla^h : A_\dol (E) \to A_\nabla(E,h)
\end{align}
whose inverse is the restriction
$\nabla \mapsto \nabla_{0,1}$. It is not difficult to verify that this
is a homeomorphism of affine spaces which is locally given by
the real linear isomorphism in \ref{lem:real_vector_space_iso}.
Note that $A_\nabla(E,h)$ depends on $h$ whereas $A_{\dol}(E)$ does
not. This justifies our choice of metric and ensures that the
moduli space is not going to depend on it.

Denote by $U(E,h)$ the group of smooth $h$-unitary sections of $\End(E)$.
That is,
\begin{align*}
  U(E,h) := \{g \in GL(E) : g^\dagger = g^{-1}\}.
\end{align*}
Note that as $C$ and the fibres $U(r)$ are both compact,
so is $U(E,h)$. Moreover, the fibres $GL(E)_x \cong GL_r(\mathbb C)$
are complex Lie groups whereas $U(E,h)_x\cong U(r)$ are real.
In fact, although we are not going to concern ourselves with this
fact any futher, it is worth mentioning that $U(E,h)\subseteq GL(E)$
is a maximal compact subgroup and $GL(E)$ may be thought of as its
complexification.

We have an action of $U(E,h)$ on $A_\nabla(E,h)$ via conjugation. That is,
for $g\in U(E,h)$ and $\nabla\in A_\nabla(E,h)$, $g\cdot\nabla$ is the map
\begin{align*}
  \Omega^0(X,E) \xlongrightarrow{\Omega^0(X,{g}^{-1})}
  \Omega^0(X,E) \xlongrightarrow{\nabla}
  \Omega^1(X,E) \xlongrightarrow{\Omega^1(X,g)}
  \Omega^1(X,E).
\end{align*}
It is not difficult to verify that this is indeed an action.
In fact, $g\cdot\nabla$ is almost immediately a holomorphic connection.
One then sees that $g\cdot\nabla$ is unitary because
$g$ and $\nabla$ are.

\todo{complex structure on the affine space}

\subsection{Higgs Connections}

We have successfully viewed the space of holomorphic bundles
$A_\dol (E)$ as an affine space modelled on
$\Omega^1_{\mathbb{R}}(C,\End(E,h))$. This is eventually going
to help us introduce an infinite dimensional manifold structure on
$A_\dol (E)$ and hence realise the quotient as such a manifold.
Let us do the same for the space of Higgs
bundles $A_\varphi(E)$. We are going to identify $A_\varphi$ with an affine
space modelled on $\Omega^1_{\mathbb{R}}(C,\End(E,h))^{\oplus 2}$.
Once again fix a Hermitian metric $h$ on $E$.

Just as holomorphic bundles correspond to unitary connections, Higgs bundles
correspond to Higgs connections.

\begin{definition}
  A \emph{Higgs connection} in $E$ is a pair $(\nabla,\Phi)$
  where $\nabla$ is a $h$-unitary connection in $E$ and
  $\Phi\in\Omega^1_{\mathbb{R}}(C,\End(E,h))$.
\end{definition}

Hence we have the space
\begin{align}\label{eq:Higgs_connections_space}
  A^{\Phi}(E,h) := A_\nabla(E,h) \times \Omega^1_{\mathbb{R}}(C,\End(E,h))
\end{align}
which naturally has the structure of an affine space modelled on
the real vector space
$\Omega^1_{\mathbb{R}}(C,\End(E,h))\oplus\Omega_{\mathbb{R}}^1(C,\End(E,h))$. Using the isomorphism (\ref{lem:real_vector_space_iso})
and the map given by (\ref{thm:chern_connection}) we have an inclusion
of topological spaces
\begin{align*}
  A_\varphi(E) \subseteq A_\dol (E) \times \Omega^{1,0}(\End(E)) \cong A_\Phi (E,h).
\end{align*}
Note that in general $A_\varphi(E)$ and $A_\Phi (E,h)$ are not homeomorphic. In
particular, a Higgs connection $(\nabla,\Phi)$ corresponds to
a pair $(\dol_E,\varphi)$ where $\dol_E$ is a Dolbeault operator and
$\varphi$ is a section in $\Omega^{1,0}(C,\End(E))$, not necessarily
holomorphic. We will have to deal with this at a later stage.

For now, observe that the group $G(E,h)$ also acts on $A_\Phi (E,h)$
by conjugation:
\begin{align*}
  G(E,h) \times A_\Phi (E,h) & \to A_\Phi (E,h)                        \\
  (g,\nabla,\Phi)            & \mapsto (g\nabla{g}^{-1},g\Phi{g}^{-1})
\end{align*}

\section{Banach Manifolds}

Our next goal is to equip the affine spaces of connections $A_\nabla(E,h)$
and $A_\Phi (E,h)$ with manifold structures. Of course, both are affine
spaces modelled on infinite dimensional vector spaces. In the next
section we are going to require our infinite dimensional manifolds to
be locally isomorphic to Banach spaces, rather than arbitrary
vector spaces. While we are going to profit from this in the long run,
it means that we have to turn $\Omega^1_{\mathbb{R}}(C,\End(E,h))$
into a Banach space. We are going to do this by using a completion of
the space of smooth sections.

\subsection{Sobolev Spaces}

In many settings, the Banach space completions of choice are the
$L^p$-spaces. While these spaces are a little bit easier to deal with,
they turn out to be too generous for our application. We are going to
consider differential operators between the spaces we construct,
i.e. maps involving differentiation. In order for differential operators
to be well defined, we need to make sure that the points of the space
have sufficiently many derivatives. Hence the Banach space completions
that we are going to use, in some sense, contain sections with a
certain number of derivatives, all of which in some $L^p$.

Consider the following setting. Let $X$ be a compact
Riemannian manifold $X$ of dimension $n$, let $E$ be a smooth complex
bundle on $X$ with Hermitian metric $h$, let $\nabla$ be a $h$-unitary
connection in $E$, and let $\ell\geq 0$ and $1<p<\infty$ be integers.
Here we are going to think of $\nabla$ as a derivative of sections and
define a space $W_{p,\ell}(X,E)$ which may be thought of as the space of
sections $X\to E$ with $\ell$ derivatives, all of which in $L^p$. As
is the case with many of the concepts we encounter throughout this paper,
these Sobolev spaces are interesting geometric objects to
contemplate but we are going to use them as a tool. The main
benefit we are going to get from using this particular Banach space
completion is the theory of elliptic regularity as presented in
\ref{sec:elliptic_regularity}.

Note that the metrics on $E$ and $X$ define norms on smooth sections
of $E$ and $T^*X$, respectively, and hence on all
$\Omega^k_{\mathbb{R}}(X,E)$.
Moreover, the metric on $X$ induces a volume form $\text{vol}_X$.
With this in mind, we are ready to define the Sobolev norm on
$\Gamma^\infty(X,E)$:

\begin{definition}
  Let $\nabla$ be a $h$-unitary connection in $E$.
  The \emph{$(p,\ell)$-Sobolev norm} with respect to $\nabla$ on
  $\Gamma^\infty(X,E)$ by
  \begin{align*}
    \Vert s\Vert_{p,\ell}^\nabla := \left(
    \sum_{i=0}^\ell \int_X \Vert \nabla^i s \Vert^p\:\text{vol}_X
    \right)^{1/p}.
  \end{align*}
\end{definition}

Observe that Sobolev norms generalise $L^p$ norms. In particular,
we have $\Vert\cdot\Vert^\nabla_{p,0} = \Vert\cdot\Vert_{L^p}$.
Further, the norm depends on the choice of connection $\nabla$. Of
course, we do not want moduli spaces to have the same dependency. Hence
we establish the following:
\begin{lemma}
  Let $\nabla$ and $\nabla'$ be $h$-unitary connections in $E$. Then the
  norms $\Vert\cdot\Vert_{p,\ell}^\nabla$ and
  $\Vert\cdot\Vert_{p,\ell}^{\nabla'}$ are equivalent.
  \begin{proof}
    \missingproof
  \end{proof}
\end{lemma}

Therefore we are justified in defining a completion that does not depend
on the connection:

\begin{definition}
  The \emph{Sobolev space} $W_{p,\ell}(X,E)$ is the Banach completion of
  $\Gamma^\infty(X,E)$ with respect to the norm
  $\Vert\cdot\Vert_{p,\ell}^\nabla$ for any $h$-unitary connection
  $\nabla$ in $E$.
\end{definition}

Observe that we immediately have a sequence of inclusions
\begin{align*}
  L^2(X,E) =
  W_{2,0}(X,E) \supseteq
  \cdots \supseteq
  W_{2,\ell}(X,E) \supseteq
  \cdots
  \supseteq
  C^\infty(X,E)
\end{align*}
where the last inclusion intuitively follows from the fact
$C^\infty \subseteq L^2$ and the observation that each derivative
$\nabla^i s$ is $C^\infty$. For a more rigorous argument, see
\cite[Corollary 3.8.3]{bc2009}.
Moreover, $C^\infty$ is dense in $L^2$ so it must be dense in each
$W^{2,\ell}$, too. Thus we will be able to regard $\Gamma^\infty(X,E)$
as a Banach space by expanding it slightly.

As usual, we are going to think of the elements of $W_{2,\ell}(X,E)$ as
sections in $\Gamma^\infty(X,E)$. However, it is worth remembering that
sections that differ on a set of measure zero are identified in the
completion.

\begin{example}
  Note that the metric on $X$ is a metric on $T^*X$. Hence we have
  a metric on $E\otimes (T^*X)^{\wedge k}$ for all $k$. This allows
  us to define
  \begin{align*}
    \Omega_{\mathbb{R}}^k(X,E)_\ell := W_{2,\ell}(X,E\otimes(T^*X)^{\wedge k}).
  \end{align*}
\end{example}

\subsection{Differential Operators}

Now that we have turned spaces of sections into Banach spaces, it is
time to make sure that maps on sections may be extended accordingly.
It is not in general true that every map
$\Gamma^\infty(X,E)\to\Gamma^\infty(X,F)$ extends to a map on
Sobolev spaces. Fortunately, there are some reasonable assumptions
we can make.

Consider a $C^\infty(X)$-linear map
\begin{align*}
  P : \Gamma^\infty(X,E) \to \Gamma^\infty(X,F).
\end{align*}
If we are given a chart $U\ni x$ in $X$ such that $P$ maps sections
$U\to E$ to sections $U\to F$ then $P$ is given by
\begin{align}\label{eq:local_diff_operator}
  \restrict{P}{U}(s) = \sum_{|I|\leq m} f_I \partial^I
\end{align}
where, for each multi index $I=\{i_1,\ldots,i_{m'}\}$, we have
$f_I\in\Gamma^\infty(U,\Hom(E,F))$, $|I| := m'$, and we write
\begin{align*}
  \partial^I = \frac{\partial^n}{\partial x_{i_1} \cdots \partial x_{i_{m'}}}.
\end{align*}
If it is possible to express $P$ in this way everywhere,
the we say that $P$ is a differential operator:

\begin{definition}\label{def:diff_operator}
  Let $E$ and $F$ be smooth bundles on $X$. A \emph{differential
    operator $P:E\to F$ of order $m$} is a $C^\infty(X)$-linear map
  \begin{align*}
    P : \Gamma^\infty(X,E) \to \Gamma^\infty(X,F)
  \end{align*}
  such that, for each $x\in X$, there is a chart $U\ni x$ trivialising $E$ and
  $F$ such that $\restrict{P}{U}$ may be expressed as
  (\ref{eq:local_diff_operator}).
\end{definition}

\begin{example}
  Consider the operator
  \begin{align}\label{eq:d_hat}
    \hat d = d + d^* : \Omega^1_{\mathbb R}(C) \to \Omega^2_{\mathbb R}(C) \oplus \Omega^0_{\mathbb R}(C).
  \end{align}
  Locally, in local coordinates $(x_1,x_2)$ around $x$ we may write
  \begin{align*}
    f_1 dx^1 + f_2 dx^2 \mapsto
    (\partial_1 f_1 - \partial_2 f_2)dx^1\wedge dx^2
    + \partial_1 f_1 + \partial_2 f_2 .
  \end{align*}
  Hence $d + d^*$ is a differential operator of order 1.
\end{example}

\begin{example}
  Let $\nabla\in A_\nabla(E,h)$ be a connection. Generalising the previous
  example slightly, consider the operator \todo{change $\nabla$ to $d_\nabla$
    and be careful about $TX\otimes TX$ vs $TX\wedge TX$}
  \begin{align}\label{eq:nabla_hat}
    \hat\nabla := \nabla + \nabla^\dagger :
    \Omega^1_{\mathbb R}(\End(E,h)) \to
    \Omega^2_{\mathbb R}(\End(E,h)) \oplus
    \Omega^0_{\mathbb R}(\End(E,h))
  \end{align}
  where we have extended the connection $\nabla$ in $E$ to a connection
  in $\End(E)$ using (\ref{ex:connection_on_end}) which
  one may verify restricts to $\End(E,h)$. This is a
  differential operator of order 1.
  \begin{proof}
    If we have a trivialisation of $E$ around $x$ with orthonormal coordinates
    $(x_1,x_2)$ we may choose an orthonormal basis $A^1,\ldots,A^{r^2}$ for
    $\mathfrak u(r)$ and write
    \begin{align}\label{eq:nabla_diff_operator}
      \nabla\left(fdx^i\otimes A^j\right)
      = \sum_k \partial_k f dx^k\wedge dx^i \otimes A^j
      - dx^i\wedge\sum_{i\ell m} f\nabla_{im\ell}dx^m\otimes A^\ell
    \end{align}
    and
    \begin{align}\label{eq:nabla_adjoint}
      \nabla^\dagger\left(fdx^i \otimes A^j\right)
      = \sum_k \partial_k fdx^k\wedge dx^i\otimes A^j
      - fdx^i \wedge \sum_{mn} \nabla_{jmn}dx^m\otimes A^n
    \end{align}
    where $\nabla_{jk}\in C^\infty(X)$ such that
    $\nabla A^i = \sum_{jk}\nabla_{ijk}dx^j\otimes A^k$. Both
    $\nabla$ (\ref{eq:nabla_diff_operator}) and $\nabla^\dagger$
    (\ref{eq:nabla_adjoint}) are differential operators of order 1 hence
    $\hat\nabla$ (\ref{eq:nabla_hat}) is, too.
  \end{proof}
\end{example}

It turns out that all differential operators extend to maps of Sobolev
completions. Of course it is intuitively clear, that a differential
operator of order $m$ involves taking $m$ derivatives and hence
it can take an element of $W_{2,\ell}$ at most to $W_{2,\ell-m}$.

\begin{theorem}
  Let $P:E\to F$ be a differential operator of order $m$ and
  let $\ell \geq 0$. Then $P$ extends uniquely to an operator
  \begin{align}\label{eq:fredholm_extension}
    P_\ell : W_{2,\ell+m}(X,E) \to W_{2,\ell}(X,F).
  \end{align}
  \begin{proof}
    See \cite[{Proposition 3.8.4}]{bc2009}.
  \end{proof}
\end{theorem}

\subsection{Banach Manifolds}

Smooth and complex manifolds are often defined using charts and
atlases. However, once the necessary theory has been established,
it is far easier to say that a complex manifold
of dimension $n$ is a locally ringed space locally isomorphic to
$(\mathbb{C}^n,\mathcal O_{\mathbb{C}^n})$ where
$\mathcal O_{\mathbb{C}^n}$ is the sheaf of holomorphic functions on
$\mathbb{C}^n$.

To define infinite dimensional manifolds, we are going to replace
$\mathbb{C}^n$ with infinite dimensional vector spaces. Of course,
the point of the usual manifold definition is to facilitate analysis
on a wider range of topological spaces. To this end, we are going to
restrict our attention to the case of real Banach spaces. While a brief
definition in the sense of the paragraph above would be correct,
it would also be unhelpful. Instead we are going to consider some
examples. The reader interested in a more holistic
introduction may consult any text on global analysis such
as \cite[Chapter 7]{kahn1982} or \cite{bc2009}.

\begin{example}
  Every smooth manifold is locally isomorphic to
  the Banach space $\mathbb R^n$ and hence a Banach manifold.
\end{example}

\begin{example}
  Let us consider our main applications. We have the affine space
  $A_\nabla(E,h)$ modelled on $\Omega^1_{\mathbb{R}}(C,\End(E,h))$.
  For each $\ell \geq 0$, we have the corresponding Sobolev completion
  \begin{align*}
    A_\nabla(E,h)_\ell := A_\nabla(E,h) + \Omega^1_{\mathbb{R}}(C,\End(E,h))_\ell
  \end{align*}
  which are naturally Banach manifolds.
  In particular, for each $\nabla\in A_\nabla(E,h)_\ell$, we have the
  tangent space
  \begin{align*}
    T_\nabla A_\nabla(E,h)_\ell
    = \Omega^1_{\mathbb{R}}(C,\End(E,h))_\ell.
  \end{align*}
  Noticing that finite direct sums of Banach spaces remain Banach spaces,
  the affine space (\ref{eq:Higgs_connections_space}) of Higgs connections also
  extends to a Banach manifold
  \begin{align*}
    A_\Phi(E,h)_\ell := A_\Phi(E,h) + \Omega^1_{\mathbb{R}}(C,\End(E,h))_\ell
    \oplus \Omega^1_{\mathbb{R}}(C,\End(E,h))_\ell
  \end{align*}
  with the obvious cotangent space at $(\nabla,\Phi)$:
  \begin{align*}
    T^*_{\nabla,\Phi} A_\Phi(E,h)_\ell
    = \Omega^1_{\mathbb{R}}(C,\End(E,h))_\ell
    \oplus \Omega^1_{\mathbb{R}}(C,\End(E,h))_\ell
  \end{align*}
\end{example}

\begin{example}
  Consider the group $U(E,h)$ of unitary
  sections of $\End(E)$. The Sobolev norms on $\Gamma^\infty(C,\End(E))$
  restrict to norms on the subset $U(E,h)$ and hence we
  may consider the completions
  \begin{align*}
    U(E,h)_\ell := W_{2,\ell}(C,U(E,h)) \subset W_{2,\ell}(C,\End(E)).
  \end{align*}
  One may show that this is a Banach manifold by applying the
  analogue of the regular value theorem. In particular,
  the tangent space at $g\in U(E,h)_\ell$ is
  \begin{align*}
    T_g U(E,h)_\ell = W_{2,\ell}(C,\End(E,h)) =: \mathfrak u(E,h)_\ell.
  \end{align*}
\end{example}

\section{Quotients of Banach Manifolds}

Having obtained a manifold structure on the affine spaces and the groups
that act on them, we are in a great position to take quotients to construct
the analytic moduli spaces of holomorphic bundles that we are after. To do this,
one needs to ensure that the group actions are sufficiently well-behaved. Moreover,
we need to ensure that the actions on affine spaces extend to a Banach Lie group
action on Banach manifolds.

We will see which properties of our actions are necessary to
obtain a geometrically well-behaved quotient. A lot of work will go into justifying why
these properties are attained. In some cases we will have to
restrict our moduli spaces to avoid special cases while in others
the theory of elliptic operators is going to be indispensable.

\subsection{Proper and Free Group Actions}

Before we extend groups and actions to the realm of infinite dimensional
Banach spaces, let us study the group actions that we have thus
far encountered a little more closely. Later we are going to require our
actions to be free and proper. Let us recall what that means.

Properness is a topological property. That is, it applies only to
topological groups:

\begin{definition}
  A continuous action $\sigma : G\times X\to X$ of a topological group $G$
  on a topological space $X$ is \emph{proper} if the map
  \begin{align}\label{eq:proper_map}
    (\sigma,\pi) : G\times X \to X\times X
  \end{align}
  is proper, i.e. the preimage of every compact set in $X\times X$
  is compact.
\end{definition}

\begin{example}
  One easily sees if $G$ is compact then every $G$-action is proper.
  In particular, the action of $U(1)$ on $U(E,h)$ is
  proper.
\end{example}

\begin{example}\label{ex:h_proper_action}
  The action of $U(E,h)$ on $A_\nabla(E,h)$ is proper.
  \begin{proof}
    We sketch the proof given in
    \cite[{Proposition 7.1.14}]{kobayashi1987}.
    The goal is to show that, if we have a sequence $(g_j,\nabla_j)$
    such that $\nabla_j\to\nabla$ and $g_j\cdot\nabla_j\to\nabla'$
    then $g_j\to g$ such that $g\cdot\nabla = \nabla'$. To do this
    one moves to the principal $U(n)$-bundle $P$ associated to $(E,h)$
    where each $g_j$ commutes with the action of the structure
    group. The next step is to focus on a single fibre by fixing $x_0\in X$ and
    $p_0\in P_{x_0}$ and considering curves $c$ starting at $x_0$ in $X$.
    One then uses $\nabla$ and $\nabla'$ to obtain lifts $\tilde c$ and
    $\tilde c'$ of $c$ in $P$. Now the limit $g$ of $(g_j)$ must satisfy
    $g(\tilde c) = \tilde c'$. This condition on $g$, together with
    commutativity with the action of the structure group, identifies it
    uniquely.
  \end{proof}
\end{example}

\begin{example}\label{ex:higgs_proper_action}
  The action of $U(E,h)$ on $A_\Phi (E,h)$ is proper.
  \begin{proof}
    \missingproof
  \end{proof}
\end{example}

Being free is not a topological property, i.e. any action of any group on any
set may be free. Nonetheless, it is particularly important in a geometric
setting as it ensures that the orbits are homogeneous and hence all points in
the quotient are locally indistinguishable, as one expects from manifolds.

\begin{definition}
  An action $\sigma : G\times X\to X$ is \emph{free} if, for all
  $g\in G$ and $x\in X$, $g\cdot x = x$ implies $g = e$.
\end{definition}

\begin{example}\label{ex:not_free}
  Consider the action of $U(E,h)$ on $A_\nabla(E)$. This is not a free action
  because scalar multiplication commutes with the connection. That is,
  for all $\lambda\in U(1)$, $g\in U(E,h)$, and $\nabla\in A_\nabla(E,h)$,
  it holds that $\lambda g\cdot \nabla = \nabla$. This flaw is easily fixed,
  however. In fact, we immediately have an action of
  \begin{align*}
    G(E,h) := U(E,h) / U(1)
  \end{align*}
  on $A_\nabla(E,h)$. Unfortunately, this action is not free either. In particular,
  if there is a non-trivial proper subbundle $E'\subseteq E$ that is preserved
  by $\nabla$ then non-trivial automorphisms of $E'$ act trivially on $\nabla$.
\end{example}

\begin{example}\label{ex:not_free_double}
  The fact that the $G(E,h)$ action on $A_\nabla(E,h)$ is not free does not
  immediately imply that the action on $A_\Phi(E,h)$ is not either. However,
  similar reasoning shows why it cannot be free: If there is a non-trivial
  proper subbundle $E'\subseteq E$ that is preserved by $\nabla$ and $\Phi$,
  respectively, then an automorphism $g\in G(E,h)$ need only be trivial
  outside of $E'$ to act trivially on $(\nabla,\Phi)$.
\end{example}


\subsection{Banach Lie Groups}

In particular, there is a natural notion of a Banach Lie group and
corresponding actions on Banach manifolds.

\begin{example}\label{ex:unitary_banach_lie_group}
  For sufficiently large $\ell$, the Banach manifold $U(E,h)_\ell$ is a
  Banach Lie group with the usual multiplication.
  \begin{proof}
    This is somewhat subtle as we have to ensure that, for any
    $g,h\in U(E,h)_\ell$, we have $gh\in U(E,h)_\ell$. However, it
    is well-known that pointwise multiplication
    in Sobolev spaces of trivial bundles yields a map
    \begin{align*}
      W_{2,\ell}(C,\mathbb R^k) \times
      W_{2,\ell}(C,\mathbb R^k) \to
      W_{2,\ell}(C,\mathbb R^k)
    \end{align*}
    whenever $\ell > k/2$. \cite[Theorem 6.1]{behzadan2021}
    Note that matrix multiplication consists of addition and
    pointwise multiplication with $k=1$. Moreover, recall that both
    the base space $C$ and the unitary gauge group $U(E,h)$ is compact.
    Using these facts, one may verify locally that the product $gh$ is indeed $W_{2,\ell}$.
  \end{proof}
\end{example}

Whenever we refer to the group $U(E,h)_\ell$ we are implicitly going to
assume that $\ell$ was chosen sufficiently large for Sobolev multiplication
to work.
In complete analogy to the finite dimensional case, the tangent space
at the identity of a Banach Lie group may be equipped with a Lie bracket
and hence is called the Lie algebra of the group.

\begin{example}
  The Lie algebra of $U(E,h)_\ell$ is $\mathfrak u(E,h)_\ell$.
\end{example}

Similarly, an action of a Banach Lie group on a Banach manifold
is a Banach Lie group action if, and only if, it is given by a
smooth map. Here we use the fact that the product of Banach manifolds
has a natural Banach manifold structure.

We observe that Sobolev completions allow us to lift group
actions:

\begin{example}
  Consider the action of $U(E,h)$ on $A_\nabla(E,h)$
  by conjugation. This extends to a proper action of $U(E,h)_{\ell+1}$
  on $A_\nabla(E,h)_\ell$ via
  \begin{align*}
    (g,\nabla) & \mapsto g\nabla g^{-1}.
  \end{align*}
  Using the Leibniz rule we calculate, for $s\in\Omega^0(C,E)$,
  \begin{align*}
    g\nabla g^{-1}(s) = g\nabla(g^{-1}s) = g(g^{-1}(\nabla s) + (\nabla g^{-1})s)
    = \nabla s + g(\nabla g^{-1})s.
  \end{align*}
  Hence we require $g\in U(E,h)_{\ell + 1}$ to make sure that $g\cdot\nabla\in A_\nabla(E,h)_\ell$.
  Analogously, we have a proper Banach Lie group action of $U(E,h)_{\ell+1}$
  on $A_\Phi (E,h)_\ell$.
\end{example}

All of our recent discussions about proper and free actions, Sobolev
completions, and Banach manifolds are motivated by the next theorem.
In particular, it will allow us to construct moduli spaces as, possibly
infinite dimensional, Banach manifolds:

\begin{theorem}\label{thm:banach_quotient}
  Consider a proper free action $\sigma : G\times X\to X$ of a Banach
  Lie group $G$ on a Banach manifold $X$. If each tangent space
  $T_x(G\cdot x) \subseteq T_x X$ is closed and complemented then
  $X/G$ is a Banach manifold and $T_x X/T_x (G\cdot x) \cong T_x(X/G)$
  for all $x\in X$.
  \begin{proof}
    The idea is to construct a slice at each point $x\in X$, i.e. a submanifold
    $x\in S_x\subseteq X$ such that $T_x X = T_xS_x \oplus T_x(G\cdot x)$. A slice is
    what induces the manifold structure of $X/G$ at $x$. More preciesley, if a slice
    $S_x$ exists at $x\in X$ then we may choose a sufficiently small chart $U\subseteq S_x$
    around $x$ such that $\pi : X \to X/G$ gives a homeomorphism onto $U\cong\pi(U)$. Doing this
    at every $x\in X$ gives us the manifold structure on $X/G$. Now the tangent space
    to $X/G$ at the equivalence class of $x$ is
    \begin{align}\label{eq:banach_quotient_tangent}
      T_x S_x = T_x X / T_x (G\cdot x).
    \end{align}
    The difficult part is making sure that a slice exists at every $x\in X$. One may use
    one of the established theorems such as \cite[Theorem 3.28]{diez2019}
    or \cite[Theorem 5.2.6]{palais1992}. The latter applies particularly well
    as the group action is Fredholm by our assumption that it be free and satisfy
    $T_x(G\cdot x)\subseteq T_x X$.
  \end{proof}
\end{theorem}

Before we consider examples, we have two remarks about the assumptions of the theorem,
in particular about the condition that the tangent space of the orbit be closed and
complemented. Firstly, note that $G\cdot x \subseteq X$ is closed then the tangent
space is a closed subspace. This follows by considering trivialisations around $g\cdot x$
in both $G\cdot x$ and $X$ where it is clear that $G\cdot x$ must be modelled on a closed
subspace of the tangent space of $X$. Secondly, as the Sobolev completions with $p=2$ are
Hilbert spaces, all closed subspaces in our examples will automatically be complemented.

\begin{example}
  Observe that the Banach Lie subgroup $U(1)\subseteq U(E,h)$ acts freely and properly
  on $U(E,h)$ via $(\lambda,g)\mapsto \lambda g$. Moreover, each tangent space
  \begin{align*}
    T_g(U(1)g) =
    i\mathbb{R} \subseteq W_{2,\ell}(C,\End(E,h))
  \end{align*}
  is finite-dimensional and hence closed and complemented.  Thus the quotient
  \begin{align*}
    G(E,h)_{\ell} := U(E,h)_\ell / U(1)
  \end{align*}
  is a Banach manifold whose tangent space at $g\in G(E,h)_\ell$ is
  \begin{align*}
    T_g G(E,h)_\ell =  W_{2,\ell}(C,\End(E,h))/i\mathbb{R} =: \mathfrak g(E,h)_\ell.
  \end{align*}
\end{example}

Unfortunately, we are not yet ready to apply the theorem \ref{thm:banach_quotient} to
the actions of $G(E,h)_{\ell+1}$ on $A_\nabla(E,h)_\ell$ and
$A_\Phi(E,h)_\ell$ as they are not free. Moreover, the tangent spaces fail to
be closed:

\begin{example}
  \missingexample
\end{example}

In the next sections we are going to rectify both issues by restricting the spaces
sufficiently. Indeed, our argument in (\ref{ex:not_free}) already partly tells us
what the correct subsets to consider will be.

\subsection{Elliptic Regularity}\label{sec:elliptic_regularity}

Passing to Sobolev spaces has allowed us to use the language of
Banach manifolds and Banach Lie groups but it has come with two
crucial drawbacks. Firstly, we have to deal with the additional
parameter $\ell$ which our final moduli space ought not to depend on.
Secondly, proper non-trivial subspaces of infinite dimensional Hilbert
spaces need not be closed and hence may not be complemented.
We will address both of these issues using elliptic regularity.

We briefly introduce the theory of elliptic regularity. While it is
impossible to treat any of the material introduced in full rigour,
we generally intend to establish a solid intuition. Unfortunately, we
cannot hope to achieve this for elliptic regularity. The reader is
invited to consult any of the many introductory texts such
as \cite{hance2014} and \cite[Chapter 6]{warner1983}. \todo{improve}

Recall the notion of a differential operator \ref{def:diff_operator}.
What we are going to require is the the matrix defined by the
$f_I$ in (\ref{eq:local_diff_operator}) is invertible.
As homogeniety is required to respect change of coordinates,
we are only going to care about top-order terms. One way to make all
this precise is to define the principal symbol of the operator by
replacing partial derivatives with variables and
call it elliptic if the symbol yields invertible matrices.
The following is a more algebraic condition:

\begin{definition}\label{def:elliptic_operator}
  A differential operator $P:E\to F$ is \emph{elliptic} if
  $\rank E = \rank F$ and, for all $x\in X$, all $s\in\Omega^0_{\mathbb C}(E)$,
  and all $f\in C^\infty(X,\mathbb R)$ such that $s(x) \neq 0$,
  $f(x) = 0$, and $df(x) \neq 0$, it holds that $P(f^m s)(x) \neq 0$.
\end{definition}

Let us make sure that this is a reasonable condition:

\begin{example}
  Consider the operator $\hat d = d + d^\dagger$ (\ref{eq:d_hat}). Let
  $x\in C$, $\omega\in\Omega^1_{\mathbb R}(C)$,
  $f\in C^\infty(C,\mathbb R)$
  satisfying the assumptions in \ref{def:elliptic_operator}.
  Recalling that $\hat d$ is of order $m=1$, we calculate
  \begin{align*}
    \hat d(f\omega)(x)= df\wedge \omega(x) - f(x)d\omega(x) + fd^\dagger\omega= df\wedge\omega(x) + fd^\dagger\omega
  \end{align*}
  which is non-zero. Indeed the calculation for
  $\hat\nabla$ (\ref{eq:nabla_hat}) is similar albeit more cumbersome.
  Hence both $\hat d$ and $\hat\nabla$ are elliptic operators of order 1.
\end{example}

Elliptic operators have many important properties. Firstly, the class
of elliptic operators is closed under taking adjoints:

\begin{lemma}\label{lem:elliptic_adjoints}
  If $P : E \to F$ is an elliptic operator then the formal adjoint
  $P^\dagger$ is elliptic.
  \begin{proof}
    \missingproof
  \end{proof}
\end{lemma}

Secondly, if we consider the extension of an elliptic operator
to Sobolev spaces then we find several desirable properties.
In particular, the image of such an operator is closed which is
exactly what we are going to use to prove that the tangent spaces
of orbits are closed and hence complemented.

\begin{theorem}\label{thm:fredholm_extension}
  If $P : E\to F$ is an elliptic operator of order $m$ and $\ell\geq 0$
  then the induced operator $P_\ell$ from (\ref{eq:fredholm_extension})
  is Fredholm. That is,
  \begin{enumerate}
    \item the image $\im P_\ell \subseteq W_{2,\ell}(X,F)$ is closed,
    \item the kernel $\ker P_\ell \subseteq W_{2,\ell+m}(X,E)$ is finite-dimensional,
    \item the cokernel $W_{2,\ell}(X,F)/\im P_\ell$ is finite-dimensional.
  \end{enumerate}
  \begin{proof}
    \missingproof
  \end{proof}
\end{theorem}

\begin{example}\label{ex:nabla_hat_elliptic}\todo{THIS EXAMPLE DOES NOT WORK HERE, WE NEED HARMONICITY.}
  Using \ref{lem:elliptic_adjoints} we have an elliptic operator
  \begin{align*}
    \hat\nabla = \nabla + \nabla^\dagger :
    \Omega^0_{\mathbb R}(\End(E,h)) \oplus
    \Omega^2_{\mathbb R}(\End(E,h)) \to
    \Omega^1_{\mathbb R}(\End(E,h)).
  \end{align*}
  In particular, the image of the operator
  \begin{align*}
    \hat\nabla_\ell :
    \Omega^0_{\mathbb R}(\End(E,h))_{\ell+1}
    \oplus \Omega^2_{\mathbb R}(\End(E,h))_{\ell-1}
    \to \Omega^1_{\mathbb R}(\End(E,h))_\ell
  \end{align*}
  is closed. As $\nabla$ is Yang-Mills, we have a chain complex \todo{elaborate?}
  \begin{equation*}
    % https://q.uiver.app/#q=WzAsNSxbMCwwLCIwIl0sWzEsMCwiXFxPbWVnYV4wX3tcXG1hdGhiYiBSfShcXEVuZChFLGgpKV97XFxlbGwrMX0iXSxbMiwwLCJcXE9tZWdhXjFfe1xcbWF0aGJiIFJ9KFxcRW5kKEUsaCkpX1xcZWxsIl0sWzMsMCwiXFxPbWVnYV4yX3tcXG1hdGhiYiBSfShcXEVuZChFLGgpKV97XFxlbGwtMX0iXSxbNCwwLCIwIl0sWzAsMV0sWzEsMiwiXFxuYWJsYV9cXGVsbCJdLFsyLDMsIlxcbmFibGFfe1xcZWxsLTF9Il0sWzMsNF1d
    \begin{tikzcd}
      0 & {\Omega^0_{\mathbb R}(\End(E,h))_{\ell+1}} & {\Omega^1_{\mathbb R}(\End(E,h))_\ell} & {\Omega^2_{\mathbb R}(\End(E,h))_{\ell-1}} & 0
      \arrow[from=1-1, to=1-2]
      \arrow["{\nabla_\ell}", from=1-2, to=1-3]
      \arrow["{\nabla_{\ell-1}}", from=1-3, to=1-4]
      \arrow[from=1-4, to=1-5]
    \end{tikzcd}
  \end{equation*}
  for all $\ell\geq 1$.  Noting $\im(\nabla_{\ell-1}^\dagger)$
  naturally is $W_{2,\ell}$-orthogonal to $\ker(\nabla_{\ell-1})$
  and hence the subspace $\im(\nabla_{\ell})\subseteq\ker(\nabla_{\ell-1})$
  in $\Omega^1_{\mathbb R}(\End(E,h))_\ell$, we find that
  $\im(\nabla_\ell)$ and $\im(\nabla_\ell^\dagger)$ must each be
  closed.
\end{example}

Finally, we have that the kernels and cokernels are not just
finite-dimensional, but also independent of the particular $\ell$
that we have chosen.

\begin{theorem}\label{eq:independence_of_l}
  If $P:E\to F$ is an elliptic operator of order 1
  and $\ell\geq 0$ then
  \begin{align*}
    \ker P_\ell \subseteq \Gamma^\infty(X,E),\hspace{1cm}
    W_{2,\ell}(X,F)/\im P_\ell \subseteq \Gamma^\infty(X,F).
  \end{align*}
  In particular, neither subspace depends on $\ell$.
  \begin{proof}[Proof idea]
    The key observation is that, for an operator
    $P':E\to E$ of order 2 and for each $\ell\geq 0$,
    there exists a constant $C$ such that
    \begin{align*}
      \Vert s\Vert^\nabla_{2,\ell+2} \leq C(\Vert P's\Vert_{2,\ell} + \Vert s\Vert_{2,0})
    \end{align*}
    for all $s\in\Gamma^\infty(X,E)$.
    This may be deduced from the Euclidean case, e.g. as presented in
    \cite[6.3.1 Theorem 2]{evans1998}. Hence, if $P's = 0$ then the
    $(2,\ell+2)$-norm is bounded
    by the $(2,0)$ norm and hence independent of $\ell$.
    One then only needs to define $P' := P^\dagger P$ which is easily
    verified to be elliptic as both $P$ and $P^\dagger$ are by
    \ref{lem:elliptic_adjoints}. We then use $\ker P = \ker P'$
    to find that $\ker P_\ell$ is independent of $\ell$. The
    statement about the cokernel follows by taking adjoints.
  \end{proof}
\end{theorem}

\subsection{Harmonic Connections}

In some sense, failure of the tangent space space to be closed is a bigger issue
than a non-free group action. Indeed, if the action is not free we simply cannot
assume that the map $g\mapsto g\cdot x$ is a diffeomorphism $G\cong G\cdot x$.
This means in particular that the calculation (\ref{eq:banach_quotient_tangent})
of the tangent space of the quotient fails. However, slices still exist so there
are ways to locally identify the quotient with a slice. In particular, we obtain
a quotient that is locally a manifold, although the tangent spaces may not be
uniform.

On the other hand, if the tangent space is not closed then it may not be complemented and
hence there may not be a slice to identify the quotient with. Hence we address this issue first.
There are multiple ways of restricting the affine spaces we are concerned with to
ensure orbits with closed tangent spaces. We do this by introducing harmonicity.

Observe that, in the presence of a Riemannian metric on a compact
manifold $X$, one may refer to the formal adjoint
\begin{align*}
  d^* : \Omega^{k+1}_{\mathbb{R}}(X)\to \Omega^{k}_{\mathbb{R}}(X)
\end{align*}
of the operator $d : \Omega^k_{\mathbb{R}}(X)\to\Omega^{k+1}_{\mathbb{R}}(X)$. This allows us to
define what it means for a form on $X$ to be harmonic:

\begin{definition}
  A $k$-form $\alpha$ on a compact Riemannian manifold $X$ is \emph{harmonic} if
  $d \alpha = d^* \alpha = 0$. The space of harmonic $k$-forms is
  \begin{align*}
    \mathcal H^k(X) := \ker d \cap \ker d^* \subseteq \Omega^k_{\mathbb{R}}(X).
  \end{align*}
\end{definition}

In particular, we have Sobolev completions $\mathcal H^k(X)_\ell$
obtained in the usual way. This makes harmonicity a suitable property to
restrict the Banach spaces we are studying:

\begin{definition}
  A connection $\nabla\in A_\nabla(E,h)$ is \emph{Yang-Mills} if
  \begin{align}\label{eq:yang_mills_condition}
    K(\nabla) \in i\mathcal H^2(C)\otimes I
  \end{align}
  where $I\in\Gamma(C,\End(E))$ is the identity section.
  I.e. there exists a harmonic form $\alpha\in\mathcal H^2(C)$ such
  that $K(\nabla) = i\alpha\otimes I$ in $\Omega^2(C,\End(E))$.
\end{definition}

Consider the space $A_\nabla^{ps}(E,h)$ of Yang-Mills connections.
We observe that gauge transformations commute with taking curvatures and
that $I\in\End(E)$ is gauge-invariant. Thus, for $\nabla\in A_\nabla^{ps}(E,h)$,
\begin{align*}
  K(g\cdot\nabla)
  = g\cdot K(\nabla)
  = i\alpha\otimes gIg^{-1}
  = i\alpha\otimes I,
\end{align*}
i.e. $A_\nabla^{ps}(E,h)\subseteq A_\nabla(E,h)$ is $U(E,h)$-invariant.
Thus we have a $G(E,h)$-action on $A_\nabla(E,h)$ which is
one may easily verify to be proper.

\begin{lemma}
  The Sobolev completion $A^{ps}_\nabla(E,h)_\ell$ is a
  Banach submanifold of $A_\nabla(E,h)_\ell$.
  \begin{proof}[Proof idea]
    There is a rather straightforward argument using an analogue
    of the regular value theorem for Banach manifolds. The map
    to consider is the moment map for the $G(E,h)_{\ell+1}$-action
    on $A_\nabla(E,h)_\ell$ considered as a symplectic
    Banach manifold, which we have not introduced for brevity.
    \cite[Proposition 5.8]{neitzke2021}
  \end{proof}
\end{lemma}
Just as in the finite dimensional case, the proper $G(E,h)_{\ell+1}$
action restricts to $A^{ps}_\nabla(E,h)_\ell$.
More importantly, we have the following:
\begin{lemma}
  Let $\nabla\in A^{ps}_\nabla(E,h)_{\ell}$. Then the tangent space
  \begin{align*}
    T_\nabla(G(E,h)_{\ell+1}\cdot\nabla) \subseteq T_\nabla A_\nabla(E,h)_\ell
  \end{align*}
  is a closed subspace.
  \begin{proof}
    The key observation is that, as a subspace of
    $\Omega^1(C,\End(E,h))$,
    \begin{align*}
      T_\nabla (G(E,h)\cdot\nabla) = \nabla\Omega^0(C,\End(E,h)).
    \end{align*}
    To see this, recall the infinitesimal action
    \begin{align*}
      \rho_\nabla(A) := \restrict{\frac{d}{dt}}{t=0} \exp(tA)\cdot\nabla.
    \end{align*}
    This linearises the action and determines the embedding
    $T_\nabla G(E,h)_{\ell+1}\subseteq T_\nabla \mathcal A(E,h)_\ell$.
    Direct calculation shows $\rho_\nabla(A) = \nabla A$
    and hence we have the diagram \todo{the $\ell$'s here
      are a little troublesome still, but there certainly is some
      Fredholm operator that $\nabla$ gives rise to whose image is what we
      are looking for so nothing's going to break}
    \begin{equation*}
      % https://q.uiver.app/#q=WzAsNSxbMSwxLCJcXE9tZWdhXjAoXFxtYXRoZnJhayB1KEUsaCkpX1xcZWxsIl0sWzQsMSwiXFxPbWVnYV4xKFxcbWF0aGZyYWsgdShFLGgpKV9cXGVsbCJdLFswLDEsIlRfXFxuYWJsYShHKEUsaClfe1xcZWxsKzF9XFxjZG90IFxcbmFibGEpIl0sWzAsMCwiVF9lIEcoRSxoKV97XFxlbGwrMX0iXSxbNCwwLCJUX1xcbmFibGFcXG1hdGhjYWwgQV9cXG5hYmxhKEUsaClfXFxlbGwiXSxbMCwxLCJcXG5hYmxhX1xcZWxsIl0sWzQsMSwiXFxjb25nIiwyXSxbMiwwLCJcXGNvbmciXSxbMywyLCIiLDIseyJzdHlsZSI6eyJoZWFkIjp7Im5hbWUiOiJlcGkifX19XSxbMyw0LCJcXHJob19cXG5hYmxhIl1d
      \begin{tikzcd}
        {T_e G(E,h)_{\ell+1}} &&&& {T_\nabla A_\nabla(E,h)_\ell} \\
        {T_\nabla(G(E,h)_{\ell+1}\cdot \nabla)} & {\Omega^0(\End(E,h))_\ell} &&& {\Omega^1(\End(E,h))_\ell}
        \arrow["{\rho_\nabla}", from=1-1, to=1-5]
        \arrow[two heads, from=1-1, to=2-1]
        \arrow["\cong"', from=1-5, to=2-5]
        \arrow["\cong", from=2-1, to=2-2]
        \arrow["{\nabla_\ell}", from=2-2, to=2-5]
      \end{tikzcd}
    \end{equation*}
    In particular,
    \begin{align*}
      T_\nabla(G(E,h)_{\ell+1}\cdot\nabla) = \im\rho_\lambda = \im\nabla_\ell.
    \end{align*}
    Now the claim follows from \ref{ex:nabla_hat_elliptic} as $\im\nabla_\ell$
    is closed in $\Omega^1(C,\End(E,h))_\ell$.
  \end{proof}
\end{lemma}

Of course, we have to find a similar harmonicity condition for
Higgs bundles.

\begin{definition}
  A Higgs connection $(\nabla,\Phi)\in A_\Phi(E,h)$
  is \emph{Hitchin} if
  \begin{align}\label{eq:Hitchin_equations}
    \nabla_{0,1}\varphi = 0,\hspace{1cm}
    K(\nabla) + [\varphi,\varphi^\dagger] \in i\mathcal H^2(X)\otimes I
  \end{align}
  for some $\alpha\in\mathcal H^2(C)$ where we write
  $\Phi = \varphi - \varphi^\dagger$ as in \ref{lem:real_vector_space_iso}.
\end{definition}

Here the first equation in (\ref{eq:Hitchin_equations}) merely says that $\Phi$
corresponds to a holomorphic Higgs field on the holomorphic bundle
$(E,\nabla_{0,1})$.  The second equation is a harmonicity condition generalising
\ref{eq:yang_mills_condition}. In particular, if $\Phi = 0$ then $\varphi =
  \varphi^\dagger$ so $(\nabla,0)$ is Hitchin if, and only if, $\nabla$ is
Yang-Mills.

Now consider the space $A_\Phi^{ps}(E,h)$ of Hitchin connections.
This once again gives rise to a Banach manifold $A_\Phi^{ps}(E,h)_\ell$
and, for sufficiently large $\ell$, an action by $G(E,h)_{\ell+1}$.

\begin{lemma}
  Let $(\nabla,\Phi)\in A_\Phi^{ps}(E,h)_\ell$. Then the tangent space
  \begin{align*}
    T_{\nabla,\Phi}(G(E,h)_{\ell+1}\cdot(\nabla,\Phi))
    \subseteq T_{\nabla,\Phi}A_{\nabla,\Phi}^{ps}(E,h)_\ell.
  \end{align*}
  is a closed subspace.
  \begin{proof}
    \missingproof
  \end{proof}
\end{lemma}

Note that in $A^{ps}_\dol(E,h)_\ell$ and $A^{ps}_\Phi(E,h)_\ell$ slices for the
$G(E,h)_{\ell+1}$-action exist everywhere. Hence the quotients
\begin{align*}
  M^{ps}_\nabla(E,h) := A^{ps}_\nabla(E,h)_\ell / G(E,h)_{\ell+1},\hspace{1cm}
  M^{ps}_\Phi(E,h) := A^{ps}_\Phi(E,h)_\ell / G(E,h)_{\ell+1}
\end{align*}
are locally manifolds. Thus one might refer to these as the analytic
moduli spaces of unitary connections. We are, however, going to go
one step further.

\subsection{Reducible Connections}

Let us now ensure that gauge transformations act freely on
our space of connections. We are going to restrict $A^{ps}_\nabla$ and
$A^{ps}_\Phi$ further to ensure that this is the case. Of course we will have
to deal with the loss of information that this entails at a later stage.

In light of our observations in \ref{ex:not_free} and \ref{ex:not_free_double}
we make the following definition:

\begin{definition}
  We define reducibility for both plain holomoprhic bundles and Higgs
  bundles:
  \begin{itemize}
    \item A connection $\nabla\in A_\nabla (E,h)$ is \emph{reducible} if there exists
          a non-trivial proper subbundle $E'\subseteq E$ such that $\nabla$ restricts
          to a connection on $E'$.
    \item Similarly, a Higgs connection $(\nabla,\Phi)\in A_\Phi(E,h)$ is
          \emph{reducible} if there exists a non-trivial proper subbundle $E'\subseteq E$
          such that $(\nabla,\Phi)$ restricts to a Higgs connection on $E'$.
  \end{itemize}
\end{definition}

Reducible connections lead to undesired behaviour. Hence we
consider the spaces of irreducible harmonic connections and their Sobolev
completions
\begin{align*}
  A^s_\nabla(E,h)_\ell \subseteq A^{ps}_\nabla(E,h)_\ell,\hspace{1cm}
  A^s_\Phi(E,h)_\ell \subseteq A^{ps}_\Phi(E,h)_\ell.
\end{align*}
Indeed, this suffices to make the $G(E,h)_{\ell+1}$ action free.

\begin{lemma}\label{lem:free_action}
  The action of $G(E,h)_{\ell+1}$ on $A_\nabla(E,h)_\ell$
  restricts to a free action on $A^s_\nabla(E,h)_\ell$.
  \begin{proof}
    Fix $g\in G(E,h)_{\ell+1}$ and $\nabla\in A^s_\nabla(E,h)_\ell$.
    We need to check that $g\cdot\nabla$ is irreducible. This is clear
    as any proper subbundle $E'$ preserved by $g\cdot\nabla$ would correspond
    to a proper subbundle $g^{-1}(E')$ \todo{check inverse} perserved by $\nabla$.

    To show that the action is free, assume $g\cdot\nabla = \nabla$.
    Then $g\nabla = \nabla g$. One may verify fibre-wise that, if $g$
    is non-trivial, this implies that $\nabla$ fixes a non-trivial
    subspace and hence a non-trivial subbundle, contradicting irreducibility.
  \end{proof}
\end{lemma}

An analogous argument also shows:

\begin{lemma}
  The action of $G(E,h)_{\ell+1}$ on $A_\Phi(E,h)_\ell$
  restricts to a free action on $A^{s}_\nabla(E,h)_\ell$.
\end{lemma}

In \ref{table:unitary_connection_spaces} we have summarised all the spaces
of connections that we have thus far defined.
\begin{table}[h]
  \begin{center}
    \begin{tabular}{|l|lllll|}
      \hline
      \textbf{Elements}   & Proper Action   &             & \begin{tabular}[c]{@{}l@{}}Closed \&\\ Complemented\\ Tangent Space\end{tabular} &             & Free Action       \\ \hline
      Unitary connections & $A_\nabla(E,h)$ & $\supseteq$ & $A^{ps}_\nabla(E,h)$                                                             & $\supseteq$ & $A^s_\nabla(E,h)$ \\ \hline
      Doubled connections & $A_\Phi(E,h)$   & $\supseteq$ & $A^{ps}_\Phi(E,h)$                                                               & $\supseteq$ & $A^s_\Phi(E,h)$   \\ \hline
    \end{tabular}
  \end{center}
  \caption{Spaces of unitary holomorphic connections and their group actions.}
  \label{table:unitary_connection_spaces}
\end{table}
Finally we are able to construct the Banach manifolds of connections
that we are after. Choose $\ell \geq 0$ sufficiently large in the
sense of \ref{ex:unitary_banach_lie_group}. Define the moduli spaces
of irreducible harmonic connections:
\begin{align*}
  M^s_\nabla(E,h) := A^s_\nabla(E,h)_\ell/G(E,h)_{\ell + 1}, \hspace{1cm}
  M^s_\Phi(E,h) := A^s_\Phi(E,h)_\ell/G(E,h)_{\ell + 1}.
\end{align*}
Note that, as the tangent space of $M^s_\nabla(E,h)$ is
contained in the kernel of the operator $\hat\nabla_{\ell-1}$ via
\begin{align*}
  T_\nabla A^s_\nabla(E,h)/G(E,h)_{\ell+1}
  = \im(\nabla_\ell) \subseteq \ker(\nabla_{\ell-1})
  \subseteq \ker(\hat\nabla_{\ell-1})
\end{align*}
In particular, it follows from (\ref{eq:independence_of_l}) that
the tangent space is independent of the choice of $\ell$.

\subsection{On Symplectic Quotients}\label{sec:symplectic_quotients}

In order to construct the moduli spaces as manifolds as directly as
possible, we have left out a lot of the context that they usually appear
in. While this has shortened the process considerably it has left
some of the choices made rather mysterious. First and foremost, it is
not at all clear why the harmonicity conditions on unitary connections and
Higgs connections are obvious choices. This question may be
answered by considering the quotients taken as K\"ahler or Hyperk\"ahler.

While we leave details to other sources such as \missingcitation,
we outline the idea. The key idea of taking such quotients is to
construct a moment map corresponding to the relevant group action.
The existence of a moment map ensures that the group action respects
the symplectic structure. A symplectic manifold $(X,\omega)$
essentially is a real smooth manifold with a choice of inclusion
$\omega : TX \inc T^*X$. If $X$ is finite-dimensional this is
automatically an isomorphism.

More precisely, a moment map of an action of a Lie group $G$ on a symplectic manifold
$(X,\omega)$ is a map $\mu : X \to \mathfrak g^*$
where $\mathfrak g$ is the Lie algebra of $G$ such that
\begin{align*}
  d\langle\mu,A\rangle = \iota_{\sigma(A)}\omega.
\end{align*}
for all $A\in\mathfrak g$ where $\langle \cdot,\cdot\rangle : \mathfrak g^* \times \mathfrak g \to \mathbb R$
is the natural pairing and $\iota_{\sigma(A)} : \Omega^2_{\mathbb R}(X)\to \Omega^1_{\mathbb R}(X)$
is the contraction by the vector field $\sigma(A)$. To take a
quotient $X/G$ then is to consider the topological quotient $\mu^{-1}(0)/G$.
Assuming some reasonable conditions, similar to the ones
in the Banach manifold quotient theorem \ref{thm:banach_quotient},
this quotient may be endowed with a symplectic structure.

One may choose a symplectic form on $A_\nabla(E,h)_\ell$ and
define a moment map for the action of $G(E,h)_{\ell+1}$ such that the
pre-image $\mu^{-1}(0)$ contains precisely the Yang-Mills connections.
Noting that, once a symplectic structure is chosen, the moment map
is essentially unique os it is indeed natural to focus on Yang-Mills
connections to construct the moduli space.

For the case of Higgs bundles, the picture is a little more complicated.
There are three distinct choices of symplectic structure on
$A_\Phi(E,h)_\ell$ hence the standard symplectic quotient is non-canonical.
To get around this, one requires all three moment maps to vanish. It is
precisely this condition that yields the defining equations of
Hitchin connections.

Considering the complex structures on $A_\nabla$ and $A_\Phi$ induced
by the real isomorphism $A_\dol\cong A_\nabla$, the quotients
thus constructed yield K\"ahler and Hyperk\"ahler manifolds, respectively.

\section{Properties of the Analytic Moduli Spaces}\label{sec:properties_of_analytic_spaces}

We have realised the moduli spaces of unitary connections as
Banach manifolds. For the remainder of the chapter, we are going to
study these spaces in more detail.

The most important question that remains is in what sense we have
achieved our original goal of constructing moduli spaces of
holomorphic bundles. In particular, it is not at all clear which subsets
of holomorphic bundles correspond to points in the moduli spaces.
Secondly, we are going to compute the tangent spaces of the moduli
spaces in hopes of gaining some understanding of the manifold structure.
Indeed, it turns out that both moduli spaces are finite-dimensional
and may be endowed with a complex structure. Hence the moduli spaces
of holomorphic bundles are complex manifolds in the usual sense.

In the process we are going to relate Higgs bundles to holomorphic bundles
by regarding Higgs fields as cotangent vectors. This will allow us to
regard the moduli space of the former as a compactification of the moduli
space of the latter. We will revisit this fact in the algebraic discussion.

\subsection{Stable Bundles}\label{sec:analytic_stable_bundles}

Our goal was to construct a moduli space of holomorphic bundles,
i.e. a geometric space whose points are isomorphism classes of
holomorphic bundles. What we did instead was observe that the space
of holomorphic bundles is isomorphic to the space of holomorphic unitary
connections and that the respective actions agree. In order to realise
the quotient of the latter into a Banach manifold, we had to discard
some points. Moving back to the original problem thus raises
the question: Which bundles did we lose in the process?

To answer this question, we define the following:
\begin{definition}\label{def:stable_bundle}
  Let $(E,\dol_E)$ be a holomorphic bundle on a compact complex
  manifold $X$. Then $(E,\dol_E)$ is
  \begin{itemize}
    \item \emph{slope semistable} if, for all holomorphic subbundles
          $(F,\dol_F)\subseteq (E,\dol_E)$,
          \begin{align}\label{eq:stability}
            \frac{\deg F}{\rank F} \leq \frac{\deg E}{\rank E};
          \end{align}
    \item \emph{slope stable} if the inequality (\ref{eq:stability})
          is strict for all holomorphic subbundles.
    \item \emph{slope polystable} if it decomposes as a direct sum
          of slope stable subbundles.
  \end{itemize}
\end{definition}

Now let $A^s_\dol(E)$ and $A^{ps}_\dol(E)$ denote the sets of slope stable
and slope semistable bundles, respectively.
Then we have the following:

\begin{theorem}[Donaldson–Uhlenbeck–Yau]\label{thm:duy}
  The correspondence (\ref{eq:chern_correspondence}) between
  Dolbeault operators on $E$ and holomorphic unitary connections in $(E,h)$
  restricts to maps
  \begin{align}\label{eq:duy}
    A^s_\dol(E) \to A^s_\nabla(E,h), \hspace{1cm}
    A^{ps}_\dol(E) \to A^{ps}_\nabla(E,h).
  \end{align}
\end{theorem}

This theorem is highly non-trivial and we are not going to attempt
to prove it. Instead, let us outline a few more general results
that it may be derived from:

\begin{itemize}
  \item The \emph{Donaldson–Uhlenbeck–Yau theorem}
        \cite{donaldson1985,uy1986}
        in its full generality extends \ref{thm:duy} to general compact
        K\"ahler manifolds. Compact Riemann surfaces
        are K\"ahler so one may simply choose such a structure to
        apply the theorem.
  \item The \emph{Kempf-Ness theorem} \cite{kn1979} gives a homeomorphism between
        symplectic quotients by compact groups and quotients by their
        complexification in the sense of geometric invariant theory (GIT). In
        \ref{sec:symplectic_quotients} we have briefly
        explained how to view $M_\nabla(E)$ as a symplectic quotient
        and in the next chapter we are going to show how to regard
        $M_\dol(E)$ as a GIT quotient.
  \item The \emph{Narasimhan–Seshadri theorem} \cite{ns1965} gives
        a correspondence between slope stable bundles and certain representations
        of the fundamental group. The original proof is algebraic in nature, but in \cite{donaldson1983} Donaldson
        uses more analytic methods. One may use the result by establishing
        a correspondence between connections and representations.
\end{itemize}


\begin{lemma}
  $A^s_{\dol}(E)$ and $A^{ps}_{\dol}(E)$ are open and dense subsets
  of \todo{not done}
\end{lemma}

Noting that the maps (\ref{eq:duy}) are equivariant with respect
to gauge transformations and that the moduli spaces corresponding
to isomorphic smooth bundles are naturally isomorphic, we may define
the moduli spaces as abstract manifolds
\begin{align}\label{eq:abstract_analytic_spaces}
  M^s_\dol(r,d) := M^s_\dol(E) := M^s_\nabla(E,h), \hspace{1cm}
  M^s_\varphi(r,d) := M^s_\varphi(E) := M^s_\Phi(E,h)
\end{align}
for any hermitian bundle $(E,h)$ of rank $r$ and degree $d$.

\subsection{Tangent Space and Dimension}

So far, we have seen that the moduli spaces of stable holomorphic
bundles may be realised as analytic spaces, in particular as
Banach manifolds. In this section we show that they are infact
finite-dimensional complex manifolds. We do this by determining
the tangent space:

\begin{theorem}
  Let $[\dol_E]\in M_\dol(E)$. Then the tangent space at $[\dol_E]$ is
  \begin{align*}
    T_{[\dol_E]}M^s_\dol(E) = H^{0,1}(C,\End(E,\dol_E)).
  \end{align*}
  \begin{proof}[Proof sketch]
    A full proof is given in \cite[223-225]{kobayashi1987}. The key idea is to look
    at curves in $M^s_\dol(E)$ to obtain the tangent space at $\nabla$ as a set of
    equivalence classes in $\Omega^1_{\mathbb{R}}(C,\End(E,h))_\ell$. One then
    identifies this space with $H^{0,1}(C,\End(E,\dol_E))$ via a sequence of calculations
    involving harmonic forms.
  \end{proof}
\end{theorem}

It is worth noting that the proof does not rely on $M^s_\dol(E)$ being a manifold
at all. Indeed, a similar calculation works for a moduli space that is not
in general a manifold, yet it makes sense to classify curves passing through a point.

Having calculated the tangent space, we may observe two key consequences.
Firstly, we see that the moduli space is indeed a complex manifold in the usual sense:

\begin{corollary}
  If $C$ is of genus $g>1$ and $E$ is of rank $r$ then
  $M^s_\dol(E)$ is a complex manifold of dimension
  \begin{align*}
    \dim_{\mathbb{C}} M^s_\dol(E) = (g-1)r^2 + 1.
  \end{align*}
  \begin{proof}
    By Dolbeault's theorem we may express the tangent space in terms of holomorphic
    sections as
    \begin{align*}
      T_{[\dol_E]}M^s_\dol(E) = \Gamma(C,\End(E,\dol_E) \otimes K_C).
    \end{align*}
    The space of holomorphic sections of a holomorphic vector bundle is finite dimensional,
    hence so is $M^s_\dol$. Moreover, the tangent space naturally is a complex vector
    space and hence we have a natural complex structure on $M^s_\dol(E)$.

    \todo{dimension}
  \end{proof}
\end{corollary}

Secondly, we find that the moduli space does not depend on the
choice of $\ell$. This follows because $A_\nabla^{s}(E,h)$
is dense in all its Sobolev completions. Hence every orbit has a
smooth representative meaning that topologically the quotients are
the same. Moreover, the complex manifold structure clearly
does not depend on $\ell$ either.

\subsection{Higgs Bundles as Cotangent Vectors}

The moduli space of Higgs bundles is also finite dimensional. One way to
see this is by considering a differential operator analogous to $\hat\nabla$,
showing that it is elliptic, and using this to prove that the
tangent space is the kernel of a Fredholm operator, i.e. finite
dimensional. We are going to take a different approach.

Recall that the space of Higgs bundles associated to a holomorphic
bundle $(E,\dol_E)$ is the space of holomorphic sections
\begin{align*}
  \Gamma(C,\End(E,\dol_E) \otimes K_C) = H^{1,0}(C,\End(E,\dol_E)).
\end{align*}
By definition of $H^{p,q}$, this is dual to
\begin{align*}
  T_{[\dol_E]}M^s_\dol(E) = H^{0,1}(C,\End(E,\dol_E))
\end{align*}
whenever $(E,\dol_E)$ is stable. Hence we have an inclusion
\begin{align}\label{eq:cotangent_inclusion}
  T^*M^s_\dol(E) \longinc M^s_\varphi(E).
\end{align}
Observe that the holomorphic bundle underlying a stable Higgs bundle need
not be stable. Hence (\ref{eq:cotangent_inclusion}) is not a bijection.
However, we have the next best thing:

\begin{theorem}\label{thm:stables_are_dense}
  $T^*M^s_\dol(E)$ is open and dense in $M^s_\varphi(E)$.
  \begin{proof}
    \missingproof
  \end{proof}
\end{theorem}

Rather than computing the manifold structure on $M_\varphi(E)$
directly, we may now simply deduce it from $T^*M_\dol(E)$.

\begin{corollary}
  $M_\varphi(E)$ is a complex manifold of dimension
  \begin{align*}
    \dim_{\mathbb C}M^s_{\varphi}(E)
    =2\dim_{\mathbb C}M^s_\dol(E)
    =2(g-1)r^2 + 2.
  \end{align*}
  \begin{proof}
    All that remains is to show that the cotangent space is a submanifold.
    The claim then follows as
    \begin{align*}
      T_{[\dol_E,\varphi]}M^s_\varphi(E) = T_{[\dol_E,\varphi]} T^*_{[\dol_E]}M^s_\dol(E)
    \end{align*}
    for $[\dol_E]\in A_\dol(E)$. The fact that the manifold
    structure $T^*M^s_\dol(E)$ agrees with that of $M^s_{\varphi}(E)$
    may be deduced from the observation that
    $M^s_\dol(E) \subseteq M^s_\varphi(E)$ is a submanifold
    via $[\dol_E] \mapsto [\dol_E,0]$.
  \end{proof}
\end{corollary}

\chapter{As Schemes}

The moduli space of holomorphic bundles on a compact Riemann
surface maybe constructed as a complex manifold exclusively
using analytic methods. Since this space was first constructed
in \missingcitation
the theory of moduli spaces has come a long way.
The goal of this section is to formally define moduli spaces
as schemes representing functors, explain how such schemes
are constructed, and hence formally define the schemes whose
points correspond to holomorphic bundles and Higgs bundles
on a compact Riemann surface, respectively.

We begin by translating fundmental notions such as compact
Riemann surfaces, holomorphic bundles, and Higgs bundles into
the language of algebraic geometry. This allows us to precisely state
the moduli problems that we aim to solve. The construction
of the related moduli spaces is achieved in multiple steps which
mirror the analytic approach: First, we observe that the
holomorphic bundles we are interested in may be thought of as
points of certain Quot schemes. Secondly, we define group
actions on these schemes whose orbits correspond to the
equivalence classes we are after. Finally, we use geometric
invariant theory (GIT) to take the quotients by these group
actions. While it is going to require a significant amount of
work to motivate and define the appropriate GIT quotients,
we will see that the same machinery can be used to solve
a plethora of similar problems without much extra work.

Once constructed, we will take some time to study the moduli
spaces which we have thus obtained. It turns out that the GIT
quotients have several desirable properties such as being
quasi-projective varieties. Moreover, we will use deformation
theory to show that holomorphic bundles may still be thought
of as cotangent vectors of holomorphic bundles.

\paragraph*{Notation}

In addtion to the notation used in the previous chapters, unless
otherwise indicated,
\begin{itemize}
  \item all schemes, maps, and products are taken over $\mathbb{C}$,
        i.e.~in the slice category
        $\Sch := \Sch/\Spec\mathbb{C}$;
  \item a projective scheme is a projective scheme over $\mathbb{C}$
        in the sense of \cite{hartshorne1977} - that is, $X$ is projective
        if there exists a closed immersion $X\inc\projective{n}{}$ for
        some $n$;
  \item sheaves on a scheme $X$ are sheaves of $\mathscr O_X$-modules,
        maps of sheaves are $\mathscr O_X$-linear, and tensor products
        of sheaves are taken over $\mathscr O_X$.
\end{itemize}

\section{Holomorphic Bundles as Locally Free Sheaves}

\missingsection

\subsection{Vector Bundles on Schemes}

Let us start the algebraic treatment by defining vector bundles on
schemes. By treating affine opens on a scheme as charts on a manifold
and by regarding the affine space $\affine{n}{}$ as the algebraic
analogue of $\mathbb C^n$, it is straightforward to translate
the definition of a complex vector bundle \ref{def:complex_bundle}
to schemes:

\begin{definition}[{\cite[Definition 11.5]{gortz2010}}]
  \label{def:vector_bundle}
  A \emph{vector bundle} of rank $r$ on a scheme $X$ is
  a map of schemes $\pi : E \to X$ such that there is an open
  cover $X = \bigcup_i U_i$ and isomorphisms
  \begin{align}\label{eq:trivialisation}
    \phi_i : {\pi}^{-1}U_i \cong \affine{r}{}\times U_i
  \end{align}
  such that, for every affine $U = \Spec R \subseteq U_i \cap U_j$,
  the map $\varphi_i \circ \varphi^{-1}_j$ restricts to a $R$-linear
  automorphism of $\affine{r}{}\times U = \Spec R[T_1,\ldots,T_r]$.
\end{definition}

Non-trivial examples are going to arise when we consider vector bundles
as locally free sheaves.

\subsection{Locally Free Sheaves}\label{sec:locally_free_sheaves}

While it was easy to translate vector bundles from manifolds
to schemes, the resulting notion may not always be the most
useful. It is much more natural to talk about sheaves. We know
that with each vector bundle $E\to X$ comes a sheaf of sections
$\Gamma(-,E)$. This has a natural $\mathscr O_X$-module structure.

\begin{lemma}
  The sheaf of sections of a vector bundle $E$ on a scheme $X$ is
  locally free.
  \begin{proof}
    If $E=\affine{r}{}\times X$ is
    trivial then the sections $s\in\Gamma(U,E)$ are just maps
    $s:U\to \affine{r}{}$. If, moreover, $U$ is affine,
    $\Gamma(U,E)$ corresponds to maps of
    $\mathbb{C}$-algebras
    $\mathbb{C}[T_1,\ldots,T_n]\to\mathscr O_X(U)$ so
    $\Gamma(U,E)\cong \mathscr O^r_X(U)$, i.e. $\Gamma(-,E)$
    is free.
    More generally, for each element $U_i$ of the cover in
    \ref{def:vector_bundle}, we can consider affines $U\subseteq U_i$
    to find $\restrict{\Gamma(-,E)}{U_i}\cong \restrict{\mathscr O_X^r}{U_i}$,
    as required.
  \end{proof}
\end{lemma}

However, much more is true. It turns out that on a scheme $X$
a vector bundle of rank $n$ is the same thing as a locally
free sheaf of rank $n$. That is, there is an equivalence
of categories given by $E \mapsto \Gamma(-,E)$.~\cite[128-129]{hartshorne1977}
We will say that a sheaf $\mathscr E$ corresponds to a vector bundle
$E$ and write $\mathscr E=\mathcal V(E)$ to mean
$\mathscr E\cong\Gamma(-,E)$.

\subsection{Compact Riemann Surfaces as Algebraic Curves}
\label{sec:surfaces_as_curves}

In the previous sections we saw that, on schemes, vector bundles are
the same thing as locally free sheaves. It is not difficult to adapt
the argument to obtain an analogous correspondence for manifolds.
However, in order to fully justify an algebraic treatment of the
moduli problem of vector bundles on compact Riemann surfaces, we
need to relate manifolds to schemes. It is not in general possible
to view every manifold as a scheme \missingcitation, it is a well
known fact that compact Riemann surfaces correspond to smooth
algebraic curves.
This fact is proven in various places, see \cite[215]{griffiths1994}
for the general idea and \cite[5-16]{harris2011}
for a comprehensive treatment.

It is worth recalling what the algebraic structure on a compact
Riemann surface $C$ of genus $g\geq 2$ is. The easiest way to obtain
this structure is by constructing an embedding into an ambient
scheme. The correct scheme to consider will be the projectivisation
of an $n$-dimensional complex vector space $V$ which is given
by
\begin{align*}
  \projective{}{}(V) := \Proj\left({
        \bigoplus_{d=0}^\infty \Sym^d(V^\vee)
      }\right)
\end{align*}
Complex points in $\projective{}{}(V)$ may be identified
with one-dimensional subspaces of $V$. \missingcitation
Now if $\mathscr L$ is
an invertible sheaf on $C$ then there is a natural map
\begin{align}\label{eq:natural_line_bundle_map}
  \Gamma(C,\mathscr L)\otimes\mathscr O_C \to \mathscr L
\end{align}
given by $s\otimes f \mapsto f\restrict{s}{U}$. We will be
particularly interested in the case where $\mathscr L$ is a
quotient of the free sheaf
$\Gamma(C,\mathscr L)\otimes\mathscr O_C$ of rank
$\dim_{\mathbb C} \Gamma(C,\mathscr L)$.

\begin{definition}
  A sheaf $\mathscr F$ on a ringed space $X$ is
  \emph{globally generated} if the induced map
  $\Gamma(X,\mathscr F) \otimes \mathscr O_X \to \mathscr F$
  is a surjection.
\end{definition}

\todo{check notation here}
In particular, if $\mathscr L$ is globally generated then,
for every $x\in C$, it induces a surjection on stalks
$\Gamma(C,\mathscr L) \surj \mathscr L_x$ whose kernel is a
subspace of codimension 1, i.e. a closed point of
$\projective{}{}(\Gamma(C,\mathscr L)^\vee)$. It turns out that this map
is an embedding \todo{we should define degree...}
whenever $\deg\mathscr L > 2g$. \cite[Proposition 2.14]{harris2011}
All that is left to do is to find invertible sheaves with large
degree. Fortunately, for any invertible sheaf $\mathscr L$,
the tensor powers $\mathscr L^{\otimes m}$ are invertible and
$\deg\mathscr L^{\otimes m} = m\deg\mathscr L$. Hence any invertible
sheaf of positive degree will do. Fortunately, the degree of
the canonical sheaf $\mathscr K=\Gamma(-,K)^*$ is $2g-2$. As we have restricted
our attention to the case $g\geq 2$, we are free to choose
$m\geq 2$ and hence $\mathscr L = \Omega_C^{\otimes m}$ to obtain
an embedding $C\inc\projective{}{}(\Gamma(C,\mathscr L)^\vee)$, as
required.
By Chow's theorem, this makes $C$ an algebraic subvariety
and hence an algebraic curve.

\begin{lemma}
  Holomorphic bundles on $C$ regarded as a manifold are equivalent
  to vector bundles on $C$ regarded as a scheme.
  \begin{proof}
    \missingproof
  \end{proof}
\end{lemma}

\subsection{Higgs Sheaves}

Under the correspondence outlined above, we may consider the canonical
bundle $K$ on $C$. This then corresponds to a sheaf
\begin{align*}
  \mathscr K := \Gamma(-,K)
\end{align*}
on $C$ regarded as a scheme. This allows us to define Higgs
sheaves on $C$:

\begin{definition}
  A \emph{Higgs sheaf} on $C$ consists of a locally free sheaf
  $\mathscr E$ on $C$ and a map of sheaves
  \begin{align*}
    \varphi : \mathscr E \to \mathscr E \otimes \mathscr K
  \end{align*}
  called a \emph{Higgs field}.
\end{definition}

Just as an analytic Higgs field may be thought of as a holomorphic
section on $\End(E)\otimes K = \Hom(E,E\otimes K)$, an algebraic
Higgs field is equivalently a section of $\Hom(\mathscr E,\mathscr E\otimes K)$.
Moreover, it is straightforward to verify that under the correspondence
$E\to\Gamma(-,E)$ a Higgs field $E\to E\otimes K$ on bundles is
equivalent to a Higgs field $\mathscr E\to\mathscr E\otimes\mathscr K$
on schemes.

Thus we are justified in treating the construction of moduli
spaces of Higgs bundles algebraically.

\subsection{Moduli Problems}

We have shown that holomorphic bundles on a compact Riemann surface
correspond to locally free sheaves on a certain smooth algebraic
curve. Now we would like to construct schemes whose points are
holomorphic bundles. By points of a scheme $M$ one usually means the
closed points. In the case where $M$ is locally of finite type,
closed points are $\mathbb{C}$-points, i.e. maps
$\Spec\mathbb{C}\to M$. \cite[Corollary 3.36]{gortz2010} As we
are hoping for our moduli spaces to be geometrically well-behaved,
this is a reasonable approximation to make. We are now justified in
thinking of a moduli problem as sending a scheme $T$ to the
$T$-points of a hypothetical moduli space.

\begin{definition}
  A \emph{moduli problem} is a functor $\mathcal M:\Sch^{op}\to\Set$.
\end{definition}

Now the obvious notion of a moduli space is a scheme whose functor
of points is precisely the moduli problem. This is called a fine
moduli space:

\begin{definition}
  A \emph{fine moduli space} of a moduli problem $\mathcal M$
  is a scheme $M$ with a natural isomorphism
  $\eta : \mathcal M \cong \Hom(-,M)$.
\end{definition}

Recall that we denote $T$-points of a scheme $S$ by
$S(T) := \Hom(T,S)$. Hence we may abuse notation and write
$\eta : \mathcal M \cong M$. We are often going to surpress the
isomorphism $\eta$ and call $M$ the fine moduli space.

\begin{example}
  \begin{itemize}
    \item projectivisation/projective space
  \end{itemize}
\end{example}

Beyond the fact that a fine moduli spaces represent the moduli
problem and thus are the best possible solution, there are some
pleasant properties to observe.
Consider an element $F\in\mathcal M(T)$, for some $T$. Observe that
under $\eta$, this corresponds to a map $f:T\to M$. By considering
closed points, we have a map sending a closed point $t\in T(\mathbb C)$
to a closed point $f\circ t\in M(\mathbb C)$. We denote th corresponding
map by $t \mapsto F_t$. This lets us consider $F$ as a family of
points in $M$ indexed by $T$.

Moreover, note that there are induced
pullback maps
\begin{align*}
  f^* : \Hom(M,M)\to\Hom(T,M), \hspace{1cm}
  F^* : \mathcal M(M)\to \mathcal M(T).
\end{align*}
Now the element $U\in\mathcal M(M)$ corresponding to the identity on $M$
induces a map $M(\mathbb C)\to M(\mathbb C)$ which allows us to
interpret $U$ as a family of all points of $M$. In particular,
every family in $\mathcal M$ is given by pulling back $U$.
Hence we make the following definition:

\begin{definition}
  Let $\mathcal M$ be a moduli problem and $M$ a fine moduli space.
  The \emph{universal element} $U\in\mathcal M(M)$ is
  $U:=\eta^{-1}_M (\identity)$.
\end{definition}

\begin{example}
  \missingexample
\end{example}

However, it turns out that this is often too much to ask. Indeed,
we really care about the $\mathbb{C}$-points and hence
we are otherwise ready accept some deviation so long as it is not
possible to do better:

\begin{definition}\missingcitation
  A \emph{coarse moduli space} of a moduli problem $\mathcal M$
  is a scheme $M$ together with a natural transformation
  $\eta : \mathcal M \to \Hom(-,M)$ such that
  \begin{itemize}
    \item $\eta : \mathcal M(\Spec\mathbb{C})\to \Hom(\Spec\mathbb{C},M)$ is a bijection;
    \item every natural transformation $\eta' : M \to \Hom(-,M')$ factors through
          $\eta$ as in the diagram
          \begin{equation*}
            % https://q.uiver.app/#q=WzAsMyxbMCwxLCJcXG1hdGhjYWwgTSJdLFsyLDAsIlxcdGV4dHtIb219KC0sTSkiXSxbMiwxLCJcXHRleHR7SG9tfSgtLE0nKSJdLFswLDEsIlxcZXRhIl0sWzAsMiwiXFxldGEnIiwyXSxbMSwyLCIiLDAseyJzdHlsZSI6eyJib2R5Ijp7Im5hbWUiOiJkYXNoZWQifX19XV0=
            \begin{tikzcd}
              && {\text{Hom}(-,M)} \\
              {\mathcal M} && {\text{Hom}(-,M')}
              \arrow[dashed, from=1-3, to=2-3]
              \arrow["\eta", from=2-1, to=1-3]
              \arrow["{\eta'}"', from=2-1, to=2-3]
            \end{tikzcd}
          \end{equation*}
  \end{itemize}
\end{definition}

\begin{example}
  \missingexample
\end{example}

Unfortunately, even coarse moduli spaces may not exist. Here is
one property that prevents a moduli space from existing. We will
see an example of this later. \todo{add reference to problem of all
  locally free sheaves}

\begin{lemma}\label{lem:no_coarse_condition}
  Let $\mathcal M$ be a moduli problem and
  $x,y\in \affine{1}{}(\mathbb{C})$. If there is an
  $F\in\mathcal M(\affine{1}{})$ such that
  $\mathcal M(x)(F) = \mathcal M(y)(F)$ if, and only if,
  $x,y\neq 0$ or $x=y=0$ then there is no coarse moduli space
  for $\mathcal M$.
  \begin{proof}
    Following \cite[Lemma 2.27]{hoskins2016}.
    \missingproof
  \end{proof}
\end{lemma}

By construction, both kinds of moduli spaces are unique up to
isomorphism and every fine moduli space is a coarse moduli
space. Hence solving moduli problems usually proceeds in
three steps:

\begin{enumerate}
  \item Formulate the problem by defining the desired functor of
        points.
  \item Construct a coarse moduli space.
  \item Check under which conditions the moduli space is fine.
\end{enumerate}

\section{Invariants of Sheaves}

Vector bundles correspond to locally free sheaves. Hence it is worth
taking some time to study locally free sheaves and, more generally,
sheaves on schemes. This is of course a very wide field
so we are going to highlight only a few key properties.

The main goal of this section is to establish some topological
invariants of coherent sheaves, generalising the grouping of
vector bundles by rank and degree. The obvious tool for studying
the topology of sheaves is cohomology. While we will have little
to do with sheaf cohomology in its raw form, we are going to
observe several properties of sheaves on curves and, more generally,
projective schemes that are going to prove useful.

\subsection{Twisting}

We begin by revisiting one of the most important family of sheaves
on projective schemes. See e.g. \cite{gortz2010} or
\cite{hartshorne1977} for detailed treatments.

Consider a projective scheme $X$. Recall that we may write
$X=\Proj R$ for some $R=\mathbb{C}[T_0,\ldots,T_n]/I$ \cite[II Corollary 5.16]{hartshorne1977}. Such a scheme comes
equipped with a particularly important family of invertible sheaves
called Serre's twisting sheaves. To define them, recall that each
graded $R$-module $M$ defines a unique sheaf $\tilde M$ on $X$
that satisfies $\tilde M (D_+(f)) = M_{(f)}$ where $M_{(f)}$
denotes the homogeneous localisation $M$ at $f\in R$. This
correspondence between modules and sheaves allows the following
definition:

\begin{definition}[{\cite[13.4]{gortz2010}}]
  For $m\in\mathbb{Z}$ and some graded ring $R$, define the graded
  $R$-module $R(m)$ by $R(m)_d := R_{m+d}$ for all $d\in\mathbb{Z}$. \emph{Serre's
    twisting sheaf} is
  \begin{align*}
    \mathscr O_X(m) := \widetilde{R(m)}.
  \end{align*}
  More generally, for a sheaf $\mathscr F$ on $X$, define
  $\mathscr F(m) := \mathscr F \otimes \mathscr O_X(m)$.
\end{definition}
The notation $\mathscr O_X(m)$ is consistent with $\mathscr F(m)$
as $\mathscr O_X(m) \cong \mathscr O_X \otimes \mathscr O_X(m)$.
Of course, the main case of interest for us is
$R=\mathbb{C}[T_0,\ldots,T_n]$, i.e. $X=\projective{n}{}$.
In this case, the twisting sheaf corresponds to a line bundle,
i.e. is invertible. That is, it has rank 1.

\begin{proposition}[{\cite[Proposition 13.15]{gortz2010}}]
  \label{thm:twisiting_sheaf_invertible}
  If $R$ is finitely generated as a $R_0$-algebra, then each
  $\mathscr O_X(m)$ is an invertible sheaf.
  \begin{proof}
    \missingproof
  \end{proof}
\end{proposition}

\begin{example}
  Let us calculate the twisting sheaf $\mathscr O(1)$ on $\projective{n}{}$
  more explicitly.
  \missingexample
\end{example}

Why are the twisting sheaves important to us? They are well-behaved
with respect to the parameters. In particular,
under the assumption of \ref{thm:twisiting_sheaf_invertible},
$(\mathscr F,m)\mapsto \mathscr F(m)$ defines an action of
the integers on the group of $\mathscr O_X$-modules under
$\otimes$:

\begin{lemma}\label{lem:additivity_twisting_sheaf}
  Let $R$ be finitely generated as a $R_0$-algebra and $m,n\in\mathbb{Z}$.
  Then $\mathscr O_X(m) \otimes \mathscr O_X(n) \cong \mathscr O_X(m + n)$. Hence
  $\mathscr F(m) \otimes \mathscr O_X(n) \cong \mathscr F(m+n)$.
  \begin{proof}
    \missingproof
  \end{proof}
\end{lemma}

We will see that topological properties of sheaves behave well
with respect to taking direct sums and tensor products and hence
with respect to twisting, too. Moreover, for suitable $R$ and
sufficiently large $m$, $\mathscr F(m)$ is globally generated. In this
case, for $m\geq 1$, the sheaves $\mathscr O_X(m)$ are ample.

\subsection{Ample sheaves}

Recall the definition of an ample sheaf:

\begin{definition}
  An invertible sheaf $\mathscr L$ on a quasi-compact quasi-separated
  scheme $X$ is \emph{ample} if, for all quasi-coherent sheaves of
  finite type $\mathscr F$, for $m$ sufficiently large,
  $\mathscr F\otimes \mathscr L^{\otimes m}$ is
  globally generated.
\end{definition}

On general schemes, ample sheaves need not exist. However, on
projective space the twisting sheaves serve as an example:

\begin{proposition}
  Let $R$ be a finitely generated $R_0$-algebra,
  write $X=\Proj R$ and let $m\geq 1$. Then
  $\mathscr O_X(m)$ is ample.
  \begin{proof}
    See \cite[Example 13.45]{gortz2010}.
  \end{proof}
\end{proposition}

More generally, any quasi-projective quasi-compact scheme has
an ample sheaf given by pulling back a twisting sheaf:

\begin{proposition}
  Let $X$ be projective, $U\subseteq X$ quasi-compact, and
  $j : U \inc X$ an open immersion. Then, for some $m\geq 1$,
  $j^*\mathscr O_X(m)$ is ample.
  \begin{proof}
    See \cite[\href{https://stacks.math.columbia.edu/tag/01Q2}{Tag 01Q2}]{stacks-project}.
  \end{proof}
\end{proposition}

\begin{corollary}
  Every quasi-projective scheme has an ample sheaf.
\end{corollary}

\todo{motivate all this a little bit more}

\subsection{Euler Characteristic}

The fundamental topological invariant of sheaves that we are going
to be interested in is the Euler characteristic. This is defined
in the obvious way, replacing regular cohomology with sheaf
cohomology:

\begin{definition}
  The \emph{Euler characteristic} of a coherent sheaf $\mathscr F$
  on a projective scheme $X$ is
  \begin{align*}
    \chi (X,\mathscr F) := \sum_{j=0}^\infty (-1)^j \dim H^j (X,\mathscr F).
  \end{align*}
\end{definition}

Note that, by finite-dimensionality of coherent cohomology
(e.g. \cite[\href{https://stacks.math.columbia.edu/tag/02O6}{Tag 02O6}]{stacks-project}) and Grothendieck's vanishing theorem (e.g.
\cite[III Theorem 2.7]{hartshorne1977}), this is sum is finite for
all $\mathscr F$.

\begin{example}
  Our primary interest will be in the case where
  $\dim X = 1$, so $H^j(X,\mathscr F)=0$ for $j\geq 2$ and
  \begin{align*}
    \chi (X,\mathscr F) = \dim H^0(X,\mathscr F)-\dim H^1(X,\mathscr F).
  \end{align*}
  If we moreover take $\mathscr F=\mathscr O_X$ then
  $\chi (X,\mathscr O_X) = 1 - g$. \todo{justify}
\end{example}

\begin{example}
  Consider the case $X=\projective{n}{}$.
  From \missingcitation we know that
  % https://swc-math.github.io/notes/files/06StillmanNotes.pdf
  % https://achinger.impan.pl/fac/fac.pdf
  \begin{align*}
    \dim H^j(\projective{n}{},\mathscr O_{\projective{n}{}}(m)) =
    \begin{cases}
      (n+1)^m         & \text{if }j = 0 \\
      0               & \text{if }0<j<n \\
      \binom{-n-1}{m} & \text{if }j=n
    \end{cases}
  \end{align*}
  Hence \todo{doublecheck}
  \begin{align*}
    \chi(\projective{n}{},\mathscr O_{\projective{n}{}}(m))
    = (n+1)^m + \binom{-n-1}{m}.
  \end{align*}
\end{example}

\begin{lemma}
  \todo{additivity on short exact sequences}
\end{lemma}

\begin{lemma}\label{lem:hilbert_base_change}
  \todo{invariance under base change}
\end{lemma}

\subsection{Degree}

Recall that for vector bundles on a compact Riemann surfaces we defined the
degree.  The Euler characteristic allows us to make a similar definition for
locally free sheaves on the curve $C$.

\begin{definition}
  The \emph{degree} of a locally free sheaf $\mathscr F$ of rank $r$
  on $C$ is
  \begin{align*}
    \deg \mathscr F := \chi (C,\mathscr F) - r\chi(C,\mathscr O_C).
  \end{align*}
\end{definition}

\begin{example}
  As the rank of $\mathscr O_C$ is $1$ its degree must be $0$.
\end{example}

\begin{lemma}\label{lem:degree_of_tensor}
  If $\mathscr E$ and $\mathscr F$ are locally free on $C$ then
  \begin{align*}
    \deg(\mathscr E\otimes\mathscr F) = \rank\mathscr E\deg\mathscr F + \rank\mathscr F\deg\mathscr E.
  \end{align*}
  \begin{proof}
    \cite[Exercise 8.24]{hoskins2016}.
    \missingproof
  \end{proof}
\end{lemma}

\begin{example}
  Using~\ref{lem:degree_of_tensor}, we find \todo{make sure $\mathscr O_X(1)$ has degree 1}
  \begin{align*}
    \deg \mathscr E(m) = \deg\mathscr E + m\rank\mathscr E.
  \end{align*}
\end{example}

\begin{theorem}[Riemann-Roch]\label{thm:riemann_roch}
  Let $\mathscr F$ be a locally free sheaf of rank $r$ and degree $d$
  on $C$. Then
  \begin{align*}
    \chi(\mathscr F) = d + r(1-g).
  \end{align*}
  \begin{proof}
    \missingproof
  \end{proof}
\end{theorem}
\subsection{Hilbert Polynomials}

While the Euler characteristic certainly is a topological invariant,
it is rather limiting. Rather than focusing on a single integer,
the Euler characteristic,
we are going to keep track of all the Euler characteristics of all
the twists of a sheaf. This turns out to be described by a polynomial.

Consider a projective scheme $X$ with an ample sheaf $\mathscr L$
and a coherent sheaf $\mathscr F$.

\begin{definition}
  The \emph{Hilbert polynomial} of $\mathscr F$ at $m\in\mathbb{Z}$
  is
  \begin{align*}
    P(\mathscr F,\mathscr L)(m) := \chi(X,\mathscr F \otimes \mathscr L^{\otimes m}).
  \end{align*}
\end{definition}

A priori, the Hilbert polynomial is a function taking integers to
integers. The fact that it is described by a polynomial is
surprising and non-trivial.

\begin{lemma}
  There exists a unique polynomial $p\in\mathbb{Q}[t]$ such that,
  for all $m$, $P(\mathscr F,\mathscr L)(m) = p(m)$.
  \begin{proof}
    See \cite[Lemma 1.2.1]{huybrechts2010}.
  \end{proof}
\end{lemma}

\begin{example}\label{ex:hilbert_polynomial}
  If $\mathscr F$ is a locally free sheaf of rank $r$ and degree $d$
  on $C$ then we use Riemann-Roch \ref{thm:riemann_roch} and
  \ref{lem:degree_of_tensor} to calculate
  \begin{align*}
    P(\mathscr F, \mathscr L)(t) = rt + d + r(1-g)
  \end{align*}
  for any degree 1 invertible sheaf $\mathscr L$ on $C$.
\end{example}

Serre's vanishing theorem \missingcitation states that,
for $m$ sufficiently large and $j>0$,
$H^j(X,\mathscr F\otimes\mathscr L^{\otimes m})=0$. Hence,
eventually,
$P(\mathscr F,\mathscr L)(m) = \dim \Gamma(X,\mathscr F\otimes \mathscr L^{\otimes m})$.

\begin{lemma}
  \todo{invariance under base change}
\end{lemma}


\missingsection

\section{Locally Free Sheaves and Quot Schemes}

The first step when constructing the analytic moduli spaces of
holomorphic bundles was to consider the slightly larger space of
all bundles of a certain rank and degree, including repeated
occurences of isomorphic instances. While this preliminary space
did not serve as a true
classification,
it provided us with a good starting point for further analysis. As the
algebraic construction follows a similar procedure, it is now
time to construct an analogous scheme.

We define a much more general moduli problem called
the Quot functor which was shown to have a fine moduli space by
Grothendieck. This is the first piece of machinery which will do a
lot of the work for us and for many other moduli problems of vector
bundles. We will show that there is a Quot scheme whose points
may be thought of as locally free sheaves of a certain rank and
degree.

\subsection{All Locally Free Sheaves}

Our approach to constructing moduli spaces of equivalence classes of
sheaves is to consider a larger space and then quotient by a suitable
group action. In the analytic case, we started with the affine space of
all holomorphic bundles. In this section we are going to see that
the same approach does not quite work in the algebraic setting.

The first step is to define the correct moduli problem. The key idea
is to associate to a scheme $T$ a family of locally free sheaves
$\mathscr E_t$. One may do so by considering a sheaf $\mathscr E$ on
$C_T := C\times T$. If we have a point $t\in T$
then we have the fibre $C_t := \Spec k(t) \times C$ and the sheaf
$\mathscr E_t := \restrict{\mathscr E}{C_t}$. In the case where
$t$ is a closed point, $C_t \cong C$ so $\mathscr E_t$ is indeed
a sheaf on $C$. Moreover, if $\mathscr E$ is flat over $T$ then
each $\mathscr E_t$ has the same Hilbert polynomial.

To truly view locally free sheaves $\mathscr E,\mathscr F$ on
$C_T$ flat over $T$ as a family, we need to adjust our notion of
equivalence. In particular, we require all the fibres to be isomorphic.

\begin{lemma}
  Let $\mathscr E$ and $\mathscr F$ be locally free sheaves on $C_T$
  flat over $T$. Then the following are equivalent:
  \begin{enumerate}
    \item For all $t\in T$, there is an isomorphism of fibres
          $\mathscr E_t\cong\mathscr F_t$.
    \item There exists an invertible sheaf $\mathscr L$ on
          $T$ such that $\mathscr E \cong \mathscr F \otimes \pi^*\mathscr L$
          where $\pi : C_T \to C$ is the base change.
  \end{enumerate}
  \begin{proof}
    \todo{not sure this is true; if it isn't then we need to find a
      different explanation}
    \missingproof
  \end{proof}
\end{lemma}

Hence we make the following definition:

\begin{definition}
  The moduli problem of locally free sheaves on $C$ is given by the
  functor $\mathcal A_h : \Sch^{\text{op}} \to \Set$
  mapping schemes $T$ to
  \begin{align*}
    \mathcal A_h(T) := \left\lbrace{\text{locally free sheaves $\mathscr E$ on $C_T$ flat over $T$}}\right\rbrace/\sim
  \end{align*}
  where $\mathscr E\sim\mathscr F$ if, and only if, there is a line bundle
  $\mathscr L$ on $T$ such that
  $\mathscr E \cong \mathscr F\otimes\pi^* \mathscr L$ and sending
  maps $f: T'\to T$ to the pullback $f^*$.
\end{definition}

However, this fails to have any moduli space at all:

\begin{lemma}\label{lem:no_coarse_moduli_space}
  The moduli problem $\mathcal A$ of locally free sheaves on $C$
  does not admit a coarse moduli space.
  \begin{proof}
    Follwing \cite[Example 2.2]{hoskins2016}. We aim to construct
    an $\mathscr E\in\mathcal A_h(\affine{1}{})$ that satisfies the
    condition in \ref{lem:no_coarse_condition}.
    \missingproof
  \end{proof}
\end{lemma}

\subsection{All Higgs Sheaves}

Let us turn to the case of Higgs sheaves. Consider the moduli problem
given by \todo{add new equivalence; make sure the canonical sheaf restricts to a canonical sheaf}
\begin{align*}
  \mathcal A_H(T) = \left\lbrace{\text{Higgs sheaves $(\mathscr E,\varphi)$
      on $X_T$}}\right\rbrace/\sim.
\end{align*}
where $(\mathscr E,\varphi)\sim(\mathscr E',\varphi')$ if, and only if,
there is an isomorphism $\mathscr E\cong\mathscr E'$ that makes the
following commute:\todo{this is not the correct canonical bundle}
\begin{equation*}
  % https://q.uiver.app/#q=WzAsNCxbMCwwLCJcXG1hdGhzY3IgRSJdLFswLDEsIlxcbWF0aHNjciBFJyJdLFsyLDEsIlxcbWF0aHNjciBFJ1xcb3RpbWVzXFxtYXRoc2NyIEtfe0NcXHRpbWVzIFR9Il0sWzIsMCwiXFxtYXRoc2NyIEVcXG90aW1lcyBcXG1hdGhzY3IgS197Q1xcdGltZXMgVH0iXSxbMCwzLCJcXHZhcnBoaSJdLFsxLDIsIlxcdmFycGhpIl0sWzAsMSwiXFxjb25nIiwyXSxbMywyLCJcXGNvbmciXV0=
  \begin{tikzcd}
    {\mathscr E} && {\mathscr E\otimes \mathscr K_{C\times T}} \\
    {\mathscr E'} && {\mathscr E'\otimes\mathscr K_{C\times T}}
    \arrow["\varphi", from=1-1, to=1-3]
    \arrow["\cong"', from=1-1, to=2-1]
    \arrow["\cong", from=1-3, to=2-3]
    \arrow["\varphi", from=2-1, to=2-3]
  \end{tikzcd}
\end{equation*}
Once again, by restricting to fibres over $t\in T(\mathbb{C})$
a family of Higgs sheaves $(\mathscr E,\varphi)$ on $X_T$ yields
a Higgs field
\begin{align*}
  \phi_t : \mathscr E_t \to \mathscr E_t \otimes \Omega^1_C.
\end{align*}
Unfortunately, we find ourselves in a similar position.
\begin{corollary}
  The moduli problem $\mathcal A_H$ of Higgs sheaves on $C$ does not admit
  a coarse moduli space.
  \begin{proof}
    Consider the natural transformation $\mathcal A_h\to\mathcal A_H$
    given by $\mathscr E \mapsto (\mathscr E,0)$. Now the sheaf
    $\mathscr E\in\mathcal A_h(\affine{1}{})$ from the proof of
    \ref{lem:no_coarse_moduli_space} pushes forward to a Higgs field
    $(\mathscr E,0)\in\mathcal A_H(\affine{1}{})$ which, moreover,
    satisfies the same property.
  \end{proof}
\end{corollary}
Thus it is not possible to construct the moduli spaces we desire
in the naive way. There are essentially two ways to get around this.
One option is to deal with
moduli stacks instead. This has the advantage of solving the problem
without discarding any information. However, it involves dealing with
algebraic stacks which are unwieldy, even more so than schemes.
See \cite{cm2017} for this point of view. We will instead take an
approach that is much in line with the analytic construction by narrowing
our attention to a subclass of locally free sheaves. Indeed, we are
going to end up with those sheaves corresponding to (semi-)stable
bundles.

\subsection{Quotient Sheaves} \todo{fix $\mathscr E$ vs $\mathscr F$}

Choosing an isomorphism $\mathbb C^n \cong \Gamma(C,\mathscr F)$
allows us to think of globally generated sheaves $\mathscr F$
as quotients $\mathscr O_X^{\oplus n} \surj \mathscr F$.
It will turn out that being globally generated is a reasonable
condition in the sense that we are not discarding too many
vector bundles. Thus it is natural to study quotients of coherent sheaves,
not to be confused with the quotients of geometric spaces that
are the overarching theme of this paper.

\begin{definition}
  Let $\mathscr E$ be a coherent sheaf on a scheme $X$.
  A \emph{family of quotients} of $\mathscr E$ over a scheme $T$
  consists of a coherent sheaf $\mathscr F$ on $X_T$ flat over $T$ and a
  surjection $q:\pi^*\mathscr E\surj\mathscr F$ where
  $\pi : X_T \to X$ is the base change.
\end{definition}

Note the condition that the quotient be flat. Recall that a sheaf
$\mathscr F$ on $X_T$ is flat over $T$ if each of the functors
$M \mapsto \mathscr F_{(x,t)} \otimes_{\mathscr O_{T,t}} M$ is exact.
In light of \cite[III Theorem 9.9]{hartshorne1977} this is a
particularly useful condition for us as it ensures that the fibres
$\mathscr F_t$ all have the same Hilbert polynomial
for all $t$ in the same connected component. In particular,
we may define the Hilbert polynomial of a family
$q : \pi^*\mathscr E \surj \mathscr F$ of quotients over $T$ to be
$P(\mathscr F_t,\mathscr L)$ for any $t\in T$.

Two families $\mathscr F$ and $\mathscr F'$ of quotients of a
sheaf $\mathscr E$ are equivalent if there is an isomorphism
$\mathscr F\cong\mathscr F'$ such that the following commutes:
\begin{equation}\label{eq:quotient_equivalence}
  % https://q.uiver.app/#q=WzAsMyxbMCwwLCJcXG1hdGhjYWwgRSJdLFsyLDAsIlxcbWF0aGNhbCBGIl0sWzIsMSwiXFxtYXRoY2FsIEYnIl0sWzEsMiwiXFxjb25nIl0sWzAsMSwicSIsMCx7InN0eWxlIjp7ImhlYWQiOnsibmFtZSI6ImVwaSJ9fX1dLFswLDIsInEnIiwyLHsic3R5bGUiOnsiaGVhZCI6eyJuYW1lIjoiZXBpIn19fV1d
  \begin{tikzcd}
    {\mathscr E} && {\mathscr F} \\
    && {\mathscr F'}
    \arrow["q", two heads, from=1-1, to=1-3]
    \arrow["{q'}"', two heads, from=1-1, to=2-3]
    \arrow["\cong", from=1-3, to=2-3]
  \end{tikzcd}
\end{equation}

\subsection{Quot Functors}

Let us now consider the moduli problem of families of
quotients of a coherent sheaf $\mathscr E$ on $X$ up to equivalence.
To do this, we need to define a functor. In particular, we need to
each map of quotients a map of sets. Fortunately, base change preserves flat families.
In particular,
if $q : \pi^*\mathscr E \surj \mathscr F$ is a family of quotients over $T$ and
$f : T' \to T$ is a morphism then we have an equality
$\pi^*\mathscr E = (\identity\times f)^*\pi^*\mathscr E$
of sheaves on $X\times T'$.
Moreover, as flatness is preserved under base change \cite[\href{https://stacks.math.columbia.edu/tag/01U9}{Tag 01U9}]{stacks-project}
$(\identity\times f)^*q : \pi^*\mathscr E \surj (\identity\times f)^*\mathscr F$
is a family of quotients over $T'$. Thus we may define the moduli problem:

\begin{definition}
  Let $\mathscr E$ be a coherent sheaf on a projective scheme $X$.
  The corresponding \emph{Quot functor} is given by sending each
  scheme $T$ to
  \begin{align*}
    \mathcal Quot (\mathscr E)(T) = \left\lbrace{
      \text{families of quotients $q : \pi^*\mathscr E \surj \mathscr F$
        over $T$}
    }\right\rbrace / \sim
  \end{align*}
  and $f : T' \to T$ to
  $\mathcal Quot(\mathscr E)(f) = (\identity \times f)^*$ where
  $\pi^*\mathscr E$ is the pullback of $\mathscr E$ along
  the base change $\pi:X\times T\to X$.
\end{definition}

As the Hilbert polynomial is invariant under base change
\ref{lem:hilbert_base_change}, the Quot functor splits as
\begin{align*}
  Quot(\mathscr E)
  = \bigsqcup_{P\in\mathbb{Q}[t]} Quot(\mathscr E)^{P,\mathscr L}
\end{align*}
where each $Quot(\mathscr E)^{P,\mathscr L}$ contains families
with Hilbert polynomial $P$. One needs to be a little bit careful
about non-connected $T$ (see e.g. the Remark in \cite[5-6]{siddharth2016}), but our main application will be
$T = \Spec\mathbb C$ where the above formula holds in the obvious way.

\begin{example}
  Recall from \ref{ex:hilbert_polynomial} that the Hilbert polynomial of a
  locally free sheaf on $C$ with respect to a fixed invertible sheaf depends
  only on its rank and degree.  Thus, for any locally free sheaf $\mathscr E$ on
  $C$, we have a splitting
  \begin{align}\label{eq:quot_functor_split_on_c}
    Quot(\mathscr E) = \bigsqcup_{r,d} Quot(\mathscr E)^{r,d}.
  \end{align}
  \missingexample
\end{example}

\subsection{Quot Schemes}

One of the reasons the Quot functor is of interest to us is because
Grothendieck constructed a fine moduli space for it. \missingcitation

\begin{theorem}
  Let $X$ be a projective scheme with an ample sheaf $\mathscr L$,
  let $\mathscr E$ be a coherent sheaf on $X$, and let
  $p\in\mathbb{Q}[t]$. Then the Quot functor
  $\mathcal Quot(\mathscr E)^{p,\mathscr L}$ has a fine moduli space
  $\Quot(\mathscr E)^{p,\mathscr L}$ called the \emph{Quot scheme}.
  Moreover, $\Quot(\mathscr E)^{p,\mathscr L}$ is projective.
  \begin{proof}[Proof idea]
    The key idea is to realise the Quot functor as a subfunctor
    of the Grassmann functor
    \begin{align*}
      Gr(V,m) := Quot(\mathscr O^{\oplus n}_{\Spec\mathbb C})^{n-m}
    \end{align*}
    of $m$-dimensional subspaces of $V\cong\mathbb C^n$. This is
    well-known to be representable by a projective variety.
    \cite[Proposition 8.14]{gortz2010}
    In particular, one obtains an embedding
    \begin{align}\label{eq:quot_embedding}
      Quot^{p,\mathscr O(1)}_{\projective{n}{}}(\mathscr O^{\oplus r}_{\projective{n}{}})
      \longinc Gr(V,p(m))
      \longinc \projective{}{}(\wedge^{p(m)}V^*)
    \end{align}
    where $V := \Gamma(\mathscr O^{\oplus r}_{\projective{n}{}})$
    in the case where $X=\projective{n}{}$, $\mathscr E=\mathscr O^{\oplus r}$,
    and $\mathscr L=\mathscr O(1)$. The general case is obtained by
    twisting.
    It then remains to show that the image of this map is closed.
    For an extensive overview see \cite[70-73]{hoskins2016}.
  \end{proof}
\end{theorem}

\begin{example}\label{ex:quot_scheme_of_lf}
  Once again, in the case $X=C$ we have schemes
  $Quot(\mathscr E)^{r,d}$ representing the functors in
  (\ref{eq:quot_functor_split_on_c}). Now consider a locally free
  sheaf $\mathscr F$ on $C$ of rank
  $r$ and degree $d$ with $H^1(C,\mathscr F)=0$. By Riemann-Roch
  \ref{thm:riemann_roch} we have
  \begin{align*}
    n := \dim\Gamma(C,\mathscr F) = \chi(C,\mathscr F) = d + r(1-g).
  \end{align*}
  If, moreover, $\mathscr F$ is globally generated then it
  corresponds to a quotient $\mathscr O^{\oplus n}_C\surj \mathscr F$,
  i.e. a closed point of
  \begin{align*}
    Q(r,d) := \Quot(\mathscr O^{\oplus n}_C)^{r,d}
  \end{align*}
  Now consider the universal quotient family
  $\hat q : \mathscr O^{\oplus n}_{C\times Q(r,d)} \surj \hat{\mathscr E}$.
  Then each quotient sheaf of $\mathscr O_C^{\oplus n}$ is given by
  $\hat{\mathscr E}_q$ for some $q\in Q(r,d)$. In light of the discussion
  above, define
  \begin{align*}
    Q_\circ(r,d) \subseteq Q(r,d) := \Quot(\mathscr O^{\oplus n}_C)^{r,d}
  \end{align*}
  to consists of all $q\in Q(r,d)$ such that $\hat{\mathscr E}_q$ is locally
  free, $H^1(C,\hat{\mathscr E}_q)=0$, and the map
  $\Gamma(C,\mathscr O^{\oplus n}_C)\surj\Gamma(C,\hat{\mathscr E}_q)$ is
  an isomorphism.
\end{example}

Some creativity is required to extend this procedure to Higgs
sheaves. What we are looking for is a scheme $F(r,d)$ whose closed points
are pairs $(q,\varphi)$ where $q$ is a closed point
in $Q_\circ(r,d)$ and $\varphi : \hat{\mathscr E}_q \to \hat{\mathscr E}_q \otimes \mathscr K$.
We are going to use the following lemma to obtain such a scheme:

\begin{lemma}\label{lem:linear_scheme}
  Let $S$ be a reduced scheme and $\mathscr F$ a coherent sheaf over
  $C\times S$, flat over $S$. Then the functor on $(\Sch/S)^{op}$ given by
  \begin{align*}
    (f : T\to S) \mapsto \Gamma(C\times T,(\identity\times f)^*\mathscr F)
  \end{align*}
  where $\pi : C\times T \to T$ is representable.
  \begin{proof}
    \cite[Lemma 3.5]{nitsure1991}
    \missingproof
  \end{proof}
\end{lemma}

\begin{example}\label{ex:quot_scheme_of_higgs}
  Fix a Higgs sheaf $(\mathscr E,\varphi)$ such that
  $\mathscr E = \hat{\mathscr E}_q$ for some $q\in Q_\circ(r,d)$.
  Note that $Q_\circ(r,d)$ is a reduced open subscheme
  \cite[281]{nitsure1991}, hence lemma
  \ref{lem:linear_scheme} applies.
  Considering the sheaf $\Hom(\hat{\mathscr E},\hat{\mathscr E}\otimes K)$ on
  $C\times Q_\circ(r,d)$ we find that the functor
  \begin{align*}
    (f:T\to Q(r,d))
    \mapsto
    \Gamma(C\times T,(\identity\times f)^*\Hom(\hat{\mathscr E},\hat{\mathscr E}\otimes\pi^*\mathscr K))
  \end{align*}
  where $\pi : C\times Q_\circ(r,d)\to C$ is represented by a
  $Q_\circ(r,d)$-scheme $F(r,d)$.
  In particular, a closed point of $F(r,d)$ corresponds to a
  closed point $q\in Q_\circ(r,d)$ and a $q$-point of $F(r,d)$.
  We may now choose $q$ such that $\mathscr E = \hat{\mathscr E}_q$ and note
  that
  \begin{align*}
    F(r,d)(q) = \Gamma(C,\Hom(\mathscr E,\mathscr E\otimes\mathscr K)).
  \end{align*}
  To see this, a few things need to be checked. Firstly, pullbacks
  commute with taking tensor products. Secondly,
  \begin{align*}
    \Gamma(C\times\Spec\mathbb C,(\identity\times q)^*\pi^*\mathscr K)
    \cong \Gamma(C,\mathscr K).
  \end{align*}
  Thus $\varphi$ corresponds to an $q$-point of $F(r,d)$
  and hence $(\mathscr E,\varphi)$ corresponds to a closed point in
  $F(r,d)$. The structure map
  \begin{align*}
    \tau : F(r,d) \longrightarrow Q_\circ(r,d)
  \end{align*}
  is just the forgetful map $(\mathscr E,\varphi) \mapsto \mathscr E$
  on closed points.
\end{example}

We have thus constructed schemes that may be thought of as containing
the sheaves that we are interested in. It is worth recalling two
shortcomings of this setup. Firstly, in both cases we had to make
minor assumptions about the sheaves, such as being globally generated
or having vanishing cohomology. We are going to have to make sure that
we did not assume too much. Secondly, it is not the case that every
closed points of these schemes corresponds to a vector bundle. This is
not going to be an issue as the quotient construction is naturally
going to result in further restrictions which are going to exclude
all the points that are not of interest to us.

\section{Quotients of Schemes}

The second major step in constructing moduli spaces of holomorphic
bundles, algebraic or analytic, is to take a quotient by an automorphism
group. In particular, we noticed that different choices of isomorphism
$\Gamma(X,\mathscr F)\cong\mathbb{C}^N$ give rise to different quotients.
The goal of this section is to eliminate these redundancies.

The first step is defining group actions in the setting of schemes.
This allows us to make precise what we mean by a quotient with
respect to such an action and what properties we would like such
quotients to posses.

There is a wide range of introductory material to group schemes.
What we need may be found in \cite{hoskins2016}. A more general
treatment is given in \cite{milne2017}.

\subsection{Group Schemes}

Category theory tells us what the correct notion of a group in the
settings of schemes is:

\begin{definition}
  A \emph{group scheme} is a group object in $\Sch$.
\end{definition}
In particular, a group scheme consists of a scheme $G$ and three maps
\begin{align}\label{eq:group_maps}
  \eta:\Spec\mathbb{C}\to G,\hspace{1cm}
  \iota:G\to G,\hspace{1cm}
  \mu:G\times G\to G
\end{align}
satisfying the usual conditions of unitality, inverses, and associativity.
As usual with classical groups, we are going to leave the maps
$\eta$, $\iota$, and $\mu$ implicit.

In the affine case $G = \Spec R$ the multiplication and inverse maps
correspond to $\mathbb{C}$-algebra homomorphisms
$\mu^\sharp : R \to R\otimes R$ and $i^\sharp : R\to R$.
If one thinks of elements $f\in R$ as functions $G\to\mathbb C$
then the comulitiplication is given by $\mu^\sharp(f)(g,h) = f(\mu(g,h))$
for $f\in R$ and $g,h\in G$.

\begin{example}\label{ex:group_schemes}
  There are several affine group schemes that we are already
  intuitively familiar with:
  \begin{enumerate}
    \item The additive group $\mathbf{G}_a := \Spec\mathbb C[t]$ with
          comulitiplication $t\mapsto t\otimes 1 + 1\otimes t$,
    \item The multiplicative group $\mathbf{G}_m := \Spec\mathbb C[t^\pm]$
          with comultiplication $t\mapsto t\otimes t$,
    \item The general linear group
          \begin{align*}
            GL_n := \Spec\mathbb C[x_{ij} : 1\leq i,j\leq n][1/\det(x_{ij})]
          \end{align*}
          where $(x_{ij})$ is the $n\times n$ matrix with entries $x_{ij}$
          with comultiplication
          \begin{align*}
            x_{ij} \mapsto \sum_{k=1}^n x_{ik}\otimes x_{kj}.
          \end{align*}
          See e.g. \cite[\href{https://stacks.math.columbia.edu/tag/022W}{Tag 022W}]{stacks-project} for details.
    \item The sepcial linear group $SL_N$ analogous to the above by
          quotienting by $\det(x_{ij})^2 - 1$.
    \item \todo{GL(V)}
  \end{enumerate}
\end{example}
But what do these group schemes have to do with their well-known
counterparts. For example, how does the scheme $GL_n$ relate to the
group $GL_n(\mathbb{C})$? The notation is no coincidence.
Group schemes induce group structures on their schematic points:
\begin{lemma}
  Let $G$ be a scheme with maps (\ref{eq:group_maps}). The following
  are equivalent:
  \begin{enumerate}
    \item $G$ is a group scheme.
    \item For every scheme $T$, $G(T)$ is a group.
  \end{enumerate}
  \begin{proof}
    The group structure on $G(T)$ is obtained by functoriality of
    $\Hom(-,G)$. The other direction the follows from a simple
    application of the Yoneda lemma.
  \end{proof}
\end{lemma}

\begin{example}
  The induced grous of the affine group schemes behave as expected:
  For a $\mathbb{C}$-algebra $R$, $\mathbf{G}_a(R) = (R,+)$,
  $\mathbf{G}_m(R) = (R^\times,\times)$, and
  $GL_n(R)$ is the usual group of invertible $n\times n$ matrices
  with coefficients in $R$.

  In more detail, consider $x\in\mathbf{G}_a(R)$ where $R$ is
  a $\mathbb{C}$-algebra. Such an $x$ is given by algebra homomorphsims
  $x^\sharp:\mathbb{C}[t] \to R$. That is, we may identify
  $\mathbf{G}_a(R)$ with $R$ using the map $x \mapsto x^\sharp(t)$. Now,
  for $x,y\in\mathbf{G}_a(R)$ and $f\in \mathbb{C}[t]$, we have the
  induced group multiplication given by
  \begin{align*}
    (xy)^\sharp(f)
    = x^\sharp(f) \cdot 1 + 1 \cdot y^\sharp(f)
    = x^\sharp(f) + y^\sharp(f).
  \end{align*}
  Thus the induced group structure on $\mathbf{G}_a(R)$ is just $(R,+)$.
\end{example}

\begin{definition}
  A group scheme is \emph{algebraic} if it is smooth and separated
  over $\mathbb{C}$.
\end{definition}

\begin{example}
  All the group schemes in \ref{ex:group_schemes} are affine algebraic
  groups. (See \cite[IV Theorem 9.3]{milne2012} for smoothness
  and \cite[Remark 3.2]{hoskins2016} for separatedness.) While group
  schemes are worth studying in full generality, the affine algebraic
  case will be sufficient for our purposes. Hence the only examples
  that the reader should have in mind are the ones already presented.
\end{example}

\subsection{Actions}

For this section, fix a scheme $X$ and a group scheme $G$.

\begin{definition}
  An \emph{action} of $G$ on $X$ is a morphism
  $\sigma : G\times X\to X$ satisfying the usual laws with respect
  to $m:G\times G\to G$.
\end{definition}

Note that an action $\sigma : G\times X\to X$ induces an action of
$G(T)$ on $X(T)$ by
\begin{equation*}
  T \xlongrightarrow{(g,x)} G\times X \xlongrightarrow{\sigma} X.
\end{equation*}

\begin{example}
  The multiplicative group $\mathbf G_m$ acts on $\affine{n}{}$
  via scalar mutliplication. That is the action on $\mathbb C$-points
  is $\lambda\cdot x = \lambda x$.
\end{example}

\begin{example}
  There is an action of $GL_n$ on $\affine{n}{}$ given by vector
  matrix multiplication. That is, for $g\in GL_n(\mathbb C)$
  and $x\in\affine{n}{}(\mathbb C)$, the action is $g\cdot x = gx$.
  There is a similar action of $GL_{n+1}$ on $\projective{n}{}$
  given by $g\cdot[x] = [gx]$ for $g\in GL_{n+1}(\mathbb C)$
  and $[x]\in\projective{n}{}(\mathbb C) \cong (\mathbb C^{n+1} - \{0\})/\mathbb C^\times$.
\end{example}

\begin{example}\label{ex:lf_action}
  Following \cite[Lemma 8.49]{hoskins2016}.
  Consider the setting of \ref{ex:quot_scheme_of_lf} and recall
  that, in order to map a locally free sheaf $\mathscr F$ to a point of
  $Q(r,d)$, we had to choose an isomorphism
  $\mathbb C^n \cong \Gamma(C,\mathscr F)$. In particular, there
  is an action of $GL_n$ on closed points of $Q(r,d)$
  given by $g\cdot q = q\circ g^{-1}$ for all $g\in GL_n(\mathbb C)$
  and $q : \mathscr O^{\oplus n}_C \surj \mathscr E$.

  To extend this to a schematic action $GL_n \times Q(r,d) \to Q(r,d)$
  we need to specify a family of quotients
  \begin{align*}
    \mathscr O^{\oplus n}_{C\times GL_n\times Q(r,d)}\surj \mathscr E
  \end{align*}
  of rank $r$ and degree $d$. We construct such a family by
  considering the universal family $\hat q:\mathscr Q^{\oplus n}_{C\times Q(r,d)}\surj \hat{\mathscr E}$,
  pulling it back along the projection $GL_n\times Q(r,d)\to Q(r,d)$,
  and composing it with the sheaf isomorphism
  \begin{align*}
    \iota : \mathscr O^{\oplus n}_{GL_n} \to \mathscr O^{\oplus n}_{GL_n}
  \end{align*}
  induces by the inversion on $GL_n$. That is, the action is defined
  by
  \begin{align*}
    \mathscr O^{\oplus n}_{C\times GL_n\times Q(r,d)}
    \xlongrightarrow{\pi^*\iota}
    \mathscr O^{\oplus n}_{C\times GL_n\times Q(r,d)}
    \xlongrightarrow{\pi^*\hat q}
    \pi^*\hat{\mathscr E}
  \end{align*}
  where each $\pi$ denotes the natural projection. As base change
  preserves Hilbert polynomials \ref{lem:hilbert_base_change} this
  has rank $r$ and degree $d$. It is then straightforward to verify
  that this is indeed an action.
\end{example}

\begin{lemma}\label{lem:orbits_are_iso_classes}
  Let $\mathscr E,\mathscr E'$ be locally free sheaves corresponding
  to closed points $q,q'$ in $Q(r,d)$ such that
  $\Gamma(X,q)$ and $\Gamma(X,q')$ are isomorphisms.
  Then $\mathscr E\cong\mathscr E'$
  if, and only if, there is a closed point $g$ of $GL_n$
  such that $g\cdot q \cong q'$.
  \begin{proof}
    If $g\cdot q \cong q'$ then there exists an isomorphism
    $\mathscr E\cong\mathscr E'$ by definition. In the other direction,
    consider an isomorphism $\psi:\mathscr E\cong\mathscr E'$.
    As both $\mathscr E$ and $\mathscr E'$ are globally generated,
    the composition
    \begin{equation*}
      % https://q.uiver.app/#q=WzAsNCxbMCwwLCJcXG1hdGhiYiBDXm4iXSxbNiwwLCJcXG1hdGhiYiBDXm4iXSxbMiwwLCJcXEdhbW1hKFgsXFxtYXRoc2NyIEUpIl0sWzQsMCwiXFxHYW1tYShYLFxcbWF0aHNjciBFJykiXSxbMCwyLCJcXEdhbW1hKFgscV57LTF9KSJdLFszLDEsIlxcR2FtbWEoWCxxJykiXSxbMiwzLCJcXEdhbW1hKFgsXFxwc2kpIl1d
      \begin{tikzcd}
        {\mathbb C^n} && {\Gamma(X,\mathscr E)} && {\Gamma(X,\mathscr E')} && {\mathbb C^n}
        \arrow["{\Gamma(X,q)^{-1}}", from=1-1, to=1-3]
        \arrow["{\Gamma(X,\psi)}", from=1-3, to=1-5]
        \arrow["{\Gamma(X,q')}", from=1-5, to=1-7]
      \end{tikzcd}
    \end{equation*}
    defines an element $g\in GL_n(\mathbb C)$ that satisfies
    $g\cdot q \cong q'$.
  \end{proof}
\end{lemma}

\begin{example}
  The $GL_n$ action on $Q(r,d)$ lifts to $F(r,d)$.
  In particular, we use \ref{lem:orbits_are_iso_classes} to make sure
  that
  \begin{align*}
    F(r,d)(q) \cong F(r,d)(g\cdot q)
  \end{align*}
  where we view $q$ as a $\mathbb C$-point of $Q(r,d)$. Hence
  $\varphi$ defines a Higgs field on the sheaf corresponding to
  $g\cdot q$. This lets us express the action on closed points
  as $g\cdot(q,\varphi) = (g\cdot q,\varphi)$.
\end{example}

\subsection{Orbits and Stabilisers}

Recall that one may use orbits or stabilisers or both to measure
how well-behaved an action is at particular points. For example,
in the analytic setting it was important that the action is free
which means that all orbits are isomorphic to the group
or, equivalently, that all stabilisers are trivial.
We are going to impose similar conditions on schematic
group actions. Hence we need to define orbits and stabilisers.
Consider an action $\sigma : G\times X\to X$ and a closed point
$x \in X(\mathbb C)$.

\begin{definition}
  The \emph{orbit} of $x$ is the set-theoretic image
  $G\cdot x := \sigma(G(\mathbb C),x)$.
\end{definition}

That is, the orbit of $x$ is just the usual orbit under the induced
action $G(\mathbb C)\times X(\mathbb C) \to X(\mathbb C)$ on closed
points. Stabilisers have a more categorical definition:

\begin{definition}
  The \emph{stabiliser} is the fibre product
  \begin{equation*}
    % https://q.uiver.app/#q=WzAsNCxbMCwwLCJHX3giXSxbMCwxLCJHIl0sWzEsMSwiWCJdLFsxLDAsIlxcU3BlY1xcbWF0aGJiICBDIl0sWzMsMiwieCJdLFsxLDIsIlxcc2lnbWEoLSx4KSIsMl0sWzAsM10sWzAsMV0sWzAsMiwiIiwxLHsic3R5bGUiOnsibmFtZSI6ImNvcm5lciJ9fV1d
    \begin{tikzcd}
      {G_x} & {\Spec\mathbb  C} \\
      G & X
      \arrow[from=1-1, to=1-2]
      \arrow[from=1-1, to=2-1]
      \arrow["\lrcorner"{anchor=center, pos=0.125}, draw=none, from=1-1, to=2-2]
      \arrow["x", from=1-2, to=2-2]
      \arrow["{\sigma(-,x)}"', from=2-1, to=2-2]
    \end{tikzcd}
  \end{equation*}
\end{definition}

Thus $G_x$ is a scheme by construction. Moreover, the stabiliser is the
preimage of the closed point $x$ under the map
$\sigma(-,x) : G\to X$, and hence a closed subscheme of $G$.

\subsection{Invariants}

The points of a quotient ought to correspond to orbits of the group
action. In other words, the quotient map should be invariant.
Fortunately, we are able to define

\begin{definition}
  Let $G$ be a group scheme that acts on $X,Y$, respectively.
  A morphism $f:X\to Y$ is \emph{$G$-invariant} if the following
  commutes:
  \begin{equation*}
    % https://q.uiver.app/#q=WzAsMyxbMCwwLCJHXFx0aW1lcyBYIl0sWzIsMCwiWCJdLFs0LDAsIlkiXSxbMCwxLCJcXHJobyIsMCx7ImN1cnZlIjotMn1dLFswLDEsIlxccGkiLDIseyJjdXJ2ZSI6Mn1dLFsxLDIsImYiXV0=
    \begin{tikzcd}
      {G\times X} && X && Y
      \arrow["\rho", curve={height=-12pt}, from=1-1, to=1-3]
      \arrow["\pi"', curve={height=12pt}, from=1-1, to=1-3]
      \arrow["f", from=1-3, to=1-5]
    \end{tikzcd}
  \end{equation*}
\end{definition}

The obvious challenge of taking quotients in geometric settings
is that there is not just a topology but also a geometric
structure to take into account. A quotient $X/G$ should satisfy the
property that any $G$-invariant $X\to Z$ uniquely factors through
$X\surj X/G$:
\begin{equation*}
  % https://q.uiver.app/#q=WzAsNCxbMiwwLCJYIl0sWzQsMCwiWC9HIl0sWzQsMSwiWiJdLFswLDAsIkdcXHRpbWVzIFgiXSxbMCwxLCIiLDAseyJzdHlsZSI6eyJoZWFkIjp7Im5hbWUiOiJlcGkifX19XSxbMCwyXSxbMSwyLCJcXGV4aXN0cyEiLDAseyJzdHlsZSI6eyJib2R5Ijp7Im5hbWUiOiJkYXNoZWQifX19XSxbMywwLCJcXHNpZ21hIiwwLHsiY3VydmUiOi0yfV0sWzMsMCwiXFxwaSIsMix7ImN1cnZlIjoyfV1d
  \begin{tikzcd}
    {G\times X} && X && {X/G} \\
    &&&& Z
    \arrow["\sigma", curve={height=-12pt}, from=1-1, to=1-3]
    \arrow["\pi"', curve={height=12pt}, from=1-1, to=1-3]
    \arrow[two heads, from=1-3, to=1-5]
    \arrow[from=1-3, to=2-5]
    \arrow["{\exists!}", dashed, from=1-5, to=2-5]
  \end{tikzcd}
\end{equation*}
Considering structure sheaves, we get the following picture:
\begin{equation*}
  % https://q.uiver.app/#q=WzAsNCxbMiwwLCJcXG1hdGhjYWwgT19YIl0sWzQsMCwiXFxtYXRoY2FsIE9fe1gvR30iXSxbNCwxLCJcXG1hdGhjYWwgT19aIl0sWzAsMCwiXFxtYXRoY2FsIE9fR1xcb3RpbWVzXFxtYXRoY2FsIE9fWCJdLFsxLDAsIiIsMix7InN0eWxlIjp7InRhaWwiOnsibmFtZSI6Imhvb2siLCJzaWRlIjoiYm90dG9tIn19fV0sWzIsMF0sWzIsMSwiIiwyLHsic3R5bGUiOnsiYm9keSI6eyJuYW1lIjoiZGFzaGVkIn19fV0sWzAsMywiXFxzaWdtYV5cXHNoYXJwIiwyLHsiY3VydmUiOjJ9XSxbMCwzLCJcXHBpXlxcc2hhcnAiLDAseyJjdXJ2ZSI6LTJ9XV0=
  \begin{tikzcd}
    {\mathscr O_G\otimes\mathscr O_X} && {\mathscr O_X} && {\mathscr O_{X/G}} \\
    &&&& {\mathscr O_Z}
    \arrow["{\sigma^\sharp}"', curve={height=12pt}, from=1-3, to=1-1]
    \arrow["{\pi^\sharp}", curve={height=-12pt}, from=1-3, to=1-1]
    \arrow[hook', from=1-5, to=1-3]
    \arrow[from=2-5, to=1-3]
    \arrow[dashed, from=2-5, to=1-5]
  \end{tikzcd}
\end{equation*}

Now we have a good idea what the subsheaf
$\mathscr O_{X/G}\inc\mathscr O_X$ should be:
\begin{definition}
  Let $\sigma : G\times X \to X$ be an action of an affine algebraic
  $G$ on a scheme $X$. The \emph{ring of invariants}
  on an affine open $U\subseteq X$ is the subring of $\mathscr O_X(U)$
  given by
  \begin{align*}
    \mathscr O_X^G(U) :=
    \left\lbrace{f \in \mathscr O_X(U) : \sigma^\sharp(f) = 1 \otimes f}\right\rbrace.
  \end{align*}
  Thus we have a sheaf $\mathscr O_X^G$ on $X$.
\end{definition}

\begin{example}
  \missingexample
\end{example}

\subsection{Quotients}

Consider a group scheme $G$ acting on $X$. What do we mean by a
quotient of $X$ by the $G$-action? We are looking for a $G$-invariant
surjective map $f : X\surj Y$ such that $Y$ contains as much of th
geometric information of $X$ as possible. In particular
we want to make sure that:
\begin{enumerate}
  \item the topology on $Y$ is induced by the topology on $X$,
  \item the affines on $Y$ correspond to affines on $X$,
  \item disjoint closed points in $Y$ correspond to disjoint
        closed orbits in $X$, and
  \item the functions on $Y$ correspond to $G$-invariant functions
        on $X$.
\end{enumerate}
With this in mind, we are ready to make the following definition:

\begin{definition}[{\cite[Definition 4.2.2]{huybrechts2010}}]\label{def:good_quotient}
  A \emph{good quotient} of $X$ by $G$-action is a $G$-invariant
  surjective map $\nu : X\surj Y$ such that the following hold:
  \begin{enumerate}
    \item $Y$ has the quotient topology.
    \item $\nu$ is affine.
    \item For every closed $G$-invariant $V\subseteq X$,
          $\nu(V)$ is closed in $Y$.
    \item For all disjoint closed $G$-invariant $V,V'\subseteq X$,
          $\nu(V)$ and $\nu(V')$ are disjoint.
    \item The map $\nu^\sharp : \mathscr O_Y \to \nu_*\mathscr O_X$
          is an isomorphism onto $(\nu_*\mathscr O_X)^G$.
  \end{enumerate}
\end{definition}


Note that our definition of a good quotient $f : X\surj Y$
ensures that $f$ is well-behaved with respect to closed orbits, but
allows for some leeway when it comes to non-closed orbits. That is,
$Y$ need not be in bijection with the topological quotient $X/G$. In particular,
all point in the closure of an orbit may be identified. It will be
possible to do better. Hence we add the following property:

\begin{definition}\label{def:geometric_quotient}
  A good quotient is \emph{geometric} if it is injective on orbits.
\end{definition}

In practice, our method of constructing quotients is going to yield
good quotients and it will always be possible to turn them into geometric
quotients. However, each step may come at the cost of having to
discard some insufficiently well-behaved orbits. Hence geometric quotients
exist but they need not be non-trivial.

We are going to turn good quotients into geometric quotients by
restricting to open subsets. Let us make sure that in doing so we
do not break any of the properties of good quotients. Direct computation
yields the following:

\begin{lemma}\label{lem:restrictions_of_good_quotients}
  For every good quotient $\nu:X\surj Y$ and every open $U\subseteq Y$
  the restriction $\nu : \nu^{-1}(U)\to U$ is
  also a good quotient.
\end{lemma}

\section{Constructing Quotients of Affine Schemes}

While we have been able to write down some conditions that we want
a map to satisfy in order to call it a geometric quotient, it is not
obvious under which conditions such quotients exist, let alone how to
construct them. To overcome these issues, we are going to
use a very powerful tool called geometric invariant theory (GIT).
In the affine case, the GIT quotients will not be very difficult
to obtain. Nonetheless, it this elementary case will
allow us to appreciate the more general construction.

\subsection{Reductive Groups}

\begin{definition}
  A \emph{linear representation} of an algebraic group $G$
  on a vector space $V$ is a homomorphism
  $\rho : G \to GL(V)$ of group-valued functors.
\end{definition}

\begin{example}
  if $V$ is finite-dimensional then its a map of algebraic groups
  \missingexample
\end{example}

\begin{example}
  affine linear rep using co-module structure
  \missingexample
\end{example}

\begin{definition}
  reductive group \missingdefinition
\end{definition}

\begin{theorem}
  The following are equivalent:
  \begin{itemize}
    \item reductive
    \item geometrically reductive
    \item linearly reductive
  \end{itemize}
  \begin{proof}
    \missingproof
  \end{proof}
\end{theorem}

\begin{example}
  \todo{$GL_n$}
\end{example}

\begin{theorem}[Nagata]\todo{rational?}
  Consider a reductive group $G$ acting on an affine scheme $X$.
  If $\mathscr O(X)$ is finitely generated then so is $\mathscr O(X)^G$.
  \begin{proof}
    \missingproof
  \end{proof}
\end{theorem}

Indeed, this condition is necessary. A theorem by Popov \missingcitation
states that $G$ is reductive if, and only if, for all affine schemes of
finite type, the ring of invariants is finitely generated.

\missingsection

\subsection{Good Quotients}

Consider a reductive group $G$ acting on an affine scheme $X$.

\begin{definition}
  The \emph{affine GIT quotient} is the map
  \begin{align*}
    \nu : X \longrightarrow X\sslash G := \Spec\mathscr O(X)^G
  \end{align*}
  arising from the inclusion $\mathscr O(X)^G \inc \mathscr O(X)$.
\end{definition}

\begin{theorem}\label{thm:affine_quotient_is_good}
  The affine GIT quotient is a good quotient.
  \begin{proof}[Proof idea]
    We follow \cite[Theorem 4.30]{hoskins2016}. Recall the definition
    \ref{def:good_quotient} of a good quotient. $G$-invariance,
    and affineness are essentially immediate and verifying that
    $\varphi^\sharp$ is an isomorphism onto $(\phi_*\mathscr O_X)^G$
    is straightforward on distinguished opens.

    To show surjectivity, one only needs to consider closed points.
    Reductivity of $G$ enables the construction of some closed point
    $x$ of $X$ for each closed point $y$ of $X\sslash G$ such that
    $\varphi(x) = y$ where $\varphi : X\to X\sslash G$ is the GIT quotient.

    Openness and separation of orbits are are addressed by considering
    $G$-invariant closed sets $W$ and $W'$ in $X$ and showing that
    it is possible to find $f\in\mathscr O(X)^G$ such that
    $f(W) = 0$ and $f(W') = 1$. \cite[Lemma 4.29]{hoskins2016}
    This then implies that $\overline{\varphi(W)}$ and $\overline{\varphi(W')}$
    are disjiont which may be sufficient to prove both remaining conditions.
  \end{proof}
\end{theorem}

While we are not yet in a position to take the quotient of the Quot
schemes with respect to the actions of $GL_n$,
we can at least contemplate some examples of good and geometric
quotients:

\begin{example}
  good and geometric
  \missingexample
\end{example}

\begin{example}
  good but not geometric
  \missingexample
\end{example}

\subsection{Geometric Quotients}

Consider an affine GIT quotient $\nu : X \surj X\sslash G$. This is
always a good quotient but it need not be geometric
(\ref{def:good_quotient} and \ref{def:geometric_quotient}, respectively).
Unfortunately, good quotients are, in particular, categorical
quotients \cite[Proposition 3.30]{hoskins2016} and hence unique
up to isomorphism. Therefore, the only way to reliably turn a good
quotient into a geometric quotient is to exclude certain orbits.

\begin{definition}\label{def:affine_stability}
  A point $x\in X$ is \emph{stable} if $G\cdot x$ is closed in $X$
  and $\dim G_x = 0$. We denote by $X^s$ the set of stable
  points. \todo{define stabiliser}
\end{definition}

\begin{lemma}\label{lem:affine_open_invariant}
  $X^s \subseteq X$ is open and $G$-invariant.
  \begin{proof}[Proof idea]
    Following \cite[Proposition 4.36]{hoskins2016}.
    The only thing to check is openness.
    Consider $x\in X(\mathbb C)$.  Once again we make use of the
    ability to separate closed $G$-invariant subsets by functions
    $f \in \mathscr O(X)^G$. \cite[Lemma 4.29]{hoskins2016}
    We choose $f$ to separate the orbit $G\cdot x$ and the
    subsets of $X$ where $\dim G_x > 0$, both may be shown to
    be closed and $G$-invariant. This implies $x\in D(f)$.
    By a dimension argument on stabilisers one then finds
    $D(f)\subseteq X^s$, showing that $X^s$ is a union of opens.
  \end{proof}
\end{lemma}

\begin{lemma}\label{lem:affine_quotient_is_open}
  $X\sslash^s G := \nu(X^s)$ is open in $X\sslash G$ and
  $\nu^{-1}(X\sslash^s G) = X^s$.
  \begin{proof}
    $X^s$ and $\nu$ are open, hence $\nu(X^s)$ is too.
    Then the isomorphism $\nu^\sharp : \mathscr O_Y \to (\nu_*\mathscr O(X))^G$
    allows us to verify $\nu^{-1}(\nu(D(f))) = D(f)$ for the open
    affine cover from the proof of \ref{lem:affine_open_invariant}.
    Hence it holds everywhere.
  \end{proof}
\end{lemma}

\begin{theorem}\label{thm:affine_quotient_is_geometric}
  The restriction $\nu : X^s \surj X\sslash^s G$ is a geometric quotient.
  \begin{proof}
    By \ref{lem:affine_open_invariant}, \ref{lem:affine_quotient_is_open},
    and \ref{lem:restrictions_of_good_quotients}, we know that
    the restriction is a good quotient. We then notice that, by construction,
    the action of $G$ on $X^s$ only has closed orbits. Now good
    quotients separate closed orbits, hence $\nu$ is injective on orbits.
  \end{proof}
\end{theorem}

\missingsection

\section{Constructing Quotients of Projective Schemes}

A this point, the main shortcoming of the GIT quotients that
we have seen so far is that they only make sense for actions on
affine schemes. In this section we extend the quotient construction
to projective schemes with certain well-behaved group actions.
The abstract theory is rather straightforward to lay out. However,
it is going to take considerable effort to equip our the actions of interest
with the additional data required in order to make sense of the quotient.

Fix an action $\sigma : G\times X\to X$ of a reductive
group on a projective scheme.

\subsection{Linearisation}

There is a wide range of actions on projective schemes. To be able to
take a quotient, one needs to consider some additional structure. This
will come in the form of linearised sheaves:

\begin{definition}
  A \emph{linearisation} of a quasi-coherent sheaf
  $\mathscr E$ on $X$ with respect to the $G$-action
  is an isomorphism $\Phi : \sigma^*\mathscr E \cong\pi^*\mathscr E$
  of sheaves on $G\times X$ satisfying the cocylce condition
  \begin{align}\label{eq:cocyle_condition}
    (\mu \times \identity)^*\Phi = \pi^*\Phi \circ (\identity\times\sigma)^*\Phi
  \end{align}
  where $\mu$ denotes the multiplication on $G$.
\end{definition}

\begin{example}\label{ex:linearised_action_on_bundle}
  Consider a $G$-linearised invertible sheaf $\mathscr L$ on $X$
  and the corresponding line bundle $\tau : L\to X$. Then the linearisation
  $\Phi : \sigma^*\mathscr L \cong \pi^*\mathscr L$ induces
  a map
  \begin{equation*}
    % https://q.uiver.app/#q=WzAsNCxbMSwwLCIoR1xcdGltZXMgWClcXHRpbWVzX3tcXHBpLFxcdGF1fSBMIl0sWzIsMCwiKEdcXHRpbWVzIFgpXFx0aW1lc197XFxzaWdtYSxcXHRhdX0gTCJdLFszLDAsIkwiXSxbMCwwLCJHXFx0aW1lcyBMIl0sWzAsMSwiXFxQaGkiXSxbMSwyXSxbMywwLCIiLDAseyJsZXZlbCI6Miwic3R5bGUiOnsiaGVhZCI6eyJuYW1lIjoibm9uZSJ9fX1dXQ==
    \begin{tikzcd}
      {G\times L} & {(G\times X)\times_{\pi,\tau} L} & {(G\times X)\times_{\sigma,\tau} L} & L
      \arrow[Rightarrow, no head, from=1-1, to=1-2]
      \arrow["\Phi", from=1-2, to=1-3]
      \arrow[from=1-3, to=1-4]
    \end{tikzcd}
  \end{equation*}
  where the equality comes from the fact that $\pi : G\times X \to X$
  is the base change of the structure map $G\to\Spec\mathbb C$
  by $X\to\Spec\mathbb C$ and the last component arises from the
  definition of the fibre product. The cocycle condition
  (\ref{eq:cocyle_condition}) ensures that this is an action. \cite[84]{huybrechts2010}
  Moreover, if $\mathscr L$ is ample then it defines a homomorphism
  $G\to GL(\Gamma(X,\mathscr L)^*)$ such that the induced embedding
  an embedding $X\inc\projective{}{}(\Gamma(X,\mathscr L)^*)$ is
  $G$-equivariant. \cite[Remark 5.20]{hoskins2016} That is, the
  linearisation ensures that the $G$-action on the projective
  scheme $X$ comes from a $GL_n$ action on a projective space.
\end{example}

To take quotients of $X$ by $G$-actions, we are going to equip
$X$ with ample $G$-linearised sheaves. That is, we are
going to construct ample sheaves $\mathscr L$ and corresponding
$G$-linearisations.

\begin{example}
  Let us define an ample $GL_n$-linearised sheaf on
  $Q(r,d)$ with respect to the action in \ref{ex:lf_action}. We follow
  \cite[75-76]{hoskins2016} and \cite[90]{huybrechts2010}.
  The key observation
  here is that, by construction of the action, we have equivalent
  families of quotients
  \begin{align*}
    \mathscr O^{\oplus n}_{C\times Q(r,d)\times GL_n}
    \xlongrightarrow{(\sigma\times\identity)^* \hat q}
    (\sigma\times\identity)^* \hat{\mathscr E}
  \end{align*}
  and
  \begin{align*}
    \mathscr O^{\oplus n}_{C\times GL_n\times Q(r,d)}
    \xlongrightarrow{\pi^*\iota}
    \mathscr O^{\oplus n}_{C\times GL_n\times Q(r,d)}
    \xlongrightarrow{\pi^*\hat q}
    \pi^*\hat{\mathscr E}
  \end{align*}
  over $GL_n\times Q(r,d)$ where
  $\hat q : \mathscr O^{\oplus n}_{C\times Q(r,d)\times GL_N} \surj \hat{\mathscr E}$
  is the universal quotient in $Q(r,d)$. In particular, we have
  an isormorphism of sheaves
  \begin{align}\label{eq:universal_linearisation}
    (\sigma^*\times\identity)\hat{\mathscr E} \cong (\pi^*\times\identity)\hat{\mathscr E}.
  \end{align}
  This is a linearisation of $\hat{\mathscr E}$ with respect to
  the action $\sigma\times\identity$ on $Q(r,d)\times C$, as the
  cocycle condition (\ref{eq:cocyle_condition}) is easily verified.

  The line bundle on $Q(r,d)$ to consider then arises
  by pulling back $\mathscr O(1)$ along the embedding (\ref{eq:quot_embedding}).
  In particular, for $\ell$ sufficiently large, we have an
  embedding of $Q(r,d)$ into the projective space
  $\projective{}{}(\wedge^m V_\ell^*)$
  where $V_\ell:=\Gamma(\mathscr O^{\oplus n}_C(\ell))$ and $m := \ell r+d+r(1-g)$.
  This embedding is constructed by factoring through the Grassmannian
  $\text{Gr}(V,m)$, a fine moduli space of locally free rank $m$ quotients of
  $V_\ell$.  The map $\text{Gr}(V_\ell,m)\inc\projective{}{}(\wedge^m V_\ell^*)$ is known as the Pl\"ucker
  embedding.  One may then calculate that the pullback of $\mathscr
    O_{\projective{}{}}(1)$ to $Q(r,d)$ is
  \begin{align*}
    \mathscr L_\ell
    := \det(\pi_*(\hat{\mathscr E} \otimes \pi^*\mathscr O_C(\ell))).
  \end{align*}
  Hence, for $\ell$ large, $\mathscr L_\ell$ is an ample sheaf
  on $Q(r,d)$. \cite[Proposition 2.2.5]{huybrechts2010}.
  The isomorphism (\ref{eq:universal_linearisation}) induces an isomorphism
  \begin{align*}
    \Phi:\sigma^*\mathscr L_\ell \cong \pi^*\mathscr L_\ell.
  \end{align*}
  For $\ell$ large, this is in fact a linearisation.
\end{example}

\begin{example}
  The $GL_n$-linearised ample sheaf $\mathscr L_\ell$ on
  $Q(r,d)$ pulls back along $\tau:F(r,d)\to Q_\circ(r,d)$ so
  we have a line bundle on $F(r,d)$ which, by abuse of
  notation, we denote by $\mathscr L_\ell$, too. The
  linearisation of $\mathscr L_\ell$ extends to the pullback by
  $GL_n$-equivariance of $u$. Moreover, the structure
  map of relative spec is affine by construction and the
  pullback of ample sheaves is ample. Thus $\mathscr L_\ell$
  is a $GL_n$-linearised ample sheaf on $F(r,d)$.
\end{example}

\begin{lemma}\label{lem:linearisations_give_reps}
  Consider a linearisation $L$ inducing the action
  $\tilde\sigma : G\times L\to L$.
  Then there is a natural linear representation $G\to GL(\Gamma(X,L))$.
  \begin{proof}[Proof idea]
    Following \cite[Lemma 5.19]{hoskins2016}. As $\Gamma(X,L)$
    is finite-dimensional, the objective is to construct a map
    $\Gamma(X,L) \to \mathscr O(G) \otimes \Gamma(X,L)$
    satisfying the relevant conditions. One may do so by considering
    $\sigma^* : \Gamma(X,L) \to \Gamma(G\times X,\sigma^*L)$ and
    composing with the isomorphism from the linearisation and
    the K\"unneth formula for sheaf cohomology.
  \end{proof}
\end{lemma}

\begin{lemma}
  If $\mathscr L$ is a linearisation then so is $\mathscr L^{\otimes k}$
  for all $k\geq 1$.
  \begin{proof}
    \missingproof
  \end{proof}
\end{lemma}

\subsection{Good Quotients}

The presence of a linearised ample sheaf enables us to construct a good
quotient.  Fix a $G$-linearised ample sheaf $\mathscr L$ with respect to an
action $\sigma : G\times X\to X$ of a reductive group on a projective scheme.

One difficulty when working with projective schemes, rather
than affine ones, is the failure of $\Proj$ to be a functor. In particular,
a graded algebra homomorphism $f : S \to R$ only induces a map
$\Proj R - N(f) \to \Proj S$ where the nullcone $N(f)\subseteq\Proj R$ is the closed
subscheme defined by \missingcitation
\begin{align*}
  N(f) := \{ p \in \Proj R : S_+ \subseteq f^{-1}p\}
\end{align*}
where $S_+\subseteq S$ is the irrelevant ideal.
Now recall that
\begin{align*}
  X \cong \Proj\bigoplus_{k\geq 0}\Gamma(X,\mathscr L^{\otimes k}).
\end{align*}
This essentially follows from \cite[Proposition 13.48]{gortz2010}. Hence
it is natural to consider the graded algebra of sections
\begin{align*}
  R(X,\mathscr L) := \bigoplus_{k\geq 0}\Gamma(X,\mathscr L^{\otimes k})
\end{align*}
and the corresponding invariants
\begin{align*}
  R(X,\mathscr L)^G := \bigoplus_{k\geq 0}\Gamma(X,\mathscr L^{\otimes k})^G.
\end{align*}
Thus the quotient is going to arise from the inclusion
$\iota : R(X,\mathscr L)^G\inc R(X,\mathscr L)$. We call the points outside the nullcone
semistable:

\begin{definition}
  A point $x\in X$ is \emph{semistable} if there exists a section
  $s\in\Gamma(X,\mathscr L^{\otimes k})^G$ for some $k>0$ such that $s(x)\neq 0$.
  Denote by $X^{ss}(\mathscr L)$ the set of all semistable points of $X$
  with respect to the linearisation $\mathscr L$.
\end{definition}

Verifying that $N(\iota) = X - X^{ss}(\mathscr L)$ is a matter of unpacking definitions.
In particular, $X^{ss}(\mathscr L)$ is open in $X$. Hence we obtain a good quotient
of semistable points in the natural way:

\begin{definition}
  The \emph{projective GIT quotient} of $X$ with respect to $\mathscr L$ is the map
  \begin{align*}
    X^{ss}(\mathscr L) \longrightarrow X \sslash^{ss}_{\mathscr L} G := \Proj R(X,{\mathscr L})^G
  \end{align*}
  corresponding to the inclusion $R(X,{\mathscr L})^G\inc R(X,{\mathscr L})$.
\end{definition}

\begin{theorem}
  The projective GIT quotient is good.
  \begin{proof}[Proof idea]
    Following \cite[Theorem 5.3]{hoskins2016}.
    Consider the projective GIT quotient
    $\nu : X^{ss}(\mathscr L)\to X\sslash^{ss}_{\mathscr L} G$.
    The key observation is that being a good quotient is local on target.
    Thus the strategy is to consider an affine cover and apply
    \ref{thm:affine_quotient_is_good}. The affines to consider
    are $D_+(f)$ in $X\sslash^{ss}_{\mathscr L} G$
    for $f\in R(X,\mathscr L)^G_+$. One then only needs to verify
    \begin{align*}
      \mathscr O_{X\sslash^{ss}_{\mathscr L} G}(D_+(f)) = \mathscr O_X(\nu^{-1} D_+(f))^G
    \end{align*}
    which is easily done by direct computation.
  \end{proof}
\end{theorem}

\begin{example}
  This allows us to define the moduli spaces that we are after:
  \begin{align*}
    M^{ss}_h(r,d)_\ell := Q_\circ(r,d)\sslash^{ss}_{{\mathscr L}_\ell} GL_n, \hspace{1cm}
    M^{ss}_H(r,d)_\ell := F(r,d)\sslash^{ss}_{{\mathscr L}_\ell} GL_n.
  \end{align*}
  Of course, we know very little about these spaces other than that
  they are good quotients. In particular, it is not clear how the moduli
  space depends on the parameter $\ell$.
\end{example}

\missingsection

\subsection{Geometric Quotients}

Although we already had to exclude some orbits to obtain a good
quotients, further restrictions are required to construct a geometric
quotient. This parallels the affine construction.
Once again fix an action $\sigma : G\times X\to X$ of a
reductive group $G$ on a projective scheme $X$ and a linearised
ample sheaf $\mathscr L$ on $X$.

\begin{definition}
  A point $x\in X$ is \emph{stable} if there exists a section
  $s\in\Gamma(X,\mathscr L^{\otimes k})^G$ for some $k>0$ such that $s(x)\neq 0$
  and the action of $G$ on the standard open $D_+(s)\subseteq X$ is closed.
  Denote by $X^s(\mathscr L)$ the set of all stable points of $X$ with respect
  to the linearisation $\mathscr L$.
\end{definition}

We find that this is completely analogous to the definition
of stability in the affine case \ref{def:affine_stability}, at
least when considering closed points.

\begin{lemma}[{\cite[Lemma 5.9]{hoskins2016}}]
  A closed point $x\in X^{ss}(\mathscr L)$ is stable if, and only if, the orbit
  $G\cdot x\subseteq X^{ss}(\mathscr L)$ is closed and $\dim G_x = 0$.
  \begin{proof}
    The `only if' direction is a matter of direct calculation.
    \missingproof
  \end{proof}
\end{lemma}

Of course, the point of restricting to a stable subset is to turn
good quotients into geometric ones. Let us make sure that we have
achieved this. The first step is to verify that we have defined
an open subscheme:

\begin{lemma}
  $X^s(\mathscr L)$ is open in $X$.
  \begin{proof}
    Recall that the set of points with zero-dimensional stabilisers is open.
    Moreover, the union $\bigcup D_+(f)$ over $f\in R(X,\mathscr L)_+^G$
    with closed action on $D_+(f)$ is open and so is the intersection
    $X^s(\mathscr L)$.
  \end{proof}
\end{lemma}

Now consider the GIT quotient $\varphi : X^{ss}(\mathscr L)\to X\sslash^{ss}_{\mathscr L} G$.
It follows from the definition \ref{def:good_quotient} that
$\varphi(X^s(\mathscr L))$ is open in $X\sslash_{\mathscr L} G$.

\begin{theorem}
  The GIT quotient $\nu : X^{ss}({\mathscr L}) \to X\sslash^{ss}_{\mathscr L} G$
  restricts to a geometric quotient
  \begin{align*}
    \nu : X^s({\mathscr L}) \to X\sslash^s_{\mathscr L} G := \nu(X^s({\mathscr L})).
  \end{align*}
  \begin{proof}
    \missingproof
  \end{proof}
\end{theorem}

\begin{example}
  We now have the stable subsets of the moduli spaces:
  \begin{align*}
    M^{s}_h(r,d)_\ell := Q_\circ(r,d)\sslash^s_{\mathscr L_\ell} GL_n, \hspace{1cm}
    M^{s}_H(r,d)_\ell := F(r,d)\sslash^s_{\mathscr L_\ell} GL_n.
  \end{align*}
  Once again, this is unsatisfactory as the moduli spaces ought
  not to depend on the parameter $\ell$.
\end{example}

\section{Stability Analysis}

Using the might of geometric invariant theory, we have been able
to construct the moduli spaces of locally free sheaves and Higgs
sheaves, respectively. Under mild assumptions, we were able to
find schemes $Q_\circ(r,d)$ and $F(r,d)$ parametrising the sheaves
that we are after. We then used these schemes and a linearised
$GL_n$-action to construct good and geometric quotients.
Unfortunately, both steps required us to discard some points.
Once again we find ourselves where we have constructed moduli spaces
but cannot yet be sure which orbits in $Q_\circ(r,d)$ and $F(r,d)$
are represented.

The first step in this section is to translate the stability
conditions of geometric invariant theory to conditions on the locally
free sheaves which the points of the moduli spaces correspond to.
In particular, we are going to find that for sufficiently large $\ell$,
the moduli spaces stabilise.

\subsection{Hilbert-Mumford Criterion}

Consider an action $\sigma : G \times X\to X$ of a reductive group $G$
on a projective scheme $X$ and a $G$-linearised ample sheaf
$\mathscr L$ on $X$. Points of $X$ are stable if they are sufficiently
well-behaved with respect to the $G$-action. The key idea to simplify
the analysis of this behaviour is to restrict the action to certain
subgroups. The simplest subgroup available is $\mathbf G_m$.

\begin{definition}
  A \emph{one-parameter subgroup} of $G$ is a non-trivial homomorphism
  of group schemes $\lambda : \mathbf G_m \to G$.
\end{definition}

We now observe that every one-parameter sugbroup $\lambda$ of $G$
induces a $\mathbf G_m$ action on $X$. Moreover, this action is
guaranteed to have a fixed point. We obtain such a fixed point
considering the morphism $\lambda_x : \mathbf G_m \to X$ given by
the action of $\mathbf G_m$ on a fixed $x \in X(\mathbb C)$.
Projective schemes are proper and hence we may use the valuative
criterion for properness to extend $\lambda_x$ uniquely along the
embedding $\mathbf G_m \inc \affine{1}{}$ to a map
$\lambda_x : \affine{1}{} \to X$.
In particular, we obtain
the new value $\lambda_x(0)$ which
we may think of as taking the limit of the action as $t\to 0$
in $\mathbf G_m$. This is the fixed point \todo{why?} of the $\mathbf G_m$
action induced by $\lambda$ we are looking for. Now recall that the
linearisation of $\mathscr L$ induces an action on the corresponding
line bundle $L$. For $s\in L_x$ and $g\in G$ we have
$g\cdot s\in L_{g\cdot x}$. In particular, $\lambda$ induces an
action of $\mathbf G_m$ on $L_{\lambda_x(0)}$. As this action is
induced by the linearisation it must be of the form \todo{why?}
$t\cdot s = t^r s$ for some $r$. \todo{why is this an integer?}

\begin{definition}
  Let $\lambda$ be a one-parameter subgroup of $G$ and let
  $x\in X$.  The \emph{Hilbert-Mumford weight} of $\lambda$
  at $x$ with respect to a linearisation
  $\Phi:\pi^*\mathscr L\cong\sigma^*\mathscr L$
  is the integer \todo{check!} $\mu^{\mathscr L}(x,\lambda)$
  such that \todo{make less ugly}
  \begin{align*}
    \Phi(\lambda(t),\lambda_x(0))
    = t^{-\mu^{\mathscr L}(x,\lambda)}\identity.
  \end{align*}
\end{definition}

It is possible to show that $x$ is stable (resp. semistable) with
respect to the $\mathbf G_m$-action induced by $\lambda$ if, and only
if, $\mu^{\mathscr L}(x,\lambda)$ and $\mu^{\mathscr L}(x,\lambda^{-1})$
positive (resp. non-negative).
See \cite[Lemma 6.9]{hoskins2016}.
The Hilbert-Mumford criterion now just says that measuring the
stability of a point with respect to every one-parameter subgroup
is indeed sufficient to determine the stability with respect to the
general action:

\begin{theorem}[Hilbert-Mumford]
  A closed point $x$ of $X$ is stable (resp. semistable) with
  respect to $\mathscr L$ if, and only if, for all one-parameter
  subgroups $\lambda$ of $G$, the Hilbert-Mumford weight
  $\mu^{\mathscr L}(x,\lambda)$ is positive (resp. non-negative).
  \begin{proof}
    2.1 in Mumford GIT, \cite[Theorem 4.9]{newstead1978}.
    \missingproof
  \end{proof}
\end{theorem}

\subsection{Stability of Locally Free Sheaves}

The Hilbert-Mumford criterion enables us to determine which orbits
are represented in our moduli spaces.
We address stability of locally free sheaves first.
The following characterises
(semi-)stable points of $Q_\circ$ fully:

\begin{theorem}
  Let $q : \mathscr O^{\oplus n}_C \surj \mathscr E$ be a closed point
  in $Q_\circ(r,d)$. Then $q\in Q^{ss}_\circ(r,d)(\mathscr L_\ell)$
  if, and only if, for all non-trivial proper subspaces
  $V\subseteq \mathbb C^n$, \todo{which line bundle?}
  \begin{align}\label{eq:git_stable_sheaf}
    \frac{\dim V}{P(\mathscr F,\ell)}\leq\frac{n}{P(\mathscr E,\ell)}
  \end{align}
  where $\mathscr F := q(V\otimes\mathscr O_C)$. Moreover,
  $q\in Q^s_\circ(\mathscr L_\ell)$ if, and only if, the inequality
  (\ref{eq:git_stable_sheaf}) is always strict.
  \begin{proof}
    \missingproof
  \end{proof}
\end{theorem}

It should not come as a surprise that (semi-)stable points of
$Q_\circ(r,d)$ correspond to slope (semi-)stable bundles
in the sense of \ref{def:stable_bundle}.
We begin by translating the notion of slope (semi-)stability to
locally free sheaves:

\begin{lemma}
  Let $\mathscr E=\Gamma(-,E)$ be a locally free sheaf on $C$.
  Then
  \begin{itemize}
    \item $E$ is slope semistable if, and only if, for all subsheaves
          $\mathscr F\subseteq\mathscr E$ with $0<\rank\mathscr F<\rank\mathscr E$,
          \begin{align}\label{eq:sheaf_slope_stability}
            \frac{\deg\mathscr F}{\rank\mathscr F}\leq\frac{\deg\mathscr E}{\rank\mathscr E};
          \end{align}
    \item $E$ is slope stable if, and only if, it is slope semistable
          and the inequality (\ref{eq:sheaf_slope_stability}) is always strict.
  \end{itemize}
  \begin{proof}
    \missingproof
  \end{proof}
\end{lemma}

Thus we are justified in calling a locally free sheaf slope (semi-)stable
whenever its corresponding vector bundle is slope (semi-)stable. Let us now turn
towards stability of sheaves in the GIT sense.

\begin{theorem}\todo{closed points?}
  If $r$ and $d$ satisfy
  \begin{align*}
    d > \max(r^2(2g-2), gr^2 + r(2g-2))
  \end{align*}
  then, for $\ell$ sufficiently large,
  $Q^s_\circ(r,d)(\mathscr L_\ell)$ and $Q^{ss}_\circ(r,d)(\mathscr L_\ell)$
  consist of all stable and semistable sheaves of $Q_\circ(r,d)$,
  respectively.
  \begin{proof}
    \missingproof
  \end{proof}
\end{theorem}

Now observe that tensoring with a line bundle does not affect
stability. That is, if $q:\mathscr O^{\oplus n}_{C\times T} \surj \mathscr E$
is a family of quotients in $Q_\circ(r,d)$ then twisting by
$\mathscr O_C(d')$ yields a family $q(d')$ in $Q_\circ(r,d+d')$.
For each $d'$, the map $q \mapsto q(d')$ is a bijection
$Q^s_\circ(r,d)\to Q_\circ(r,d+d')$. Thus we have the following:

\begin{corollary}\label{thm:stability_of_lf_sheaves}
  For $\ell$ sufficiently large,
  $Q^s_\circ(r,d)(\mathscr L_\ell)$ and $Q^{ss}_\circ(r,d)(\mathscr L_\ell)$
  consist of all slope stable and slope semistable sheaves of $Q_\circ(r,d)$,
  respectively.
\end{corollary}

In particular, we find that the moduli spaces stabilise as $\ell \to \infty$.
Hence we define
\begin{align}\label{eq:plain_algebraic_space}
  M^s_h(r,d) := M^s_h(r,d)_\ell = Q_\circ(r,d)\sslash^s_{\mathscr L_\ell} GL_n
\end{align}
for large $\ell$ in the sense of \ref{thm:stability_of_lf_sheaves}.

\subsection{Stability of Higgs Sheaves}

Now the stability analysis for Higgs sheaves is analagous.
\missingsection

\begin{corollary}\label{thm:stability_of_higgs_sheaves}
  For $\ell$ sufficiently large,
  $F^s(r,d)(\mathscr L_\ell)$ and $F^{ss}(r,d)(\mathscr L_\ell)$
  consist of all slope stable and slope semistable sheaves of $F(r,d)$,
  respectively.
\end{corollary}

In particular, the moduli spaces of Higgs sheaves also stabilise as
$\ell\to\infty$. Hence we define
\begin{align}\label{eq:higgs_algebraic_space}
  M^s_H(r,d) := M^s_H(r,d)_\ell = F(r,d)\sslash^s_{\mathscr L_\ell} GL_n
\end{align}
for large $\ell$ in the sense of \ref{thm:stability_of_higgs_sheaves}.

\section{Properties of the Algebraic Moduli Spaces}

In \ref{sec:properties_of_analytic_spaces} we observed that the
analytic moduli spaces $M^s_\dol(r,d)$ and $M^s_\varphi(r,d)$ have
several desirable properties. In particular, although
these spaces were constructed as smooth Banach manifolds, the results
are complex finite-dimensional manifolds. Moreover, we calculated
the tangent spaces and observed that the cotangent space of
$M^s_\dol(r,d)$ is open and dense in $M^s_\varphi(r,d)$.

This section is going to establish analogous results in the algebraic
setting. We begin by defining tangent spaces of schemes. We then
determine the tangent spaces of the algebraic moduli spaces and
relate cotangent vectors of locally free sheaves to Higgs field. Finally,
we establish that both algebraic spaces are in fact smooth varieties.

\subsection{Tangent Space}

\begin{definition}
  Let $X$ be a scheme and $x\in X$ a point. The \emph{Zariski cotangent
  space} of $X$ at $x$ is
  \begin{align*}
    T_x^* X := \mathfrak m_x /\mathfrak m_x^2
  \end{align*}
  where $\mathfrak m_x\subseteq\mathscr O_{X,x}$ is the maximal ideal.
\end{definition}

\begin{theorem}
  $T_{[q]} M^s_h(r,d) = \Ext^1(\mathscr E,\mathscr E)$.
  \begin{proof}
    \missingproof
  \end{proof}
\end{theorem}



\begin{corollary}\label{thm:cotangent_is_open}
  $TM^s_h(r,d)\subseteq M^s_H(r,d)$ is an open subscheme.
  \begin{proof}
    \missingproof
  \end{proof}
\end{corollary}

\subsection{Varieties}

\begin{theorem}
  $M^s_h(r,d)$ is a smooth quasi-projective variety.
  \begin{proof}
    \cite[Proposition 8.65]{hoskins2016}
    \missingproof
  \end{proof}
\end{theorem}

\begin{theorem}
  $M^s_H(r,d)$ is a smooth quasi-projective variety.
  \begin{proof}
    \cite[Proposition 7.4]{nitsure1991}
    \missingproof
  \end{proof}
\end{theorem}

\subsection{Analytification}

In various places throughout this paper we witnessed the interplay
between analytic and algebraic geometry. For instance, we saw that
holomorphic vector bundles on compact Riemann surfaces are equivalent
to locally free sheaves on smooth algebraic curves. More generally,
even though in one chapter we constructed manifolds and on the other
we constructed schemes, we found that the analytic and algebraic
constructions of the corresponding moduli spaces follow the same
fundamental steps. Further, we saw that the properties we observed
in the analytic setting, especially with respect to the relation
of plain holomorphic bundles to Higgs bundles, were present in the
algebraic setting, too.

Recall that the correspondence between holomorphic bundles on
compact Riemann surfaces and locally free sheaves on smooth algebraic
curves gives bijections
\begin{align*}
  M^s_{\dol}(r,d) \to M^s_h(r,d)(\mathbb C), \hspace{1cm}
  M^s_{\varphi}(r,d) \to M^s_H(r,d)(\mathbb C)
\end{align*}
between the spaces constructed in \ref{eq:abstract_analytic_spaces},
\ref{eq:plain_algebraic_space}, and \ref{eq:higgs_algebraic_space}.
We would like to turn these maps into homeomorphisms. If we consider
the subspace topology on the sets of complex points, we have no hope
of achieving this. Fortunately, observe that there is another choice of
topology. In particular, the set of closed points $\projective{n}{}(\mathbb C)$
is in bijection with the complex manifold $\mathbf{CP}^n$. Any quasi-projective
scheme also inherits this topology:

\begin{definition}
  Let $X$ be a quasi-projective scheme. Then the \emph{analytic topology} on
  $X(\mathbb C)\subseteq \mathbf{CP}^n$ is the subspace topology induced by an
  immersion $X\inc \mathbf P^n$. Denote by $X^{an}$ the corresponding
  topological space.
\end{definition}

Note that the choice of immersion does not matter.
Moreover, it is not difficult to verify the following:

\begin{lemma}
  If $U\subseteq X(\mathbb C)$ is open then it is open and dense in $X^{an}$.
\end{lemma}

\begin{example}
  By \ref{thm:cotangent_is_open}, $TM_h^s(r,d)$ is open in $M_H^s(r,d)$. Thus,
  in particular, $TM_h^s(r,d)(\mathbb C)$ is open and dense in $M^s_H(r,d)^{an}$.
\end{example}

The key observation now is that $X^{an}$ inherits more than just a
topology from $\mathbf{CP}^n$. The holomorphic functions on $\mathbf{CP}^n$
define a structure sheaf $\mathscr O_{\mathbf{CP}^n}$. Now $X^{an}$
obtains the structure of a locally ringed subspace:

\begin{definition}
  The \emph{analytification} of a quasi-projective scheme $X$
  is the locally ringed space
  $(X^{an},\mathscr O_X^{an}) := (X^{an},\restrict{\mathscr O_{\mathbf{CP}^n}}{X^{an}})$.
\end{definition}

Observe that, being a locally ringed subspace of $\mathbf{CP}^n$, 
the space $X^{an}$ has many desirable properties. In particular, it is
an analytic space. However, it is in general not guaranteed to
be a manifold. 

\begin{example}
  A classical example of this is $X=\Spec \mathbb C[x,y]/(y^2-x^3)$
  which naturally embeds into $\affine{2}{}$ and hence $\projective{2}{}$
  but fails to have a chart around $[0:0:1]$.
\end{example}

Moreover, analytification extends to maps. That is, there is a
functor $(-)^{an}$ from quasi-projective schemes to analytic spaces.
Fortunately, this restricts to a functor from smooth projective varieties
to complex manifolds.

Using the analytification functor we have now obtained two
manifolds whose points are stable holomorphic bundles on $C$:
$M^s_\dol(r,d)$ and $M^s_h(r,d)^{an}$. Similarly,
both $M^s_\varphi(r,d)$ and $M^s_H(r,d)^{an}$ are complex manifolds
whose points are stable Higgs bundles.

The key question is: Did we obtain the same manifolds in both cases?
Fortunately, \cite[Theorem C]{fan2020} answers this question affirmatively:

\begin{theorem}
  The bijection
  $M^s_\varphi(r,d) \cong M^s_H(r,d)^{an}$
  is a biholomorphism.
  \begin{proof}
    \missingproof
  \end{proof}
\end{theorem}

We may consider $M^s_\dol(r,d)$ and $M^s_h(r,d)^{an}$ as submanifolds
of $M^s_\varphi(r,d)$ and $M^s_H(r,d)^{an}$, respectively, by taking
the zero Higgs field. Hence the theorem also answers the question for plain
holomorphic bundles:

\begin{corollary}
  The bijection
  $M^s_\dol(r,d) \cong M^s_h(r,d)^{an}$
  is a biholomorphism.
\end{corollary}

\chapter{Conclusion}

In the preceeding chapters we explored two fundamentally different
approaches to constructing moduli spaces of holomorphic vector
bundles on a fixed compact Riemann surface. Our goal was to
outline the processes of constructing such spaces and highlight where
the main difficulties with each of the methods arise.
In this chapter we spend some time reflecting, both on the methods
themselves as well as our presentation of the material.
Furthermore, we sketch some ways in which this account may be
extended.

\section{Comparison of the Constructions}


The analytic approach
quickly lead us to consider the quotients of spaces of unitary connections
by the action of a compact gauge group. We were forced to restrict these
spaces twice, firstly to ensure that the quotient is locally a manifold
and secondly to make sure that this is true globally. We then had to
backtrack to translate the conditions we imposed on unitary connections
to the setting on holomorphic vector bundles. Overall it was clear
why each step was necessary, but less obvious why it was correct or
how one might apply the principles to an unsolved problem.

The algebraic construction then provided a much more systematic
point of view. The categorical language naturally enabled us to make
precise what properties we expect quotients of geometric spaces to
have. Moreover, we were able to construct the moduli spaces as GIT
quotients. While we were required to exclude exactly the same set of
vector bundles from our construction to obtain good and geometric
quotients, this time around we were able to view each step as an
instance of a much broader theory. Moreover, we were able to recover
the analytic constructions by applying the anlytification functor.

Unfortunately, there is no such thing as a free lunch. \cite{friedman1977}
Taking the algebraic point of view turned out to be no easier than
doing things analytically. It was merely a case of moving difficulties
from one place to another. In particular, obtaining suitable schemes with
a linearised ample sheaves requires a large amount of creativity.
Moreover, many of the proofs ended up being analytic, in particular
stability analysis using one parameter subgroups
and tangent space calculation using deformation theory come to mind.

\section{Evaluation}

Presenting the constructions of moduli spaces comes with two main
difficulties. Firstly, each construction takes a lot of space on its
own. In particular, the analytic and algebraic approaches each
require a lot of theory. This makes it difficult to present both
processes alongside each other. Due to this, most resources cover
either one or the other. This introduces the second hurdle: There is a
fundamental language difference between analytic and algebraic treatments.
In this section we discuss some decisions that had to be made in
order to deal with these problems.

\subsection{Bypassing Banach Manifolds}

\missingsection

\subsection{Language Barrier}

\missingsection

\section{Future Work}

\missingsection

\pagebreak
\renewcommand{\bibname}{References}
\addcontentsline{toc}{chapter}{References}
\printbibliography

\end{document}
