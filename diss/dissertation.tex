\documentclass[12pt]{ociamthesis}  % default square logo
%\documentclass[12pt,beltcrest]{ociamthesis} % use old belt crest logo
%\documentclass[12pt,shieldcrest]{ociamthesis} % use older shield crest logo

\usepackage{dissertation}

%input macros (i.e. write your own macros file called mymacros.tex
%and uncomment the next line)
%\include{mymacros}

\title{Moduli Spaces of\\[1ex]Holomorphic Bundles}   %note \\[1ex] is a line break in the title

\author{Franz Miltz}             %your name
\college{Lady Margaret Hall}  %your college

%\renewcommand{\submittedtext}{change the default text here if needed}
\degree{Master of Science}     %the degree
\degreedate{2024}         %the degree date

\addbibresource{references.bib}

%end the preamble and start the document
\begin{document}

%this baselineskip gives sufficient line spacing for an examiner to easily
%markup the thesis with comments
\baselineskip=18pt plus1pt

%set the number of sectioning levels that get number and appear in the contents
\setcounter{secnumdepth}{3}
\setcounter{tocdepth}{3}


\maketitle                  % create a title page from the preamble info
\begin{dedication}
  To somebody
\end{dedication}
\begin{acknowledgements}

\end{acknowledgements}
\begin{abstract}

\end{abstract}

\begin{romanpages}          % start roman page numbering
  \tableofcontents            % generate and include a table of contents
\end{romanpages}            % end roman page numbering

\chapter{Introduction}

\begin{itemize}
  \item Riemann surfaces
  \item complex geometry at the intersection between the analytic and algebraic settings
\end{itemize}

\chapter{As Manifolds}

Consider the set of holomorphic bundles on a compact Riemann surface,
up to isomorphism. What can we say about this set? What additional
structure does it have? How does this structure change if we equip
the bundles with additional data?
In this chapter we are going to construct complex manifolds whose
points correspond to equivalence classes of holomorphic bundles
and Higgs bundles, a type of decorated holomorphic bundle. These
manifolds are referred to as moduli spaces. While it
is possible to construct topological spaces representing all such bundles,
the result is non-Hausdorff and hence does not provide us with a
satisfying answer. The majority of the chapter is going to be spent
on rectifying this issue.

The key observation will be to ignore certain `unstable' bundles. We
are going to find that doing so does not lose much information but
allows for the construction of the moduli spaces. Although unstable
bundles are the main obstacle, removing them does not trivialise the
problem. To obtain the moduli spaces, we have to consider quotients
of infinite dimensional manifolds by group actions. Such considerations
require us to introduce the theory of Banach manifolds.

The first part of this chapter is going to be spent on the kinds of
objects that we aim to classify, i.e. holomorphic bundles and Higgs
bundles on compact Riemann surfaces. We recall some facts and establish
notation. The next to sections are going to correspond to the two
main steps in obtaining a complex manifold structure on the topological
space of equivalence classes of holomorphic bundles and Higgs bundles,
respectively. Firstly, we construct a Banach manifolds of stable
bundles. Secondly, we take a quotient with respect to the appropriate
group action. Towards the end of the chapter we establish some
further facts about the moduli spaces and how they relate to each other.

\paragraph*{Notation}

Before we begin, let us establish some notation. Unless otherwise
indicated,

\begin{itemize}
  \item all vector spaces are complex, maps of vector spaces are
    $\mathbb{C}$-linear, and tensor products of vector spaces are
    over $\mathbb{C}$;
  \item all manifolds are smooth;
  \item dimensions are taken over $\mathbb{C}$;
  \item the word `bundle' refers to a vector bundle.
\end{itemize}

\section{Holomorphic Bundles on Compact Riemann Surfaces}

Holomorphic maps are smooth. Hence, holomorphic vector bundles
are smooth vector bundles. In contrast to maps, in the case of vector
bundles the word `holomorphic' does not refer to a property, but
a structure. In particular, there may be many holomorphic bundles with
the same underlying smooth bundle.

\subsection{Complex Vector Bundles}

Recall the definition of a smooth bundle on a complex
manifold:

\begin{definition}
  Let $X$ be a complex manifold of dimension $n$. A
  \emph{smooth vector bundle} $E$ of rank $r$ on $X$ is a smooth map
  $\pi : E\to X$ such that there
  is an open cover $X = \bigcup_i U_i$ and diffeomorphisms
  \begin{equation}\label{eq:smooth_trivialisation}
    \phi_i : {\pi}^{-1}U_i \cong \mathbb{C}^r \times U_i
  \end{equation}
  such that, for every chart $U\subseteq U_i\cap U_j$, the
  map $\phi_i \circ {\phi_j}^{-1}$ restricts to a linear
  automorphism of $\mathbb{C}^r\times U \cong \mathbb{C}^{r+n}$.
\end{definition}

\begin{example}
  A complex manifold $X$ of dimension $n$ may be regarded as a
  real manifold of dimension $2n$. Any smooth real vector bundle
  $E$ on $X$ corresponds to a \emph{complexified} smooth bundle
  $E_{\mathbb{C}} := E\otimes_{\mathbb{R}} \mathbb{C}$ of equal rank.
  In particular, we may complexify the real vector bundles
  $TX$ and $T^*X$ to obtain smooth vector bundles
  $T_{\mathbb{C}} X$ and $T_{\mathbb{C}}^*X$
  on $X$ of rank
  \begin{align*}
    \rank(T_\mathbb{C} X)
    = \rank_{\mathbb{R}} (TX) = 2n.
  \end{align*}
\end{example}

Now a holomorphic bundle is a smooth bundle where the trivialisations
(\ref{eq:smooth_trivialisation}) are biholomorphic. As holomorphic
maps are smooth, each holomorphic bundle has an underlying smooth
structure.

\begin{example}
  smooth bundle with multiple holomorphic structures
  \missingexample
\end{example}

\begin{example}
  Recall that a complex manifold $X$ of dimension $n$ induces a
  complex structure on the real smooth rank $2n$ vector bundle $TX$.
  That is, there is a map of smooth real vector bundles
  $J : TX \to TX$ such that $J^2 = -1$. Note that this extends to
  an automorphism of the complexification $T_{\mathbb{C}}X$.
  This defines a splitting into subbundles
  \begin{align}\label{eq:tangent_decomposition}
    T_{\mathbb{C}} X = T_{1,0}X \oplus T_{0,1}X
  \end{align}
  whose fibres are eigenspaces of $i$ and $-i$, respectively. The
  complex bundle $T_{1,0} X$ of rank $n$ is called the
  \emph{holomorphic tangent bundle} of $X$. Indeed, if we regard $TX$
  as a complex vector bundle via $J$ then the map
  \begin{align*}
    TX
    \longrightarrow T_{\mathbb{C}} X
    = T_{1,0}X \oplus T_{0,1}X
    \longrightarrow T_{1,0}X
  \end{align*}
  is a $\mathbb{C}$-linear isomorphism and hence induces a holomorphic
  structure on $TX$, justifying the name.
\end{example}

One is usually interested in what kinds of sections a vector bundle
admits. Indeed, in the next chapter we are going to make extensive use
of the fact that a vector bundle is uniquely determined by its sections.
In the complex case, there are a few definitions:

\begin{definition}
  Consider a complex vector bundle $\pi : E \to X$. A section of $E$
  on an open $U\subseteq X$ is a map $s : U \to E$ such that
  $\pi \circ s = \identity_U$.
  \begin{itemize}
    \item If $E$ is smooth then $\Gamma(U,E)$ denotes the vector space
      of smooth sections on $U$.
    \item If $E$ is holomorphic then $H^0(U,E)$ denotes the vector space
      of holomorphic sections on $U$.
  \end{itemize}
\end{definition}

The notation $H^0(U,E)$ for holomorphic sections comes from sheaf
cohomology. The bundle $E$ corresponds to a locally free sheaf
whose cohomology in degree $0$ when restricted to $U$ is precisely
$H^0(U,E)$. More on this in the case of schemes in
\ref{sec:locally_free_sheaves}.

Note that there are some important differences between $\Gamma(U,E)$
and $H^0(U,E)$. While both are complex vector spaces $\Gamma(U,E)$ is
typically infinite dimensional while $H^0(U,E)$ is always finite
dimensional.~\cite[Theorem 1.4.1]{ma2007}

\subsection{Dolbeault Cohomology}
\missingcitation

Consider a complex manifold $X$ of dimension $n$. Recall the de Rahm
complex
\begin{align*}
  \cdots \xlongrightarrow{d}
  \Omega^{k-1}(X)\xlongrightarrow{d}
  \Omega^{k}(X)\xlongrightarrow{d}
  \Omega^{k+1}(X)\xlongrightarrow{d}
  \cdots
\end{align*}
where $\Omega^0(X) = C^\infty(X)$ and, for $k > 0$,
\begin{align*}
  \Omega^k(X) := \Gamma(X,\Lambda^k T^*X).
\end{align*}
The splitting $T^*_\mathbb{C} X = T^*_{1,0}X \oplus T^*_{0,1}X$
leads us to define
\begin{align*}
  \Omega^{p,q}(X)
  := \Gamma(X,\Lambda^p T^*_{1,0}X \wedge\Lambda^q T^*_{0,1}X)
\end{align*}
and hence induces a splitting
\begin{align*}
  \Omega^k_{\mathbb{C}}(X) = \bigoplus_{p+q=k} \Omega^{p,q}(X).
\end{align*}
Moreover, the operator $d : \Omega^{p,q}(X) \to \Omega^{p+q+1}_{\mathbb{C}}(X)$ splits into operators $d = \partial + \dol$ where
\begin{align*}
  \partial : \Omega^{p,q}(X) \to \Omega^{p+1,q}(X),\hspace{1cm}
  \dol : \Omega^{p,q}(X) \to \Omega^{p,q+1}(X).
\end{align*}
In particular, we now have a family of complexes
\begin{align}\label{eq:dolbeault_complex}
  \cdots \xlongrightarrow{d}
  \Omega^{p,q-1}(X)\xlongrightarrow{\dol}
  \Omega^{p,q}(X)\xlongrightarrow{\dol}
  \Omega^{p,q+1}(X)\xlongrightarrow{\dol}
  \cdots
\end{align}
\begin{definition}
  The cohomology of the complex \ref{eq:dolbeault_complex} is
  called \emph{Dolbeault cohomology}. To be precise, the
  $(p,q)$ Dolbeault cohomology group is
  \begin{align*}
    H^{p,q}(X) := \frac{
      \ker(\dol : \Omega^{p,q}(X) \to \Omega^{p,q+1}(X))
    }{
      \im(\dol : \Omega^{p,q-1}(X) \to \Omega^{p,q}(X))
    }
  \end{align*}
\end{definition}

\subsection{Dolbeault Cohomology of Holomorphic Bundles}

\todo{explain how holomorphic bundles give rise to $\dol$-operators}
Now consider a holomorphic bundle $E$ of rank $r$ and the space of
$E$-valued holomorphic $(p,q)$-forms
\begin{align*}
  \Omega^{p,q}(X,E) := \Omega^{p,q}(X)\otimes\Gamma(X,E).
\end{align*}
It is straightforward to verify that the operator $\dol$ extends
to an operator
\begin{align}\label{eq:general_dolbeault_operator}
  \dol_E : \Omega^{p,q}(X,E) \to \Omega^{p,q+1}(X,E)
\end{align}
by considering local trivialisations where we have
\begin{align*}
  \dol_E \left({\sum_{i=1}^k \alpha_i \otimes e_i}\right)
  = \sum_{i=1}^k \dol \alpha_i \otimes e_i.
\end{align*}
Moreover, the operator $\dol_E$ satisfies
\begin{align*}
  \dol_E(\alpha \otimes s) = \dol\alpha \otimes s + (-1)^{p+q} \alpha \wedge \dol_E s.
\end{align*}
Hence it is uniquely determined by the component
$\Gamma(X,E) \to \Omega^{0,1}(X,E)$. This leads us to the following
definition:

\begin{definition}
  A \emph{Dolbeault operator} on a smooth bundle $E$ on a complex
  manifold $X$ is a linear operator
  \begin{align*}
    \dol_E : \Gamma(X,E) \to \Omega^{0,1}(X,E)
  \end{align*}
  such that $\dol_E^2 = 0$ and, for all $s\in\Gamma(X,E)$ and
  $f\in C^\infty(X)$,
  \begin{align*}
    \dol_E (fs) = \dol f\otimes s + f \dol_E s.
  \end{align*}
\end{definition}

The process outlined above is reversible. That is, a holomorphic
bundle is just a smooth bundle with a Dolbeault operator. To make
this precise, we need to say what morphisms of such operators are.
Consider a map $f : E \to F$ of smooth bundles on a fixed complex
manifold $X$. This induces a map $\Gamma(X,E)\to\Gamma(X,F)$
given by $s \mapsto f\circ s$. Hence $f$ is a map of Dolbeault operators
if, and only if, it commutes with the operators:

\begin{definition}
  A \emph{morphism of Dolbeault operators} $f : \dol_E \to \dol_F$
  is a map of smooth bundles $f : E\to F$ such that the following commutes:
  \begin{equation*}
    % https://q.uiver.app/#q=WzAsNCxbMCwwLCJcXEdhbW1hKFgsRSkiXSxbMCwxLCJcXEdhbW1hKFgsRikiXSxbMiwxLCJcXE9tZWdhXnswLDF9KFgpXFxvdGltZXMgXFxHYW1tYShYLEYpIl0sWzIsMCwiXFxPbWVnYV57MCwxfShYKVxcb3RpbWVzIFxcR2FtbWEoWCxFKSJdLFswLDMsIlxcYmFyXFxwYXJ0aWFsX0UiXSxbMSwyLCJcXGJhclxccGFydGlhbF9GIl0sWzAsMSwiXFxHYW1tYShYLGYpIiwyXSxbMywyLCJcXHRleHR7aWR9XFwsXFxvdGltZXMgXFxHYW1tYShYLGYpIl1d
    \begin{tikzcd}
      {\Gamma(X,E)} && {\Omega^{0,1}(X)\otimes \Gamma(X,E)} \\
      {\Gamma(X,F)} && {\Omega^{0,1}(X)\otimes \Gamma(X,F)}
      \arrow["{\bar\partial_E}", from=1-1, to=1-3]
      \arrow["{\Gamma(X,f)}"', from=1-1, to=2-1]
      \arrow["{\identity\,\otimes \Gamma(X,f)}", from=1-3, to=2-3]
      \arrow["{\bar\partial_F}", from=2-1, to=2-3]
    \end{tikzcd}
  \end{equation*}
\end{definition}

\begin{theorem}
  Fix a complex manifold $X$. There is an equivalence of categories
  between holomorphic bundles on $X$ and smooth bundles with corresponding
  Dolbeault operators.
  \begin{proof}
    This is essentially \cite[Theorem 3.2]{moroianu2004}. One only
    needs to verify that a map of smooth bundles is a map of
    holomorphic bundles if, and only if, it respects the holomorphic
    structure. \todo{maybe we should do this proof?}
  \end{proof}
\end{theorem}

Note that the operator $\dol_E$ on a holomorphic bundle $E$ is
unique only up to isomorphism. Fortunately, taking cohomologies
is functorial and thus we obtain the Dolbeault cohomology of
holomorphic vector bundles:

\begin{definition}
  Let $E$ be a holomorphic vector bundle on $X$. The \emph{$(p,q)$
  Dolbeault cohomology group of $X$ with coefficients in $E$} is
  \begin{align*}
    H^{p,q}(X,E) := \frac{
      \ker(\dol_E : \Omega^{p,q}(X,E) \to \Omega^{p,q+1}(X,E))
    }{
      \im(\dol_E : \Omega^{p,q-1}(X,E) \to \Omega^{p,q}(X,E))
    }.
  \end{align*}
\end{definition}

\subsection{Higgs Bundles}

\todo{write about why Higgs bundles exist, what they're good for, yadda
yadda}

\begin{definition}
  The \emph{canonical bundle} $K_X$ on a complex manifold $X$
  of dimension $n$ is the holomorphic vector bundle defined as
  \begin{align*}
    K_X := \det T^*_{0,1} X := \wedge^n T^*_{0,1} X.
  \end{align*}
\end{definition}

Note that the canonical bundle is a line bundle on every $X$.
\todo{explain / cite}

\begin{definition}
  Let $E$ be a holomorphic bundle on a complex manifold $X$
  of dimension $n$.
  A \emph{Higgs field} on $E$ is a map of holomorphic bundles
  \begin{align*}
    \phi : E \to E \otimes K_X.
  \end{align*}
  A \emph{Higgs bundle} is a holomorphic bundle equipped with
  a Higgs field.
\end{definition}

Consider Higgs bundles $(E,\phi)$ and $(E',\phi')$ and a map
$f:E\to E'$ of holomorphic bundles. We say $f$ is a map of
Higgs bundles if it respects the Higgs fields, i.e. the following
commutes:
\begin{equation*}
  % https://q.uiver.app/#q=WzAsNCxbMCwwLCJFIl0sWzIsMCwiRVxcb3RpbWVzIEtfWCJdLFswLDEsIkYiXSxbMiwxLCJGXFxvdGltZXMgS19YIl0sWzEsMywiZlxcb3RpbWVzIFxcdGV4dHtpZH0iXSxbMCwyLCJmIiwyXSxbMiwzLCJcXHBoaSciLDJdLFswLDEsIlxccGhpIl1d
  \begin{tikzcd}
    E && {E\otimes K_X} \\
    F && {F\otimes K_X}
    \arrow["\phi", from=1-1, to=1-3]
    \arrow["f"', from=1-1, to=2-1]
    \arrow["{f\otimes \identity}", from=1-3, to=2-3]
    \arrow["{\phi'}"', from=2-1, to=2-3]
  \end{tikzcd}
\end{equation*}
This means in particular that, if $(E,\phi)\cong(E',\phi')$ are
isomorphic Higgs bundles then we have isomorphic holomorphic bundles
$E\cong E'$. This must then mean that $E$ and $E'$ are isomorphic
as smooth bundles.

Note that a Higgs field may be thought of as a holomorphic section
of $\End E \otimes K_X$ where $\End E := E^\vee \otimes E$.
In particular, if we fix a holomorphic bundle $E$ then each section
$\phi$ defines a unique Higgs bundle $(E,\phi)$ so the space of Higgs
bundles with underlying holomorphic bundle $E$ is
$H^0(X,\End E \otimes K_X)$.

\section{Topology of Holomorphic Bundles}

Our main goal is gain an understanding of the geometry of
holomorphic bundles on compact Riemann surfaces. Topology underlies
geometry and so we must understand the topology before we can study
the geometry. There are going to be two main takeaways from this
section. Firstly, we are going to observe that smooth vector
bundles on a compact Riemann surface are characterised by their
rank and a topological invariant called the degree. Secondly, we
will introduce sheaf cohomology, a powerful tool that is going to
play a large role in the next chapter where we take a more
algebraic approach.

\subsection{Chern Classes}

We noticed that in order for two holomorphic bundles to be isomorphic
they must have the same underlying smooth bundles. Hence it is worth
investigating when smooth bundles are isomorphic. Fix a complex
manifold $X$.

The first observation we make is that, if we have isomorphic smooth
bundles $E\cong E'$ then $\rank E = \rank E'$. However, the converse
is not true. To see this it suffices to consider any non-trivial
bundle and compare it to the trivial bundle of equal rank. E.g.
$TX$ on $S^2$ will do.

This leads us to consider another invariant: Chern classes. For
every smooth bundles $E$ on $X$ of rank $r$ there are elements
$c_j(E)\in H^{2j}(X,\mathbb{Z})$ for $j\geq 0$ called the Chern
classes of $E$. There are different ways of constructing these classes,
e.g. \cite{fine2013} and \cite{griffiths1994}, for our purposes
it is sufficient to know a few key properties:

\begin{lemma}\missingcitation
  Let $E$ be a smooth bundle on $X$ of rank $r$. Then
  \begin{enumerate}
    \item $c_j(E) = 0$ whenever $j<1$ or $j>r$.
    \item If $F$ is a smooth bundle on $X$ such that $E\cong F$
      then $c_i(E) = c_i(F)$ for all $i$.
    \item For every other smooth bundle $F$ on $X$,
      \begin{align*}
        c_i(E\oplus F) = \bigoplus_{j=0}^{i} c_j(E)\cup c_{i-j}(F).
      \end{align*}
  \end{enumerate}
\end{lemma}

\subsection{Degree}

Our goal is to classify holomorphic bundles on a compact Riemann surface,
i.e. a compact connected complex manifold of dimension 1. Hence we
ought to consider this case more closely. To this end, fix a compact
Riemann surface $C$.

\todo{genus}

Note that the real manifold of dimension $2$ underlying $C$ is orientable. \missingcitation
Hence the cohomology group $H^2(X,\mathbb{Z})$ is generated by the
fundamental class $[C]$.

\begin{definition}
  The \emph{degree} of a smooth bundle $E$ on $C$ is the integer
  \begin{align*}
    \deg E = c_1(E)\cdot[C].
  \end{align*}
\end{definition}

\begin{theorem}
  Smooth bundles on $C$ are isomorphic if, and only if, they have
  equal rank and degree.
\end{theorem}

\missingsection

\section{Spaces of Holomorphic Bundles}

Having established holomorphic bundles on compact Riemann surfaces,
we are ready to contemplate the spaces that they form. While it is
quite straightforward to obtain topological spaces of equivalence
classes of holomorphic bundles, it turns out that neither is Hausdorff.

In light of previous results, fix a smooth bundle $E$ of rank $r$
and degree $d$ on a compact Riemann surface $C$.

\subsection{The Naive Quotients}

Recall that a holomorphic bundle with underlying smooth structure $E$
is given by a Dolbeault operator $\dol_E$. Hence we have a set of
\begin{align*}
  \Dol(E) := \left\lbrace{
      \text{Dolbeault operators $\dol_E : \Gamma(C,E) \to \Omega^{0,1}(C,E)$}
  }\right\rbrace.
\end{align*}
This has an obvious topology given by the following observations:
\begin{lemma}
  Consider linear maps
  \begin{align*}
    \dol_E,\alpha : \Gamma(C,E) \to \Omega^{0,1}(C,E).
  \end{align*}
  Then $\dol_E\in\Dol(E)$ if, and only if, $\dol_E+\alpha\in\Dol(E)$.
  \begin{proof}
    Consider a trivialisation ${\pi}^{-1}U \cong \mathbb{C}^r \times U$.
    Then, for all $s\in\Gamma(U,E)\cong C^\infty(U,\mathbb{C}^r)$,
    \begin{align*}
      (\dol_E + \alpha)^2(s)
      = \alpha(\dol_E s) + \dol_E(\alpha s) + \alpha^2 s.
    \end{align*}
    \missingproof
  \end{proof}
\end{lemma}

\missingsection

\section{Quotients and Banach Manifolds}

\missingsection
\section{Properties of the Analytic Moduli Spaces}

\missingsection
\chapter{As Schemes}

The moduli space of holomorphic bundles on a compact Riemann
surface maybe constructed as a complex manifold exclusively
using analytic methods. Since this space was first constructed
in \missingcitation
the theory of moduli spaces has come a long way.
The goal of this section is to formally define moduli spaces
as schemes representing functors, explain how such schemes
are constructed, and hence formally define the schemes whose
points correspond to holomorphic bundles and Higgs bundles
on a compact Riemann surface, respectively.

We begin by translating fundmental notions such as compact
Riemann surfaces, holomorphic bundles, and Higgs bundles into
the language of algebraic geometry. This allows us to state
precisely the moduli problems that we aim to solve. The construction
of the related moduli spaces is achieved in multiple steps which
mirror the analytic approach: Firstly, we observe that the
holomorphic bundles we are interested in may be thought of as
points of certain Quot schemes. Secondly, we define group
actions on these schemes whose orbits correspond to the
equivalence classes we are after. Finally, we use geometric
invariant theory (GIT) to take the quotients by these group
actions. While it is going to require a significant amount of
work to motivate and define the appropriate GIT quotients,
we will see that the same machinery can be used to solve
a plethora of similar problems without much extra work.

Once constructed, we will take some time to study the moduli
spaces which we have thus obtained. It turns out that the GIT
quotients have several desriable properties such as the being
quasi-projective varieties. Moreover, we will use deformation
theory to show that holomorphic bundles may still be thought
of as cotangent vectors of holomorphic bundles.

\paragraph*{Notation}

In addtion to the notation used in the previous chapters, unless
otherwise indicated,
\begin{itemize}
  \item all schemes, maps, and products are taken over $\mathbb{C}$,
    i.e.~in the slice category
    $\Sch := \Sch_{\mathbb{Z}}/\Spec\mathbb{C}$;
  \item a projective scheme is a projective scheme over $\mathbb{C}$
    in the sense of \cite{hartshorne1977} - that is, $X$ is projective
    if there exists a closed immersion $X\inc\projective{n}{}$ for
    some $n$;
  \item sheaves on a scheme $X$ are sheaves of $\mathcal O_X$-modules,
    maps of sheaves are $\mathcal O_X$-linear, and tensor products
    of sheaves are taken over $\mathcal O_X$.
\end{itemize}

\section{Holomorphic Bundles as Locally Free Sheaves}

\missingsection

\subsection{Compact Riemann Surfaces as Algebraic Curves}
\label{sec:surfaces_as_curves}

We are going to view moduli spaces of holomorphic bundles
as schemes. The first step towards this is translating the
base space. While it is not in general possible to view
every manifold as a scheme \missingcitation, it is a well
known fact that compact Riemann surfaces correspond to smooth
algebraic curves.
This fact is proven in various places, see \cite[215]{griffiths1994}
for the general idea and \cite[5-16]{harris2011}
for a comprehensive treatment.

It is worth recalling what the algebraic structure on a compact
Riemann surface $C$ of genus $g\geq 2$ is. The easiest way to obtain
this structure is by constructing an embedding into an ambient
scheme. The correct scheme to consider will be the projectivisation
of an $n$-dimensional complex vector space $V$ which is given
by
\begin{align*}
  \projective{}{}(V) := \Proj\left({
      \bigoplus_{d=0}^\infty \Sym^d(V^\vee)
  }\right)
\end{align*}
Complex points in $\projective{}{}(V)$ may be identified
with one-dimensional subspaces of $V$. \missingcitation
Now if $\mathcal L$ is
an invertible sheaf on $C$ then there is a natural map
\begin{align}\label{eq:natural_line_bundle_map}
  H^0(C,\mathcal L)\otimes\mathcal O_C \to \mathcal L
\end{align}
given by $s\otimes f \mapsto f\restrict{s}{U}$. We will be
particularly interested in the case where $\mathcal L$ is a
quotient of the free sheaf $H^0(C,\mathcal L)\otimes\mathcal O_C$.

\begin{definition}
  A sheaf $\mathcal F$ on a ringed space $X$ is
  \emph{globally generated} if the induced map
  $H^0(X,\mathcal F) \otimes \mathcal O_X \to \mathcal F$
  is a surjection.
\end{definition}

\todo{check notation here}
In particular, if $\mathcal L$ is globally generated then,
for every $x\in C$, it induces a surjection on stalks
$H^0(C,\mathcal L) \surj \mathcal L_x$ whose kernel is a
subspace of codimension 1, i.e. a closed point of
$\projective{}{}(H^0(C,\mathcal L)^\vee)$. It turns out that this map is an embedding
whenever $\deg\mathcal L > 2g$. \cite[Proposition 2.14]{harris2011}
All that is left to do is to find invertible sheaves with large
degree. Fortunately, for any invertible sheaf $\mathcal L$,
the tensor powers $\mathcal L^{\otimes m}$ are invertible and
$\deg\mathcal L^{\otimes m} = m\deg\mathcal L$. Hence any invertible
sheaf of positive degree will do. Fortunately, the degree of
the canonical sheaf $\Omega_C$ is $2g-2$. As we have restricted
our attention to the case $g\geq 2$, we are free to choose
$m\geq 2$ and hence $\mathcal L = \Omega_C^{\otimes m}$ to obtain
an embedding $C\inc\projective{}{}(H^0(C,\mathcal L)^\vee)$, as
required.
By Chow's theorem, this makes $C$ an algebraic subvariety
and hence an algebraic curve.

\subsection{Vector Bundles on Schemes}

As we are now able to regard our base space, a compact Riemann
surface, as a scheme, it makes sense to translate vector bundles
into this setting. It is not very difficult to come up with a
sensible definition based on the usual setting of manifolds:

\begin{definition}[{\cite[Definition 11.5]{gortz2010}}]
  \label{def:vector_bundle}
  A \emph{vector bundle} of rank $r$ on a scheme $X$ is
  a map of schemes $\pi : E \to X$ such that there is an open
  cover $X = \bigcup_i U_i$ and isomorphisms
  \begin{align}\label{eq:trivialisation}
    \phi_i : {\pi}^{-1}U_i \cong \affine{r}{}\times U_i
  \end{align}
  such that, for every affine $U = \Spec R \subseteq U_i \cap U_j$,
  the map $\phi_i \circ \phi^{-1}_j$ restricts to a $R$-linear
  automorphism of $\affine{r}{}\times U = \Spec R[T_1,\ldots,T_r]$.
\end{definition}

\begin{example}
  \begin{itemize}
    \item trivial bundle
    \item bundle on curve from previous section, e.g. $T^1$ or
      $T^1 \# T^1$.
  \end{itemize}
\end{example}

\begin{itemize}
  \item show that holomorphic bundles on a compact Riemann surface $C$
    are the same thing as vector bundles on $C$, regarded as a smooth algebraic curve and hence a scheme
\end{itemize}

\missingsection

\subsection{Locally Free Sheaves}\label{sec:locally_free_sheaves}

While it was easy to translate vector bundles from manifolds
to schemes, the resulting notion may not always be the most
useful. It is much more natural to talk about sheaves. We know
that with each vector bundle $E\to X$ comes a sheaf of sections
$\Gamma(-,E)$. This has a natural $\mathcal O_X$-module structure.

\begin{lemma}
  The sheaf of sections of a vector bundle $E$ on a scheme $X$ is
  locally free.
  \begin{proof}
    If $E=\affine{r}{}\times X$ is
    trivial then the sections $s\in\Gamma(U,E)$ are just maps
    $s:U\to \affine{r}{}$. If, moreover, $U$ is affine,
    $\Gamma(U,E)$ corresponds to maps of
    $\mathbb{C}$-algebras
    $\mathbb{C}[T_1,\ldots,T_n]\to\mathcal O_X(U)$ so
    $\Gamma(U,E)\cong \mathcal O^r_X(U)$, i.e. $\Gamma(-,E)$
    is free.
    More generally, for each element $U_i$ of the cover in
    \ref{def:vector_bundle}, we can consider affines $U\subseteq U_i$
    to find $\restrict{\Gamma(-,E)}{U_i}\cong \restrict{\mathcal O_X^r}{U_i}$,
    as required.
  \end{proof}
\end{lemma}

However, much more is true. It turns out that on a scheme $X$
a vector bundle of rank $n$ is the same thing as a locally
free sheaf of rank $n$. That is, there is an equivalence
of categories given by $E \mapsto \Gamma(-,E)^\vee$.~\cite[128-129]{hartshorne1977}
We will say that a sheaf $\mathcal E$ corresponds to a vector bundle
$E$ and write $\mathcal E=\mathscr V(E)$ to mean
$\mathcal E\cong\Gamma(-,E)^\vee$.

\subsection{Higgs Sheaves}

\missingsection

\subsection{Moduli Problems}

We have shown that holomorphic bundles on a compact Riemann surface
correspond to locally free sheaves on a certain smooth algebraic
curve. Now we would like to construct schemes whose points are
holomorphic bundles. By points of a scheme $M$ one usually means the
closed points. In the case where $M$ is locally of finite type,
closed points are $\mathbb{C}$-points, i.e. maps
$\Spec\mathbb{C}\to M$. \cite[Corollary 3.36]{gortz2010} As we
are hoping for our moduli spaces to be geometrically well-behaved,
this is a reasonable approximation to make. We are now justified in
thinking of a moduli problem as sending a scheme $T$ to the
$T$-points of a hypothetical moduli space.

\begin{definition}
  A \emph{moduli problem} is a functor $\mathscr M:\Sch^{op}\to\Set$.
\end{definition}

\begin{example}
  \missingexample
  \begin{itemize}
    \item lines in a vector space
    \item holomorphic bundles up to isomorphism
    \item Higgs bundles up to isomorphism
    \item some coarse example
  \end{itemize}
\end{example}

Now the obvious notion of a moduli space is a scheme whose functor
of points is precisely the moduli problem. This is called a fine
moduli space:

\begin{definition}
  A \emph{fine moduli space} of a moduli problem $\mathscr M$
  is a scheme $M$ with a natural isomorphism
  $\eta : \mathscr M \cong \Hom(-,M)$.
\end{definition}

Slightly abusing notation, we may simply write
$\eta : \mathscr M \cong M$. We are often going to surpress the
isomorphism $\eta$ and call $M$ the fine moduli space.

\begin{example}
  \begin{itemize}
    \item projectivisation/projective space
  \end{itemize}
\end{example}

Beyond the fact that a fine moduli spaces represent the moduli
problem and thus are the best possible solution, there are some
pleasant properties to observe. Consider an element
$F\in\mathscr M(T)$, for some $T$. Observe that under $\eta$,
this corresponds to a map $f:T\to M$ which induces pullback maps
$f^* : \Hom(M,M)\to\Hom(T,M)$ and hence
$F^* : \mathscr M(M)\to \mathscr M(T)$.
Now note that there is an element $U\in\mathscr M(M)$ that
corresponds to the identity on $M$. Thus $F^* U = F$, so every
element $\mathscr M(T)$ is given by pulling back $U$. Hence
we make the following definition:

\begin{definition}
  Let $\mathscr M$ be a moduli problem and $M$ a fine moduli space.
  The \emph{universal element} $U\in\mathscr M(M)$ is
  $U:=\eta^{-1}_M (\identity)$.
\end{definition}

\begin{example}
  \missingexample
\end{example}

However, it turns out that this is often too much to ask. Indeed,
we really care about the $\mathbb{C}$-points and hence
we are otherwise ready accept some deviation so long as it is not
possible to do better:

\begin{definition}\missingcitation
  A \emph{coarse moduli space} of a moduli problem $\mathscr M$
  is a scheme $M$ together with a natural transformation
  $\eta : \mathscr M \to \Hom(-,M)$ such that
  \begin{itemize}
    \item $\eta : \mathscr M(\Spec\mathbb{C})\to M(\Spec\mathbb{C})$ is a bijection;
    \item for every scheme $N$, every natural transformation
      $\nu : M \to \Hom(-,N)$ factors through $\eta$ as in the diagram
      \missingdefinition{universal property}
  \end{itemize}
\end{definition}

\begin{example}
  \missingexample
\end{example}

Unfortunately, even coarse moduli spaces may not exist. Here is
one property that prevents a moduli space from existing. We will
see an example of this later. \todo{add reference to problem of all
locally free sheaves}

\begin{lemma}\label{lem:no_coarse_condition}
  Let $\mathscr M$ be a moduli problem and
  $x,y\in \affine{1}{}(\mathbb{C})$. If there is an
  $F\in\mathscr M(\affine{1}{})$ such that
  $\mathscr M(x)(F) = \mathscr M(y)(F)$ if, and only if,
  $x,y\neq 0$ or $x=y=0$ then there is no coarse moduli space
  for $\mathscr M$.
  \begin{proof}
    Following \cite[Lemma 2.27]{hoskins2016}.
    \missingproof
  \end{proof}
\end{lemma}

By construction, both kinds of moduli spaces are unique up to
isomorphism and every fine moduli space is a coarse moduli
space. Hence solving moduli problems usually proceeds in
three steps:

\begin{enumerate}
  \item Formulate the problem by defining the desired functor of
    points.
  \item Construct a coarse moduli space.
  \item Check under which conditions the moduli space is fine.
\end{enumerate}

\section{Topology of Sheaves on Schemes}

Vector bundles correspond to locally free sheaves. Hence it is worth
taking some time to study locally free sheaves and, more generally,
sheaves on schemes. This is of course a very wide field
so we are going to highlight only a few key properties.

The main goal of this section is to establish some topological
invariants of coherent sheaves, generalising the grouping of
vector bundles by rank and degree. The obvious tool for studying
the topology of sheaves is cohomology. While we will have little
to do with sheaf cohomology in its raw form, we are going to
observe several properties of sheaves on curves and, more generally,
projective schemes that are going to prove useful.

\subsection{Twisting}

We begin by revisiting one of the most important family of sheaves
on projective schemes. See e.g. \cite{gortz2010} or
\cite{hartshorne1977} for detailed treatments.

Consider a projective scheme $X$. Recall that we may write
$X=\Proj R$ for some $R=\mathbb{C}[T_0,\ldots,T_n]/I$ \cite[II Corollary 5.16]{hartshorne1977}. Such a scheme comes
equipped with a particularly important family of invertible sheaves
called Serre's twisting sheaves. To define them, recall that each
graded $R$-module $M$ defines a unique sheaf $\tilde M$ on $X$
that satisfies $\tilde M (D_+(f)) = M_{(f)}$ where $M_{(f)}$
denotes the homogeneous localisation $M$ at $f\in R$. This
correspondence between modules and sheaves allows the following
definition:

\begin{definition}[{\cite[13.4]{gortz2010}}]
  For $m\in\mathbb{Z}$, define the graded $R$-module $R(m)$ by
  $R(d)_d := R_{m+d}$ for all $d\in\mathbb{Z}$. \emph{Serre's
  twisting sheaf} is
  \begin{align*}
    \mathcal O_X(m) := \widetilde{R(m)}.
  \end{align*}
  More generally, for a sheaf $\mathcal F$ on $X$, define
  $\mathcal F(m) := \mathcal F \otimes \mathcal O_X(m)$.
\end{definition}
The notation $\mathcal O_X(m)$ is justified by the observation
$\mathcal O_X(m) \cong \mathcal O_X \otimes \mathcal O_X(m)$.
Of course, the main case of interest for us is
$R=\mathbb{C}[T_0,\ldots,T_n]$, i.e. $X=\projective{n}{}$.
In this case, the twisting sheaf corresponds to a line bundle,
i.e. is invertible. That is, it has rank 1.

\begin{proposition}[{\cite[Proposition 13.15]{gortz2010}}]
  \label{thm:twisiting_sheaf_invertible}
  If $R$ is finitely generated as a $R_0$-algebra, then each
  $\mathcal O_X(m)$ is an invertible sheaf.
  \begin{proof}
    \missingproof
  \end{proof}
\end{proposition}

Why are the twisting sheaves important to us? They are well-behaved
with respect to the parameters. In particular,
under the assumption of \ref{thm:twisiting_sheaf_invertible},
$(\mathcal F,m)\mapsto \mathcal \mathcal F(m)$ defines an action of
the integers on the group of sheaves on $X$:

\begin{lemma}\label{lem:additivity_twisting_sheaf}
  Let $R$ be finitely generated as a $R_0$-algebra and $m,n\in\mathbb{Z}$.
  Then $\mathcal O_X(m) \otimes \mathcal O_X(n) \cong \mathcal O_X(m + n)$. Hence
  $\mathcal F(m) \otimes \mathcal O_X(n) \cong \mathcal F(m+n)$.
  \begin{proof}
    \missingproof
  \end{proof}
\end{lemma}

We will see that topological properties of sheaves behave well
with respect to taking direct sums and tensor products and hence
with respect to twisting, too. Moreover, for suitable $R$ and
sufficiently large $m$, $\mathcal F(m)$ is globally generated. In this
case, for $m\geq 1$, the sheaves $\mathcal O_X(m)$ are ample.

\subsection{Ample sheaves}

Recall the definition of an ample sheaf:

\begin{definition}
  An invertible sheaf $\mathcal L$ on a quasi-compact quasi-separated
  scheme $X$ is \emph{ample} if, for all quasi-coherent sheaves of
  finite type $\mathcal F$, for $m$ sufficiently large,
  $\mathcal F\otimes \mathcal L^{\otimes m}$ is
  globally generated.
\end{definition}

On general schemes, ample sheaves need not exist. However, on
projective space the twisting sheaves serve as an example:

\begin{proposition}
  Let $R$ be a finitely generated $R_0$-algebra,
  write $X=\Proj R$ and let $m,n\geq 1$. Then
  $\mathcal O_X(m)$ is ample.
  \begin{proof}
    See \cite[Example 13.45]{gortz2010}.
  \end{proof}
\end{proposition}

More generally, any quasi-projective quasi-compact scheme has
an ample sheaf given by pulling back a twisted sheaf:

\begin{proposition}
  Let $X$ be projective, $U\subseteq X$ quasi-compact, and
  $j : U \inc X$ an open immersion then. Then, for some $m\geq 1$,
  $j^*\mathcal O_X(m)$ is ample.
  \begin{proof}
    See \cite[\href{https://stacks.math.columbia.edu/tag/01Q2}{Tag 01Q2}]{stacks-project}.
  \end{proof}
\end{proposition}

\begin{corollary}
  Every quasi-projective scheme has an ample sheaf.
\end{corollary}

\todo{motivate all this a little bit more}

\subsection{Euler Characteristic}

The fundamental topological invariant of sheaves that we are going
to be interested in is the Euler characteristic. This is defined
in the obvious way, replacing regular cohomology with sheaf
cohomology:

\begin{definition}
  The \emph{Euler characteristic} of a coherent sheaf $\mathcal F$
  on a projective scheme $X$ is
  \begin{align*}
    \chi (X,\mathcal F) := \sum_{j=0}^\infty (-1)^j \dim H^j (X,\mathcal F).
  \end{align*}
\end{definition}

Note that, by finite-dimensionality of coherent cohomology
(e.g. \cite[\href{https://stacks.math.columbia.edu/tag/02O6}{Tag 02O6}]{stacks-project}) and Grothendieck's vanishing theorem (e.g.
\cite[III Theorem 2.7]{hartshorne1977}), this is sum is finite for
all $\mathcal F$.

\begin{example}
  Our primary interest will be in the case where
  $\dim X = 1$, so $H^j(X,\mathcal F)=0$ for $j\geq 2$ and
  \begin{align*}
    \chi (X,\mathcal F) = \dim H^0(X,\mathcal F)-\dim H^1(X,\mathcal F).
  \end{align*}
  If we moreoever take $\mathcal F=\mathcal O_X$ then
  $\chi (X,\mathcal O_X) = 1 - g$. \todo{justify}
\end{example}

\begin{example}
  Consider the case $X=\projective{n}{}$.
  From \missingcitation we know that
  % https://swc-math.github.io/notes/files/06StillmanNotes.pdf
  % https://achinger.impan.pl/fac/fac.pdf
  \begin{align*}
    \dim H^j(\projective{n}{},\mathcal O_{\projective{n}{}}(m)) =
    \begin{cases}
      (n+1)^m & \text{if }j = 0 \\
      0 &\text{if }0<j<n \\
      \binom{-n-1}{m} &\text{if }j=n
    \end{cases}
  \end{align*}
  Hence \todo{doublecheck}
  \begin{align*}
    \chi(\projective{n}{},\mathcal O_{\projective{n}{}}(m))
    = (n+1)^m + \binom{-n-1}{m}.
  \end{align*}
\end{example}

\begin{lemma}
  \todo{additivity on short exact sequences}
\end{lemma}

\begin{lemma}
  \todo{invariance under base change}
\end{lemma}

\subsection{Hilbert Polynomials}

While the Euler characteristic certainly is a topological invariant,
it is rather limiting. Rather than focusing on a single integer,
the Euler characteristic,
we are going to keep track of all the Euler characteristics of all
the twists of a sheaf. This turns out to be described by a polynomial.

Consider a projective scheme $X$ with an ample sheaf $\mathcal L$
and a coherent sheaf $\mathcal F$.

\begin{definition}
  The \emph{Hilbert polynomial} of $\mathcal F$ at $m\in\mathbb{Z}$
  is
  \begin{align*}
    P(\mathcal F,\mathcal L)(m) := \chi(X,\mathcal F \otimes \mathcal L^{\otimes m}).
  \end{align*}
\end{definition}

A priori, the Hilbert polynomial is a function taking integers to
integers. The fact that it is described by a polynomial is
surprising and non-trivial.

\begin{lemma}
  There exists a unique polynomial $p\in\mathbb{Q}[t]$ such that,
  for all $m$, $P(\mathcal F,\mathcal L)(m) = p(m)$.
  \begin{proof}
    See \cite[Lemma 1.2.1]{huybrechts2010}.
  \end{proof}
\end{lemma}

\begin{example}
  \missingexample
\end{example}

Serre's vanishing theorem \missingcitation states that,
for $m$ sufficiently large and $j>0$,
$H^j(X,\mathcal F\otimes\mathcal L^{\otimes m})=0$. Hence,
eventually,
$P(\mathcal F,\mathcal L)(m) = \dim H^0(X,\mathcal F\otimes \mathcal L^{\otimes m})$.

\begin{lemma}
  \todo{invariance under base change}
\end{lemma}

\subsection{Degree}

While the Hilbert polynomial of a coherent sheaf is in general the
correct invariant to consider, we are working with locally free
sheaves on a curve. Hence some simplifications are in order.
Recall that for vector bundles on a curve we defined the degree.
A similar definition is possible for locally free sheaves:

\begin{definition}[{\cite[\href{https://stacks.math.columbia.edu/tag/0AYQ}{Tag 0AYQ}]{stacks-project}}]
  The \emph{degree} of a locally free sheaf $\mathcal F$ of rank $r$
  on $C$ is $\deg \mathcal F := \chi (C,\mathcal F) - r\chi(C,\mathcal O_C)$.
\end{definition}

\begin{example}
  As the rank of $\mathcal O_C$ is $1$ its degree must be $0$.
\end{example}

\begin{lemma}\label{lem:degree_of_tensor}
  If $\mathcal E$ and $\mathcal F$ are locally free on $C$ then
  \begin{align*}
    \deg(\mathcal E\otimes\mathcal F) = \rank\mathcal E\deg\mathcal F + \rank\mathcal F\deg\mathcal E.
  \end{align*}
  \begin{proof}
    \cite[Exercise 8.24]{hoskins2016}.
    \missingproof
  \end{proof}
\end{lemma}

\begin{example}
  Using~\ref{lem:degree_of_tensor}, we find \todo{make sure $\mathcal O_X(1)$ has degree 1}
  \begin{align*}
    \deg \mathcal E(m) = \deg\mathcal E + m\rank\mathcal E.
  \end{align*}
\end{example}

\missingsection

\section{Locally Free Sheaves as Points of Quot Schemes}

The first step in constructing the analytic moduli spaces of
holomorphic bundles was to consider the slightly larger space of
all bundles bundles of a certain rank and degree, including repeated
occurences of isomorphic instances. While this preliminary space
did not serve as a true
classification,
it provided us with a good starting point for future analysis. As the
algebraic construction follows a similar procedure, it is now
time to construct an analogous scheme.

We define a much more general moduli problem called
the Quot functor which was shown to have a fine moduli space by
Grothendieck. This is the first piece of machinery which will do a
lot of the work for us and for many other moduli problems of vector
bundles. We will show that there is a Quot scheme whose points
may be thought of as locally free sheaves of a certain rank and
degree.

\subsection{All Locally Free Sheaves}

Our approach to constructing moduli spaces of equivalence classes of
sheaves is to consider a larger space and then quotient by a suitable
group action. In the analytic case, we started with the affine space of
all holomorphic bundles. On the algebraic side, this does not quite work.
Let us see why that is.

The first step is to define the correct moduli problem. The key idea
is to associate to a scheme $T$ a family of locally free sheaves
$\mathcal E_t$. We may view a sheaf $\mathcal E$ on $C_T := C\times T$
as a family of sheaves on $C$ indexed by $T$. If we have a point $t\in T$
then we have the fibre $C_t := \Spec k(t) \times C$ and the sheaf
$\mathcal E_t := \restrict{\mathcal E}{C_t}$. In the case where
$t$ is a $\mathbb{C}$-point, $C_t \cong C$ so $\mathcal E_t$ is indeed
a sheaf on $C$. Moreover, if $\mathcal E$ is flat over $T$ then
each $\mathcal E_t$ has the same Hilbert polynomial.

To truly view locally free sheaves $\mathcal E,\mathcal F$ on
$C_T$ flat over $T$ as a family, we need to adjust our notion of
equivalence. In particular, we require all the fibres to be isomorphic.

\begin{lemma}
  Let $\mathcal E$ and $\mathcal F$ be locally free sheaves on $C_T$
  flat over $T$. Then $\mathcal E_t\cong\mathcal F_t$ for all $t\in T$
  if, and only if, there exists an invertible sheaf $\mathcal L$ on
  $T$ such that $\mathcal E \cong \mathcal F \otimes \pi^*\mathcal L$
  where $\pi : C_T \to C$ is the base change.
  \begin{proof}
    \todo{not sure this is true; if it isn't then we need to find a
    different explanation}
    \missingproof
  \end{proof}
\end{lemma}

Hence we make the following definition:

\begin{definition}
  Define the moduli problem $\mathscr A : \Sch^{\text{op}} \to \Set$
  bysending a scheme $T$ to
  \begin{align*}
    \mathscr A(T) := \left\lbrace{\text{locally free sheaves $\mathcal E$ on $C_T$ flat over $T$}}\right\rbrace/\sim
  \end{align*}
  where $\mathcal E\sim\mathcal F$ if, and only if, there is a line bundle
  $\mathcal L$ on $T$ such that
  $\mathcal E \cong \mathcal F\otimes\pi^* \mathcal L$. Maps
  $f: T'\to T$ get sent to the pullback map
  $\mathcal E\mapsto f^*\mathcal E$.
\end{definition}

However, this fails to have any moduli space at all:

\begin{lemma}\label{lem:no_coarse_moduli_space}
  The moduli problem $\mathscr A$ of locally free sheaves on $C$
  does not admit a coarse moduli space.
  \begin{proof}
    Follwing \cite[Example 2.2]{hoskins2016}. We aim to construct
    an $\mathcal E\in\mathscr A(\affine{1}{})$ that satisfies the
    condition in \ref{lem:no_coarse_condition}.
    \missingproof
  \end{proof}
\end{lemma}

Let us think about the case of Higgs sheaves. Consider the moduli problem
given by \todo{add new equivalence; make sure the canonical sheaf restricts to a canonical sheaf}
\begin{align*}
  \mathscr A_H(T) = \left\lbrace{\text{Higgs sheaves $(\mathcal E,\phi)$
  on $X_T$}}\right\rbrace/\sim.
\end{align*}
where $(\mathcal E,\phi)\sim(\mathcal E',\phi')$ if, and only if,
there is an isomorphism $\mathcal E\cong\mathcal E'$ that makes the
following commute:
\begin{equation*}
  % https://q.uiver.app/#q=WzAsNCxbMCwwLCJcXG1hdGhjYWwgRSJdLFswLDEsIlxcbWF0aGNhbCBFJyJdLFsyLDEsIlxcbWF0aGNhbCBFJ1xcb3RpbWVzXFxPbWVnYV4xX3tDXFx0aW1lcyBUfSJdLFsyLDAsIlxcbWF0aGNhbCBFXFxvdGltZXMgXFxPbWVnYV4xX3tDXFx0aW1lcyBUfSJdLFswLDMsIlxccGhpIl0sWzEsMiwiXFxwaGknIl0sWzAsMSwiXFxjb25nIiwyXSxbMywyLCJcXGNvbmciXV0=
  \begin{tikzcd}
    {\mathcal E} && {\mathcal E\otimes \Omega^1_{C\times T}} \\
    {\mathcal E'} && {\mathcal E'\otimes\Omega^1_{C\times T}}
    \arrow["\phi", from=1-1, to=1-3]
    \arrow["\cong"', from=1-1, to=2-1]
    \arrow["\cong", from=1-3, to=2-3]
    \arrow["{\phi'}", from=2-1, to=2-3]
  \end{tikzcd}
\end{equation*}
Once again, by restricting to fibres over $t\in T(\mathbb{C})$
a family of Higgs sheaves $(\mathcal E,\phi)$ on $X_T$ yields
a Higgs field
\begin{align*}
  \phi_t : \mathcal E_t \to \mathcal E_t \otimes \Omega^1_C.
\end{align*}
Unfortunately, we find ourselves in a similar position.
\begin{corollary}
  The moduli problem $\mathscr A_H$ of Higgs sheaves on $C$ does not admit
  a coarse moduli space.
  \begin{proof}
    Consider the natural transformation $\mathscr A\to\mathscr A_H$
    given by $\mathcal E \mapsto (\mathcal E,0)$. Now the sheaf
    $\mathcal E\in\mathscr A(\affine{1}{})$ from the proof of
    \ref{lem:no_coarse_moduli_space} pushes forward to a Higgs field
    $(E,0)\in\mathscr A_H(\affine{1}{})$ which, moreoever, satisfies the
    same property.
  \end{proof}
\end{corollary}
Thus it is not possible to construct the moduli spaces we desire
in the naive way. There are essentially two ways to get around this.
One option is to deal with
moduli stacks instead. This has the advantage of solving the problem
without discarding any information. However, it involves dealing with
algebraic stacks which are unwieldy, even more so than schemes.
See \cite{cm2017} for this point of view. We will instead take an
approach that is much in line with the analytic construction by narrowing
our attention to a subclass of locally free sheaves. Indeed, we are
going to end up with those sheaves corresponding to (semi-)stable
bundles.

\subsection{Quotients}

If a sheaf $\mathcal F$ is globally generated and
$\dim H^0(X,\mathcal F) = N$ then $\mathcal F$ may be thought
of as a quotient of the free sheaf $\mathcal O_X^{\oplus N}$
by choosing an isomorphism
\begin{align*}
  H^0(X,\mathcal F) \otimes \mathcal O_X \cong \mathbb{C}^N \otimes \mathcal O_X \cong \mathcal O_X^{\oplus N}.
\end{align*}
It will turn out that being globally generated will be a reasonable
condition in the sense that we are not missing too many vector
bundles. Thus it is sensible to study quotients of coherent sheaves.

\begin{definition}
  Let $\mathcal E$ be a coherent sheaf on a scheme $X$.
  A \emph{family of quotients} of $\mathcal E$ over a scheme $T$
  consists of a sheaf $\mathcal F$ on $X_T$ flat over $T$ and a
  surjection $q:\pi^*\mathcal E\surj\mathcal F$ where
  $\pi : X_T \to X$ is the base change.
\end{definition}

Note the condition that the quotient be flat. Recall that a sheaf
$\mathcal F$ on $X_T$ is flat over $T$ if each of the functors
$M \mapsto \mathcal F_{(x,t)} \otimes_{\mathcal O_{T,t}} M$ is exact.
In light of \cite[III Theorem 9.9]{hartshorne1977} this is a
particularly useful condition for us as it ensures that the fibres
$\mathcal F_t$ all have the same Hilbert polynomial. Hence we may
refer to the Hilbert polynomial $P(q,\mathcal L)$ of the family
$q:\pi^*\mathcal E\surj\mathcal F$.

\begin{example}
  We saw earlier that a globally generated locally free sheaf
  $\mathcal F$ of rank $r$ and degree $d$ on $C$ together with
  a choice of isomorphism $\mathbb{C}^N \cong H^0(C,\mathcal F)$
  give rise to a family of quotients
  \begin{align*}
    H^0(C,\mathcal F)\otimes\mathcal O_C \cong \mathcal O^{\oplus N}_C
    \surj \mathcal F
  \end{align*}
  over $\mathbb{C}$ where $N := \dim H^0(X,\mathcal F)$.
\end{example}

Consider two quotients $\mathcal F$ and $\mathcal F'$ of a
sheaf $\mathcal E$.
Such quotients are equivalent if there is an isomorphism
$\mathcal F\cong\mathcal F'$ such that the following commutes
\begin{equation}\label{eq:quotient_equivalence}
  % https://q.uiver.app/#q=WzAsMyxbMCwwLCJcXG1hdGhjYWwgRSJdLFsyLDAsIlxcbWF0aGNhbCBGIl0sWzIsMSwiXFxtYXRoY2FsIEYnIl0sWzEsMiwiXFxjb25nIl0sWzAsMSwicSIsMCx7InN0eWxlIjp7ImhlYWQiOnsibmFtZSI6ImVwaSJ9fX1dLFswLDIsInEnIiwyLHsic3R5bGUiOnsiaGVhZCI6eyJuYW1lIjoiZXBpIn19fV1d
  \begin{tikzcd}
    {\mathcal E} && {\mathcal F} \\
                 && {\mathcal F'}
                 \arrow["q", two heads, from=1-1, to=1-3]
                 \arrow["{q'}"', two heads, from=1-1, to=2-3]
                 \arrow["\cong", from=1-3, to=2-3]
  \end{tikzcd}
\end{equation}

We would now like to consider the moduli problem of equivalence
classes of quotients of a coherent sheaf $\mathcal E$ on $X$.
We note that base change preserves quotients. In particular,
if $q : \mathcal E \surj \mathcal F$ is a quotient and
$f : Y \to X$ is a base change then
$f^*q : f^*\mathcal E \surj f^*\mathcal F$ is a quotient
by \cite[\href{https://stacks.math.columbia.edu/tag/01U9}{Tag 01U9}]{stacks-project}.

\subsection{Quot functors}

Thus we may consider the moduli problem of equivalence classes of
quotients of a given sheaf:

\begin{definition}\missingcitation
  Let $\mathcal E$ be a coherent sheaf on a projective scheme $X$.
  The corresponding \emph{Quot functor} is given by sending each
  scheme $T$ to
  \begin{align*}
    \mathscr Q(\mathcal E)(T) = \left\lbrace{
        \text{quotients $q : \pi^*\mathcal E \surj \mathcal F$
        over $T$}
    }\right\rbrace / \sim
  \end{align*}
  and $f : T' \to T$ to
  $\mathscr Q(\mathcal E)(f) = (\identity \times f)^*$ where
  $\pi^*\mathcal E$ is the pullback of $\mathcal E$ along
  the base change $\pi:X_T\to X$.
\end{definition}

Now note that this functor splits
\begin{align*}
  \mathscr Q(\mathcal E)
  = \bigsqcup_{P\in\mathbb{Q}[t]} \mathscr Q(\mathcal E)^{P,\mathcal L}.
\end{align*}
where each $\mathscr Q(\mathcal E)^{P,\mathcal L}$ contains families
with Hilbert polynomial $P$.

\begin{example}
  Recall on $C$, we have
  \begin{align*}
    P(\mathcal F,\mathcal O_C(1))(t) = rt + d + r(1-g)
  \end{align*}
  for any locally free sheaf $\mathcal F$ of rank $r$ and degree $d$.
  Thus, for any coherent sheaf $\mathcal E$, we have a splitting
  \begin{align*}
    \mathscr Q(\mathcal E) = \bigsqcup_{r,d} \mathscr Q(\mathcal E)^{r,d}.
  \end{align*}
  \missingexample
\end{example}

\subsection{Quot schemes}

One of the reasons the Quot functor is of interest to us is because
Grothendieck constructed a moduli space for it. \missingcitation

\begin{theorem}
  Let $X$ be a projective scheme with an ample sheaf $\mathcal L$,
  let $\mathcal E$ be a coherent sheaf on $X$, and let
  $p\in\mathbb{Q}[t]$. Then the Quot functor
  $\mathscr Q(\mathcal E)^{p,\mathcal L}$ has a fine moduli space
  $Q(\mathcal E)$ called the \emph{Quot scheme}.
  \begin{proof}
    For an extensive overview see \cite{hoskins2016}.
  \end{proof}
\end{theorem}

\begin{example}
  Once again, in the case $X=C$ we have schemes
  $Q(\mathcal E)^{r,d}$.
  \missingexample
\end{example}

\missingsection

\section{Geometric Invariant Theory}

The second major step in constructing moduli spaces of holomorphic
bundles, algebraic or analytic, is to take a quotient by an automorphism
group. In particular, we noticed that different choices of isomorphism
$H^0(X,\mathcal F)\cong\mathbb{C}^N$ give rise to different quotients.
The goal of this section is to eliminate these redundancies.

The first step is defining group actions in the setting of schemes.
This allows us to make precise what we mean by a quotient with
respect to such an action and what properties we would like such
quotients to satisfy. To construct such quotients we are going to
use a very powerful tool called geometric invariant theory (GIT).
In the affine case, the GIT quotients will not be very difficult
to obtain. However, a lot of work will be required to generalise the
construction sufficiently to suit our needs. We will find that
GIT naturally gives rise to a notion of stability which coincides
with the stability of vector bundles.

\subsection{Group Schemes}

\begin{definition}
  A \emph{group scheme} is a group object in $\Sch$.
\end{definition}
In particular, a group scheme consists of a scheme $G$ and three maps
\begin{align}\label{eq:group_maps}
  e:\Spec\mathbb{C}\to G,\hspace{1cm}
  i:G\to G,\hspace{1cm}
  m:G\times G\to G
\end{align}
satisfying the usual conditions of unitality, inverses, and associativity.
As usual with classical groups, we are going to leave the maps
$e$, $i$, and $m$ implicit.

\begin{example}\label{ex:group_schemes}
  In the affine case $G = \Spec R$ we may specify the multiplication
  and inverse maps in terms of $\mathbb{C}$-algebra homomorphisms
  $m^* : R \to R\otimes R$ and $i^* : R\to R$.
  There are several affine group schemes that we are already
  intuitively familiar with:
  \begin{enumerate}
    \item The additive group $\mathbf{G}_a := \Spec k[t]$ with
      comulitiplication $t\mapsto t\otimes 1 + 1\otimes t$,
    \item The multiplicative group $\mathbf{G}_m := \Spec k[t^\pm]$
      with comultiplication $t\mapsto t\otimes t$,
    \item The general linear group
      \begin{align*}
        GL_n := \Spec k[x_{ij} : 1\leq i,j\leq n][1/\det(x_{ij})]
      \end{align*}
      where $(x_{ij})$ is the $n\times n$ matrix with entries $x_{ij}$
      with comultiplication
      \begin{align*}
        x_{ij} \mapsto \sum_{k=1}^n x_{ik}\otimes x_{kj}.
      \end{align*}
      See e.g. \cite[\href{https://stacks.math.columbia.edu/tag/022W}{Tag 022W}]{stacks-project} for details.
    \item The sepcial linear group $SL_N$ analogous to the above by
      quotienting by $\det(x_{ij})^2 - 1$.
  \end{enumerate}
\end{example}
But what do these group schemes have to do with their well-known
counterparts. For example, how does the scheme $GL_n$ relate to the
group $GL_n(\mathbb{C})$? The notation is no coincidence.
Group schemes induce group structures on their points:
\begin{lemma}
  Let $G$ be a scheme with maps (\ref{eq:group_maps}). The following
  are equivalent:
  \begin{enumerate}
    \item $G$ is a group scheme.
    \item For every scheme $T$, $G(T)$ is a group.
  \end{enumerate}
  \begin{proof}

    \missingproof
  \end{proof}
\end{lemma}

\begin{example}
  The induced grous of the affine group schemes behave as expected:
  For a $\mathbb{C}$-algebra $R$, $\mathbf{G}_a(R) = (R,+)$,
  $\mathbf{G}_m(R) = (R^\times,\times)$, and
  $GL_n(R)$ is the usual group of invertible $n\times n$ matrices
  with coefficients in $R$.

  In more detail, consider $x\in\mathbf{G}_a(R)$ where $R$ is
  a $\mathbb{C}$-algebra. Such an $x$ is given by algebra homomorphsims
  $x^\sharp:\mathbb{C}[t] \to R$. That is, we may identify
  $\mathbf{G}_a(R)$ with $R$ using the map $x \mapsto x^\sharp(t)$. Now,
  for $x,y\in\mathbf{G}_a(R)$ and $f\in \mathbb{C}[t]$, we have the
  induced group multiplication given by
  \begin{align*}
    (xy)^\sharp(f)
    = x^\sharp(f) \cdot 1 + 1 \cdot y^\sharp(f)
    = x^\sharp(f) + y^\sharp(f).
  \end{align*}
  Thus the induced group structure on $\mathbf{G}_a(R)$ is just $(R,+)$.
\end{example}

\begin{definition}
  A group scheme is \emph{algebraic} if it is smooth and separated
  over $\mathbb{C}$.
\end{definition}

\begin{example}
  All the group schemes in \ref{ex:group_schemes} are affine algebraic
  groups. (See \cite[IV Theorem 9,3]{milne2012} for smoothness
  and \cite[Remark 3.2]{hoskins2016} for separatedness.) While group
  schemes are worth studying their full generality, the affine algebraic
  case will be sufficient for our purposes. Hence the only examples
  that the reader should have in mind are the ones already presented.
\end{example}

\subsection{Actions}

For this section, fix a scheme $X$ and a group scheme $G$.

\begin{definition}
  An \emph{action} of $G$ on $X$ is a morphism
  $\sigma : G\times X\to X$ satisfying the usual laws with respect
  to $m:G\times G\to G$.
\end{definition}

Note that an action $\sigma : G\times X\to X$ induces an action of
$G(T)$ on $X(T)$ by
\begin{equation*}
  T \xlongrightarrow{(g,x)} G\times X \xlongrightarrow{\sigma} X.
\end{equation*}

\begin{example}
  On locally free sheaves.
  \missingexample
\end{example}

\begin{example}
  On Higgs sheaves.
  \missingexample
\end{example}


\subsection{Invariants}

The points of a quotient ought to correspond to orbits of the group
action. In other words, the quotient map should be invariant.
Fortunately, we are able to define

\begin{definition}
  Let $G$ be a group scheme that acts on $X,Y$, respectively.
  A morphism $f:X\to Y$ is \emph{$G$-invariant} if the following
  commutes:
  \begin{equation*}
    % https://q.uiver.app/#q=WzAsMyxbMCwwLCJHXFx0aW1lcyBYIl0sWzIsMCwiWCJdLFs0LDAsIlkiXSxbMCwxLCJcXHJobyIsMSx7ImN1cnZlIjotMn1dLFswLDEsIlxccGkiLDEseyJjdXJ2ZSI6Mn1dLFsxLDIsImYiLDFdXQ==
    \begin{tikzcd}
      {G\times X} && X && Y
      \arrow["\sigma"{description}, curve={height=-12pt}, from=1-1, to=1-3]
      \arrow["\pi"{description}, curve={height=12pt}, from=1-1, to=1-3]
      \arrow["f"{description}, from=1-3, to=1-5]
    \end{tikzcd}
  \end{equation*}
\end{definition}

The obvious challenge of taking quotients in geometric settings
is that there is not just a topology but also a geometric
structure to take into account. A quotient $X/G$ should satisfy the
property that any $G$-invariant $X\to Z$ uniquely factors through
$X\surj X/G$:
\begin{equation*}
  % https://q.uiver.app/#q=WzAsNCxbMiwwLCJYIl0sWzQsMCwiWC9HIl0sWzQsMSwiWiJdLFswLDAsIkdcXHRpbWVzIFgiXSxbMCwxLCIiLDAseyJzdHlsZSI6eyJoZWFkIjp7Im5hbWUiOiJlcGkifX19XSxbMCwyXSxbMSwyLCJcXGV4aXN0cyEiLDAseyJzdHlsZSI6eyJib2R5Ijp7Im5hbWUiOiJkYXNoZWQifX19XSxbMywwLCJcXHNpZ21hIiwwLHsiY3VydmUiOi0yfV0sWzMsMCwiXFxwaSIsMix7ImN1cnZlIjoyfV1d
  \begin{tikzcd}
    {G\times X} && X && {X/G} \\
                &&&& Z
                \arrow["\sigma", curve={height=-12pt}, from=1-1, to=1-3]
                \arrow["\pi"', curve={height=12pt}, from=1-1, to=1-3]
                \arrow[two heads, from=1-3, to=1-5]
                \arrow[from=1-3, to=2-5]
                \arrow["{\exists!}", dashed, from=1-5, to=2-5]
  \end{tikzcd}
\end{equation*}
Considering structure sheaves, we get the following picture:
\begin{equation*}
  % https://q.uiver.app/#q=WzAsNCxbMiwwLCJcXG1hdGhjYWwgT19YIl0sWzQsMCwiXFxtYXRoY2FsIE9fe1gvR30iXSxbNCwxLCJcXG1hdGhjYWwgT19aIl0sWzAsMCwiXFxtYXRoY2FsIE9fR1xcb3RpbWVzXFxtYXRoY2FsIE9fWCJdLFsxLDAsIiIsMix7InN0eWxlIjp7InRhaWwiOnsibmFtZSI6Imhvb2siLCJzaWRlIjoiYm90dG9tIn19fV0sWzIsMF0sWzIsMSwiIiwyLHsic3R5bGUiOnsiYm9keSI6eyJuYW1lIjoiZGFzaGVkIn19fV0sWzAsMywiXFxzaWdtYV5cXHNoYXJwIiwyLHsiY3VydmUiOjJ9XSxbMCwzLCJcXHBpXlxcc2hhcnAiLDAseyJjdXJ2ZSI6LTJ9XV0=
  \begin{tikzcd}
    {\mathcal O_G\otimes\mathcal O_X} && {\mathcal O_X} && {\mathcal O_{X/G}} \\
                                      &&&& {\mathcal O_Z}
                                      \arrow["{\sigma^\sharp}"', curve={height=12pt}, from=1-3, to=1-1]
                                      \arrow["{\pi^\sharp}", curve={height=-12pt}, from=1-3, to=1-1]
                                      \arrow[hook', from=1-5, to=1-3]
                                      \arrow[from=2-5, to=1-3]
                                      \arrow[dashed, from=2-5, to=1-5]
  \end{tikzcd}
\end{equation*}

Now we have a good idea what the subsheaf
$\mathcal O_{X/G}\inc\mathcal O_X$ should be:
\begin{definition}
  Let $\sigma : G\times X \to X$ be an action of an affine algebraic
  $G$ on a scheme $X$. The \emph{ring of invariants}
  on an affine open $U\subseteq X$ is the subring of $\mathcal O_X(U)$
  given by
  \begin{align*}
    \mathcal O_X^G(U) :=
    \left\lbrace{f \in \mathcal O_X(U) : \sigma^\sharp(f) = 1 \otimes f}\right\rbrace.
  \end{align*}
  Thus we have a sheaf $\mathcal O_X^G$ on $X$.
\end{definition}

\begin{example}
  \missingexample
\end{example}

\subsection{Quotients}

Consider a group scheme $G$ acting on $X$. What do we mean by a
quotient of $X$ by the $G$-action? We are looking for a $G$-invariant
surjective map $f : X\surj Y$ such that $Y$ contains as much of th
geometrical information of $X$ as possible.

\todo{explain better why these conditions are relevant}

\begin{definition}[{\cite[Definition 4.2.2]{huybrechts2010}}]
  A \emph{good quotient} of $X$ by $G$-action is a map
  $f : X\surj Y$ such that the following hold:
  \todo{fix definition}
  \begin{enumerate}
    \item $f$ is affine.
    \item $f$ is surjective and $Y$ has the quotient topology.
    \item The map $f^\sharp : \mathcal O_Y \to \mathcal O_X$
      is an isomorphism $\mathcal O_Y \cong \mathcal O_X^G$.
    \item For every closed $G$-invariant $V\subseteq X$,
      $f(V)$ is closed in $Y$.
    \item For all disjoint closed $G$-invariant $V,V'\subseteq X$,
      $f(V)$ and $f(V')$ are disjoint.
  \end{enumerate}
\end{definition}

\begin{definition}
  geometric quotient
\end{definition}

\subsection{Reductive Groups}

\missingsection

\subsection{Of Affine Schemes}

\missingsection

\subsection{By Linear Actions}

\missingsection

\subsection{By Linearised Actions}

\missingsection

\chapter{Analytification}

\missingsection

\chapter{Future Work}

\missingsection

\pagebreak
\renewcommand{\bibname}{References}
\addcontentsline{toc}{chapter}{References}
\printbibliography

\end{document}
